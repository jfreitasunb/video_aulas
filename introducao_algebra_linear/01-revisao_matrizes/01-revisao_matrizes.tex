%!TEX program = xelatex


\def\autor{Jos\'e Ant\^onio O. Freitas}
\def\instituto{MAT-UnB}

\documentclass{beamer}
\usetheme{Madrid}
\usecolortheme{beaver}
% \mode<presentation>
\usepackage{caption}
\usepackage{textpos}
\usepackage{amssymb}
\usepackage{amsmath,amsfonts,amsthm,amstext}
\usepackage[brazil]{babel}
% \usepackage[latin1]{inputenc}
\usepackage{graphicx}
\graphicspath{{/home/jfreitas/GitHub_Repos/video_aulas/logo/}{D:/Dropbox/imagens-latex/}}
\usepackage{enumitem}
\usepackage{multicol}
\usepackage{answers}
\usepackage{tikz,ifthen}
\usetikzlibrary{lindenmayersystems}
\usetikzlibrary[shadings]
\newtheorem{definicao}{Defini\c{c}\~ao}[section]
\newtheorem{definicoes}{Defini\c{c}\~oes}[section]
\newtheorem{exemplo}{Exemplo}[section]
\newtheorem{exemplos}{Exemplos}[section]
\newtheorem{exercicio}{Exerc{\'\i}cio}
\newtheorem{observacao}{Observa{\c c}{\~a}o:}[section]
\newtheorem{observacoes}{Observa{\c c}{\~o}es:}[section]
\newtheorem*{solucao}{Solu{\c c}{\~a}o:}
\newtheorem{proposicao}{Proposi\c{c}\~ao}
\newtheorem{lema}{Lema}
\newtheorem{teorema}{Teorema}
\newtheorem{corolario}{Corol\'ario}
\newenvironment{prova}[1][Prova]{\noindent\textbf{#1:} }{\qedsymbol}%{\ \rule{0.5em}{0.5em}}
\newcommand{\nsub}{\varsubsetneq}
\newcommand{\vaz}{\emptyset}
\newcommand{\im}{{\rm Im\,}}
\newcommand{\sub}{\subseteq}
\newcommand{\n}{\mathbb{N}}
\newcommand{\z}{\mathbb{Z}}
\newcommand{\rac}{\mathbb{Q}}
\newcommand{\real}{\mathbb{R}}
\newcommand{\complex}{\mathbb{C}}
\newcommand{\cp}[1]{\mathbb{#1}}
\newcommand{\ch}{\mbox{\textrm{car\,}}\nobreak}
\newcommand{\vesp}[1]{\vspace{ #1  cm}}
\newcommand{\compcent}[1]{\vcenter{\hbox{$#1\circ$}}}
\newcommand{\comp}{\mathbin{\mathchoice
{\compcent\scriptstyle}{\compcent\scriptstyle}
{\compcent\scriptscriptstyle}{\compcent\scriptscriptstyle}}}

\title{Matrizes}
\author[\autor]{\autor}
%\institute[\instituto]{\instituto}
\date{}

\begin{document}
    \begin{frame}
        \maketitle
    \end{frame}

    %\logo{\includegraphics[scale=.1]{logo-MAT.png}\vspace*{8.5cm}}

    \begin{frame}

        \begin{observacao}
            Vamos utilizar os seguintes conjuntos numéricos:
            \begin{enumerate}[label={\arabic*})]
                \item O conjunto dos números racionais: $\rac$

                \item O conjunto dos números reais: $\real$

                \item O conjunto dos números complexos: $\complex = \{a + bi \mid a, b \in \real,\ i^2 = -1\}$.
            \end{enumerate}
        \end{observacao}

        \begin{teorema}
            O conjunto dos reais, $\real$, com as operações de soma e multiplicação usual satisfaz as seguintes propriedades:
            \begin{enumerate}[label={\roman*})]
                \seti
                \item Para todos $x$, $y \in \real$ temos: $x + y = y + x$.

                \item Para todos $x$, $y$, $z \in \real$ temos: $(x + y) + z = x + (y + z)$.

                \item Existe 0 em $\real$ tal que $0 + x = x$, para todo $x \in \real$.

                \item Para cada $x \in \real$, existe um único $-x \in \real$ tal que
                    $x + (-x) = 0$.
            \end{enumerate}
        \end{teorema}
    \end{frame}

    \begin{frame}
      \begin{teorema}
            \begin{enumerate}[label={\roman*})]
              \conti
              \item Para todos $x$, $y \in \real$ vale: $xy = yx$.

              \item Para todos $x$, $y$, $z \in \real$ vale: $(xy)z = x(yz)$.

              \item Existe 1 em $\real$ tal que $1\cdot x = x$, para todo $x \in \real$.

              \item Para cada $x \in \real$, $x \ne 0$, existe $x^{-1} \in \real$ tal
                que $x\cdot x^{-1} = 1$.
            \end{enumerate}
            Com essas propriedades dizemos que a terna $(\real, +, \cdot)$ é um corpo ou simplesmente que $\real$ é um corpo.
      \end{teorema}

      \begin{observacao}
        As propriedades do teorema anterior também são verdadeiras nas ternas
        $(\rac, +, \cdot)$ e $(\complex, +, \cdot)$, onde para o caso dos
        complexos temos:
        \begin{align*}
          (a + bi) + (c + di) &= (a + c) + (b + d)i\\
          (a + bi)\cdot(c + di) &= (ad + bc) + (ac - bd)i.
        \end{align*}
        Assim $(\rac, +, \cdot)$ e $(\complex, +, \cdot)$ também são exemplos de
        corpos.
      \end{observacao}
    \end{frame}

    \begin{frame}
      \begin{definicao}
        Uma matriz $A$ com entrandas num corpo $\mathbb{K}$, é uma tabela contendo $m\cdot n$ números de $\mathbb{K}$ organizados em $m$ linhas e $n$ colunas, onde $m$ e $n$ são números naturais com $m$, $n \ge 1$ e escrita como
            \begin{displaymath}
                A = \begin{pmatrix}
                a_{11} & a_{12} & \cdots & a_{1n}\\
                a_{21} & a_{22} & \cdots & a_{2n}\\
                \vdots & \vdots & \cdots & \vdots\\
                a_{m1} & a_{m2} & \cdots & a_{mn}
            \end{pmatrix}\end{displaymath}
        ou
        \begin{displaymath}
          A = \begin{bmatrix}
            a_{11} & a_{12} & \cdots & a_{1n}\\
            a_{21} & a_{22} & \cdots & a_{2n}\\
            \vdots & \vdots & \cdots & \vdots\\
            a_{m1} & a_{m2} & \cdots & a_{mn}
          \end{bmatrix}.
        \end{displaymath}

      \end{definicao}
    \end{frame}

    \begin{frame}
      A $i$-\textbf{ésima linha} de $A$ é
      \[
        \begin{bmatrix} a_{i1} & a_{i2} & \cdots & a_{in} \end{bmatrix}
      \]
      para $i = 1$, \dots, $m$.

      Agora a $j$-\textbf{ésima coluna} de $A$ é
      \[
        \begin{bmatrix}
          a_{1j}\\
          a_{2j}\\
          \vdots\\
          a_{mj}
        \end{bmatrix}
      \]
      para $j=1$, \dots, $n$.
    \end{frame}

    \begin{frame}

      Uma matriz $A$ com $m$ \textbf{linhas} e $n$ \textbf{colunas} é dita possuir \textbf{ordem} ou \textbf{tamanho} $m \times n$.

      Usaremos também a notação
      \[
        A = (a_{ij})_{m \times n}
      \]
      para denotar uma matriz $A$ de $m$ linhas por $n$ colunas.

      Nesse caso diremos que $a_{ij}$ ou $[A]_{ij}$ é o \textbf{elemento} ou a \textbf{entrada} na posição $i$, $j$ da matriz A.

      Quando $m = n$, diremos que $A$ é uma \textbf{matriz quadrada de ordem} $n$ e que os elementos $a_{11}$, $a_{22}$, \dots, $a_{nn}$ formam a \textbf{diagonal principal} de $A$.
    \end{frame}

    \begin{frame}
      \begin{exemplos}
        Considere as seguintes matrizes:
        \begin{align*}
          A = \begin{bmatrix}
            1 & 2\\
            3 & 4
          \end{bmatrix},
          B = \begin{bmatrix}
            -3 & 0\\
            0 & 0
          \end{bmatrix},
          C = \begin{bmatrix}
            1 & 2 & -\sqrt{2}\\
            1/3 & 4\pi & -3/2
          \end{bmatrix},\\\\
          D = \begin{bmatrix}
            1 & 2 & 0 & -3 & 8
          \end{bmatrix},
          E = \begin{bmatrix}
            12\\
            3\\
            0,3\\
            2,5
          \end{bmatrix},
          F = \begin{bmatrix}
            1
          \end{bmatrix}
        \end{align*}
      \end{exemplos}
    \end{frame}

    \begin{frame}
    \begin{definicao}
      Uma matriz que possui uma única linha, isto é, uma matriz de tamanho $1 \times n$ é chamada de \textbf{matriz linha}. Uma matriz que possui uma única coluna, isto é, uma matriz de tamanho $m \times 1$  é chamada de uma \textbf{matriz coluna}.
    \end{definicao}

    \begin{definicao}
      Dadas duas matrizes $A = (a_{ij})_{m \times n}$ e $B = (b_{ij})_{p \times q}$, dizemos que $A$ e $B$ são \textbf{iguais}
      e escrevemos $A = B$ se:
      \begin{enumerate}[label={\arabic*})]
        \item $m = p$ e $n = q$,
        \item $a_{ij} = b_{ij}$ para $i = 1$, \dots, $m$ e $j = 1$, \dots, $n$.
      \end{enumerate}
    \end{definicao}
  \end{frame}

  \begin{frame}
    \begin{observacao}
      Denotaremos o conjunto de todas as matrizes com entradas em $\real$ por $M_{m \times n}(\real)$. Assim por exemplo,
      \[
          M_{2 \times 3}(\real) = \left\{\begin{bmatrix}a_{11} & a_{12} & a_{13}\\ a_{21} & a_{22} & a_{23}\end{bmatrix} \mid a_{ij} \in \real,\ i,j = 1,2,3\right\}.
      \]

      Quando $m = n$ vamos escrever $M_{n \times n}(\real) = M_n(\real)$.

    \end{observacao}
  \end{frame}

  \begin{frame}
    \begin{definicao}
      Dadas duas matrizes de mesmo tamanho, $A = (a_{ij})_{m \times n}$ e $B = (b_{ij})_{m \times n}$ definimos a soma de $A$ com $B$
      como sendo a matriz
      \[
        C = A + B = (c_{ij})_{m \times n}
      \]
      onde
      \[
        c_{ij} = a_{ij} + b_{ij}
      \]
      para $i = 1$, \dots, $m$ e $j = 1$, \dots, $n$.
    \end{definicao}
  \end{frame}

  \begin{frame}
    \begin{exemplos}
      Considere as matrizes:
      \[
        A = \begin{bmatrix}
          \phantom{-}1 & 3\\
          \phantom{-}2 & 4\\
          -3 & 0
        \end{bmatrix},
       B = \begin{bmatrix}
          -2 & \phantom{-}0\\
          \phantom{-}1 & \phantom{-}3\\
          \phantom{-}5 & -4
        \end{bmatrix}.
    \]
    Denotando por $C$ a soma de $A$ e $B$, então
    \begin{align*}
     C = A + B = \begin{bmatrix}
       1 + (-2) & 3 + 0\\
       2 + 1 & 4 + 3\\
       -3 + 5 & 0 + (-4)
     \end{bmatrix} = \begin{bmatrix}
      -1 & \phantom{-}3\\
      \phantom{-}3 & \phantom{-}7\\
      \phantom{-}2 & -4
     \end{bmatrix}
    \end{align*}
    \end{exemplos}
  \end{frame}

  \begin{frame}
    \begin{definicao}
      Dada uma matriz $A = (a_{ij})_{m \times n}$. A \textbf{multiplicação de $A$ por um escalar(número) $\alpha$} é definida
      como sendo a matriz
      \[
        B = \alpha A
      \]
      obtida multiplicando cada entrada da matriz $A$ pelo número $\alpha$, ou seja,
      \[
        b_{ij} = \alpha a_{ij}
      \]
      para $i = 1$, \dots, $m$ e $j = 1$, \dots, $n$. Nesse caso dizemos que a matriz $B$ é um \textbf{múltiplo escalar} da matriz $A$.
    \end{definicao}
  \end{frame}

  \begin{frame}
    \begin{exemplo}
      O produto da matriz
      \[A = \begin{bmatrix}
        -1 & \phantom{-}3\\
        \phantom{-}3 & \phantom{-}7\\
        \phantom{-}2 & -4
      \end{bmatrix}
      \]
      pelo escalar $-2$ é dada por
      \[
        -2A = \begin{bmatrix}
            (-2)(-1) & (-2)(-3)\\
            (-2)3 & (-2)7\\
            (-2)2 & (-2)(-4)
        \end{bmatrix} =
        \begin{bmatrix}
          \phantom{-}2 & \phantom{-}6\\
          -6 & -14\\
          -4 & \phantom{-}8
        \end{bmatrix}.
      \]
    \end{exemplo}
  \end{frame}

  \begin{frame}
    \begin{definicao}
      Dadas duas matrizes $A$ e $B$ tais que $A = (a_{ij})_{m \times p}$ e $B = (b_{ij})_{p \times n}$, isto é,
      \textbf{o número de colunas de $A$ é igual ao número de linhas de $B$} podemos definir o \textbf{produto de $A$ por $B$}
      denotado por
      \[
        C = AB.
      \]
      A matriz $C$ será uma matriz $m \times n$ obtida da seguinte forma
      \begin{equation}\label{multiplicacao_de_matrizes}
        c_{ij} = a_{i1}b_{1j} + a_{i2}b_{2j} + \cdots + a_{ip}b_{pj}
      \end{equation}
      para $i = 1$, \dots, $m$ e $j = 1$, \dots, $n$.
    \end{definicao}
  \end{frame}


  \begin{frame}
    A equação \eqref{multiplicacao_de_matrizes} está dizendo que para obtermos o elemento na posição $i$, $j$ da matriz $C$ precisamos
    realizar a soma dos produtos dos elementos da $i$-ésima linha de $A$ pelos elementos correspondentes da $j$-ésima coluna de $B$:
    \[
      \begin{bmatrix}
        c_{11} & \cdots & c_{1n}\\
        \vdots & c_{ij} & \vdots\\
        c_{m1} & \cdots & c_{mn}
      \end{bmatrix} = \begin{bmatrix}
        a_{11} & a_{12} & \cdots & a_{1n}\\
        \vdots & \vdots & \vdots & \vdots\\
        a_{i1} & a_{i2} & \cdots & a_{ip}\\
        \vdots & \vdots & \vdots & \vdots\\
        a_{m1} & \cdots & \cdots & a_{mn}
      \end{bmatrix} \begin{bmatrix}
        b_{11} & \cdots & b_{1j} & \cdots & b_{1n}\\
        b_{21} & \cdots & b_{2j} & \cdots & b_{2n}\\
        \vdots & \cdots & \vdots & \cdots & \vdots\\
        b_{p1} & \cdots & b_{pj} & \cdots & b_{pn}
      \end{bmatrix}
    \]
  \end{frame}

  \begin{frame}
    \begin{exemplo}
      Considere as seguintes matrizes:
      \[
        A = \begin{bmatrix}
          1 & 3\\
          2 & 4\\
          -3 & 0
        \end{bmatrix},\quad
        B = \begin{bmatrix}
          -2 & 1 & 0\\
          0 & 3 & 1\\
          1 & 2 & 0
        \end{bmatrix}
      \]
      Nessse caso não podemos fazer o produto de $A$ por $B$ pois $A$ é uma matriz $3 \times 2$ e $B$ é uma matriz $3 \times 3$.
      Mas podemos realizar o produto $BA$. Nesse caso fazendo $C = BA$ obtemos
      \begin{align*}
        C &= BA = \begin{bmatrix}
          -2 & 1 & 0\\
          \phantom{-}0 & 3 & 1\\
          \phantom{-}1 & 2 & 0
        \end{bmatrix}\begin{bmatrix}
          \phantom{-}1 & 3\\
          \phantom{-}2 & 4\\
          -3 & 0
        \end{bmatrix} \\
        &= \begin{bmatrix}
          (-2)1 + 1\cdot 1 + 0(-3) & (-2)3 + 1\cdot 3 + 0\cdot 3\\
          0\cdot 1 + 3\cdot 2 + 1(-3) & 0\cdot 3 + 3\cdot 4 + 1\cdot 0\\
          1 \cdot 1 + 2 \cdot 2 + 0(-3) & 1 \cdot 3 + 2 \cdot 4 + 0\cdot 0
        \end{bmatrix}
        = \begin{bmatrix}
          -1 & -3\\
          \phantom{-}3 & \phantom{-}12\\
          \phantom{-}5 & \phantom{-}11
        \end{bmatrix}
      \end{align*}
    \end{exemplo}
  \end{frame}

  \begin{frame}
    \begin{definicao}
      A \textbf{transposta} de uma matriz $A = (a_{ij})_{m \times n}$ é definida como sendo a matriz $n \times m$
      \[
        B = A^t
      \]
      obtida trocando-se as linhas com as colunas de $A$, ou seja,
      \[
        b_{ij} = a_{ji}
      \]
      para $i = 1$, \dots, $n$ e $j = 1$, \dots, $m$.
    \end{definicao}
  \end{frame}

  \begin{frame}
    \begin{exemplos}
      As transpostas das matrizes
      \begin{align*}
        A &= \begin{bmatrix}
            -3 & 1\\
            0 & 0
          \end{bmatrix},
          B = \begin{bmatrix}
            1 & 2 & -\sqrt{2}\\
            1/3 & 4\pi & -3/2
          \end{bmatrix},
          C = \begin{bmatrix}
            1 & 2 & 0 & -3 & 8
          \end{bmatrix},\\
          &D = \begin{bmatrix}
            12\\
            3\\
            0,3\\
            2,5
          \end{bmatrix},
          E = \begin{bmatrix}
            1
          \end{bmatrix}
      \end{align*}
      são as matrizes
      \begin{align*}
        A^t &= \begin{bmatrix}
            -3 & 0\\
            1 & 0
          \end{bmatrix},
          B^t = \begin{bmatrix}
            1 & 1/3\\
            2 & 4\pi\\
            -\sqrt{2} & -3/2
          \end{bmatrix},
          C^t= \begin{bmatrix}
            1 \\ 2 \\ 0 \\ -3 \\ 8
          \end{bmatrix},\\
          D^t &= \begin{bmatrix}
            12 & 3 & 0,3 & 2,5
          \end{bmatrix},
          E^t = \begin{bmatrix}
            1
          \end{bmatrix}.
      \end{align*}
    \end{exemplos}
  \end{frame}

  \begin{frame}
   \begin{teorema}
      Sejam $A$, $B$ e $C$ matrizes com tamanhos apropriados, $\alpha$ e $\beta$ escalares. Então as seguintes propriedades são válidas:
      \begin{enumerate}[label={\arabic*})]
        \item \textit{(Comutatividade)} $A + B = B + A$

        \item \textit{(Associatividade)} $(A + B) + C = A + (B + C)$

        \item \textit{(Elemento Neutro)} A matriz denotada por $\overline{0}$, $m \times n$, definida por $\overline{0} = (0)_{m \times n}$, para todo $i = 1$, \dots, $m$
        e todo $j = 1$, \dots, $m$ é tal que
        \[
          A + \overline{0} = A
        \]
        para toda matriz $A$, $m \times n$. Tal matriz é chamada de \textbf{matriz nula} $m \times n$.
        \seti
      \end{enumerate}
    \end{teorema}
  \end{frame}

  \begin{frame}
    \begin{teorema}
      \begin{enumerate}[label={\arabic*)}]
        \conti

        \item \textit{(Elemento simétrico ou oposto)} Para cada matriz $A = (a_{ih})$, existe uma única matriz $-A$, definida por
          \[ -A = (-a_{ij})\]
        tal que
        \[ A + (-A) = \overline{0}.\]

        \item \textit{(Associatividade)} $(\alpha\beta)A = \alpha(\beta A)$

        \item \textit{(Distributividade)} $(\alpha + \beta)A = \alpha A + \beta A$

        \item \textit{(Distributividade)} $\alpha(A + B) = \alpha A + \alpha B$

        \item \textit{(Associatividade)} $(AB)C = A(BC)$
        \seti
      \end{enumerate}
    \end{teorema}
  \end{frame}

  \begin{frame}
    \begin{teorema}
      \begin{enumerate}[label={\arabic*)}]
        \conti

        \item \textit{(Elemento Neutro)} Para cada inteiro positivo $p$ a matrix $p \times p$ dada por
          \[
            I_p = \begin{bmatrix}
              1 & 0 & \cdots & 0\\
              0 & 1 & \cdots & 0\\
              \vdots & & \ddots & \vdots\\
              0 & 0 & \cdots & 1
            \end{bmatrix}
          \]
        chamada de \textbf{matriz identidade} e é tal que
        \[
          AI_n = I_mA = A
        \]
        para toda matriz $A = (a_{ij})_{m \times n}$.
      \seti
      \end{enumerate}
    \end{teorema}
  \end{frame}

  \begin{frame}
    \begin{teorema}
      \begin{enumerate}[label={\arabic*)}]
        \conti

        \item \textit{(Distributividade)} $A(B + C) = AB + AC$ e $(B + C)A = BA + CA$

        \item $\alpha(AB) = (\alpha A)B = A(\alpha B)$

        \item $(A^t)^t = A$

        \item $(A + B)^t = A^t + B^t$

        \item $(\alpha A)^t = \alpha A^t$

        \item $(AB)^t = B^tA^t$
      \end{enumerate}
    \end{teorema}
  \end{frame}
\end{document}
