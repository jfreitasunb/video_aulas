%!TEX program = xelatex


\def\autor{Jos\'e Ant\^onio O. Freitas}
\def\instituto{MAT-UnB}

\documentclass{beamer}
\usetheme{Madrid}
\usecolortheme{beaver}
% \mode<presentation>
\usepackage{caption}
\usepackage{textpos}
\usepackage{amssymb}
\usepackage{amsmath,amsfonts,amsthm,amstext}
\usepackage[brazil]{babel}
% \usepackage[latin1]{inputenc}
\usepackage{graphicx}
\graphicspath{{/home/jfreitas/GitHub_Repos/video_aulas/logo/}{D:/Dropbox/imagens-latex/}}
\usepackage{enumitem}
\usepackage{multicol}
\usepackage{answers}
\usepackage{tikz,ifthen}
\usetikzlibrary{lindenmayersystems}
\usetikzlibrary[shadings]
\newtheorem{definicao}{Defini\c{c}\~ao}[section]
\newtheorem{definicoes}{Defini\c{c}\~oes}[section]
\newtheorem{exemplo}{Exemplo}[section]
\newtheorem{exemplos}{Exemplos}[section]
\newtheorem{exercicio}{Exerc{\'\i}cio}
\newtheorem{observacao}{Observa{\c c}{\~a}o:}[section]
\newtheorem{observacoes}{Observa{\c c}{\~o}es:}[section]
\newtheorem*{solucao}{Solu{\c c}{\~a}o:}
\newtheorem{proposicao}{Proposi\c{c}\~ao}
\newtheorem{lema}{Lema}
\newtheorem{teorema}{Teorema}
\newtheorem{corolario}{Corol\'ario}
\newenvironment{prova}[1][Prova]{\noindent\textbf{#1:} }{\qedsymbol}%{\ \rule{0.5em}{0.5em}}
\newcommand{\nsub}{\varsubsetneq}
\newcommand{\vaz}{\emptyset}
\newcommand{\im}{{\rm Im\,}}
\newcommand{\sub}{\subseteq}
\newcommand{\n}{\mathbb{N}}
\newcommand{\z}{\mathbb{Z}}
\newcommand{\rac}{\mathbb{Q}}
\newcommand{\real}{\mathbb{R}}
\newcommand{\complex}{\mathbb{C}}
\newcommand{\cp}[1]{\mathbb{#1}}
\newcommand{\ch}{\mbox{\textrm{car\,}}\nobreak}
\newcommand{\vesp}[1]{\vspace{ #1  cm}}
\newcommand{\compcent}[1]{\vcenter{\hbox{$#1\circ$}}}
\newcommand{\comp}{\mathbin{\mathchoice
{\compcent\scriptstyle}{\compcent\scriptstyle}
{\compcent\scriptscriptstyle}{\compcent\scriptscriptstyle}}}

\title{Matrizes}
\author[\autor]{\autor}
%\institute[\instituto]{\instituto}
\date{}

\begin{document}
    \begin{frame}
        \maketitle
    \end{frame}

    %\logo{\includegraphics[scale=.1]{logo-MAT.png}\vspace*{8.5cm}}

    \begin{frame}

      \begin{definicao}
        Uma matriz $A$, $m \times n$ é uma tabela contendo $mn$ números organizados em $m$ linhas e $n$ colunas sendo escrita como
        \begin{displaymath}
         A = \begin{pmatrix}
           a_{11} & a_{12} & \cdots & a_{1n}\\
           a_{21} & a_{22} & \cdots & a_{2n}\\
           \vdots & \vdots & \cdots & \vdots\\
           a_{m1} & a_{m2} & \cdots & a_{mn}
         \end{pmatrix}\end{displaymath}
         ou
        \begin{displaymath}
          A = \begin{bmatrix}
            a_{11} & a_{12} & \cdots & a_{1n}\\
            a_{21} & a_{22} & \cdots & a_{2n}\\
            \vdots & \vdots & \cdots & \vdots\\
            a_{m1} & a_{m2} & \cdots & a_{mn}
          \end{bmatrix}.
        \end{displaymath}
        
      \end{definicao}
    \end{frame}

    \begin{frame}
      A $i$-\textbf{ésima linha} de $A$ é
      \[
        \begin{bmatrix} a_{i1} & a_{i2} & \cdots & a_{in} \end{bmatrix}
      \]
      para $i = 1$, \dots, $m$.

      Agora a $j$-\textbf{ésima coluna} de $A$ é
      \[
        \begin{bmatrix}
          a_{1j}\\
          a_{2j}\\
          \vdots\\
          a_{mj}
        \end{bmatrix}
      \]
      para $j=1$, \dots, $n$.
    \end{frame}

    \begin{frame}
      Usaremos também a notação
      \[
        A = (a_{ij})_{m \times n}
      \]
      para denotar uma matriz $A$ de $m$ linhas por $n$ colunas. 

      Nesse caso diremos que $a_{ij}$ ou $[A]_{ij}$ é o \textbf{elemento} ou a \textbf{entrada} na posição $i$, $j$ da matriz A.

      Quando $m = n$, diremos que $A$ é uma \textbf{matriz quadrada de ordem} $n$ e que os elementos $a_{11}$, $a_{22}$, \dots, $a_{nn}$ 
      formam a \textbf{diagonal principal} de $A$.
    \end{frame}

    \begin{frame}
      \begin{exemplos}
        Considere as seguintes matrizes:
        \begin{eqnarray*}
          A = \begin{bmatrix}
            1 & 2\\
            3 & 4
          \end{bmatrix},
          B = \begin{bmatrix}
            -3 & 0\\
            0 & 0
          \end{bmatrix},
          C = \begin{bmatrix}
            1 & 2 & -\sqrt{2}\\
            1/3 & 4\pi & -3/2
          \end{bmatrix},\\\\
          D = \begin{bmatrix}
            1 & 2 & 0 & -3 & 8
          \end{bmatrix},
          E = \begin{bmatrix}
            12\\
            3\\
            0,3\\
            2,5
          \end{bmatrix},
          F = \begin{bmatrix}
            1
          \end{bmatrix}
        \end{eqnarray*} 
      \end{exemplos}
    \end{frame}

    \begin{frame}
    \begin{definicao}
      Uma matriz que possui uma única linha é chamada de \textbf{matriz linha}, e um matriz que possui uma única coluna
      é chamada de \textbf{matriz coluna}.
    \end{definicao}

    \begin{definicao}
      Dadas duas matrizes $A = (a_{ij})_{m \times n}$ e $B = (b_{ij})_{p \times q}$, dizemos que $A$ e $B$ são \textbf{iguais}
      e escrevemos $A = B$ se:
            \begin{enumerate}[label={\arabic*})]
              \item $m = p$ e $n = q$,
              \item $a_{ij} = b_{ij}$ para $i = 1$, \dots, $m$ e $j = 1$, \dots, $n$.
            \end{enumerate}se:
                  \begin{enumerate}[label={\arabic*})]
                    \item $m = p$ e $n = q$,
                    \item $a_{ij} = b_{ij}$ para $i = 1$, \dots, $m$ e $j = 1$, \dots, $n$.
                  \end{enumerate}se:
      \begin{enumerate}[label={\arabic*})]
        \item $m = p$ e $n = q$,
        \item $a_{ij} = b_{ij}$ para $i = 1$, \dots, $m$ e $j = 1$, \dots, $n$.
      \end{enumerate}
    \end{definicao}
  \end{frame}
  
  \begin{frame}
    \begin{definicao}
      Dadas duas matrizes de mesmo tamanho, $A = (a_{ij})_{m \times n}$ e $B = (b_{ij})_{m \times n}$ definimos a soma de $A$ com $B$ 
      como sendo a matriz
      \[
        C = A + B = (c_{ij})_{m \times n}
      \]
      onde
      \[
        c_{ij} = a_{ij} + b_{ij}
      \]
      para $i = 1$, \dots, $m$ e $j = 1$, \dots, $n$.
    \end{definicao}
  \end{frame}

  \begin{frame}
    \begin{exemplos}
      Considere as matrizes:
      \[
        A = \begin{bmatrix}
          \phantom{-}1 & 3\\
          \phantom{-}2 & 4\\
          -3 & 0
        \end{bmatrix}, 
       B = \begin{bmatrix}
          -2 & \phantom{-}0\\
          \phantom{-}1 & \phantom{-}3\\
          \phantom{-}5 & -4 
        \end{bmatrix}.
    \]
    Denotando por $C$ a soma de $A$ e $B$, então
    \begin{eqnarray*}
     C = A + B = \begin{bmatrix}
       1 + (-2) & 3 + 0\\
       2 + 1 & 4 + 3\\
       -3 + 5 & 0 + (-4)
     \end{bmatrix} = \begin{bmatrix}
      -1 & \phantom{-}3\\
      \phantom{-}3 & \phantom{-}7\\
      \phantom{-}2 & -4
     \end{bmatrix} 
    \end{eqnarray*}
    \end{exemplos}
  \end{frame}

  \begin{frame}
    \begin{definicao}
      Dada uma matriz $A = (a_{ij})_{m \times n}$. A \textbf{multiplicação de $A$ por um escalar(número) $\alpha$} é definida
      como sendo a matriz
      \[
        B = \alpha A 
      \]
      obtida multiplicando cada entrada da matriz $A$ pelo número $\alpha$, ou seja,
      \[
        b_{ij} = \alpha a_{ij}
      \]
      para $i = 1$, \dots, $m$ e $j = 1$, \dots, $n$. Nesse caso dizemos que a matriz $B$ é um \textbf{múltiplo escalar} da matriz $A$.
    \end{definicao}
  \end{frame}

  \begin{frame}
    \begin{exemplo}
      O produto da matriz
      \[A = \begin{bmatrix}
        -1 & \phantom{-}3\\
        \phantom{-}3 & \phantom{-}7\\
        \phantom{-}2 & -4
      \end{bmatrix}
      \]
      pelo escalar $-2$ é dada por
      \[
        -2A = \begin{bmatrix}
            (-2)(-1) & (-2)(-3)\\
            (-2)3 & (-2)7\\
            (-2)2 & (-2)(-4)
        \end{bmatrix} =
        \begin{bmatrix}
          \phantom{-}2 & \phantom{-}6\\
          -6 & -14\\
          -4 & \phantom{-}8
        \end{bmatrix}.
      \]
    \end{exemplo}
  \end{frame}
\end{document}
