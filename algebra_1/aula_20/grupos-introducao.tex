%!TEX program = xelatex
\def\ano{2020}
\def\semestre{1}
\def\disciplina{\'Algebra 1}
\def\turma{C}
\def\autor{Jos\'e Ant\^onio O. Freitas}
\def\instituto{MAT-UnB}

\documentclass{beamer}
\usetheme{Madrid}
\usecolortheme{beaver}
% \mode<presentation>
\usepackage{caption}
\usepackage{textpos}
\usepackage{amssymb}
\usepackage{amsmath,amsfonts,amsthm,amstext}
\usepackage[brazil]{babel}
% \usepackage[latin1]{inputenc}
\usepackage{graphicx}
\graphicspath{{/home/jfreitas/GitHub_Repos/video_aulas/logo/}{D:/Dropbox/imagens-latex/}}
\usepackage{enumitem}
\usepackage{multicol}
\usepackage{answers}
\usepackage{tikz,ifthen}
\usetikzlibrary{lindenmayersystems}
\usetikzlibrary[shadings]
\newtheorem{definicao}{Defini\c{c}\~ao}[section]
\newtheorem{definicoes}{Defini\c{c}\~oes}[section]
\newtheorem{exemplo}{Exemplo}[section]
\newtheorem{exemplos}{Exemplos}[section]
\newtheorem{exercicio}{Exerc{\'\i}cio}
\newtheorem{observacao}{Observa{\c c}{\~a}o:}[section]
\newtheorem{observacoes}{Observa{\c c}{\~o}es:}[section]
\newtheorem*{solucao}{Solu{\c c}{\~a}o:}
\newtheorem{proposicao}{Proposi\c{c}\~ao}
\newtheorem{lema}{Lema}
\newtheorem{teorema}{Teorema}
\newtheorem{corolario}{Corol\'ario}
\newenvironment{prova}[1][Prova]{\noindent\textbf{#1:} }{\qedsymbol}%{\ \rule{0.5em}{0.5em}}
\newcommand{\nsub}{\varsubsetneq}
\newcommand{\vaz}{\emptyset}
\newcommand{\im}{{\rm Im\,}}
\newcommand{\sub}{\subseteq}
\newcommand{\n}{\mathbb{N}}
\newcommand{\z}{\mathbb{Z}}
\newcommand{\rac}{\mathbb{Q}}
\newcommand{\real}{\mathbb{R}}
\newcommand{\complex}{\mathbb{C}}
\newcommand{\cp}[1]{\mathbb{#1}}
\newcommand{\ch}{\mbox{\textrm{car\,}}\nobreak}
\newcommand{\vesp}[1]{\vspace{ #1  cm}}
\newcommand{\compcent}[1]{\vcenter{\hbox{$#1\circ$}}}
\newcommand{\comp}{\mathbin{\mathchoice
{\compcent\scriptstyle}{\compcent\scriptstyle}
{\compcent\scriptscriptstyle}{\compcent\scriptscriptstyle}}}

\title{Grupos - Introdu\c{c}\~ao}
\author[\autor]{\autor}
\institute[\instituto]{\instituto}
\date{}

\begin{document}
    \begin{frame}
        \maketitle
    \end{frame}

    \logo{\includegraphics[scale=.1]{logo-MAT.png}\vspace*{8.5cm}}

    \begin{frame}
        \begin{definicao}
            Seja $G \ne \emptyset$ \pause um conjunto no qual est\'a definida uma opera{\c c}{\~a}o bin{\'a}ria $*$ \pause tal que:\pause
            \begin{enumerate}
                \item[i)] Para todos $x$, $y$, $z\in G$:\pause
                \[
                    (x*y)*z \pause = x*(y*z).\pause
                \]

                \item[ii)] Existe $e \in G$ \pause tal que\pause
                \[
                    x*e \pause = x = \pause e*x\pause
                \]
                para todo $x \in G$. \pause Tal elemento $e$ \pause {\'e} chamado de \textbf{elemento neutro} \pause ou \textbf{unidade} \pause de $G$.\pause

            \end{enumerate}
        \end{definicao}
    \end{frame}

    \begin{frame}
        \begin{definicao}
            \begin{enumerate}
                \item[iii)] Para cada $x \in G$, \pause existe $y \in G$ \pause tal que\pause
                \[
                    x*y \pause = e = \pause y*x.\pause
                \]
                O elemento $y$ \pause {\'e} chamado de \textbf{inverso} \pause ou \textbf{oposto} \pause de $x$.\pause
            \end{enumerate}
            Nesse caso dizemos que o par $(G, *)$ \pause \'e um \textbf{grupo}.\pause
        \end{definicao}
    \end{frame}

    \begin{frame}
        \begin{observacao}
            Quando $*$ {\'e} uma ``soma", \pause dizemos que $(G,*)$ {\'e} um \textbf{grupo aditivo}.\pause

            \vspace{.3cm}

            Se $*$ {\'e} uma ``multiplica{\c c}{\~a}o", \pause dizemos que $(G,*)$ {\'e} um \textbf{grupo multiplicativo}.\pause

            \vspace{.3cm}
            Al\'em disso, quando n\~ao houver chance de confus\~ao com rela\c{c}\~ao \`a opera\c{c}\~ao do grupo $(G, *)$ \pause vamos dizer simplesmente que $G$ \'e um grupo.\pause
        \end{observacao}

        \begin{definicao}
            Um grupo $(G,*)$ \pause {\'e} chamado de \textbf{grupo comutativo} \pause ou \textbf{abeliano} \pause quando $*$ {\'e} comutativa, \pause ou seja, quando \pause
            \[
                x*y = \pause y*x \pause
            \]
            para todos $x$, $y \in G$.
        \end{definicao}
    \end{frame}

    \begin{frame}
        \begin{exemplos}
            \begin{enumerate}[label={\arabic*})]
                \item $(\z,+)$ {\'e} um grupo abeliano.\pause

                \vspace{.3cm}

                \item $(\z, \cdot)$ n\~ao \'e grupo.\pause

                \vspace{.3cm}

                \item $(\rac,+)$ {\'e} um grupo abeliano.\pause

                \vspace{.3cm}

                \item $(\rac^*,\cdot)$ {\'e} um grupo abeliano.\pause

                \vspace{.3cm}

                \item $(\real,+)$ {\'e} um grupo abeliano.\pause

                \vspace{.3cm}

                \item $(\real^*,\cdot)$ {\'e} um grupo abeliano.\pause

                \vspace{.3cm}

                \item $(\complex,+)$ {\'e} um grupo abeliano.\pause

                \vspace{.3cm}

                \item $(\complex^*,\cdot)$ {\'e} um grupo abeliano.
                \seti
            \end{enumerate}
        \end{exemplos}
    \end{frame}

    \begin{frame}
        \begin{exemplos}
            \begin{enumerate}[label={\arabic*})]
                \conti

                \item $(\z_m,\oplus)$ {\'e} grupo abeliano.\pause

                \vspace{.3cm}

                \item Considere o conjunto dos n{\'u}meros reais $\mathbb{R}$ \pause com a opera{\c c}{\~a}o $*$ \pause definida por\pause
                \[
                    x*y = \pause x + y - 3\pause
                \]
                para $x$, $y \in \mathbb{R}$. \pause Ent{\~a}o $(\mathbb{R}, *)$ \pause {\'e} um grupo abeliano.\pause

                \vspace{.3cm}

                \item $(\z_m-\{\overline{0}\},\otimes)$ \pause {\'e} grupo?\pause

                \vspace{.3cm}

                \item $(\real, *)$ \pause onde $x * y = y$ \pause para todos $x$, $y \in \real$ \pause \'e grupo?
                \seti
            \end{enumerate}
        \end{exemplos}
    \end{frame}

    \begin{frame}
        \begin{exemplos}
            \begin{enumerate}[label={\arabic*})]
                \conti

                \item Denote por $\mathbb{K}$ \pause um dos conjuntos $\z$, \pause $\rac$, \pause $\real$ \pause ou $\complex$, \pause indistintamente. Seja \pause
                \begin{center}
                    $M_{r \times s}(\mathbb{K}) \pause = \{A \mid $ A \'e uma matriz \pause \\de $r$ linhas \pause por $s$ \pause colunas cujas entradas est\~ao em $\mathbb{K} \pause\}$.
                \end{center}
                Ent\~ao $(M_{r \times s}(\mathbb{K}), +)$ \pause onde $+$ \'e a soma de matrizes \pause \'e um grupo abeliano. \pause

                \vspace{.3cm}

                \item Denote por $\mathbb{K}$ \pause um dos conjuntos $\rac$, \pause $\real$ \pause ou $\complex$, \pause indistintamente. Seja \pause
                \begin{center}
                    $GL_n(\mathbb{K}) \pause = \{A \in M_{n \times n}(\mathbb{K}) \pause \mid \det(A) = 1\}$. \pause
                \end{center}
                Ent\~ao $(GL_n(\mathbb{K}), \cdot)$ \pause onde $\cdot$ \'e a multiplica\c{c}\~ao de matrizes \'e um grupo \pause n\~ao abeliano. \pause
            \end{enumerate}
        \end{exemplos}
    \end{frame}

    \begin{frame}
        \begin{proposicao}
            Seja $(G,*)$ um grupo. \pause Ent\~ao:\pause
            \begin{enumerate}[label={\roman*})]
                \item O elemento neutro de $G$ {\'e} {\'u}nico.\pause

                \vspace{.3cm}

                \item Existe um {\'u}nico inverso para cada $x \in G$.

                \seti
            \end{enumerate}
        \end{proposicao}
    \end{frame}

    \begin{frame}
        \begin{proposicao}
            \begin{enumerate}[label={\roman*})]

                \conti

                \item Para todos $x$, $y \in G$,\pause
                \[
                    (x*y)^{-1} \pause = y^{-1}*x^{-1}.\pause
                \]
                Por indu{\c c}{\~a}o, \pause $x_1$, $x_2$, \dots ,$x_{n-1}$, $x_n \in G$,\pause
                \[
                    (x_1*x_2*\cdots *x_{n-1}*x_{n})^{-1} = \pause x^{-1}_{n}\pause *x^{-1}_{n-1}\pause *\cdots *\pause x^{-1}_2\pause *x^{-1}_1\pause
                \]
                \item Para todo $x \in G$, \pause
                \[
                    (x^{-1})^{-1} \pause = x.
                \]
            \end{enumerate}
        \end{proposicao}
    \end{frame}

\end{document}
