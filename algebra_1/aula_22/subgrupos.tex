%!TEX program = xelatex
\def\ano{2020}
\def\semestre{1}
\def\disciplina{\'Algebra 1}
\def\turma{C}
\def\autor{Jos\'e Ant\^onio O. Freitas}
\def\instituto{MAT-UnB}

\documentclass{beamer}
\usetheme{Madrid}
\usecolortheme{beaver}
% \mode<presentation>
\usepackage{caption}
\usepackage{textpos}
\usepackage{amssymb}
\usepackage{amsmath,amsfonts,amsthm,amstext}
\usepackage[brazil]{babel}
% \usepackage[latin1]{inputenc}
\usepackage{graphicx}
\graphicspath{{/home/jfreitas/GitHub_Repos/video_aulas/logo/}{D:/Dropbox/imagens-latex/}}
\usepackage{enumitem}
\usepackage{multicol}
\usepackage{answers}
\usepackage{tikz,ifthen}
\usetikzlibrary{lindenmayersystems}
\usetikzlibrary[shadings]
\newtheorem{definicao}{Defini\c{c}\~ao}[section]
\newtheorem{definicoes}{Defini\c{c}\~oes}[section]
\newtheorem{exemplo}{Exemplo}[section]
\newtheorem{exemplos}{Exemplos}[section]
\newtheorem{exercicio}{Exerc{\'\i}cio}
\newtheorem{observacao}{Observa{\c c}{\~a}o:}[section]
\newtheorem{observacoes}{Observa{\c c}{\~o}es:}[section]
\newtheorem*{solucao}{Solu{\c c}{\~a}o:}
\newtheorem{proposicao}{Proposi\c{c}\~ao}
\newtheorem{lema}{Lema}
\newtheorem{teorema}{Teorema}
\newtheorem{corolario}{Corol\'ario}
\newenvironment{prova}[1][Prova]{\noindent\textbf{#1:} }{\qedsymbol}%{\ \rule{0.5em}{0.5em}}
\newcommand{\nsub}{\varsubsetneq}
\newcommand{\vaz}{\emptyset}
\newcommand{\im}{{\rm Im\,}}
\newcommand{\sub}{\subseteq}
\newcommand{\n}{\mathbb{N}}
\newcommand{\z}{\mathbb{Z}}
\newcommand{\rac}{\mathbb{Q}}
\newcommand{\real}{\mathbb{R}}
\newcommand{\complex}{\mathbb{C}}
\newcommand{\cp}[1]{\mathbb{#1}}
\newcommand{\ch}{\mbox{\textrm{car\,}}\nobreak}
\newcommand{\vesp}[1]{\vspace{ #1  cm}}
\newcommand{\compcent}[1]{\vcenter{\hbox{$#1\circ$}}}
\newcommand{\comp}{\mathbin{\mathchoice
{\compcent\scriptstyle}{\compcent\scriptstyle}
{\compcent\scriptscriptstyle}{\compcent\scriptscriptstyle}}}

\title{Subgrupos}
\author[\autor]{\autor}
\institute[\instituto]{\instituto}
\date{}

\begin{document}
    \begin{frame}
        \maketitle
    \end{frame}

    \logo{\includegraphics[scale=.1]{logo-MAT.png}\vspace*{8.5cm}}

    \begin{frame}
        \begin{definicao}
            Seja $(G,*)$ um grupo. \pause Se $G$ {\'e} um conjunto com uma quantidade finita de elementos, \pause dizemos que $G$ {\'e} um \textbf{grupo finito}. \pause Denotamos por $|G|$ \pause o n{\'u}mero de elementos de $G$ \pause e que ser{\'a} chamado de \textbf{ordem} de $G$ \pause ou \textbf{cardinalidade} de $G$. \pause Quando o conjunto $G$ n{\~a}o {\'e} finito, \pause dizemos que $G$ {\'e} um \textbf{grupo infinito}.\pause
        \end{definicao}

        \begin{exemplos}
            \begin{enumerate}[label={\roman*})]
                \item $(\z_m, +)$ {\'e} um grupo finito para todo $m>1$ \pause e $|G| = m$.\pause
                \item $(S_n, \circ)$ \'e um grupo finito \pause e $|G| = n!$ elementos.\pause
                \item $(\z, +)$ {\'e} um grupo infinito.
            \end{enumerate}
        \end{exemplos}
    \end{frame}

    \begin{frame}
        \begin{definicao}
            Seja $(G,*)$ um grupo. \pause Um subconjunto n{\~a}o vazio \pause $H\sub G$ \pause {\'e} chamado de \textbf{subgrupo} de $G$ \pause se, e somente se, $(H,*)$ \pause {\'e} um grupo.\pause
        \end{definicao}

        \begin{proposicao}
            Seja $(G, *)$ um grupo. \pause Um subconjunto n{\~a}o vazio \pause $H\subseteq G$ {\'e} um subgrupo de $G$ \pause se, e somente se\pause
            \begin{enumerate}[label={\roman*})]
                \item\label{subgrupo_condicao_1} $x^{-1}\in H$, \pause para todo $x \in H$;
                \item\label{subgrupo_condicao_2} $x*y\in H$, \pause para todos $x$, $y \in H$.\pause
            \end{enumerate}
        \end{proposicao}
    \end{frame}

    \begin{frame}
        \begin{exemplos}
            \begin{enumerate}[label={\roman*})]
                \item Dado $(G,*)$ grupo, \pause $H=\{e\}$ \pause e $H=G$ \pause s{\~a}o subgrupos de $G$, \pause chamados de \textbf{subgrupos triviais}.\pause

                \item Seja $(\mathbb{Z},+)$ um grupo. \pause Tomando $H = m\z$, \pause onde $m > 1$, ent{\~a}o $H$ {\'e} subgrupo de $\z$.\pause

                \item $G = U(\z_8) = \{\overline{1}, \overline{3}, \overline{5}, \overline{7}\}$. \pause Ent\~ao $(G,\odot)$ {\'e} um grupo \pause com $|G| = 4$. \pause Al\'em disso,\pause
                \begin{center}
                    \begin{tabular}{l}
                        $H_1 = \{\overline{1}, \overline{3}\}\pause$\\
                        $H_2 = \{\overline{1}, \overline{5}\}\pause$\\
                        $H_3 = \{\overline{1}, \overline{7}\}\pause$
                    \end{tabular}
                \end{center}
                S\~ao subgrupos de $G$.
                \seti
            \end{enumerate}
        \end{exemplos}
    \end{frame}

    \begin{frame}
        \begin{exemplos}
            \begin{enumerate}[label={\roman*})]
                \conti
                \item Considere o grupo aditivo $M_2(\real)$. \pause Mostre que o conjunto\pause
                \[
                    H = \left\{\begin{pmatrix}
                        a & b\\c & d
                    \end{pmatrix} \in M_2(\real) \mid a + d = 0\right\} \pause
                \]
                \'e um subgrupo de $M_2(\real)$.
            \end{enumerate}
        \end{exemplos}
    \end{frame}

    \begin{frame}
        Seja $(G, *)$ um grupo. \pause Para simplificar a nota\c{c}\~ao \pause vamos adotar uma nota\c{c}\~ao multiplicativa \pause e escrever $(G, *) = \pause (G, \cdot)$. \pause Assim, dados $x$, $y \in G$ vamos denotar\pause
        \[
            x * y = \pause x \cdot y = \pause xy.\pause
        \]

        Nesse caso vamos dizer simplesmente que $G$ \'e um grupo.
    \end{frame}

    \begin{frame}
        \begin{proposicao}
            Seja $G$ um grupo. \pause Dado $H \subset G$ um subgrupo \pause defina \pause
            \[
                x \sim y \pause \mbox{ se, e somente se, } \pause x^{-1}y \in H \pause
            \]
            para todos $x$, $y \in G$. \pause
            \begin{enumerate}[label={\roman*})]
                \item A rela\c{c}\~ao $\sim$ \pause sobre $G$ definida acima \'e uma rela\c{c}\~ao de equival\^encia. \pause

                \item Se $a \in G$, \pause ent\~ao a classe de equival\^encia determinada por $a$ \pause \'e o conjunto \pause
                \[
                    aH = \pause \{ah \pause \mid h \in H\}.
                \]
            \end{enumerate}
        \end{proposicao}
    \end{frame}

    \begin{frame}
        \begin{proposicao}
            Seja $H$ um subgrupo de um grupo $G$. \pause Ent\~ao duas classes laterais quaisquer \pause m\'odulo $H$ \pause s\~ao subconjuntos de $G$ que possuem a mesma cardinalidade, \pause isto \'e, a mesma quantidade de elementos. \pause
        \end{proposicao}
    \end{frame}

    \begin{frame}
        \begin{exemplos}
            \begin{enumerate}[label={\roman*})]
                \item No grupo multiplicativo $G = \{1, -1, i, -i\}$, \pause onde $i^2 = -1$. \pause Considere o conjunto $H = \{1, -1\}$. \pause Ent\~ao $H$ \'e um sugbrupo de $G$ \pause e as classes laterais ser\~ao:

                \seti
            \end{enumerate}
        \end{exemplos}
    \end{frame}

    \begin{frame}
        \begin{exemplos}
            \begin{enumerate}[label={\roman*})]
                \conti
                \item Considere o grupo multiplicativo $\real^*$ \pause e $H = \{ x \in \real^* \mid x > 0\} \pause \subset \real^*$. \pause Ent\~ao $H$ \'e subgrupo de $\real^*$ \pause e as classes laterais ser\~ao:
                \seti
            \end{enumerate}
        \end{exemplos}
    \end{frame}

    \begin{frame}
        \begin{exemplos}
            \begin{enumerate}[label={\roman*})]
                \conti
                \item Considere agora o grupo sim\'etrico $G = S_3$. \pause Denote por \pause
                \[
                    a = \begin{pmatrix}
                        1 & 2 & 3\\2 & 3 & 1
                    \end{pmatrix}, \pause \quad
                    b = \begin{pmatrix}
                        1 & 2 & 3\\1 & 3 & 2
                    \end{pmatrix}. \pause
                \]
                Fica como exerc{\'\i}cio verificar que $\{e, a, a^2 , b, ba, ba^2\} = \pause S_3$. \pause Aqui $e$ \'e a fun\c{c}\~ao identidade, \pause $a^2 = a \circ a$, \pause $ba = b \circ a$ e \pause $ba^2 = b\circ(a\circ a)$. \pause Seja $H = \{e, a , a^2\}$. \pause Ent\~ao $H$ \'e subgrupo de $S_3$ \pause e as classes laterais ser\~ao:
            \end{enumerate}
        \end{exemplos}
    \end{frame}
\end{document}
