%!TEX program = xelatex
% !TEX encoding = ISO-8859-1
\def\ano{2020}
\def\semestre{1}
\def\disciplina{\'Algebra 1}
\def\turma{C}
\def\autor{Jos\'e Ant\^onio O. Freitas}
\def\instituto{MAT-UnB}

\documentclass{beamer}
\usetheme{Madrid}
\usecolortheme{beaver}
% \mode<presentation>
\usepackage{caption}
\usepackage{textpos}
\usepackage{amssymb}
\usepackage{amsmath,amsfonts,amsthm,amstext}
\usepackage[brazil]{babel}
% \usepackage[latin1]{inputenc}
\usepackage{graphicx}
\graphicspath{{/home/jfreitas/GitHub_Repos/video_aulas/logo/}{D:/Dropbox/imagens-latex/}}
\usepackage{enumitem}
\usepackage{multicol}
\usepackage{answers}
\usepackage{tikz,ifthen}
\usetikzlibrary{lindenmayersystems}
\usetikzlibrary[shadings]
\newtheorem{definicao}{Defini\c{c}\~ao}[section]
\newtheorem{definicoes}{Defini\c{c}\~oes}[section]
\newtheorem{exemplo}{Exemplo}[section]
\newtheorem{exemplos}{Exemplos}[section]
\newtheorem{exercicio}{Exerc{\'\i}cio}
\newtheorem{observacao}{Observa{\c c}{\~a}o:}[section]
\newtheorem{observacoes}{Observa{\c c}{\~o}es:}[section]
\newtheorem*{solucao}{Solu{\c c}{\~a}o:}
\newtheorem{proposicao}{Proposi\c{c}\~ao}
\newtheorem{lema}{Lema}
\newtheorem{teorema}{Teorema}
\newtheorem{corolario}{Corol\'ario}
\newenvironment{prova}[1][Prova]{\noindent\textbf{#1:} }{\qedsymbol}%{\ \rule{0.5em}{0.5em}}
\newcommand{\nsub}{\varsubsetneq}
\newcommand{\vaz}{\emptyset}
\newcommand{\im}{{\rm Im\,}}
\newcommand{\sub}{\subseteq}
\newcommand{\n}{\mathbb{N}}
\newcommand{\z}{\mathbb{Z}}
\newcommand{\rac}{\mathbb{Q}}
\newcommand{\real}{\mathbb{R}}
\newcommand{\complex}{\mathbb{C}}
\newcommand{\cp}[1]{\mathbb{#1}}
\newcommand{\ch}{\mbox{\textrm{car\,}}\nobreak}
\newcommand{\vesp}[1]{\vspace{ #1  cm}}
\newcommand{\compcent}[1]{\vcenter{\hbox{$#1\circ$}}}
\newcommand{\comp}{\mathbin{\mathchoice
{\compcent\scriptstyle}{\compcent\scriptstyle}
{\compcent\scriptscriptstyle}{\compcent\scriptscriptstyle}}}

\title{Grupos Cíclicos}
\author[\autor]{\autor}
\institute[\instituto]{\instituto}
\date{\today}

\begin{document}
    \begin{frame}
        \maketitle
    \end{frame}

    \logo{\includegraphics[scale=.1]{logo-MAT.png}\vspace*{8.5cm}}

    \begin{frame}
        Seja $(G, *)$ um grupo. \pause

        Caso a operação $*$ seja do tipo multiplicativa, vamos escrever $(G, *) = \pause (G, \cdot)$. \pause Assim, dados $x$, $y \in G$ vamos denotar\pause
        \[
            x * y = \pause x \cdot y = \pause xy.
        \]

        Caso a operação $*$ seja do tipo aditiva, vamos escrever $(G, *) = \pause (G, +)$. \pause Assim, dados $x$, $y \in G$ vamos denotar\pause
        \[
            x * y = \pause x + y
        \]

        Com a notação multiplicativa \pause o inverso de um elemento $x \in G$ \pause será denotado por $x^{-1}$ \pause e no caso da notação aditiva \pause o oposto de $x \in G$ \pause será denotado por $-x$.
    \end{frame}

    \begin{frame}
        Seja $G$ um grupo multiplicativo e denote por $e$ o elemento neutro de $G$. Se $x \in G$ e $m \in \z$, a \textbf{potência $m$-ésima} de $x$, ou \textbf{potência de $x$ de expoente $m$}, é o elemento de $G$ denotado por
        \[
            x^m
        \]
        e definido por:
        \[
            x^m = \begin{cases}
                    e, & \mbox{se m = 0},\\
                    x^{m-1}x, & \mbox{ se } m \ge 1\\
                    (x^{-m})^{-1}, & \mbox{ se } m < 1. 
                   \end{cases}
        \]
        
    \end{frame}

    \begin{frame}
        \begin{exemplos}
            \begin{enumerate}[label={\arabic*})]
                \item No grupo multiplicativo $GL_2(\real)$ seja
                \[
                    A = \begin{pmatrix}
                        1 & 1\\2 & 3
                    \end{pmatrix}.
                \]
                Então:
                \seti
            \end{enumerate}
        \end{exemplos}
    \end{frame}
    
    \begin{frame}
        \begin{exemplos}
            \begin{enumerate}[label={\arabic*})]
                \conti
                \item No grupo multiplicativo $\z_5^*$ seja $a = \overline{2}$. Então:
                \seti
            \end{enumerate}
        \end{exemplos}
    \end{frame}

    \begin{frame}
        \begin{exemplos}
            \begin{enumerate}[label={\arabic*})]
                \conti
                \item No grupo multiplicativo $S_3$ seja
                \[
                    a = \begin{pmatrix}
                        1 & 2 & 3\\ 2 & 3 & 1
                    \end{pmatrix}
                \]
                Então:
            \end{enumerate}
        \end{exemplos}
    \end{frame}

    \begin{frame}
        \begin{proposicao}
            Seja $G$ um grupo multiplicativo. Se $m$ e $n$ são números inteiros e $x \in G$, então
            \begin{enumerate}[label={\arabic*})]
                \item $x^mx^n = x^{m + n}$

                \item $x^{-m} = (x^m)^{-1}$

                \item $(x^m)^n = x^{mn}$

                \item $x^mx^n = x^{m + n}$

                \item $x^mx^n = x^nx^m$
            \end{enumerate}
        \end{proposicao}
    \end{frame}

    \begin{frame}
        Seja $G$ um grupo aditivo e denote por $e$ o elemento neutro de $G$. Se $x \in G$ e $m \in \z$, o \textbf{múltiplo $m$-ésimo} de $x$ é o elemento de $G$ denotado por
        \[
            m \cdot x
        \]
        e definido por:
        \[
            m \cdot x = \begin{cases}
                    e, & \mbox{se m = 0},\\
                    (m - 1)\cdot x + x, & \mbox{ se } m \ge 1\\
                    -[(-m) \cdot x], & \mbox{ se } m < 1. 
                   \end{cases}
        \]
    \end{frame}

    \begin{frame}
        \begin{proposicao}
            Seja $G$ um grupo aditivo. Se $m$ e $n$ são números inteiros e $x \in G$, então
            \begin{enumerate}[label={\arabic*})]
                \item $m \cdot x + n \cdot x = (m + n) \cdot x$

                \item $(-m) \cdot x = -(m \cdot x)$

                \item $n\cdot (m \cdot x) = (nm)\cdot x$
            \end{enumerate}
        \end{proposicao}
    \end{frame}
\end{document}