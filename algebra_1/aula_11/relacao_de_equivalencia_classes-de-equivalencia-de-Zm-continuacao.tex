%!TEX program = xelatex
% !TEX encoding = ISO-8859-1
\def\ano{2020}
\def\semestre{1}
\def\disciplina{\'Algebra 1}
\def\turma{C}
\def\autor{Jos\'e Ant\^onio O. Freitas}
\def\instituto{MAT-UnB}

\documentclass{beamer}
\usetheme{Madrid}
\usecolortheme{beaver}
% \mode<presentation>
\usepackage{caption}
\usepackage{textpos}
\usepackage{amssymb}
\usepackage{amsmath,amsfonts,amsthm,amstext}
\usepackage[brazil]{babel}
% \usepackage[latin1]{inputenc}
\usepackage{graphicx}
\graphicspath{{/home/jfreitas/GitHub_Repos/video_aulas/logo/}{D:/Dropbox/imagens-latex/}}
\usepackage{enumitem}
\usepackage{multicol}
\usepackage{answers}
\usepackage{tikz,ifthen}
\usetikzlibrary{lindenmayersystems}
\usetikzlibrary[shadings]
\newtheorem{definicao}{Defini\c{c}\~ao}[section]
\newtheorem{definicoes}{Defini\c{c}\~oes}[section]
\newtheorem{exemplo}{Exemplo}[section]
\newtheorem{exemplos}{Exemplos}[section]
\newtheorem{exercicio}{Exerc{\'\i}cio}
\newtheorem{observacao}{Observa{\c c}{\~a}o:}[section]
\newtheorem{observacoes}{Observa{\c c}{\~o}es:}[section]
\newtheorem*{solucao}{Solu{\c c}{\~a}o:}
\newtheorem{proposicao}{Proposi\c{c}\~ao}
\newtheorem{lema}{Lema}
\newtheorem{teorema}{Teorema}
\newtheorem{corolario}{Corol\'ario}
\newenvironment{prova}[1][Prova]{\noindent\textbf{#1:} }{\qedsymbol}%{\ \rule{0.5em}{0.5em}}
\newcommand{\nsub}{\varsubsetneq}
\newcommand{\vaz}{\emptyset}
\newcommand{\im}{{\rm Im\,}}
\newcommand{\sub}{\subseteq}
\newcommand{\n}{\mathbb{N}}
\newcommand{\z}{\mathbb{Z}}
\newcommand{\rac}{\mathbb{Q}}
\newcommand{\real}{\mathbb{R}}
\newcommand{\complex}{\mathbb{C}}
\newcommand{\cp}[1]{\mathbb{#1}}
\newcommand{\ch}{\mbox{\textrm{car\,}}\nobreak}
\newcommand{\vesp}[1]{\vspace{ #1  cm}}
\newcommand{\compcent}[1]{\vcenter{\hbox{$#1\circ$}}}
\newcommand{\comp}{\mathbin{\mathchoice
{\compcent\scriptstyle}{\compcent\scriptstyle}
{\compcent\scriptscriptstyle}{\compcent\scriptscriptstyle}}}

\title{Rela\c{c}\~ao de Equival\^encia - Classes de Equival\^encia nos Inteiros - Continua\c{c}\~ao}
\author[\autor]{\autor}
\institute[\instituto]{\instituto}
\date{29 de agosto de 2020}

\begin{document}
    \begin{frame}
        \maketitle
    \end{frame}

    \logo{\includegraphics[scale=.1]{logo-MAT.png}\vspace*{8.5cm}}

    \begin{frame}
        Dado $n \in \z$, temos\pause
        \[
            \overline{n} = \pause C(n) = \pause \{x \in \z \mid \pause x\equiv n \pmod{m}\}.\pause
        \]

        Vamos dentoar $C(n)$ \pause por $R_{m}(n)$ \pause ou $\overline{n}$, \pause quando n{\~a}o houver possibilidade de confus{\~a}o. \pause Assim fixando $m > 1$ vamos escrever\pause
        \begin{center}
            \begin{tabular}{l}
                $R_{m}(0) = \pause \{x \in \z \mid \pause x\equiv 0 \pmod{m}\} \pause =\{x \in \z \mid \pause x = mk, k \in \z\}\pause = m\z$\pause\\
                \\
                $R_{m}(1) = \pause \{x \in \z \mid \pause x\equiv 1 \pmod{m}\} \pause = \{x \in \z \mid \pause x = 1 + km, k \in \z \}$\pause \\
                \\
                \vdots\\
                \\
                $R_{m}(n) = \pause \{x \in \z \mid \pause x = n + km, k \in \z \}$\pause    
            \end{tabular}
            
        \end{center}
    \end{frame}

    \begin{frame}
        \begin{proposicao}
            As classes de equival{\^e}ncia definidas pela congru{\^e}ncia m{\'o}dulo $m$ \pause s{\~a}o determinadas pelos restos da divis{\~a}o inteira por $m$. \pause Em outras palavras, $R_{m}(n)$ \pause {\'e} o conjunto dos n{\'u}meros inteiros \pause cujo resto na divis{\~a}o inteira por $m$ {\'e} $n$.\pause
        \end{proposicao}

        \begin{corolario}
            $R_{m}(k) = R_{m}(l)$ \pause se, e somente se, $k\equiv l \pmod{m}$.\pause
        \end{corolario}
    \end{frame}
    \begin{frame}
        \begin{exemplos}
            \begin{enumerate}[label={\arabic*})]
                \item Se $m=2$, \pause ent{\~a}o os poss{\'\i}veis restos na divis{\~a}o inteira por 2 s{\~a}o 0 e 1. \pause Logo, existem duas classes de equival{\^e}ncia, a saber\pause
                \begin{center}
                    \begin{tabular}{l}
                        $R_{2}(0) = \pause \{x \in \z \mid \pause x\equiv 0 \pmod{2}\} = \pause \{x\in \z \mid \pause x = 2k, k \in \z \}$\pause \\
                        \\
                        $R_{2}(1) = \pause \{x \in \z \mid \pause x\equiv 1 \pmod{2}\} = \pause \{x \in \z \mid \pause x = 1 + 2k, k\in \z \}.$\pause
                    \end{tabular}
                \end{center}
                \vspace{.3cm}
                \item Se $m = 3$, \pause ent{\~a}o os poss{\'\i}veis restos da divis{\~a}o inteira s{\~a}o 0, 1 e 2. \pause Da{\'\i}\pause
                \begin{center}
                    \begin{tabular}{l}
                        $R_{3}(0) = \pause \{x \in \z \mid \pause x\equiv 0 \pmod{3}\} = \pause \{x \in \z \mid \pause x = 3k, k \in \z\}$\pause\\
                        \\
                        $R_{3}(1)  = \pause \{x \in \z \mid \pause x\equiv 1 \pmod{3}\} = \pause \{x \in \z \mid \pause x = 3k + 1, k \in \z \}$\pause\\
                        \\
                        $R_{3}(2) = \pause \{x \in \z \mid \pause x \equiv 2 \pmod{3}\} = \pause \{x \in \z \mid \pause x = 3k + 2, k \in \z\}$\pause
                    \end{tabular}
                \end{center}
            \end{enumerate} 
        \end{exemplos}
    \end{frame}

    \begin{frame}    
        \begin{proposicao}
            Na rela{\c c}{\~a}o de equival{\^e}ncia m{\'o}dulo $m$ existem $m$ classes de equival{\^e}ncia.\pause
        \end{proposicao}
        \noindent \textbf{Prova:}
            Os poss{\'\i}veis restos na divis{\~a}o inteira por $m$ \pause s{\~a}o $0,1,...,(m-1)$. \pause Como cada poss{\'\i}vel resto define uma classe de equival{\^e}ncia diferente, \pause existem exatamente $m$ classes de equival{\^e}ncia.\hspace{.5cm}\qedsymbol\pause
    \end{frame}

    \begin{frame}
        \begin{observacao}
        Fixado $m$ inteiro positivo, \pause denotaremos\pause
        \begin{center}
            \begin{tabular}{l}
                $R_{m}(0) = \overline{0}$\pause \\
                \\
                $R_{m}(1) = \overline{1}$\pause \\
                \\
                $\vdots$\\
                \\
                $R_{m}(m-1) = \overline{m-1}$\pause
            \end{tabular}
        \end{center}

        O conjunto quociente \pause desta rela{\c c}{\~a}o ser{\'a} denotado por $\displaystyle\frac{\z}{m\z}$ \pause ou $\z_m$. \pause Assim\pause
        \[
            \z_m = \pause \displaystyle\frac{\z}{m\z} = \pause \{\overline{0},\overline{1},...,\overline{m-1}\}.\pause
        \]
        \end{observacao}
    \end{frame}

    \begin{frame}
        Vamos definir um meio de somar \pause e multiplicar os elementos de $\z_m$. \pause Por exemplo, em $\z_2 = \pause \{\overline{0},\overline{1}\}$ \pause temos que a soma de pares {\'e} par, \pause soma de par com {\'\i}mpar \pause {\'e} {\'\i}mpar \pause e a soma de {\'\i}mpares {\'e} par. \pause Assim podemos escrever\pause

        \begin{table}[h]
           \centering 
           \setlength{\arrayrulewidth}{0,5\arrayrulewidth}
           \begin{tabular}{|c|c|c|} 
              \hline
              $\oplus$ & $\overline{0}$ & $\overline{1}$ \T\\
              \hline
              $\overline{0}$ & \phantom{abc} & \phantom{abc}\T\\
              \hline
              $\overline{1}$ & \phantom{abc} & \phantom{abc}\T\\
              \hline
           \end{tabular}
        \end{table}

        Para multiplica{\c c}{\~a}o, temos\pause

        \begin{table}[h]
           \centering 
           \setlength{\arrayrulewidth}{0,5\arrayrulewidth}
           \begin{tabular}{|c|c|c|} 
              \hline
              $\otimes$ & $\overline{0}$ & $\overline{1}$\T\\
              \hline
              $\overline{0}$ & \phantom{abc} & \phantom{abc}\T\\
              \hline
              $\overline{1}$ & \phantom{abc} & \phantom{abc}\T\\
              \hline
           \end{tabular}
        \end{table}
    \end{frame}

    \begin{frame}
        \begin{definicao}
            Dados $\overline{a}$, $\overline{b} \in \z_m$ definimos\pause
            \begin{center}
                \begin{tabular}{l}
                    $\overline{a}\oplus\overline{b} = \pause \overline{a + b}\label{soma_modulo_m}$\pause \\
                    \\
                    $\overline{a}\otimes\overline{b} = \pause \overline{ab}.\label{multiplicacao_modulo_m}$\pause
                \end{tabular}
            \end{center}
        \end{definicao}

        \begin{proposicao}
            As opera{\c c}{\~o}es de soma \pause e produto \pause definidas acima s{\~a}o independentes dos representantes das classes.\pause
        \end{proposicao}
        \noindent \textbf{Prova:}
            Dadas duas classes em $\z_m$ com representantes diferentes, \pause $\overline{a}_{1} = \overline{a}_{2}$ e \pause $\overline{b}_{1} = \overline{b}_{2}$, \pause com $a_{1}\ne a_{2}$ \pause e $b_{1}\ne b_{2}$, \pause temos:\pause
            \begin{center}
                $a_1 \equiv a_2 \pmod m$\pause\\
                $b_1 \equiv b_2 \pmod m$.\pause
            \end{center}
            Daí,
            \begin{center}
                $a_1 + b_1 \equiv a_2 + b_2 \pmod m$\pause\\
                $a_1b_1 \equiv a_2b_2 \pmod m\pause$\\
            \end{center}
    \end{frame}

    \begin{frame}
        Mas de $a_1 + b_1 \equiv a_2 + b_2 \pmod m$ \pause segue que $\overline{a_1 + b_1} =  \pause \overline{a_2 + b_2}$\pause. Assim
        \begin{center}
            $\overline{a}_{1}\oplus \overline{b}_{1} = \pause \overline{a_{1}+b_{1}} = \pause \overline{a_{2} + b_{2}} = \pause \overline{a}_{2}\oplus \overline{b}_{2}$.\pause
        \end{center}

        Agora de $a_1b_1 \equiv a_2b_2 \pmod m$ \pause segue que $\overline{a_1b_2} = \pause \overline{a_2b_2}$\pause. Assim
        \begin{center}
            $\overline{a}_{1}\otimes \overline{b}_{1} = \pause \overline{a_{1}b_{1}} = \pause \overline{a_{2}b_{2}} = \pause \overline{a}_{2}\otimes\overline{b}_{2}$.\pause
        \end{center}

        Portanto a soma e a multiplicação \pause não dependem dos representantes que escolhemos para as classes de equivalência, \pause como queríamos.\hspace{.3cm} \qedsymbol
    \end{frame}

    \begin{frame}
        \begin{exemplo}
            A soma e a multiplica{\c c}{\~a}o em $\z_4 = \{\overline{0},\overline{1},\overline{2},\overline{3}\}$\pause
            s\~ao dadas nas tabelas abaixo:\pause
                \begin{table}[!htb]
                  \caption{Soma em $\z_4$}
                  \begin{minipage}{.5\linewidth}
                    \centering
                    \begin{tabular}{|c|c|c|c|c|} 
                        \hline
                        $\oplus$ & $\overline{0}$ & $\overline{1}$ & $\overline{2}$ & $\overline{3}$\T\\
                        \hline
                        $\overline{0}$ & \phantom{abc} & \phantom{abc} & \phantom{abc} & \phantom{abc}\T\\
                        \hline
                        $\overline{1}$ & \phantom{abc} & \phantom{abc}& \phantom{abc} & \phantom{abc}\T\\
                        \hline
                        $\overline{2}$ & \phantom{abc} & \phantom{abc} & \phantom{abc} & \phantom{abc}\T\\
                        \hline
                        $\overline{3}$ & \phantom{abc} & \phantom{abc} & \phantom{abc} & \phantom{abc}\T\\
                        \hline
                    \end{tabular}
                  \end{minipage}
                \end{table}
        \end{exemplo}
    \end{frame}

    \begin{frame}
        \begin{exemplo}
                \begin{table}[!htb]
                  \caption{Multiplica\c{c}\~ao em $\z_4$}
                  \begin{minipage}{.5\linewidth}
                  \centering
                    \begin{tabular}{|c|c|c|c|c|} 
                      \hline
                      $\otimes$ & $\overline{0}$ & $\overline{1}$ & $\overline{2}$ & $\overline{3}$\T\\
                      \hline
                      $\overline{0}$ & \phantom{abc} & \phantom{abc} & \phantom{abc} & \phantom{abc}\T\\
                      \hline
                      $\overline{1}$ & \phantom{abc} & \phantom{abc} & \phantom{abc} & \phantom{abc}\T\\
                      \hline
                      $\overline{2}$ & \phantom{abc} & \phantom{abc} & \phantom{abc} & \phantom{abc}\T\\
                      \hline
                      $\overline{3}$ & \phantom{abc} & \phantom{abc} & \phantom{abc} & \phantom{abc}\T\\
                      \hline
                    \end{tabular}
                \end{minipage}
            \end{table}
        \end{exemplo}
    \end{frame}

    \begin{frame}
        \begin{proposicao}
            As opera\c{c}\~oes de soma $\oplus$ \pause e multiplica\c{c}\~ao $\otimes$ \pause em $\z_m$ satisfazem as seguintes propriedades:\pause
            \begin{enumerate}[label={\roman*})]
                \item Para todos $\overline{x}$, $\overline{y} \in \z_m$: $\overline{x} \oplus \overline{y} = \overline{y} \oplus \overline{x}$. \vspace{.2cm} \pause

                \item Para todos $\overline{x}$, $\overline{y}$ e $\overline{z} \in \z_m$: $(\overline{x} \oplus \overline{y}) \oplus \overline{z} = \overline{x} \oplus (\overline{y} \oplus \overline{z})$. \vspace{.2cm} \pause

                \item Para todo $\overline{x} \in \z_m$, $\overline{x} \oplus \overline{0} = \overline{x}$. \vspace{.2cm} \pause

                \item Para todo $\overline{x} \in \z_m$, existe $\overline{y} \in \z_m$ tal que $\overline{x} \oplus \overline{y} = \overline{0}$. \vspace{.2cm} \pause

                \item Para todos $\overline{x}$, $\overline{y} \in \z_m$: $\overline{x} \otimes \overline{y} = \overline{y} \otimes \overline{x}$. \vspace{.2cm} \pause

                \item Para todos $\overline{x}$, $\overline{y}$ e $\overline{z} \in \z_m$: $(\overline{x} \otimes \overline{y}) \otimes \overline{z} = \overline{x} \otimes (\overline{y} \otimes \overline{z})$. \vspace{.2cm} \pause

                \item Para todo $\overline{x} \in \z_m$: $\overline{x} \otimes \overline{1} = \overline{x}$. \pause
            \end{enumerate}
        \end{proposicao}
    \end{frame}

    \begin{frame}
        \noindent \textbf{Prova: }
            \begin{enumerate}[label={\roman*})]
                \item $\overline{x} \oplus \overline{y} = \pause \overline{x + y} = \pause \overline{y + x} = \pause \overline{y} \oplus \overline{x}$. \vspace{.2cm} \pause
                
                \item $(\overline{x} \oplus \overline{y}) \oplus \overline{z} = \pause \overline{x + y} \pause \oplus \overline{z} = \pause \overline{(x + y) + z} = \pause \overline{x + (y + z)} \pause = \overline{x} \oplus \pause \overline{y + z}\pause  = \overline{x} \oplus \pause (\overline{y} \oplus \overline{z})$. \vspace{.2cm} \pause

                \item $\overline{x} \oplus \overline{0} = \pause \overline{x + 0} = \pause \overline{x}$. \vspace{.2cm} \pause

                \item Dado $\overline{x} \in \z_m$ \pause escolha $\overline{y} = \pause \overline{m - x} \in \z_m$. \pause Assim $\overline{x} \oplus \overline{y} = \pause \overline{x} \oplus \pause \overline{m - x} = \pause \overline{x + (m - x)} = \pause \overline{m} = \overline{0}$. \vspace{.2cm} \pause

                \item $\overline{x} \otimes \overline{y} = \pause \overline{x \cdot y} = \pause \overline{y \cdot x} = \pause \overline{y} \otimes \overline{x}$. \vspace{.2cm} \pause

                \item $(\overline{x} \otimes \overline{y}) \otimes \overline{z} = \pause \overline{x \cdot y} \otimes \pause \overline{z} = \pause \overline{(x \cdot y)\cdot z} \pause = \overline{x\cdot(y \cdot z)} = \pause \overline{x} \otimes \pause \overline{y \cdot z} = \pause \overline{x} \otimes \pause (\overline{y}\otimes \overline{z})$. \vspace{.2cm} \pause

                \item $\overline{x} \otimes \overline{1} = \pause \overline{x \cdot 1} = \pause \overline{x}$. \pause
            \end{enumerate}
        \qedsymbol
    \end{frame}

    \begin{frame}
        \begin{definicao}
            Um elemento $\overline{a} \in \z_m$ {\'e} \textbf{invers{\'\i}vel} \pause se, e somente se, existe $\overline{b} \in \z_m$ \pause tal que $\overline{a} \otimes \overline{b} = \pause \overline{1}$. \pause Neste caso, $\overline{b}$ \pause {\'e} chamado \textbf{inverso} de $\overline{a}$ \pause e denotaremos $\overline{b} = (\overline{a})^{-1}$.\pause
        \end{definicao}

        \begin{proposicao}
            Se o inverso existe, \pause ent\~ao ele {\'e} {\'u}nico.\pause
        \end{proposicao}
        \noindent \textbf{Prova:}
            De fato, \pause dado $\overline{a} \in \z_m$, \pause suponha que existem $\overline{b}$, \pause $\overline{d} \in \z_m$\pause tais que $\overline{a} \otimes \overline{b} = \overline{1} \pause = \overline{a} \otimes \overline{d}$, \pause ent{\~a}o\pause
            \begin{center}
                \begin{tabular}{l}
                    $\overline{b} = \pause \overline{b} \otimes \overline{1} = \pause \overline{b} \otimes \pause (\overline{a} \otimes \overline{d}) = \pause (\overline{b} \otimes \overline{a}) \pause \otimes \overline{d} = \pause \overline{1} \otimes \overline{d} = \pause \overline{d}$\pause
                \end{tabular}
            \end{center}
        Portanto o inverso de $\overline{a}$ \'e \'unico, como quer{\'\i}amos. \hspace{.5cm}\qedsymbol\pause
    \end{frame}

    \begin{frame}
        \begin{proposicao}
            Um elemento $\overline{a} \in \z_m$ {\'e} \pause invers{\'\i}vel \pause se, e somente se, $mdc(a,m) = 1$.\pause
        \end{proposicao}

        \begin{corolario}
            Se $m$ \'e um n\'umero primo, \pause ent\~ao para todo $\overline{x} \in \z_m$, \pause $\overline{x} \ne \overline{0}$, \pause existe inverso.\pause
        \end{corolario}

        \begin{exemplos}
            \begin{enumerate}[label={\arabic*})]
                \item Em $\z_4$ existem dois elementos invers{\'\i}veis \pause que s{\~a}o $\overline{1}$, \pause cujo inverso {\'e} $\overline{1}$, \pause e o $\overline{3}$, \pause cujo inverso {\'e} $\overline{3}$.\pause
                \item Em $\z_{11}$, \pause todos elementos, exceto $\overline{0}$, \pause possuem inverso:
                \begin{table}[h]
                    \caption{\it Inversos em $\z_{11}$}
                   \begin{tabular}{|c|c|c|c|c|c|c|c|c|c|c|} 
                        \hline
                        Elemento & $\overline{1}$ & $\overline{2}$ & $\overline{3}$ & $\overline{4}$ & $\overline{5}$ & $\overline{6}$ & $\overline{7}$ & $\overline{8}$ & $\overline{9}$ & $\overline{10}$\T \\
                        \hline
                        Inverso & \phantom{ab} & \phantom{ab} & \phantom{ab} & \phantom{ab} & \phantom{ab} & \phantom{ab} & \phantom{ab} & \phantom{ab} & \phantom{ab} & \phantom{ab} \T\\
                        \hline
                   \end{tabular}
                \end{table}
            \end{enumerate}
        \end{exemplos}
    \end{frame}
\end{document}