%!TEX program = xelatex
% !TEX encoding = ISO-8859-1
\def\ano{2020}
\def\semestre{1}
\def\disciplina{\'Algebra 1}
\def\turma{C}
\def\autor{Jos\'e Ant\^onio O. Freitas}
\def\instituto{MAT-UnB}

\documentclass{beamer}
\usetheme{Madrid}
\usecolortheme{beaver}
% \mode<presentation>
\usepackage{caption}
\usepackage{textpos}
\usepackage{amssymb}
\usepackage{amsmath,amsfonts,amsthm,amstext}
\usepackage[brazil]{babel}
% \usepackage[latin1]{inputenc}
\usepackage{graphicx}
\graphicspath{{/home/jfreitas/GitHub_Repos/video_aulas/logo/}{D:/Dropbox/imagens-latex/}}
\usepackage{enumitem}
\usepackage{multicol}
\usepackage{answers}
\usepackage{tikz,ifthen}
\usetikzlibrary{lindenmayersystems}
\usetikzlibrary[shadings]
\newtheorem{definicao}{Defini\c{c}\~ao}[section]
\newtheorem{definicoes}{Defini\c{c}\~oes}[section]
\newtheorem{exemplo}{Exemplo}[section]
\newtheorem{exemplos}{Exemplos}[section]
\newtheorem{exercicio}{Exerc{\'\i}cio}
\newtheorem{observacao}{Observa{\c c}{\~a}o:}[section]
\newtheorem{observacoes}{Observa{\c c}{\~o}es:}[section]
\newtheorem*{solucao}{Solu{\c c}{\~a}o:}
\newtheorem{proposicao}{Proposi\c{c}\~ao}
\newtheorem{lema}{Lema}
\newtheorem{teorema}{Teorema}
\newtheorem{corolario}{Corol\'ario}
\newenvironment{prova}[1][Prova]{\noindent\textbf{#1:} }{\qedsymbol}%{\ \rule{0.5em}{0.5em}}
\newcommand{\nsub}{\varsubsetneq}
\newcommand{\vaz}{\emptyset}
\newcommand{\im}{{\rm Im\,}}
\newcommand{\sub}{\subseteq}
\newcommand{\n}{\mathbb{N}}
\newcommand{\z}{\mathbb{Z}}
\newcommand{\rac}{\mathbb{Q}}
\newcommand{\real}{\mathbb{R}}
\newcommand{\complex}{\mathbb{C}}
\newcommand{\cp}[1]{\mathbb{#1}}
\newcommand{\ch}{\mbox{\textrm{car\,}}\nobreak}
\newcommand{\vesp}[1]{\vspace{ #1  cm}}
\newcommand{\compcent}[1]{\vcenter{\hbox{$#1\circ$}}}
\newcommand{\comp}{\mathbin{\mathchoice
{\compcent\scriptstyle}{\compcent\scriptstyle}
{\compcent\scriptscriptstyle}{\compcent\scriptscriptstyle}}}

\title{Rela\c{c}\~ao de Equival\^encia - Classes de Equivalência nos Inteiros - Continuação}
\author[\autor]{\autor}
\institute[\instituto]{\instituto}
\date{\today}

\begin{document}
    \begin{frame}
        \maketitle
    \end{frame}

    \logo{\includegraphics[scale=.1]{logo-MAT.png}\vspace*{8.5cm}}

    \begin{frame}
        Como a congru{\^e}ncia m{\'o}dulo $m$ {\'e} uma rela{\c c}{\~a}o de equival{\^e}ncia, podemos determinar suas classes de equival{\^e}ncia. Assim, dado $n \in \z$, temos
        \[
            \overline{n} = C(n) = \{x \in \z \mid x\equiv n \pmod{m}\}.
        \]

        Denotaremos $C(n)$ por $R_{m}(n)$ ou $\overline{n}$, quando n{\~a}o houver possibilidade de confus{\~a}o.

        Por exemplo, fixando $m > 1$
        \begin{align*}
            R_{m}(0) &= \{x \in \z \mid x\equiv 0 \pmod{m}\}=\{x\in \z \mid x = mk, k\in\z\}=m\z\\
            R_{m}(1) &= \{x\in\z \mid x\equiv 1 \pmod{m}\}=\{x\in\z \mid x = 1 + km, k\in\z\}\\
            R_{m}(n) &= \{x\in\z \mid x = n + km, k\in\z\}
        \end{align*}
    \end{frame}
    \begin{frame}
        \begin{proposicao}
            As classes de equival{\^e}ncia definidas pela congru{\^e}ncia m{\'o}dulo $m$ s{\~a}o determinadas pelos restos da divis{\~a}o inteira por $m$. Em outras palavras, $R_{m}(n)$ {\'e} o conjunto dos n{\'u}meros inteiros cujo resto na divis{\~a}o inteira por $m$ {\'e} $n$.
        \end{proposicao}

        \begin{corolario}
            $R_{m}(k) = R_{m}(l)$ se, e somente se, $k\equiv l \pmod{m}$.
        \end{corolario}
    \end{frame}
    \begin{frame}
        \begin{exemplos}
            \begin{enumerate}[label={\arabic*})]
                \item Se $m=2$, ent{\~a}o os poss{\'\i}veis restos na divis{\~a}o inteira por 2 s{\~a}o 0 e 1. Logo, existem duas classes de equival{\^e}ncia, a saber
                \begin{align*}
                    R_{2}(0) &= \{x \in \z \mid x\equiv 0 \pmod{2}\} = \{x\in \z \mid x = 2k, k\in\z\}\\
                    R_{2}(1) &= \{x\in\z \mid x\equiv 1 \pmod{2}\} = \{x\in\z \mid x = 1 + 2k, k\in\z\}.
                \end{align*}
                
                \item Se $m = 3$, ent{\~a}o os poss{\'\i}veis restos da divis{\~a}o inteira s{\~a}o 0, 1 e 2. Da{\'\i}
                \begin{align*}
                    R_{3}(0) &= \{x \in \z \mid x\equiv 0 \pmod{3}\} = \{x\in \z \mid x = 3k, k \in \z\}\\
                    R_{3}(1) & = \{x \in \z \mid x\equiv 1 \pmod{3}\} = \{x\in\z \mid x = 3k + 1, k \in \z\}\\
                    R_{3}(2) &= \{x \in \z \mid x\equiv 2 \pmod{3}\} = \{x\in\z \mid x = 3k + 2, k \in \z\}
                \end{align*}
            \end{enumerate} 
        \end{exemplos}
    \end{frame}

    \begin{frame}    
        \begin{proposicao}
            Na rela{\c c}{\~a}o de equival{\^e}ncia m{\'o}dulo $m$ existem $m$ classes de equival{\^e}ncia.
        \end{proposicao}
        \begin{prova}
            Os poss{\'\i}veis restos na divis{\~a}o inteira por $m$ s{\~a}o $0,1,...,(m-1)$. Como cada poss{\'\i}vel resto define uma classe de equival{\^e}ncia diferente, existem exatamente $m$ classes de equival{\^e}ncia
        \end{prova}
    \end{frame}

    \begin{frame}
        \begin{observacao}
        Fixado $m$ inteiro positivo, denotaremos
        \begin{align*}
            R_{m}(0) &= \overline{0}\\
            R_{m}(1) &= \overline{1}\\
            &\vdots\\
            R_{m}(m-1) &= \overline{m-1}
        \end{align*}

        O conjunto quociente desta rela{\c c}{\~a}o ser{\'a} denotado por $\displaystyle\frac{\z}{m\z}$ ou $\z_m$. Assim
        \[
            \z_m = \displaystyle\frac{\z}{m\z}=\{\overline{0},\overline{1},...,\overline{m-1}\}.
        \]
        \end{observacao}
    \end{frame}

    \begin{frame}
        Queremos definir um meio de somar e multiplicar os elementos de $\z_m$. Por exemplo, em $\z_2 = \{\overline{0},\overline{1}\}$ temos que a soma de pares {\'e} par, soma de par com {\'\i}mpar {\'e} {\'\i}mpar e a soma de {\'\i}mpares {\'e} par. Assim podemos escrever

        \begin{table}[h]
           \centering 
           \setlength{\arrayrulewidth}{0,5\arrayrulewidth}
           \begin{tabular}{|c|c|c|} 
              \hline
              $\oplus$ & $\overline{0}$ & $\overline{1}$ \T\\
              \hline
              $\overline{0}$ & $\overline{0}$ & $\overline{1}$\T\\
              \hline
              $\overline{1}$ & $\overline{1}$ & $\overline{0}$\T\\
              \hline
           \end{tabular}
        \end{table}

        Para multiplica{\c c}{\~a}o, temos

        \begin{table}[h]
           \centering 
           \setlength{\arrayrulewidth}{0,5\arrayrulewidth}
           \begin{tabular}{|c|c|c|} 
              \hline
              $\otimes$ & $\overline{0}$ & $\overline{1}$\T\\
              \hline
              $\overline{0}$ & $\overline{0}$ & $\overline{0}$\T\\
              \hline
              $\overline{1}$ & $\overline{0}$ & $\overline{1}$\T\\
              \hline
           \end{tabular}
        \end{table}
    \end{frame}

    \begin{frame}
        \begin{definicao}
            Dados $\overline{a}$, $\overline{b} \in \z_m$ definimos
            \begin{align}
                \overline{a}\oplus\overline{b} &= \overline{a + b}\label{soma_modulo_m}\\
                \overline{a}\otimes\overline{b} &= \overline{ab}.\label{multiplicacao_modulo_m}
            \end{align}
        \end{definicao}

        \begin{proposicao}
            As opera{\c c}{\~o}es de soma e produto definidas em \eqref{soma_modulo_m} e \eqref{multiplicacao_modulo_m} s{\~a}o independentes dos representantes das classes.
        \end{proposicao}
        \begin{prova}
            Dadas duas classes em $\z_m$ com representantes diferentes, $\overline{a}_{1} = \overline{a}_{2}$, $\overline{b}_{1} = \overline{b}_{2}$, com $a_{1}\ne a_{2}$ e $b_{1}\ne b_{2}$, temos:
            \begin{align*}
                \overline{a}_{1}\oplus \overline{b}_{1} &= \overline{a_{1}+b_{1}} = \overline{a_{2} + b_{2}} =  \overline{a}_{2}\oplus \overline{b}_{2}\\
                \overline{a}_{1}\otimes \overline{b}_{1} &= \overline{a_{1}b_{1}} = \overline{a_{2}b_{2}} = \overline{a}_{2}\otimes\overline{b}_{2}.
            \end{align*}
        \end{prova}
    \end{frame}

    \begin{frame}
        \begin{exemplo}
            A some e a multiplica{\c c}{\~a}o em $\z_4 = \{\overline{0},\overline{1},\overline{2},\overline{3}\}$
            s\~ao dadas nas tabelas abaixo:
                \begin{table}[!htb]
                  \caption{Soma em $\z_4$}
                  \begin{minipage}{.5\linewidth}
                    \centering
                    \begin{tabular}{|c|c|c|c|c|} 
                        \hline
                        $\oplus$ & $\overline{0}$ & $\overline{1}$ & $\overline{2}$ & $\overline{3}$\T\\
                        \hline
                        $\overline{0}$ & $\overline{0}$ & $\overline{1}$ & $\overline{2}$ & $\overline{3}$\T\\
                        \hline
                        $\overline{1}$ & $\overline{1}$ & $\overline{2}$ & $\overline{3}$ & $\overline{0}$\T\\
                        \hline
                        $\overline{2}$ & $\overline{2}$ & $\overline{3}$ & $\overline{0}$ & $\overline{1}$\T\\
                        \hline
                        $\overline{3}$ & $\overline{3}$ & $\overline{0}$ & $\overline{1}$ & $\overline{2}$\T\\
                        \hline
                    \end{tabular}
                  \end{minipage}
                \end{table}
        \end{exemplo}
    \end{frame}

    \begin{frame}
        \begin{exemplo}
                \begin{table}[!htb]
                  \caption{Multiplica\c{c}\~ao em $\z_4$}
                  \begin{minipage}{.5\linewidth}
                  \centering
                    \begin{tabular}{|c|c|c|c|c|} 
                      \hline
                      $\otimes$ & $\overline{0}$ & $\overline{1}$ & $\overline{2}$ & $\overline{3}$\T\\
                      \hline
                      $\overline{0}$ & $\overline{0}$ & $\overline{0}$ & $\overline{0}$ & $\overline{0}$\T\\
                      \hline
                      $\overline{1}$ & $\overline{0}$ & $\overline{1}$ & $\overline{2}$ & $\overline{3}$\T\\
                      \hline
                      $\overline{2}$ & $\overline{0}$ & $\overline{2}$ & $\overline{0}$ & $\overline{2}$\T\\
                      \hline
                      $\overline{3}$ & $\overline{0}$ & $\overline{3}$ & $\overline{2}$ & $\overline{1}$\T\\
                      \hline
                    \end{tabular}
                \end{minipage}
            \end{table}
        \end{exemplo}
    \end{frame}

    \begin{frame}
        \begin{proposicao}
            As opera\c{c}\~oes de soma $\oplus$ e multiplica\c{c}\~ao $\otimes$ em $\z_m$ satisfazem as seguintes propriedades:
            \begin{enumerate}[label={\roman*})]
                \item Para todos $\overline{x}$, $\overline{y} \in \z_m$: $\overline{x} \oplus \overline{y} = \overline{y} \oplus \overline{x}$.
                \item Para todos $\overline{x}$, $\overline{y}$ e $\overline{z} \in \z_m$: $(\overline{x} \oplus \overline{y}) \oplus \overline{z} = \overline{x} \oplus (\overline{y} \oplus \overline{z})$.
                \item Para todo $\overline{x} \in \z_m$, $\overline{x} \oplus \overline{0} = \overline{x}$.
                \item Para todo $\overline{x} \in \z$, existe $\overline{y} \in \z$ tal que $\overline{x} \oplus \overline{y} = \overline{0}$.
                \item Para todos $\overline{x}$, $\overline{y} \in \z_m$: $\overline{x} \otimes \overline{y} = \overline{y} \otimes \overline{x}$.
                \item Para todos $\overline{x}$, $\overline{y}$ e $\overline{z} \in \z_m$: $(\overline{x} \otimes \overline{y}) \otimes \overline{z} = \overline{x} \otimes (\overline{y} \otimes \overline{z})$.
                \item Para todo $\overline{x} \in \z_m$: $\overline{x} \otimes \overline{1} = \overline{x}$.
            \end{enumerate}
        \end{proposicao}
    \end{frame}

    \begin{frame}
        \begin{prova}
            \begin{enumerate}[label={\roman*})]
                \item $\overline{x} \oplus \overline{y} = \overline{x + y} = \overline{y + x} = \overline{y} \oplus \overline{x}$.
                
                \item $(\overline{x} \oplus \overline{y}) \oplus \overline{z} = \overline{x + y} \oplus \overline{z} = \overline{(x + y) + z} = \overline{x + (y + z)} = \overline{x} \oplus \overline{y + z} = \overline{x} \oplus (\overline{y} \oplus \overline{z})$.

                \item $\overline{x} \oplus \overline{0} = \overline{x + 0} = \overline{x}$.

                \item Dado $\overline{x} \in \z_m$ escolha $\overline{y} = \overline{m - x} \in \z_m$. Assim $\overline{x} \oplus \overline{y} = \overline{x} \oplus \overline{m - x} = \overline{x + (m - x)} = \overline{m} = \overline{0}$.

                \item $\overline{x} \otimes \overline{y} = \overline{x \cdot y} = \overline{y \cdot x} = \overline{y} \otimes \overline{x}$.

                \item $(\overline{x} \otimes \overline{y}) \otimes \overline{z} = \overline{x \cdot y} \otimes \overline{z} = \overline{(x \cdot y)\cdot z} = \overline{x\cdot(y \cdot z)} = \overline{x} \otimes \overline{y \cdot z} = \overline{x} \otimes (\overline{y}\otimes \overline{z})$.

                \item $\overline{x} \otimes \overline{1} = \overline{x \cdot 1} = \overline{x}$.
            \end{enumerate}
        \end{prova}
    \end{frame}

    \begin{frame}
        \begin{definicao}
            Um elemento $\overline{a} \in \z_m$ {\'e} \textbf{invers{\'\i}vel} se, e somente se, existe $\overline{b} \in \z_m$ tal que $\overline{a} \otimes \overline{b} = \overline{1}$. Neste caso, $\overline{b}$ {\'e} chamado \textbf{inverso} de $\overline{a}$ e denotaremos $\overline{b} = (\overline{a})^{-1}$.
        \end{definicao}

        \begin{proposicao}
            Se o inverso existe, ent\~ao ele {\'e} {\'u}nico.
        \end{proposicao}
        \begin{prova}
            De fato, dado $\overline{a} \in \z_m$, suponha que existem $\overline{b}$, $\overline{d} \in \z_m$ tais que $\overline{a} \otimes \overline{b} = \overline{1} = \overline{a} \otimes \overline{d}$, ent{\~a}o
            \begin{align*}
                \overline{b} &= \overline{b} \otimes \overline{1} = \overline{b} \otimes (\overline{a} \otimes \overline{d})\\ &= (\overline{b} \otimes \overline{a}) \otimes \overline{d} = \overline{1} \otimes \overline{d} = \overline{d}
            \end{align*}
        \end{prova}
    \end{frame}

    \begin{frame}
        \begin{proposicao}
            Um elemento $\overline{a} \in \z_m$ {\'e} invers{\'\i}vel se, e somente se, $mdc(a,m)=1$.
        \end{proposicao}

        \begin{corolario}
            Se $m$ \'e um n\'umero primo, ent\~ao para todo $\overline{x} \in \z_m$, $\overline{x} \ne \overline{0}$, existe inverso.
        \end{corolario}

        \begin{exemplos}
            \begin{enumerate}[label={\arabic*})]
                \item Em $\z_4$ existem dois elementos invers{\'\i}veis que s{\~a}o $\overline{1}$, cujo inverso {\'e} $\overline{1}$, e o $\overline{3}$, cujo inverso {\'e} $\overline{3}$.
                \item Em $\z_{11}$, todos elementos, exceto $\overline{0}$, possuem inverso:

                \begin{table}[h]
                    \centering 
                    \setlength{\arrayrulewidth}{0,5\arrayrulewidth}
                    \caption{\it Inversos em $\z_{11}$}
                   \begin{tabular}{|c|c|c|c|c|c|c|c|c|c|c|} 
                        \hline
                        Elemento & $\overline{1}$ & $\overline{2}$ & $\overline{3}$ & $\overline{4}$ & $\overline{5}$ & $\overline{6}$ & $\overline{7}$ & $\overline{8}$ & $\overline{9}$ & $\overline{10}$\T \\
                        \hline
                        Inverso & $\overline{1}$ & $\overline{6}$ & $\overline{4}$ & $\overline{3}$ & $\overline{9}$ & $\overline{2}$ & $\overline{8}$ & $\overline{7}$ & $\overline{5}$ & $\overline{10}$\T \\
                        \hline
                   \end{tabular}
                \end{table}
            \end{enumerate}
        \end{exemplos}
    \end{frame}
\end{document}