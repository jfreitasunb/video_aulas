%!TEX program = xelatex
% !TEX encoding = ISO-8859-1
\def\ano{2020}
\def\semestre{1}
\def\disciplina{\'Algebra 1}
\def\turma{C}
\def\autor{Jos\'e Ant\^onio O. Freitas}
\def\instituto{MAT-UnB}

\documentclass{beamer}
\usetheme{Madrid}
\usecolortheme{beaver}
% \mode<presentation>
\usepackage{caption}
\usepackage{textpos}
\usepackage{amssymb}
\usepackage{amsmath,amsfonts,amsthm,amstext}
\usepackage[brazil]{babel}
% \usepackage[latin1]{inputenc}
\usepackage{graphicx}
\graphicspath{{/home/jfreitas/GitHub_Repos/video_aulas/logo/}{D:/Dropbox/imagens-latex/}}
\usepackage{enumitem}
\usepackage{multicol}
\usepackage{answers}
\usepackage{tikz,ifthen}
\usetikzlibrary{lindenmayersystems}
\usetikzlibrary[shadings]
\newtheorem{definicao}{Defini\c{c}\~ao}[section]
\newtheorem{definicoes}{Defini\c{c}\~oes}[section]
\newtheorem{exemplo}{Exemplo}[section]
\newtheorem{exemplos}{Exemplos}[section]
\newtheorem{exercicio}{Exerc{\'\i}cio}
\newtheorem{observacao}{Observa{\c c}{\~a}o:}[section]
\newtheorem{observacoes}{Observa{\c c}{\~o}es:}[section]
\newtheorem*{solucao}{Solu{\c c}{\~a}o:}
\newtheorem{proposicao}{Proposi\c{c}\~ao}
\newtheorem{lema}{Lema}
\newtheorem{teorema}{Teorema}
\newtheorem{corolario}{Corol\'ario}
\newenvironment{prova}[1][Prova]{\noindent\textbf{#1:} }{\qedsymbol}%{\ \rule{0.5em}{0.5em}}
\newcommand{\nsub}{\varsubsetneq}
\newcommand{\vaz}{\emptyset}
\newcommand{\im}{{\rm Im\,}}
\newcommand{\sub}{\subseteq}
\newcommand{\n}{\mathbb{N}}
\newcommand{\z}{\mathbb{Z}}
\newcommand{\rac}{\mathbb{Q}}
\newcommand{\real}{\mathbb{R}}
\newcommand{\complex}{\mathbb{C}}
\newcommand{\cp}[1]{\mathbb{#1}}
\newcommand{\ch}{\mbox{\textrm{car\,}}\nobreak}
\newcommand{\vesp}[1]{\vspace{ #1  cm}}
\newcommand{\compcent}[1]{\vcenter{\hbox{$#1\circ$}}}
\newcommand{\comp}{\mathbin{\mathchoice
{\compcent\scriptstyle}{\compcent\scriptstyle}
{\compcent\scriptscriptstyle}{\compcent\scriptscriptstyle}}}

\title{Anéis}
\author[\autor]{\autor}
\institute[\instituto]{\instituto}
\date{\today}

\begin{document}
    \begin{frame}
        \maketitle
    \end{frame}

    \logo{\includegraphics[scale=.1]{logo-MAT.png}\vspace*{8.5cm}}
    
    \begin{frame}
        \begin{definicao}
            Seja $A$ um conjunto n{\~a}o vazio. Dizemos que $A$ est{\'a} munido (ou equipado) de uma \textbf{opera{\c c}{\~a}o bin{\'a}ria} quando existe uma fun{\c c}{\~a}o
            \begin{align*}
                &\Delta : A \times A \to A\\
                &(a,b) \longmapsto a\Delta b        
            \end{align*}
            Uma opera{\c c}{\~a}o bin{\'a}ria tamb{\'e}m {\'e} chamada de uma \textbf{opera{\c c}{\~a}o interna} em $A$.
        \end{definicao}
    \end{frame}

    \begin{frame}
        \begin{exemplos}
            \begin{enumerate}[label={\arabic*})]
                \item A soma usual nos conjuntos $\z$, $\rac$, $\real$ e $\complex$ {\'e} uma opera{\c c}{\~a}o bin{\'a}ria.

                \vspace{.3cm}

                \item A multiplica\c{c}\~ao usual nos conjuntos $\z$, $\rac$, $\real$ e $\complex$ {\'e} uma opera{\c c}{\~a}o bin{\'a}ria.

                \vspace{.3cm}

                \item Seja $m > 1$, $m \in \z$ fixo. A soma e a multiplica\c{c}\~ao definidos em $\z_m = \{\overline{0},\overline{1},...,\overline{m-1}\}$ são opera\c{c}ões bin\'arias.
                
                \vspace{.3cm}

                \item A opera\c{c}\~ao $\div$ em $\rac^{*}$ {\'e} uma opera{\c c}{\~a}o bin{\'a}ria.
                
                \vspace{.3cm}

                \item J\'a em $\n$, $\z$, $\z^{*}$ e em $\rac$ a opera\c{c}\~ao $\div$ n{\~a}o {\'e} uma opera{\c c}{\~a}o bin{\'a}ria.
                
                \vspace{.3cm}

            \end{enumerate}
        \end{exemplos}
    \end{frame}

    \begin{frame}
        \begin{definicao}
            Seja $A$ um conjunto n{\~a}o vazio $A$ no qual est\~ao definidas duas opera{\c c}{\~o}es bin\'arias $\oplus$ e $\otimes$, chamadas \textit{soma} e \textit{produto}.  Dizemos que $(A, \oplus, \otimes)$ {\'e} um \textbf{anel} quando as seguintes condi{\c c}{\~o}es s{\~a}o verdadeiras:
            \begin{enumerate}[label={\roman*})]
                \item \textbf{Associatividade}: para todos $x$, $y$, $z \in A$ vale que
                \[
                    (x \oplus y) \oplus z = x \oplus (y \oplus z)
                \]
                Essa propriedade {\'e} chamada \textbf{propriedade associativa} da soma.

                \vspace{.7cm}

                \item \textbf{Comutatividade}: Para todos $x$, $y \in A$ vale
                \[
                    x \oplus y = y \oplus x.
                \]

                \vspace{.7cm}

                \seti
            \end{enumerate}
        \end{definicao}
    \end{frame}

    \begin{frame}
        \begin{definicao}
            \begin{enumerate}[label={\roman*})]
                \conti

                \item \textbf{Elemento Neutro}: Existe em $A$ um elemento denotado por $0$ (zero) ou $0_{A}$ tal que para todo elemento $x \in A$ vale
                \[
                    x \oplus 0_A = x = 0_A \oplus x.
                \]
                Tal elemento $0_A$ \'e chamado de \textbf{elemento neutro da soma} ou simplesmente \textbf{elemento neutro}.

                \vspace{.7cm}

                \item \textbf{Elemento Oposto}: Para cada elemento $x \in A$, existe $y \in A$ tal que
                \[
                    x \oplus y = 0_A = y \oplus x.
                \]
                Tal elemento $y$ \'e chamado de \textbf{oposto aditivo} de $x$ ou simplesmente \textbf{oposto} de $x$.

                \vspace{.7cm}

                \seti
    \end{enumerate}
        \end{definicao}
    \end{frame}

    \begin{frame}
        \begin{definicao}
            \begin{enumerate}[label={\roman*})]
                \conti

                \item \textbf{Associatividade}: Para todos $x$, $y$, $z \in A$, vale que
                \[
                    (x\otimes y) \otimes z = x\otimes (y\otimes z).
                \]

                \vspace{.7cm}

                \item \textbf{Distributividade}: Para todos $x$, $y$, $z \in A$ vale
                \[
                    (x \oplus y)\otimes z = x\otimes z \oplus y\otimes z.
                \]
                Essa propriedade {\'e} chamada \textbf{distributiva da soma em rela{\c c}{\~a}o ao produto}.

                \vspace{.7cm}

                \seti
            \end{enumerate}
        \end{definicao}
    \end{frame}

    \begin{frame}
        \begin{definicao}
            \begin{enumerate}[label={\roman*})]
                \conti

                \item \textbf{Distributividade}: Para todos $x$, $y$, $z \in A$ vale
                \[
                    x\otimes(y \oplus z) = x\otimes y \oplus x\otimes z.
                \]
                Essa {\'e} a propriedade \textbf{distributiva do produto em rela{\c c}{\~a}o {\`a} soma}.
            \end{enumerate}
        \end{definicao}
    \end{frame}

    \begin{frame}
        \begin{observacoes}
            Seja $(A, \oplus, \otimes)$ uma anel.
            \begin{enumerate}[label={\arabic*})]
                \item \textbf{Comutatividade}: Se para todos $x$, $y \in A$ vale
                \[
                    x \otimes y = y \otimes x.
                \]
                Dizemos que $(A, \oplus, \otimes)$ {\'e} um \textbf{anel comutativo}.

                \vspace{.7cm}

                \item \textbf{Unidade}: Se existe em $A$ um elemento denotado por $1$ ou $1_{A}$ tal que
                \[
                    x \otimes 1 = x = 1 \otimes x,
                \]
                para todo $x \in A$, ent{\~a}o dizemos que $(A, \oplus, \otimes)$ \'e um \textbf{anel com unidade} ou um \textbf{anel unit{\'a}rio}. O elemento $1_A$ {\'e} chamado de \textbf{unidade} de $A$ ou \textbf{elemento neutro da multiplica\c{c}\~ao} de $A$.

                \vspace{.7cm}

                \seti
             \end{enumerate}
        \end{observacoes}
    \end{frame}

    \begin{frame}
        \begin{observacoes}
            \begin{enumerate}[label={\arabic*})]
                \conti

                \item Se um anel $(A, \oplus, \otimes)$ satisfaz as duas propriedades anteriores dizemos que $(A, \oplus, \otimes)$ \'e um \textbf{anel comutativo com unidade} ou um \textbf{anel comutativo unit\'ario}.

                \vspace{.5cm}

                \item Seja $(A, \oplus, \otimes)$ uma anel. Quando n\~ao houver chance de confus\~ao com rela\c{c}\~ao \`as opera\c{c}\~oes envolvidas diremos simplesmente que $A$ \'e uma anel.
            \end{enumerate}
        \end{observacoes}
    \end{frame}

    \begin{frame}
        \begin{exemplos}
            \begin{enumerate}[label={\arabic*})]
                \item $(\z,+,.)$, $(\rac,+,.)$, $(\real,+,.)$, $(\complex,+,.)$, $(\z_m, \oplus, \otimes)$ s{\~a}o an{\'e}is associativos, comutativos e com unidade.

                \seti
            \end{enumerate}
        \end{exemplos}
    \end{frame}

    \begin{frame}
        \begin{exemplos}
            \begin{enumerate}[label={\arabic*})]
                \conti

                \item  Consideremos em $\z \times \z$ as opera\c{c}\~oes $\oplus$ e $\otimes$ definidas por
                \vspace{.3cm}
                \begin{center}
                    \begin{tabular}{l}
                        $(a, b) \oplus (c, d) = (a + c, b + d)$\\
                        \\
                        $(a ,b) \otimes (c, d) = (ac - bd, ad + bc)$.    
                    \end{tabular}
                \end{center}
                \vspace{.3cm}
                Mostre que $(\z \times \z, \oplus, \otimes)$ \'e um anel comutativo e com unidade.
            \end{enumerate}
        \end{exemplos}
    \end{frame}

    \begin{frame}
        \begin{observacao}
            Seja $(A, \oplus, \cdot)$ um anel. Para simplificar a nota\c{c}\~ao vamos denotar a opera\c{c}\~ao $\oplus$
            por $+$ e a opera\c{c}\~ao $\otimes$ por $\cdot$ e assim escrever simplesmente que $(A, +, \cdot)$ \'e um anel.
        \end{observacao}
    \end{frame}

    \begin{frame}
        \begin{proposicao}
            Seja $(A, + , \cdot)$ uma anel. Ent\~ao:
            \begin{enumerate}[label={\roman*})]
                \item O elemento neutro {\'e} {\'u}nico.

                \vspace{.5cm}

                \item Para cada $x \in A$ existe um {\'u}nico oposto.

                \vspace{.5cm}
                
                \item Para todo $x \in A$, $-(-x) = x$.

                \vspace{.5cm}
                
                \item Dados $x_{1}$, $x_{2}$, \dots, $x_n \in A$, $n \geqslant 2$, ent{\~a}o
                \[
                    -(x_1 + x_2 + \dots + x_n) = (-x_1) + (-x_2) + \dots + (-x_n).
                \]

                \vspace{.5cm}

                \seti
            \end{enumerate}
        \end{proposicao}
    \end{frame}

    \begin{frame}
        \begin{proposicao}
            \begin{enumerate}[label={\roman*})]
                \conti
                
                \item Para todos $a$, $x$, $y \in A$, se $a + x = a + y$, ent{\~a}o $x = y$.

                \vspace{.5cm}
                
                \item Para todo $x \in A$, $x\cdot 0_A = 0_A = 0_A\cdot x$.

                \vspace{.5cm}
                
                \item Para todos $x$, $y \in A$, temos $x(-y) = (-x)y = -(xy)$.

                \vspace{.5cm}
                
                \item Para todos $x$, $y \in A$, $xy = (-x)(-y)$.

                \vspace{.5cm}
            \end{enumerate}
        \end{proposicao}
    \end{frame}

    \begin{frame}
        \noindent \textbf{\textit{Prova:} }\pause
        \begin{enumerate}[label={\roman*})]
            \item Suponha que existam $0_1$, $0_2\in A$ elementos neutros de $A$. Assim
                
            \[
                x + 0_1 = x \quad \mbox{e}\quad x + 0_2 = x 
            \]
            para todo $x \in A$. Assim

            \[
                   0_1 = 0_1 + 0_2 = 0_2
            \]
            
            e portanto o elemento neutro \'e \'unico.
                
            \vspace{.5cm}

            \seti
        \end{enumerate}
    \end{frame}

    \begin{frame}
        \begin{enumerate}[label={\roman*})]
            \conti

            \item De fato, dado $x \in A$ suponha que existam $y_1$, $y_2\in A$ tais que
            \[
                x + y_1 = 0_A \quad \mbox{e}\quad x + y_2 = 0_A.
            \]
            Da{\'\i}
            \[
                y_1 = y_2 + 0_A = y_1 + (x + y_2) = (y_1 + x) + y_2 = 0_A + y_2 =y_2.
            \]
            Logo o oposto de $x$ \'e \'unico  e da{\'\i} ser\'a denotado por $-x$.
                
            \vspace{.5cm}

            \item Dado $x \in A$, ent\~ao $-x$ {\'e} oposto de $x$, isto {\'e}, $x + (-x) = 0_A$. Logo o oposto de $(-x)$ {\'e} $x$, ou seja, $-(-x) = x$.
            
            \seti
        \end{enumerate}
    \end{frame}

    \begin{frame}
            \begin{enumerate}[label={\roman*})]
                \conti

                \item Segue usando indu\c{c}\~ao sobre $n$.

                \vspace{.5cm}

                \item Suponha que $a + x = a + y$. Seja $-a$ o oposto de $a$ da{\'\i}
                \begin{align*}
                    x &= 0_A + x \\ &= [(-a) + a] + x\\ &= (-a) + (a + x) \\ &= (-a) + (a + y) \\ &= [(-a) + a] + y \\ &= 0_A + y = y
                \end{align*}
                como quer{\'\i}amos.
                \seti
            \end{enumerate}
    \end{frame}

    \begin{frame}
            \begin{enumerate}[label={\roman*})]
                \conti
                \item Temos $0_A + x\cdot 0_A = a\cdot 0_A = a(0_A + 0_A) = a\cdot 0_A + a\cdot 0_A$. Assim do item anterior segue que $x\cdot 0_A = 0_A$.

                \vspace{.5cm}

                \item Provemos que $x(-y) = -(xy)$:
                \[
                    x(-y) + xy = x((-y) + y) = x\cdot 0_A = 0_A,
                \]
                portanto $-xy = x(-y)$.

                \vspace{.5cm}
                
                \item Basta usar o caso anterior.
            \end{enumerate}
    \end{frame}

    \begin{frame}
        \begin{definicao}
            Um anel comutativo $(A, + , \cdot)$ {\'e} dito ser um \textbf{anel de integridade} quando para todos 
            $x$, $y \in A$, se $xy = 0_A$, ent{\~a}o $x = 0_A$ ou $y = 0_a$. Um anel de integridade tamb{\'e}m {\'e} chamado de \textbf{dom{\'\i}nio de integridade} ou simplesmente de \textbf{dom{\'\i}nio}.
        \end{definicao}

        \begin{observacao}
            Se $x$ e $y$ s{\~a}o elementos n{\~a}o nulos de um anel $A$ tais que $xy = 0_A$, ent{\~a}o $x$ e $y$ s{\~a}o chamados de \textbf{divisores pr{\'o}prios de zero}.
        \end{observacao}
    \end{frame}

    \begin{frame}
        \begin{exemplos}
            \begin{enumerate}[label={\arabic*})]
                \item Os an{\'e}is $\z$, $\rac$, $\real$, $\complex$ s{\~a}o an{\'e}is de integridade.
                
                \vspace{.5cm}

                \item Em geral $\z_m$ n{\~a}o {\'e} anel de integridade, por exemplo, em $\z_4$, $\overline{2} \neq \overline{0}$, no entanto $\overline{2}\otimes \overline{2} = \overline{4} = \overline{0}$.

                \vspace{.5cm}

                \seti
            \end{enumerate}
        \end{exemplos}
    \end{frame}

    \begin{frame}
        \begin{exemplos}
            \begin{enumerate}[label={\arabic*})]
                \conti

                \item $M_{n}(\real)$ n{\~a}o {\'e} um anel de integridade, por exemplo, em $M_{2}(\real)$
                \begin{align*}
                    A &= \begin{bmatrix}
                        1 & 0\\
                        0 & 0
                    \end{bmatrix} \neq \begin{bmatrix}
                        0 & 0\\
                        0 & 0       
                    \end{bmatrix},\qquad 
                    B = \begin{bmatrix}
                        0 & 0\\
                        1 & 0
                    \end{bmatrix} \neq \begin{bmatrix}
                        0 & 0\\
                        0 & 0
                    \end{bmatrix}\\
                    AB & =\begin{bmatrix}
                        0 & 0\\
                        0 & 0
                    \end{bmatrix}
                \end{align*}

                \vspace{.5cm}
                
                \item Suponha que $m = nk$, $m > n > 1$ e $m > k > 1$. Logo, em $\z_m$, $\overline{n} \neq \overline{0}$ e $\overline{k} \neq \overline{0}$ e no entanto $\overline{n} \otimes \overline{k} = \overline{m} = \overline{0}$. Logo, se $m$ n{\~a}o {\'e} primo, ent{\~a}o $\z_m$ n{\~a}o {\'e} um anel de integridade. Agora, suponha que $m = p$ primo. Sejam $\overline{x}$, $\overline{y} \in \z_m$ tais que $\overline{x}\otimes \overline{y} = \overline{0}$, ou seja, $xy \equiv 0 \pmod p$. Da{\'\i} $p\mid xy$. Logo $p\mid x$ ou $p\mid y$. Portanto, $\overline{x} = \overline{0}$ ou $\overline{y} = \bar{0}$. Assim, $\z_m$ {\'e} anel de integridade se, e somente se, $m$ {\'e} primo.
            \end{enumerate}
        \end{exemplos}
    \end{frame}
\end{document}