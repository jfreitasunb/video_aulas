%!TEX program = xelatex
% !TEX encoding = ISO-8859-1
\def\ano{2020}
\def\semestre{1}
\def\disciplina{\'Algebra 1}
\def\turma{C}
\def\autor{Jos\'e Ant\^onio O. Freitas}
\def\instituto{MAT-UnB}

\documentclass{beamer}
\usetheme{Madrid}
\usecolortheme{beaver}
% \mode<presentation>
\usepackage{caption}
\usepackage{textpos}
\usepackage{amssymb}
\usepackage{amsmath,amsfonts,amsthm,amstext}
\usepackage[brazil]{babel}
% \usepackage[latin1]{inputenc}
\usepackage{graphicx}
\graphicspath{{/home/jfreitas/GitHub_Repos/video_aulas/logo/}{D:/Dropbox/imagens-latex/}}
\usepackage{enumitem}
\usepackage{multicol}
\usepackage{answers}
\usepackage{tikz,ifthen}
\usetikzlibrary{lindenmayersystems}
\usetikzlibrary[shadings]
\newtheorem{definicao}{Defini\c{c}\~ao}[section]
\newtheorem{definicoes}{Defini\c{c}\~oes}[section]
\newtheorem{exemplo}{Exemplo}[section]
\newtheorem{exemplos}{Exemplos}[section]
\newtheorem{exercicio}{Exerc{\'\i}cio}
\newtheorem{observacao}{Observa{\c c}{\~a}o:}[section]
\newtheorem{observacoes}{Observa{\c c}{\~o}es:}[section]
\newtheorem*{solucao}{Solu{\c c}{\~a}o:}
\newtheorem{proposicao}{Proposi\c{c}\~ao}
\newtheorem{lema}{Lema}
\newtheorem{teorema}{Teorema}
\newtheorem{corolario}{Corol\'ario}
\newenvironment{prova}[1][Prova]{\noindent\textbf{#1:} }{\qedsymbol}%{\ \rule{0.5em}{0.5em}}
\newcommand{\nsub}{\varsubsetneq}
\newcommand{\vaz}{\emptyset}
\newcommand{\im}{{\rm Im\,}}
\newcommand{\sub}{\subseteq}
\newcommand{\n}{\mathbb{N}}
\newcommand{\z}{\mathbb{Z}}
\newcommand{\rac}{\mathbb{Q}}
\newcommand{\real}{\mathbb{R}}
\newcommand{\complex}{\mathbb{C}}
\newcommand{\cp}[1]{\mathbb{#1}}
\newcommand{\ch}{\mbox{\textrm{car\,}}\nobreak}
\newcommand{\vesp}[1]{\vspace{ #1  cm}}
\newcommand{\compcent}[1]{\vcenter{\hbox{$#1\circ$}}}
\newcommand{\comp}{\mathbin{\mathchoice
{\compcent\scriptstyle}{\compcent\scriptstyle}
{\compcent\scriptscriptstyle}{\compcent\scriptscriptstyle}}}

\title{Rela\c{c}\~ao de Equival\^encia}
\author[\autor]{\autor}
\institute[\instituto]{\instituto}
\date{\today}

\begin{document}
    \begin{frame}
        \maketitle
    \end{frame}

    \logo{\includegraphics[scale=.1]{logo-MAT.png}\vspace*{8.5cm}}

    \begin{frame}
        \begin{definicao}
            Seja $A$ um conjunto n{\~a}o vazio e $R\subseteq A \times A$. Dizemos que $R$ {\'e} uma \textbf{rela{\c c}{\~a}o de equival{\^e}ncia} se:
            \begin{enumerate}[label={\roman*})]
                \item Para todo $x \in A$, $(x,x) \in R$. \textit{(Propriedade Reflexiva)}
                \item Se $(x, y) \in R$, ent\~ao $(y, x) \in R$. \textit{(Propriedade Sim\'etrica)}
                \item Se $(x, y) \in R$ e $(y, z) \in R$, ent\~ao $(x, z)\in R$. \textit{(Propriedade Transitiva)}
            \end{enumerate}
        \end{definicao}

        Quando $R\subseteq A \times A$ {\'e} uma rela{\c c}{\~a}o de equival{\^e}ncia, dizemos que $R$ {\'e} uma rela{\c c}{\~a}o de equival{\^e}ncia em $A$. Quando dois elementos $x$, $y \in A$ s{\~a}o tais que $(x,y) \in R$, dizemos que $x$ e $y$ \textbf{s{\~a}o relacionados} ou que $x$ e $y$ \textbf{est\~ao relacionados}.
    \end{frame}

    \begin{frame}
        \begin{exemplos}\label{exemplos_relacoes_equivalencia}
            \begin{enumerate}[label={\arabic*})]
                \item Seja A=\{1,2,3,4\}. Temos
                \begin{align*}
                    A\times A = &\{(1,1);(1,2);(1,3);(1,4);(2,1);(2,2);(2,3);(2,4);\\ &(3,1);(3,2);(3,3);(3,4);(4,1);(4,2);(4,3);(4,4)\}.
                \end{align*}
                Quais dos seguintes conjuntos s\~ao exemplos de rela{\c c}{\~o}es de equival{\^e}ncia?
                \begin{itemize}
                    \item $R_{1}= A\times A$
                    \item $R_{2}=\{(1,1);(2,2);(3,3)\}$
                    \item $R_{3}=\{(1,1);(2,2);(3,3);(4,4);(1,2);(2,1)\}$
                    \item $R_{4}=\{(1,1);(2,2);(3,3);(4,4)\}$
                    \item $R_{5}=\{(1,1);(2,2);(3,3);(4,4);(1,2);(2,1);(2,4);(4;2)\}$
                \end{itemize}
                \begin{solucao}
                    $R_2$ n\~ao \'e rela\c{c}\~ao de equival\^encia pois $(4,4) \notin R_2$.

                    $R_5$ n\~ao \'e rela\c{c}\~ao de equival\^encia pois, por exemplo, $(1,4) \notin R_5$.

                    Os demais s\~ao exemplos de rela\c{c}\~oes de equival\^encia.
                \end{solucao}
                
                \item Seja $A = \z$ e $R\subseteq \z\times \z$ definida por $R = \{(x,y)\in \z \times \z \mid x = y\}$.
                Ent\~ao $R$ {\'e} uma rela{\c c}{\~a}o de equival{\^e}ncia.
                \begin{solucao}
                    De fato,
                    \begin{itemize}
                        \item Para todo $x \in \z$ temos $x = x$ da{\'\i} $(x,x) \in R$.
                        \item Se $(x,y)\in R$, ent\~ao pela defini\c{c}\~ao de $R$ temos $x = y$. Logo $y = x$ e ent\~ao $(y,x)\in R$.
                        \item Se $(x,y) \in R$ e $(y,z) \in R$, ent\~ao  $x = y$ e $y = z$. Logo $x = z$ e assim $(x,z)\in R$ como quer{\'\i}amos.
                    \end{itemize}
                    Portanto $R$ \'e uma rela\c{c}\~ao de equival\^encia sobre $\z$.
                \end{solucao}
                
                \item Seja $A = \z$ e tome $R = \{(x,y)\in \z \times \z \mid x - y = 2k, \mbox{ para algum } k \in \z\}$. Mostre que $R$
                \'e uma rela{\c c}{\~a}o de equival{\^e}ncia sobre $\z$.
                \begin{solucao}
                    De fato,
                    \begin{itemize}
                        \item Para todo $x\in\z$ temos $x - x = 2\cdot0$ e com isso $(x,x) \in R$.
                        \item Se $(x,y) \in R$ ent\~ao existe $k \in \z$ tal que $x - y = 2k$. Agora $y - x = -(x - y) = -2k = 2 (-k)$ 
                        e como $-k \in \z$ segue que $(y,x) \in R$.
                        \item Se $(x,y) \in R$ e $(y,z) \in R$, ent\~ao existem $k$, $l\in \z$ tais que $x - y = 2k$ e $y - z = 2l$.
                        Somando essas duas equa\c{c}\~oes obtemos
                        \begin{align*}
                            (x - y) + (y - z) &= 2k + 2l\\
                            x - z &= 2(k + l)
                        \end{align*}
                        e como $k + l \in \z$ segue que $(x,z) \in \z$.
                    \end{itemize}
                    Assim $R$ \'e uma rela\c{c}\~ao de equival\^encia.
                \end{solucao}
            \end{enumerate}
        \end{exemplos}
    \end{frame}
\end{document}