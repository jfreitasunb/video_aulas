%!TEX program = xelatex
\def\ano{2020}
\def\semestre{1}
\def\disciplina{\'Algebra 1}
\def\turma{C}
\def\autor{Jos\'e Ant\^onio O. Freitas}
\def\instituto{MAT-UnB}

\documentclass{beamer}
\usetheme{Madrid}
\usecolortheme{beaver}
\usepackage{pgfplots}
\pgfplotsset{compat=1.15}
\usepackage{mathrsfs}
\usetikzlibrary{arrows}

% \mode<presentation>
\usepackage{caption}
\usepackage{textpos}
\usepackage{amssymb}
\usepackage{amsmath,amsfonts,amsthm,amstext}
\usepackage[brazil]{babel}
% \usepackage[latin1]{inputenc}
\usepackage{graphicx}
\graphicspath{{/home/jfreitas/GitHub_Repos/video_aulas/logo/}{D:/Dropbox/imagens-latex/}}
\usepackage{enumitem}
\usepackage{multicol}
\usepackage{answers}
\usepackage{tikz,ifthen}
\usetikzlibrary{lindenmayersystems}
\usetikzlibrary[shadings]
\newtheorem{definicao}{Defini\c{c}\~ao}[section]
\newtheorem{definicoes}{Defini\c{c}\~oes}[section]
\newtheorem{exemplo}{Exemplo}[section]
\newtheorem{exemplos}{Exemplos}[section]
\newtheorem{exercicio}{Exerc{\'\i}cio}
\newtheorem{observacao}{Observa{\c c}{\~a}o:}[section]
\newtheorem{observacoes}{Observa{\c c}{\~o}es:}[section]
\newtheorem*{solucao}{Solu{\c c}{\~a}o:}
\newtheorem{proposicao}{Proposi\c{c}\~ao}
\newtheorem{lema}{Lema}
\newtheorem{teorema}{Teorema}
\newtheorem{corolario}{Corol\'ario}
\newenvironment{prova}[1][Prova]{\noindent\textbf{#1:} }{\qedsymbol}%{\ \rule{0.5em}{0.5em}}
\newcommand{\nsub}{\varsubsetneq}
\newcommand{\vaz}{\emptyset}
\newcommand{\im}{{\rm Im\,}}
\newcommand{\sub}{\subseteq}
\newcommand{\n}{\mathbb{N}}
\newcommand{\z}{\mathbb{Z}}
\newcommand{\rac}{\mathbb{Q}}
\newcommand{\real}{\mathbb{R}}
\newcommand{\complex}{\mathbb{C}}
\newcommand{\cp}[1]{\mathbb{#1}}
\newcommand{\ch}{\mbox{\textrm{car\,}}\nobreak}
\newcommand{\vesp}[1]{\vspace{ #1  cm}}
\newcommand{\compcent}[1]{\vcenter{\hbox{$#1\circ$}}}
\newcommand{\comp}{\mathbin{\mathchoice
{\compcent\scriptstyle}{\compcent\scriptstyle}
{\compcent\scriptscriptstyle}{\compcent\scriptscriptstyle}}}

\title{Fun\c{c}\~oes - Imagem Direta e Inversa}
\author[\autor]{\autor}
\institute[\instituto]{\instituto}
\date{}

\begin{document}
    \begin{frame}
        \maketitle
    \end{frame}

    \logo{\includegraphics[scale=.1]{logo-MAT.png}\vspace*{8.5cm}}

    \begin{frame}
        \begin{definicao}
            Seja $f : A \to B$ \pause uma fun{\c c}{\~a}o.\pause
            \begin{enumerate}[label={\roman*})]
                \item Dado $P \sub A$, \pause chama-se \textbf{imagem direta} \pause de $P$ \pause \textbf{segundo} $f$ \pause e indica-se por $f(P)$ \pause o subconjunto de $B$ \pause dado por\pause
                \[
                    f(P) = \pause \{f(x) \pause \mid x \in P\},\pause
                \]
                isto {\'e}, \pause $f(P)$ \pause {\'e} o conjunto das imagens por $f$ \pause dos elementos de $P$.\pause

                \vspace{.5cm}

                \item Dado $Q \sub B$, \pause chama-se \textbf{imagem inversa} \pause de $Q$ \textbf{segundo} $f$ \pause e indica-se por \pause $f^{-1}(Q)$ \pause o subconjunto de $A$ \pause dado por\pause
                \[
                    f^{-1}(Q) \pause = \{x \in A \pause \mid f(x) \in Q\},\pause
                \]
                isto {\'e}, \pause $f^{-1}(Q)$ \pause {\'e} o conjunto dos elementos de $A$ \pause que tem imagem em $Q$ \pause atrav{\'e}s de $f$.\pause
            \end{enumerate}
        \end{definicao}
    \end{frame}

    \begin{frame}
        \begin{center}
            \definecolor{qqwuqq}{rgb}{0,0.39215686274509803,0}
            \begin{tikzpicture}[line cap=round,line join=round,>=triangle 45,x=1cm,y=1cm]
                \begin{axis}[
                    x=1cm,y=1cm,
                    axis lines=middle,
                    xmin=-3.799766551541622,
                    xmax=3.66180069974375,
                    ymin=-3.772843851129456,
                    ymax=4.427922943806761,
                    xtick={-1.5708, 0, 1.5708},
                    xticklabels={$-\frac{\pi}{2}$, 0 , $\frac{\pi}{2}$},
                    ytick={-3.5,-3,...,4},
                    yticklabels={,,},
                ]
                    \clip(-5.799766551541622,-3.772843851129456) rectangle (3.66180069974375,4.427922943806761);
                    \draw[line width=2pt,color=qqwuqq,smooth,samples=100,domain=-1.499766551541622:1.46180069974375] plot(\x,{tan(((\x))*180/pi)});

                    \begin{scriptsize}
                        \draw[color=qqwuqq] (-2.673499485506525,-3.627346375735427) node {$f(x) = \tan(x)$};
                    \end{scriptsize}

                \end{axis}
            \end{tikzpicture}
        \end{center}
    \end{frame}

    \begin{frame}
        \begin{center}
            \definecolor{qqwuqq}{rgb}{0,0.39215686274509803,0}
            \begin{tikzpicture}[line cap=round,line join=round,>=triangle 45,x=1cm,y=1cm]
                \begin{axis}[
                    x=1cm,y=1cm,
                    axis lines=middle,
                    xmin=-3.799766551541622,
                    xmax=3.66180069974375,
                    ymin=-3.772843851129456,
                    ymax=4.427922943806761,
                    xtick={-1.5708, 0, 1.5708},
                    xticklabels={$-\frac{\pi}{2}$, 0 , $-\frac{\pi}{2}$},
                    ytick={-3.5,-3,...,4},
                    yticklabels={,,}
                ]
                    \clip(-5.799766551541622,-3.772843851129456) rectangle (3.66180069974375,4.427922943806761);
                    \draw[line width=2pt,color=qqwuqq,smooth,samples=100,domain=-1.499766551541622:1.46180069974375] plot(\x,{tan(((\x))*180/pi)});

                    \begin{scriptsize}
                        \draw[color=qqwuqq] (-2.673499485506525,-3.627346375735427) node {$f(x) = \tan(x)$};
                    \end{scriptsize}

                \end{axis}
            \end{tikzpicture}
        \end{center}
    \end{frame}

    \begin{frame}
        \definecolor{qqwuqq}{rgb}{0,0.39215686274509803,0}
        \begin{tikzpicture}[line cap=round,line join=round,>=triangle 45,x=1cm,y=1cm]
        \begin{axis}[
            x=1cm,y=1cm,
            axis lines=middle,
            xmin=-6.091971853434739,
            xmax=5.498177280367296,
            ymin=-3.365807745222352,
            ymax=3.816557833474226,
            xtick={
                    -6.28318, -4.7123889, -3.14159, -1.5708,
                    1.5708, 3.14159, 4.7123889, 6.28318
                },
            xticklabels={
                    $-2\pi$, $-\frac{3\pi}{2}$, $-\pi$, $-\frac{\pi}{2}$,
                    $\frac{\pi}{2}$, $\pi$, $\frac{3\pi}{2}$, $2\pi$
                },
            ytick={-4,-3,...,4},
            yticklabels={,,}
        ]
            \clip(-7.091971853434739,-7.365807745222352) rectangle (5.498177280367296,6.816557833474226);
            \draw[line width=2pt,color=qqwuqq,smooth,samples=100,domain=-6.091971853434739:5.498177280367296] plot(\x,{3*sin(((\x))*180/pi)});
            \begin{scriptsize}
                \draw[color=qqwuqq] (-4.8,-3) node {$g(x) = 3\sin(x)$};
            \end{scriptsize}
            \end{axis}
        \end{tikzpicture}
    \end{frame}

    \begin{frame}
        \definecolor{qqwuqq}{rgb}{0,0.39215686274509803,0}
        \begin{tikzpicture}[line cap=round,line join=round,>=triangle 45,x=1cm,y=1cm]
        \begin{axis}[
            x=1cm,y=1cm,
            axis lines=middle,
            xmin=-6.091971853434739,
            xmax=5.498177280367296,
            ymin=-3.365807745222352,
            ymax=3.816557833474226,
            xtick={
                    -6.28318, -4.7123889, -3.14159, -1.5708,
                    1.5708, 3.14159, 4.7123889, 6.28318
                },
            xticklabels={
                    $-2\pi$, $-\frac{3\pi}{2}$, $-\pi$, $-\frac{\pi}{2}$,
                    $\frac{\pi}{2}$, $\pi$, $\frac{3\pi}{2}$, $2\pi$
                },
            ytick={-4,-3,...,4},
            yticklabels={,,}
        ]
            \clip(-7.091971853434739,-7.365807745222352) rectangle (5.498177280367296,6.816557833474226);
            \draw[line width=2pt,color=qqwuqq,smooth,samples=100,domain=-6.091971853434739:5.498177280367296] plot(\x,{3*sin(((\x))*180/pi)});
            \begin{scriptsize}
                \draw[color=qqwuqq] (-4.8,-3) node {$g(x) = 3\sin(x)$};
            \end{scriptsize}
            \end{axis}
        \end{tikzpicture}
    \end{frame}

    \begin{frame}
        \begin{exemplos}
            \begin{enumerate}
                \item[1)] Seja $A = \{1, 3, 5, 7, 9 \}$ \pause e $B = \{0, 1, 2, 3, \dots, 10\}$ \pause e $f : A \to B$ \pause dada por $f(x) = x + 1$. \pause Temos:\pause

                \vspace{.5cm}

                \begin{itemize}
                    \item $f(\{1\}) = \pause \{f(1)\} \pause= \{2\}$\pause

                    \vspace{.5cm}

                    \item $f(\{3, 5, 7\}) \pause= \{f(3), \pause f(5), \pause f(7)\} \pause = \{4, 6, 8\}$\pause

                    \vspace{.5cm}

                    \item $f(A) \pause = \{f(1), \pause f(3), \pause f(5), \pause f(7), \pause f(9)\} = \pause \{2, 4, 6, 8, 10\}$\pause

                    \vspace{.5cm}

                    \item $f(\emptyset) \pause = \emptyset$\pause

                    \vspace{.5cm}

                    \item $f^{-1}(\{2, 4, 10\}) = \pause \{x \in A \pause \mid f(x) \pause \in \{2, 4, 10\}\pause\} = \{1, 3, 9\}$\pause

                    \vspace{.5cm}

                    \item $f^{-1}(\{0, 1, 3, 5, 7, 9\}) \pause = \{x \in A \pause \mid f(x) \pause \in \{0, 1, 3, 5, 7, 9\}\pause\} = \emptyset$\pause

                    \vspace{.5cm}
                \end{itemize}
            \end{enumerate}
        \end{exemplos}
    \end{frame}

    \begin{frame}
        \begin{exemplos}
            \begin{enumerate}
                \item[2)] Sejam $A = B = \real$ \pause e $f : \real \to \real$ \pause dada por $f(x) = x^2$. \pause Temos:\pause

                \vspace{.5cm}

                \begin{itemize}
                    \item $f(\{1, 2, 3\}) \pause = \{1, 4, 9\}$\pause

                    \vspace{.5cm}

                    \item $f([0,2]) \pause = \{f(x) \pause \in \real \pause \mid 0 \le x \pause \le 2 \pause \} = \{x^2 \pause \mid 0 \le x \le 2\} \pause = [0, 4]$\pause

                    \vspace{.5cm}

                    \item $f^{-1}([1, 9]) \pause = \{x \in \real \pause \mid f(x) \in [1, 9]\pause \} = \{ x \in \real \pause \mid 1 \le f(x) \le 9\} \pause = \{x \in \real \pause \mid 1 \le x^2 \pause \le 9\} \pause = [-1, -3] \pause \cup [1, 3]$\pause

                    \vspace{.5cm}
                \end{itemize}
            \end{enumerate}
        \end{exemplos}
    \end{frame}

    \begin{frame}
        \begin{proposicao}
            Seja $f : A \to B$ uma fun{\c c}{\~a}o \pause e sejam $P$, \pause $Q \sub A$, \pause $R$, \pause $S \sub B$.\pause
            \begin{enumerate}[label={\roman*})]
                \item Se $P \sub Q$, \pause ent{\~a}o $f(P) \sub f(Q)$.\pause

                \vspace{.5cm}

                \item $f^{-1}(R \cup S) \pause = f^{-1}(R) \pause \cup f^{-1}(S)$.\pause
            \end{enumerate}
        \end{proposicao}
    \end{frame}
\end{document}
