%!TEX program = xelatex
% !TEX encoding = ISO-8859-1
\def\ano{2020}
\def\semestre{1}
\def\disciplina{\'Algebra 1}
\def\turma{C}
\def\autor{Jos\'e Ant\^onio O. Freitas}
\def\instituto{MAT-UnB}

\documentclass{beamer}
\usetheme{Madrid}
\usecolortheme{beaver}
% \mode<presentation>
\usepackage{caption}
\usepackage{textpos}
\usepackage{amssymb}
\usepackage{amsmath,amsfonts,amsthm,amstext}
\usepackage[brazil]{babel}
% \usepackage[latin1]{inputenc}
\usepackage{graphicx}
\graphicspath{{/home/jfreitas/GitHub_Repos/video_aulas/logo/}{D:/Dropbox/imagens-latex/}}
\usepackage{enumitem}
\usepackage{multicol}
\usepackage{answers}
\usepackage{tikz,ifthen}
\usetikzlibrary{lindenmayersystems}
\usetikzlibrary[shadings]
\newtheorem{definicao}{Defini\c{c}\~ao}[section]
\newtheorem{definicoes}{Defini\c{c}\~oes}[section]
\newtheorem{exemplo}{Exemplo}[section]
\newtheorem{exemplos}{Exemplos}[section]
\newtheorem{exercicio}{Exerc{\'\i}cio}
\newtheorem{observacao}{Observa{\c c}{\~a}o:}[section]
\newtheorem{observacoes}{Observa{\c c}{\~o}es:}[section]
\newtheorem*{solucao}{Solu{\c c}{\~a}o:}
\newtheorem{proposicao}{Proposi\c{c}\~ao}
\newtheorem{lema}{Lema}
\newtheorem{teorema}{Teorema}
\newtheorem{corolario}{Corol\'ario}
\newenvironment{prova}[1][Prova]{\noindent\textbf{#1:} }{\qedsymbol}%{\ \rule{0.5em}{0.5em}}
\newcommand{\nsub}{\varsubsetneq}
\newcommand{\vaz}{\emptyset}
\newcommand{\im}{{\rm Im\,}}
\newcommand{\sub}{\subseteq}
\newcommand{\n}{\mathbb{N}}
\newcommand{\z}{\mathbb{Z}}
\newcommand{\rac}{\mathbb{Q}}
\newcommand{\real}{\mathbb{R}}
\newcommand{\complex}{\mathbb{C}}
\newcommand{\cp}[1]{\mathbb{#1}}
\newcommand{\ch}{\mbox{\textrm{car\,}}\nobreak}
\newcommand{\vesp}[1]{\vspace{ #1  cm}}
\newcommand{\compcent}[1]{\vcenter{\hbox{$#1\circ$}}}
\newcommand{\comp}{\mathbin{\mathchoice
{\compcent\scriptstyle}{\compcent\scriptstyle}
{\compcent\scriptscriptstyle}{\compcent\scriptscriptstyle}}}

\title{Fun\c{c}\~oes - Continua\c{c}\~ao}
\author[\autor]{\autor}
\institute[\instituto]{\instituto}
\date{\today}

\begin{document}
    \begin{frame}
        \maketitle
    \end{frame}

    \logo{\includegraphics[scale=.1]{logo-MAT.png}\vspace*{8.5cm}}
    
    \begin{frame}
        \begin{definicao}
            Seja $f : A \to B$ \pause uma fun{\c c}{\~a}o.\pause
            \begin{enumerate}[label={\roman*})]
                \item Dado $P \sub A$, \pause chama-se \textbf{imagem direta} \pause de $P$ \pause \textbf{segundo} $f$ \pause e indica-se por $f(P)$ \pause o subconjunto de $B$ \pause dado por\pause
                \[
                    f(P) = \pause \{f(x) \pause \mid x \in P\},\pause
                \]
                isto {\'e}, \pause $f(P)$ \pause {\'e} o conjunto das imagens por $f$ \pause dos elementos de $P$.\pause
                
                \vspace{.5cm}

                \item Dado $Q \sub B$, \pause chama-se \textbf{imagem inversa} \pause de $Q$ \textbf{segundo} $f$ \pause e indica-se por \pause $f^{-1}(Q)$ \pause o subconjunto de $A$ \pause dado por\pause
                \[
                    f^{-1}(Q) \pause = \{x \in A \pause \mid f(x) \in Q\},\pause
                \]
                isto {\'e}, \pause $f^{-1}(Q)$ \pause {\'e} o conjunto dos elementos de $A$ \pause que tem imagem em $Q$ \pause atrav{\'e}s de $f$.\pause
            \end{enumerate}
        \end{definicao}
    \end{frame}

    \begin{frame}
        \vspace{5cm}
    \end{frame}

    \begin{frame}
        \vspace{5cm}
    \end{frame}

    \begin{frame}
        \begin{exemplos}
            \begin{enumerate}
                \item[1)] Seja $A = \{1, 3, 5, 7, 9 \}$ \pause e $B = \{0, 1, 2, 3, \dots, 10\}$ \pause e $f : A \to B$ \pause dada por $f(x) = x + 1$. \pause Temos:\pause

                \vspace{.5cm}

                \begin{itemize}
                    \item $f(\{1\}) = \pause \{f(1)\} \pause= \{2\}$\pause

                    \vspace{.5cm}

                    \item $f(\{3, 5, 7\}) \pause= \{f(3), \pause f(5), \pause f(7)\} \pause = \{4, 6, 8\}$\pause

                    \vspace{.5cm}

                    \item $f(A) \pause = \{f(1), \pause f(3), \pause f(5), \pause f(7), \pause f(9)\} = \pause \{2, 4, 6, 8, 10\}$\pause

                    \vspace{.5cm}

                    \item $f(\emptyset) \pause = \emptyset$\pause

                    \vspace{.5cm}

                    \item $f^{-1}(\{2, 4, 10\}) = \pause \{x \in A \pause \mid f(x) \pause \in \{2, 4, 10\}\pause\} = \{1, 3, 9\}$\pause

                    \vspace{.5cm}

                    \item $f^{-1}(\{0, 1, 3, 5, 7, 9\}) \pause = \{x \in A \pause \mid f(x) \pause \in \{0, 1, 3, 5, 7, 9\}\pause\} = \emptyset$\pause

                    \vspace{.5cm}
                \end{itemize}
            \end{enumerate}
        \end{exemplos}
    \end{frame}

    \begin{frame}
        \begin{exemplos}
            \begin{enumerate}
                \item[2)] Sejam $A = B = \real$ \pause e $f : \real \to \real$ \pause dada por $f(x) = x^2$. \pause Temos:\pause

                \vspace{.5cm}

                \begin{itemize}
                    \item $f(\{1, 2, 3\}) \pause = \{1, 4, 9\}$\pause

                    \vspace{.5cm}

                    \item $f([0,2]) \pause = \{f(x) \pause \in \real \pause \mid 0 \le x \pause \le 2 \pause \} = \{x^2 \pause \mid 0 \le x \le 2\} \pause = [0, 4]$\pause

                    \vspace{.5cm}

                    \item $f^{-1}([1, 9]) \pause = \{x \in \real \pause \mid f(x) \in [1, 9]\pause \} = \{ x \in \real \pause \mid 1 \le f(x) \le 9\} \pause = \{x \in \real \pause \mid 1 \le x^2 \pause \le 9\} \pause = [-1, -3] \pause \cup [1, 3]$\pause

                    \vspace{.5cm}
                \end{itemize}
            \end{enumerate}
        \end{exemplos}
    \end{frame}

    \begin{frame}
        \begin{proposicao}
            Seja $f : A \to B$ uma fun{\c c}{\~a}o \pause e sejam $P$, \pause $Q \sub A$, \pause $X$, \pause $Y \sub B$.\pause
            \begin{enumerate}[label={\roman*})]
                \item Se $P \sub Q$, \pause ent{\~a}o $f(P) \sub f(Q)$.\pause

                \vspace{.5cm}

                \item $f^{-1}(X \cup Y) \pause = f^{-1}(X) \pause \cup f^{-1}(Y)$.\pause
            \end{enumerate}
        \end{proposicao}
            \noindent \textbf{\textit{Prova: }}\pause
            \begin{enumerate}
                \item[i)] Se $y \in f(P)$, \pause ent{\~a}o existe $x \in P$ \pause tal que $f(x) = y$. \pause Mas como $P \sub Q$, \pause ent{\~a}o $x \in Q$ \pause e da{\'\i} $y \in f(Q)$. \pause Logo $f(P) \sub f(Q)$.\pause
            \end{enumerate}
        \end{frame}

        \begin{frame}
            \begin{enumerate}
                \item[ii)] Seja $z \in f^{-1}(X \cup Y)$. \pause Ent{\~a}o $f(z) \in X \cup Y$. \pause Se $f(z) \in X$, \pause ent\~ao $z \in f^{-1}(X)$ \pause e da{\'\i} $z \in f^{-1}(X) \cup \pause f^{-1}(Y)$. \pause Se $f(z) \in Y$, \pause ent{\~a}o $z \in f^{-1}(Y)$ \pause e assim $z \in f^{-1}(X) \cup \pause f^{-1}(Y)$. \pause Logo, $f^{-1}(X \cup Y) \sub f^{-1}(X) \cup f^{-1}(Y)$.\pause

                \vspace{.3cm}

                Agora, seja $z \in f^{-1}(X) \cup f^{-1}(Y)$. \pause Se $z \in f^{-1}(X)$, \pause ent{\~a}o $f(z) \in X$, \pause da{\'\i} $f(z) \in X \cup Y$, \pause isto {\'e}, \pause $z \in f^{-1}(X \cup Y)$. \pause Se $z \in f^{-1}(Y)$, \pause ent{\~a}o $f(z) \in Y$ \pause e assim $f(z) \in X \cup Y$, \pause isto {\'e}, \pause $z \in f^{-1}(X \cup Y)$. \pause Logo $f^{-1}(X) \cup f^{-1}(Y) \sub f^{-1}(X \cup Y)$.\pause

                \vspace{.3cm}

                Portanto, \pause $f^{-1}(X \cup Y) = \pause f^{-1}(X) \cup f^{-1}(Y)$.\pause \hspace{.5cm} \qedsymbol
            \end{enumerate}
    \end{frame}
\end{document}