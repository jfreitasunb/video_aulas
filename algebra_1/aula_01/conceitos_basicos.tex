%!TEX program = xelatex
% !TEX encoding = ISO-8859-1
\def\ano{2020}
\def\semestre{1}
\def\disciplina{\'Algebra 1}
\def\turma{C}
\def\autor{Jos\'e Ant\^onio O. Freitas}
\def\instituto{MAT-UnB}

\documentclass{beamer}
\usetheme{Madrid}
\usecolortheme{beaver}
% \mode<presentation>
\usepackage{caption}
\usepackage{textpos}
\usepackage{amssymb}
\usepackage{amsmath,amsfonts,amsthm,amstext}
\usepackage[brazil]{babel}
% \usepackage[latin1]{inputenc}
\usepackage{graphicx}
\graphicspath{{/home/jfreitas/GitHub_Repos/video_aulas/logo/}{D:/Dropbox/imagens-latex/}}
\usepackage{enumitem}
\usepackage{multicol}
\usepackage{answers}
\usepackage{tikz,ifthen}
\usetikzlibrary{lindenmayersystems}
\usetikzlibrary[shadings]
\newtheorem{definicao}{Defini\c{c}\~ao}[section]
\newtheorem{definicoes}{Defini\c{c}\~oes}[section]
\newtheorem{exemplo}{Exemplo}[section]
\newtheorem{exemplos}{Exemplos}[section]
\newtheorem{exercicio}{Exerc{\'\i}cio}
\newtheorem{observacao}{Observa{\c c}{\~a}o:}[section]
\newtheorem{observacoes}{Observa{\c c}{\~o}es:}[section]
\newtheorem*{solucao}{Solu{\c c}{\~a}o:}
\newtheorem{proposicao}{Proposi\c{c}\~ao}
\newtheorem{lema}{Lema}
\newtheorem{teorema}{Teorema}
\newtheorem{corolario}{Corol\'ario}
\newenvironment{prova}[1][Prova]{\noindent\textbf{#1:} }{\qedsymbol}%{\ \rule{0.5em}{0.5em}}
\newcommand{\nsub}{\varsubsetneq}
\newcommand{\vaz}{\emptyset}
\newcommand{\im}{{\rm Im\,}}
\newcommand{\sub}{\subseteq}
\newcommand{\n}{\mathbb{N}}
\newcommand{\z}{\mathbb{Z}}
\newcommand{\rac}{\mathbb{Q}}
\newcommand{\real}{\mathbb{R}}
\newcommand{\complex}{\mathbb{C}}
\newcommand{\cp}[1]{\mathbb{#1}}
\newcommand{\ch}{\mbox{\textrm{car\,}}\nobreak}
\newcommand{\vesp}[1]{\vspace{ #1  cm}}
\newcommand{\compcent}[1]{\vcenter{\hbox{$#1\circ$}}}
\newcommand{\comp}{\mathbin{\mathchoice
{\compcent\scriptstyle}{\compcent\scriptstyle}
{\compcent\scriptscriptstyle}{\compcent\scriptscriptstyle}}}

\title{No\c{c}\~oes de L\'ogica}
\author[\autor]{\autor}
\institute[\instituto]{\instituto}
\date{\today}

\begin{document}
    \begin{frame}
        \maketitle
    \end{frame}

    \logo{\includegraphics[scale=.1]{logo-MAT.png}\vspace*{8.5cm}}

    \begin{frame}
        \begin{definicao}
            Uma \textbf{proposi\c{c}\~ao} \'e enunciado, por meio de palavras ou s{\'\i}mbolos, ao qual podemos atribuir um \textbf{valor l\'ogico}.\pause
        \end{definicao}

        \begin{definicao}
            Diz-se que o \textbf{valor l\'ogico} de uma proposi\c{c}\~ao \'e ``verdade'' (V) se a proposi\c{c}\~ao \'e verdadeira ou ``falsidade'' (F) se a proposi\c{c}\~ao \'e falsa.
        \end{definicao}
    \end{frame}

    \begin{frame}
        \begin{exemplos}
            Julgue se as seguintes senten\c{c}as s\~ao ou n\~ao proposi\c{c}\~oes:\pause
            \begin{itemize}
                \item[1)] Todo n\'umero primo \'e {\'\i}mpar.\pause
                \item[2)] $x^2 + y^2 \ge 0$ para todos $x$, $y \in \real$.\pause
                \item[3)] $x$ \'e um n\'umero real maior que 2.
            \end{itemize}
        \end{exemplos}
    \end{frame}

    \begin{frame}
        \begin{center}
            ``Toda proposi\c{c}\~ao tem um, e um s\'o, dos valores l\'ogicos \textbf{verdade} ou \textbf{falsidade}.''\pause
        \end{center}
        Isso \'e conhecido como \textbf{Princ{\'\i}pio da n\~ao contradi\c{c}\~ao e do terceiro exclu{\'\i}do}.
    \end{frame}

    \begin{frame}
        Vamos considerar com proposi\c{c}\~oes da forma:\pause

        \begin{center}
            Se $\mathbb{H}$, ent\~ao $\mathbb{T}$.\pause
        \end{center}

        $\mathbb{H}$ \'e a hip\'otese\pause

        $\mathbb{T}$ \'e a tese.\pause

        \begin{center}
            $\mathbb{H}$ se, e somente se, $\mathbb{T}$\pause

            ou

            $\mathbb{H}$ se, e s\'o se, $\mathbb{T}$.\pause
        \end{center}

        Essa proposi\c{c}\~ao poder decomposta em duas proposi\c{c}\~oes:\pause
        \begin{itemize}
            \item[1)] Se $\mathbb{H}$, ent\~ao $\mathbb{T}$.\pause
            \item[2)] Se $\mathbb{T}$, ent\~ao $\mathbb{H}$.
        \end{itemize}
    \end{frame}

    \begin{frame}
        Por exemplo, sejam $x$, $y \in \real$.\pause
        \[
            x^2 + y^2 = 0 \quad\mbox{ se, e somente se, }\quad x = y = 0\pause
        \]
        Nesse caso podemos escrever:
        \begin{itemize}
            \item Se $x^2 + y^2 = 0$, ent\~ao $x = y = 0$.\pause
            \item Se $x = y = 0$, ent\~ao $x^2 + y^2 = 0$.
        \end{itemize}
    \end{frame}

    \begin{frame}
        Temos 3 caminhos para tentar provar uma proposi\c{c}\~ao do tipo:
        \begin{center}
            Se $\mathbb{H}$, ent\~ao $\mathbb{T}$.\pause
        \end{center}

        \begin{itemize}
            \item[1)] Demonstra\c{c}\~ao direta;\pause
            \item[2)] Demonstra\c{c}\~ao por contraposi\c{c}\~ao;\pause
            \item[3)] Demonstra\c{c}\~ao por contradi\c{c}\~ao ou redu\c{c}\~ao ao absurdo.\pause
        \end{itemize}
    \end{frame}
\end{document}