%!TEX program = xelatex
% !TEX encoding = ISO-8859-1
\def\ano{2020}
\def\semestre{1}
\def\disciplina{\'Algebra 1}
\def\turma{C}
\def\autor{Jos\'e Ant\^onio O. Freitas}
\def\instituto{MAT-UnB}

\documentclass{beamer}
\usetheme{Madrid}
\usecolortheme{beaver}
% \mode<presentation>
\usepackage{caption}
\usepackage{textpos}
\usepackage{amssymb}
\usepackage{amsmath,amsfonts,amsthm,amstext}
\usepackage[brazil]{babel}
% \usepackage[latin1]{inputenc}
\usepackage{graphicx}
\graphicspath{{/home/jfreitas/GitHub_Repos/video_aulas/logo/}{D:/Dropbox/imagens-latex/}}
\usepackage{enumitem}
\usepackage{multicol}
\usepackage{answers}
\usepackage{tikz,ifthen}
\usetikzlibrary{lindenmayersystems}
\usetikzlibrary[shadings]
\newtheorem{definicao}{Defini\c{c}\~ao}[section]
\newtheorem{definicoes}{Defini\c{c}\~oes}[section]
\newtheorem{exemplo}{Exemplo}[section]
\newtheorem{exemplos}{Exemplos}[section]
\newtheorem{exercicio}{Exerc{\'\i}cio}
\newtheorem{observacao}{Observa{\c c}{\~a}o:}[section]
\newtheorem{observacoes}{Observa{\c c}{\~o}es:}[section]
\newtheorem*{solucao}{Solu{\c c}{\~a}o:}
\newtheorem{proposicao}{Proposi\c{c}\~ao}
\newtheorem{lema}{Lema}
\newtheorem{teorema}{Teorema}
\newtheorem{corolario}{Corol\'ario}
\newenvironment{prova}[1][Prova]{\noindent\textbf{#1:} }{\qedsymbol}%{\ \rule{0.5em}{0.5em}}
\newcommand{\nsub}{\varsubsetneq}
\newcommand{\vaz}{\emptyset}
\newcommand{\im}{{\rm Im\,}}
\newcommand{\sub}{\subseteq}
\newcommand{\n}{\mathbb{N}}
\newcommand{\z}{\mathbb{Z}}
\newcommand{\rac}{\mathbb{Q}}
\newcommand{\real}{\mathbb{R}}
\newcommand{\complex}{\mathbb{C}}
\newcommand{\cp}[1]{\mathbb{#1}}
\newcommand{\ch}{\mbox{\textrm{car\,}}\nobreak}
\newcommand{\vesp}[1]{\vspace{ #1  cm}}
\newcommand{\compcent}[1]{\vcenter{\hbox{$#1\circ$}}}
\newcommand{\comp}{\mathbin{\mathchoice
{\compcent\scriptstyle}{\compcent\scriptstyle}
{\compcent\scriptscriptstyle}{\compcent\scriptscriptstyle}}}

\title{An\'eis - Homomorfismos}
\author[\autor]{\autor}
\institute[\instituto]{\instituto}
\date{\today}

\begin{document}
    \begin{frame}
        \maketitle
    \end{frame}

    \logo{\includegraphics[scale=.1]{logo-MAT.png}\vspace*{8.5cm}}
    
    \begin{frame}
        \begin{definicao}
            Sejam $(A, +, \cdot)$ \pause e $(B, \oplus, \otimes)$ \pause anéis. \pause Uma fun{\c c}{\~a}o $f : A \to B$ \pause é chamada de um \textbf{homomorfismo de anéis}, \pause ou simplesmente \textbf{homomorfismo}, \pause se satisfaz:\pause
            \begin{enumerate}[label={\roman*})]
                \item $f(x + y) \pause = f(x) \pause \oplus f(y)$\pause

                \vspace{.5cm}
                
                \item $f(x \cdot y) \pause = f(x) \pause \otimes f(y)$\pause
                
                \vspace{.5cm}
            \end{enumerate}
            para todos $x$, $y \in A$.\pause
        \end{definicao}
    \end{frame}

    \begin{frame}
        \begin{exemplos}
            \begin{exemplos}
            Verifique se as seguintes funções $f : A \to B$, \pause são homomorfismos de anéis:\pause
            \begin{enumerate}[label={\roman*})]
                \item $A = \z$, \pause $B = \z$ \pause e $f(x) = x + 1$\pause
                \item $A = \z$, \pause $B = M_2(\z_5)$ \pause e
                \[
                    f(a) = \begin{bmatrix}
                        \overline{a} & \overline{0}\\
                        \overline{0} & \overline{a}
                    \end{bmatrix}.\pause
                \]
            \end{enumerate}
        \end{exemplos}
        \end{exemplos}
    \end{frame}
    \begin{frame}
        \begin{proposicao}
            Sejam $(A, +, \cdot)$ e $(B, \oplus, \otimes)$ an\'eis. \pause Se $f : A \to B$ é um homomorfismo, \pause ent{\~a}o:\pause
            \begin{enumerate}[label={\roman*})]
                \item $f(0_{A}) \pause = 0_{B}$\pause

                \vspace{.5cm}

                \item $f(-x) \pause = -f(x)$, \pause para todo $x \in A$.\pause
            \end{enumerate}
        \end{proposicao}

        \noindent \textbf{\textit{Prova: }}
    \end{frame}

    \begin{frame}
        \begin{observacao}
            A condição \textit{(i)} da proposição anterior \pause serve para determinar quando uma função $f : A \to B$, \pause onde $A$ e $B$ são anéis, não é um homomorfismo. \pause Caso $f(0_A) \ne 0_B$, \pause então $f$ não é um homomorfismo. \pause Mas pode acontecer de $f(0_A) = 0_B$ \pause e mesmo assim $f$ não é um homomorfismo de anéis, \pause como o exemplo a seguir mostra:\pause
        \end{observacao}
    \end{frame}

    \begin{frame}
        \begin{exemplo}
            Sejam $A = M_2(\real)$, \pause $B = \real$ \pause anéis com as operações usuais. \pause A função\pause
            \[
                f\left(\begin{bmatrix}x & y\\z & t\end{bmatrix}\right) \pause = x
            \]
            é tal que $f(0_A) = 0_B$ \pause e no entanto $f$ não é um homomorfismo de anéis.\pause
        \end{exemplo}
    \end{frame}

    \begin{frame}
        \begin{definicao}Seja $f:A\rightarrow B$ um homomorfismo, onde $A$ e $B$ s{\~a}o an{\'e}is. Dizemos que:
            \begin{enumerate}[label={\roman*})]
                \item $f$ {\'e} um \textbf{epimorfismo} \pause se $f$ for sobrejetora.\pause

                \vspace{.5cm}

                \item $f$ {\'e} um \textbf{monomorfismo} \pause se $f$ for injetora.\pause

                \vspace{.5cm}
                
                \item $f$ {\'e} um \textbf{isomorfismo} \pause se $f$ for bijetora.\pause

                \vspace{.5cm}
                
                \item Quando $A=B$ \pause e $f$ {\'e} um isomorfismo, \pause ent{\~a}o $f$ {\'e} um \textbf{automorfismo}.\pause

                \vspace{.5cm}
                
            \end{enumerate}
        \end{definicao}
    \end{frame}

    \begin{frame}
        \begin{definicao}
            Sejam $(A, +, \cdot)$ e $(B, \oplus, \otimes)$ an\'eis \pause e $f : A \to B$ um homomorfismo de an\'eis. \pause Ent\~ao o subconjunto de $A$ \pause definido por\pause
            \[
                \ker(f) = \pause N(f) =\pause \{ x \in A \pause \mid f(x) = 0_B\}\pause
            \]
            \'e chamado de \textbf{kernel} \pause ou \textbf{n\'ucleo} \pause de $f$.\pause
        \end{definicao}
    \end{frame}

    \begin{frame}
        \begin{exemplos}
            Determine o \textbf{kernel} dos seguintes homomorfismo de anéis:
            \begin{enumerate}
                \item[i)] $f : \z \to M_2(\z_5)$ \pause tal que\pause
                \[
                    f(a) = \begin{bmatrix}
                        \overline{a} & \overline{0}\\
                        \overline{0} & \overline{a}
                    \end{bmatrix}.\pause
                \]

                \item[ii)] Seja $(\z\times\z, +, \cdot)$ \pause um anel com as seguintes opera\c{c}\~oes\pause
                \begin{center}
                    $(a, b) + (c, d) = (a + c, b + d)\pause$\\
                    $(a, b)\cdot (c, d) = (ac, ad + bc)\pause$
                \end{center}
                para todos $(a, b)$, $(c, d) \in \z\times\z$. \pause
                O homomorfismo $ f : \z \to \z \times \z$ é dado por $f(a, b) = a$.
            \end{enumerate}
        \end{exemplos}
    \end{frame}

    \begin{frame}
        \begin{exemplos}
            \begin{enumerate}
                \item[iii)] $f : \rac \to M_3(\rac)$ \pause dada por\pause
                \[
                    f(x) = \begin{pmatrix}
                        x & 0 & 0\\
                        0 & x & 0\\
                        0 & 0 & x
                    \end{pmatrix}\pause
                \]
            \end{enumerate}
        \end{exemplos}

        \noindent \textbf{Solução:}
        \vspace{3cm}
    \end{frame}

    \begin{frame}
        \begin{proposicao}
            Sejam $(A, +, \cdot)$ e $(B, \oplus, \otimes)$ an\'eis \pause e $f : A \to B$ um homomorfismo de an\'eis. \pause Ent\~ao:\pause
            \begin{enumerate}[label={\roman*})]
                \item $\ker(f)$ \'e um subanel de $A$.\pause

                \vspace{.5cm}

                \item $f$ \'e injetora \pause se, e somente se, \pause $\ker(f) = \{0_A\}$.\pause

                \vspace{.5cm}
            \end{enumerate}
        \end{proposicao}
    \end{frame}

    \begin{frame}
        \noindent \textbf{\textit{Prova: }}

        \vspace{6.5cm}
    \end{frame}

    \begin{frame}
        \begin{definicao}
            Seja $(A, +, \cdot)$ um anel unitário. \pause Dado $x \in A$, \pause dizemos que $x$ é \textbf{inversível} \pause ou que $x$ \textbf{possui inverso} \pause se existe $y \in A$ \pause tal que\pause
            \[
                x \cdot y = 1_A \pause = y \cdot x.\pause
            \]
        \end{definicao}
    \end{frame}

    \begin{frame}
        \begin{proposicao}
            Seja $(A, +, \cdot)$ um anel unitário. \pause O inverso multiplicativo de um elemento $x \in A$, \pause se existir, \pause é único.\pause
        \end{proposicao}
    
        \noindent \textbf{Prova:}
        \vspace{4.7cm}
    \end{frame}

    \begin{frame}
        \begin{proposicao}
            Sejam $(A, +, \cdot)$ e $(B, \oplus, \otimes)$ an\'eis \pause e seja $f : A \to B$ um homomorfismo sobrejetor de an\'eis.\pause
            \begin{enumerate}[label={\roman*})]
                \item Se $A$ tem unidade, \pause ent\~ao $B$ tem unidade e\pause
                \[
                    f(1_A) = 1_B.\pause
                \]

                \vspace{.5cm}

                \item Se $A$ tem unidade \pause e $x \in A$ \pause possui inverso multiplicativo, \pause ent\~ao $f(x)$ \pause tem inverso e\pause
                \[
                    [f(x)]^{-1} = f(x^{-1}).\pause
                \]

                \vspace{.5cm}
            \end{enumerate}
        \end{proposicao}
    \end{frame}

    \begin{frame}
        \noindent \textbf{\textit{Prova: }}
        \vspace{6.5cm}
    \end{frame}
\end{document}