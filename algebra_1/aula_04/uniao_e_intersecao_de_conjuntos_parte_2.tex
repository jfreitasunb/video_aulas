%!TEX program = xelatex
% !TEX encoding = ISO-8859-1
\def\ano{2020}
\def\semestre{1}
\def\disciplina{\'Algebra 1}
\def\turma{C}
\def\autor{Jos\'e Ant\^onio O. Freitas}
\def\instituto{MAT-UnB}

\documentclass{beamer}
\usetheme{Madrid}
\usecolortheme{beaver}
% \mode<presentation>
\usepackage{caption}
\usepackage{textpos}
\usepackage{amssymb}
\usepackage{amsmath,amsfonts,amsthm,amstext}
\usepackage[brazil]{babel}
% \usepackage[latin1]{inputenc}
\usepackage{graphicx}
\graphicspath{{/home/jfreitas/GitHub_Repos/video_aulas/logo/}{D:/Dropbox/imagens-latex/}}
\usepackage{enumitem}
\usepackage{multicol}
\usepackage{answers}
\usepackage{tikz,ifthen}
\usetikzlibrary{lindenmayersystems}
\usetikzlibrary[shadings]
\newtheorem{definicao}{Defini\c{c}\~ao}[section]
\newtheorem{definicoes}{Defini\c{c}\~oes}[section]
\newtheorem{exemplo}{Exemplo}[section]
\newtheorem{exemplos}{Exemplos}[section]
\newtheorem{exercicio}{Exerc{\'\i}cio}
\newtheorem{observacao}{Observa{\c c}{\~a}o:}[section]
\newtheorem{observacoes}{Observa{\c c}{\~o}es:}[section]
\newtheorem*{solucao}{Solu{\c c}{\~a}o:}
\newtheorem{proposicao}{Proposi\c{c}\~ao}
\newtheorem{lema}{Lema}
\newtheorem{teorema}{Teorema}
\newtheorem{corolario}{Corol\'ario}
\newenvironment{prova}[1][Prova]{\noindent\textbf{#1:} }{\qedsymbol}%{\ \rule{0.5em}{0.5em}}
\newcommand{\nsub}{\varsubsetneq}
\newcommand{\vaz}{\emptyset}
\newcommand{\im}{{\rm Im\,}}
\newcommand{\sub}{\subseteq}
\newcommand{\n}{\mathbb{N}}
\newcommand{\z}{\mathbb{Z}}
\newcommand{\rac}{\mathbb{Q}}
\newcommand{\real}{\mathbb{R}}
\newcommand{\complex}{\mathbb{C}}
\newcommand{\cp}[1]{\mathbb{#1}}
\newcommand{\ch}{\mbox{\textrm{car\,}}\nobreak}
\newcommand{\vesp}[1]{\vspace{ #1  cm}}
\newcommand{\compcent}[1]{\vcenter{\hbox{$#1\circ$}}}
\newcommand{\comp}{\mathbin{\mathchoice
{\compcent\scriptstyle}{\compcent\scriptstyle}
{\compcent\scriptscriptstyle}{\compcent\scriptscriptstyle}}}

\title{Uni\~ao e Interse\c{c}\~ao de Conjuntos - Parte 2}
\author[\autor]{\autor}
\institute[\instituto]{\instituto}
\date{22 de julho de 2020}

\begin{document}
    \begin{frame}
        \maketitle
    \end{frame}

    \logo{\includegraphics[scale=.1]{logo-MAT.png}\vspace*{8.5cm}}

    \begin{frame}
        \begin{proposicao} Sejam $A,$ $B$ e $C$ tr{\^e}s conjuntos, ent{\~a}o:\pause
            \begin{enumerate}[label={\roman*})]
                \item $A \cap ( B \cup C) = (A \cap B) \cup (A \cap C)$\pause
                \item $A \cup (B \cap C) = (A \cup B) \cap (A \cup C)$\pause
            \end{enumerate}
        \end{proposicao}
        \textit{Prova:} \pause
        Para mostrar a primeira igualdade precisamos mostrar que\pause
        \begin{enumerate}[label=({\arabic*})]
            \item $A \cap (B \cup C) \subseteq (A \cap B) \cup ( A \cap C)$;\label{intersecao_unicao_1}\pause
            \item $(A \cap B) \cup (A \cap C) \subseteq A \cap (B \cup C).$\label{intersecao_unicao_2}\pause
        \end{enumerate}

        Para provar \ref{intersecao_unicao_1} seja $x\in A \cap (B \cup C)$. \pause Logo $x \in A$ \pause e $x \in B \cup C$. \pause Agora, de $x \in B \cup C$, \pause segue que $x \in B$ \pause ou $x \in C$. \pause Suponha que $x \in B$. \pause Como $x \in A$ e $x \in B$, \pause ent\~ao $x \in A \cap B$. \pause Assim, $x \in (A \cap B) \cup (A \cap C)$, \pause ou seja, $A \cap (B \cup C) \subseteq (A \cap B) \cup (A \cap C)$. \pause Por outro lado, se $x \in C$, \pause como $x \in A$, ent{\~a}o $x \in A \cap C$ \pause e da{\'\i} $x \in (A \cap B) \cup (A \cap C)$, \pause logo $A \cap (B \cup C)\subseteq (A \cap B) \cup (A \cap C)$.\pause

        Portanto,
        \[
            A \cap (B \cup C) \subseteq (A \cap B) \cup (A \cap C).\pause
        \]
    \end{frame}
    \begin{frame}
        Agora para provar \ref{intersecao_unicao_2}, \pause seja $y \in (A \cap B) \cup (A \cap C)$. \pause Da{\'\i}, $y \in A\cap B$ \pause ou $y \in A \cap C$. \pause Suponha que $y \in A \cap B$. \pause Assim, $y \in A$ \pause e $y \in B$. \pause Como $y \in B$, \pause segue que $y \in B \cup C$ \pause e ent{\~a}o $y \in A \cap (B \cup C)$, \pause ou seja, $(A \cap B) \cup (A \cap C) \subseteq A \cap (B \cup C)$. \pause Agora, suponha que $y \in A \cap C$. \pause Com isso $y \in A$ \pause e $y \in C$. \pause Desse modo, $y \in B \cup C$ \pause e ent{\~a}o $y \in A \cap (B \cup C)$ \pause e da{\'\i}\pause
        \[
            (A \cap B) \cup (A \cap C) \subseteq A \cap (B \cup C).\pause
        \]

        Portanto
        \[
            A \cap (B \cup C)=(A \cap B) \cup (A \cap C),\pause
        \]
        como quer{\'\i}amos.
    \end{frame}
    \begin{frame}
        Para mostrar a segunda igualdade de conjuntos, precisamos mostrar que\pause
        \begin{enumerate}[label=({\arabic*})]
            \item $A \cup (B \cap C) \subseteq (A \cup B) \cap (A \cup C)$;\label{uniao_intersecao_1}\pause
            \item $(A \cup B) \cap (A \cup C) \subseteq A \cup (B \cap C)$.\label{uniao_intersecao_2}\pause
        \end{enumerate}

        Para mostrar \ref{uniao_intersecao_1} seja $x \in A \cup (B \cap C)$. \pause Da{\'\i} $x \in A$ \pause ou $x \in B \cap C$. \pause Suponha que $x \in A$, \pause assim $x \in A \cup B$ \pause e tamb\'em $x \in A \cup C$. Logo $x \in (A \cup B) \cap (A \cup C)$ \pause e com isso $A \cup (B \cap C) \subseteq (A \cup B) \cap (A \cup C)$.\pause Agora suponha que $x \in B \cap C$. \pause Ent\~ao $x \in B$ e $x \in C$. Assim $x \in A \cup B$ \pause e $x \in A \cup C$. \pause Logo $x \in (A \cup B) \cap (A \cup C)$. \pause Com isso $A \cup (B \cap C) \subseteq (A \cup B) \cap (A \cup C)$.\pause

        Agora para mostrar \ref{uniao_intersecao_2} seja $y \in (A \cup B) \cap (A \cup C)$. \pause Assim $y \in A \cup B$ \pause e $y \in A \cup C$. \pause Assim $y \in A$ \pause ou $y \in B$ \pause e tamb\'em $y \in A$ \pause ou $y \in C$. \pause Se $y \in A$, \pause ent\~ao $y \in A \cup (B \cap C)$. Agora, suponha que $y \notin A$, \pause logo $y \in B$ e $y \in C$, \pause isto \'e, $y \in B \cap C$ e com isso $y \in A \cup (B \cap C)$. Assim, \pause independente do caso sempre temos $y \in A \cup (B \cap C)$. \pause Logo, $(A \cup B) \cap (A \cup C) \subseteq A \cup (B \cap C)$.\pause

        Portanto,\pause
        \[
            (A\cup B) \cap (A\cup C) = A \cup (B\cap C),\pause
        \]
        como quer{\'\i}amos. \qedsymbol \pause
    \end{frame}
\end{document}
