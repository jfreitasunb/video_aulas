%!TEX program = xelatex
\def\ano{2020}
\def\semestre{1}
\def\disciplina{\'Algebra 1}
\def\turma{C}
\def\autor{Jos\'e Ant\^onio O. Freitas}
\def\instituto{MAT-UnB}

\documentclass{beamer}
\usetheme{Madrid}
\usecolortheme{beaver}
% \mode<presentation>
\usepackage{caption}
\usepackage{textpos}
\usepackage{amssymb}
\usepackage{amsmath,amsfonts,amsthm,amstext}
\usepackage[brazil]{babel}
% \usepackage[latin1]{inputenc}
\usepackage{graphicx}
\graphicspath{{/home/jfreitas/GitHub_Repos/video_aulas/logo/}{D:/Dropbox/imagens-latex/}}
\usepackage{enumitem}
\usepackage{multicol}
\usepackage{answers}
\usepackage{tikz,ifthen}
\usetikzlibrary{lindenmayersystems}
\usetikzlibrary[shadings]
\newtheorem{definicao}{Defini\c{c}\~ao}[section]
\newtheorem{definicoes}{Defini\c{c}\~oes}[section]
\newtheorem{exemplo}{Exemplo}[section]
\newtheorem{exemplos}{Exemplos}[section]
\newtheorem{exercicio}{Exerc{\'\i}cio}
\newtheorem{observacao}{Observa{\c c}{\~a}o:}[section]
\newtheorem{observacoes}{Observa{\c c}{\~o}es:}[section]
\newtheorem*{solucao}{Solu{\c c}{\~a}o:}
\newtheorem{proposicao}{Proposi\c{c}\~ao}
\newtheorem{lema}{Lema}
\newtheorem{teorema}{Teorema}
\newtheorem{corolario}{Corol\'ario}
\newenvironment{prova}[1][Prova]{\noindent\textbf{#1:} }{\qedsymbol}%{\ \rule{0.5em}{0.5em}}
\newcommand{\nsub}{\varsubsetneq}
\newcommand{\vaz}{\emptyset}
\newcommand{\im}{{\rm Im\,}}
\newcommand{\sub}{\subseteq}
\newcommand{\n}{\mathbb{N}}
\newcommand{\z}{\mathbb{Z}}
\newcommand{\rac}{\mathbb{Q}}
\newcommand{\real}{\mathbb{R}}
\newcommand{\complex}{\mathbb{C}}
\newcommand{\cp}[1]{\mathbb{#1}}
\newcommand{\ch}{\mbox{\textrm{car\,}}\nobreak}
\newcommand{\vesp}[1]{\vspace{ #1  cm}}
\newcommand{\compcent}[1]{\vcenter{\hbox{$#1\circ$}}}
\newcommand{\comp}{\mathbin{\mathchoice
{\compcent\scriptstyle}{\compcent\scriptstyle}
{\compcent\scriptscriptstyle}{\compcent\scriptscriptstyle}}}

\title{Homomorfismo de Grupos - Continua\c{c}\~ao}
\author[\autor]{\autor}
\institute[\instituto]{\instituto}
\date{}

\begin{document}
    \begin{frame}
        \maketitle
    \end{frame}

    \logo{\includegraphics[scale=.1]{logo-MAT.png}\vspace*{8.5cm}}

    \begin{frame}
        \begin{definicao}
            Sejam $(G, *)$, \pause $(H, \triangle)$ grupos \pause e $f : G \to H$ um homomorfismo de grupos. \pause Chama-se de \textbf{n\'ucleo} \pause ou \textbf{kernel} \pause de $f$ e denota-se por \pause $N(f)$ \pause ou $\ker(f)$ \pause o seguinte subconjunto de $G$:\pause
            \[
                \ker(f) = \pause \{x \in G \pause \mid f(x) = 1_H\}.
            \]
        \end{definicao}
    \end{frame}

    \begin{frame}
        \begin{exemplos}
            \begin{enumerate}[label={\arabic*})]
                \item Considere o homomorfismo $f : \z \to \complex^*$ \pause dado por $f(x) = i^x$. \pause O kernel de $f$ \'e:
                \seti
            \end{enumerate}
        \end{exemplos}
        \vspace{2cm}
    \end{frame}
    \begin{frame}
        \begin{exemplos}
            \begin{enumerate}[label={\arabic*})]
            \conti
                \item Considere o homomorfismo $g : \real^*_+ \to \real$ \pause dado por $g(x) = \ln(x)$. \pause O n\'ucleo de $g$ \'e:
                \seti
            \end{enumerate}
        \end{exemplos}
        \vspace{2cm}
    \end{frame}
    \begin{frame}
        \begin{exemplos}
            \begin{enumerate}[label={\arabic*})]
            \conti
                \item Considere o homomorfismo $h : \z \to \z_m$ \pause dado por $h(x) = \overline{x}$, \pause $m > 0$ fixo. \pause O kernel de $h$ \'e:
            \end{enumerate}
        \end{exemplos}
        \vspace{2cm}
    \end{frame}

    \begin{frame}
        \begin{proposicao}
            Sejam $(G, *)$, \pause $(H, \triangle)$ grupos \pause e $f : G \to H$ um homomorfismo de grupos. \pause Ent\~ao: \pause
            \vspace{.5cm}

            \begin{enumerate}[label={\roman*})]
                \item $\ker(f)$ \'e um subgrupo de $G$. \pause

                \vspace{.5cm}

                \item $f$ \'e um monomorfismo se, e somente se, $\ker(f) = \{1_G\}$.

                \vspace{.5cm}
            \end{enumerate}
        \end{proposicao}
    \end{frame}

    \begin{frame}
        \begin{proposicao}
            Sejam $H$, $J$ e $L$ grupos. \pause Se $f : H \to J$ \pause e $g : J \to L$ \pause s\~ao homomorfismos de grupos, \pause ent\~ao $g \circ f : H \to L$ \pause tamb\'em \'e um homomorfismo de grupos.
        \end{proposicao}
        \vspace{2cm}
    \end{frame}

    \begin{frame}
        \begin{corolario}
            Se $f$ e $g$ s\~ao homomorfismo \pause injetores \pause (sobrejetores), ent\~ao $g \circ f$ \pause tamb\'em \'e um homomorfismo injetor \pause (sobrejetor).
        \end{corolario}
        \vspace{2cm}
    \end{frame}

    \begin{frame}
        \begin{proposicao}
            Se $f : G \to H$ \'e um isomorfismo de grupos, \pause ent\~ao $f^{-1} : H \to G$ \pause tamb\'em \'e um isomorfismo de grupos.
        \end{proposicao}
    \end{frame}
    \vspace{2cm}
\end{document}
