%!TEX program = xelatex
% !TEX encoding = ISO-8859-1
\def\ano{2020}
\def\semestre{1}
\def\disciplina{\'Algebra 1}
\def\turma{C}
\def\autor{Jos\'e Ant\^onio O. Freitas}
\def\instituto{MAT-UnB}

\documentclass{beamer}
\usetheme{Madrid}
\usecolortheme{beaver}
% \mode<presentation>
\usepackage{caption}
\usepackage{textpos}
\usepackage{amssymb}
\usepackage{amsmath,amsfonts,amsthm,amstext}
\usepackage[brazil]{babel}
% \usepackage[latin1]{inputenc}
\usepackage{graphicx}
\graphicspath{{/home/jfreitas/GitHub_Repos/video_aulas/logo/}{D:/Dropbox/imagens-latex/}}
\usepackage{enumitem}
\usepackage{multicol}
\usepackage{answers}
\usepackage{tikz,ifthen}
\usetikzlibrary{lindenmayersystems}
\usetikzlibrary[shadings]
\newtheorem{definicao}{Defini\c{c}\~ao}[section]
\newtheorem{definicoes}{Defini\c{c}\~oes}[section]
\newtheorem{exemplo}{Exemplo}[section]
\newtheorem{exemplos}{Exemplos}[section]
\newtheorem{exercicio}{Exerc{\'\i}cio}
\newtheorem{observacao}{Observa{\c c}{\~a}o:}[section]
\newtheorem{observacoes}{Observa{\c c}{\~o}es:}[section]
\newtheorem*{solucao}{Solu{\c c}{\~a}o:}
\newtheorem{proposicao}{Proposi\c{c}\~ao}
\newtheorem{lema}{Lema}
\newtheorem{teorema}{Teorema}
\newtheorem{corolario}{Corol\'ario}
\newenvironment{prova}[1][Prova]{\noindent\textbf{#1:} }{\qedsymbol}%{\ \rule{0.5em}{0.5em}}
\newcommand{\nsub}{\varsubsetneq}
\newcommand{\vaz}{\emptyset}
\newcommand{\im}{{\rm Im\,}}
\newcommand{\sub}{\subseteq}
\newcommand{\n}{\mathbb{N}}
\newcommand{\z}{\mathbb{Z}}
\newcommand{\rac}{\mathbb{Q}}
\newcommand{\real}{\mathbb{R}}
\newcommand{\complex}{\mathbb{C}}
\newcommand{\cp}[1]{\mathbb{#1}}
\newcommand{\ch}{\mbox{\textrm{car\,}}\nobreak}
\newcommand{\vesp}[1]{\vspace{ #1  cm}}
\newcommand{\compcent}[1]{\vcenter{\hbox{$#1\circ$}}}
\newcommand{\comp}{\mathbin{\mathchoice
{\compcent\scriptstyle}{\compcent\scriptstyle}
{\compcent\scriptscriptstyle}{\compcent\scriptscriptstyle}}}

\title{An\'eis}
\author[\autor]{\autor}
\institute[\instituto]{\instituto}
\date{\today}

\begin{document}
    \begin{frame}
        \maketitle
    \end{frame}

    \logo{\includegraphics[scale=.1]{logo-MAT.png}\vspace*{8.5cm}}
    
    \begin{frame}
        \begin{definicao}
            Seja $A \ne \emptyset$ um conjunto. \pause Dizemos que $A$ est{\'a} munido \pause (ou equipado) \pause de uma \textbf{opera{\c c}{\~a}o bin{\'a}ria} \pause quando existe uma fun{\c c}{\~a}o\pause
            \begin{align*}
                &\Delta : A \times A \to A\\
                &(a,b) \longmapsto a\Delta b        
            \end{align*}
            Uma opera{\c c}{\~a}o bin{\'a}ria tamb{\'e}m {\'e} chamada de uma \textbf{opera{\c c}{\~a}o interna} em $A$.\pause
        \end{definicao}
    \end{frame}

    \begin{frame}
        \begin{exemplos}
            \begin{enumerate}[label={\arabic*})]
                \item A soma usual \pause nos conjuntos $\z$, \pause $\rac$, \pause $\real$ \pause e $\complex$ \pause {\'e} uma opera{\c c}{\~a}o bin{\'a}ria.\pause

                \vspace{.3cm}

                \item A multiplica\c{c}\~ao usual \pause nos conjuntos $\z$, \pause $\rac$, \pause $\real$ \pause e $\complex$ {\'e} uma opera{\c c}{\~a}o bin{\'a}ria.\pause

                \vspace{.3cm}

                \item Seja $m > 1$, \pause $m \in \z$ fixo. \pause A soma \pause e a multiplica\c{c}\~ao definidos em $\z_m = \pause \{\overline{0}, \overline{1}, ..., \overline{m-1}\}$ \pause s\~ao opera\c{c}\~oes bin\'arias.\pause
                
                \vspace{.3cm}

                \item A opera\c{c}\~ao $\div$ \pause em $\rac^{*}$ \pause {\'e} uma opera{\c c}{\~a}o bin{\'a}ria.\pause
                
                \vspace{.3cm}

                \item J\'a em $\n$, \pause $\z$, \pause $\z^{*}$ \pause e em $\rac$ \pause a opera\c{c}\~ao $\div$ \pause n{\~a}o {\'e} uma opera{\c c}{\~a}o bin{\'a}ria.\pause
                
                \vspace{.3cm}

            \end{enumerate}
        \end{exemplos}
    \end{frame}

    \begin{frame}
        \begin{definicao}
            Seja $A \ne \emptyset$ um conjunto \pause no qual est\~ao definidas duas opera{\c c}{\~o}es bin\'arias \pause $\oplus$ \pause e $\otimes$, \pause chamadas \textbf{soma} \pause e \textbf{produto} \pause ou \textbf{multiplicação}. \pause Dizemos que $(A, \oplus, \otimes)$ \pause {\'e} um \textbf{anel} \pause quando as seguintes condi{\c c}{\~o}es s{\~a}o verdadeiras:\pause
            \begin{enumerate}[label={\roman*})]
                \item \textbf{Associatividade}: \pause para todos $x$, \pause $y$, \pause $z \in A$ \pause vale\pause
                \[
                    (x \oplus y) \pause \oplus z \pause = x \oplus \pause (y \oplus z).\pause
                \]
                Essa propriedade {\'e} chamada de \pause \textbf{propriedade associativa} \pause da soma.\pause

                \vspace{.7cm}

                \item \textbf{Comutatividade}: \pause Para todos $x$, \pause $y \in A$ \pause vale\pause
                \[
                    x \oplus y = \pause y \oplus x.\pause
                \]

                \vspace{.7cm}

                \seti
            \end{enumerate}
        \end{definicao}
    \end{frame}

    \begin{frame}
        \begin{definicao}
            \begin{enumerate}[label={\roman*})]
                \conti

                \item \textbf{Elemento Neutro}: \pause Existe em $A$ \pause um elemento denotado por $0$ \pause (zero) ou $0_{A}$ \pause tal que para todo elemento $x \in A$ \pause vale\pause
                \[
                    x \oplus 0_A \pause = x \pause = 0_A \oplus x.\pause
                \]
                Tal elemento $0_A$ \pause \'e chamado de \textbf{elemento neutro da soma} \pause ou simplesmente \textbf{elemento neutro}.\pause

                \vspace{.7cm}

                \item \textbf{Elemento Oposto}: \pause Para cada elemento $x \in A$, \pause existe $y \in A$ \pause tal que\pause
                \[
                    x \oplus y \pause = 0_A \pause = y \oplus x.\pause
                \]
                Tal elemento $y$ \pause \'e chamado de \textbf{oposto aditivo} \pause de $x$ \pause ou simplesmente \textbf{oposto} de $x$.\pause

                \vspace{.7cm}

                \seti
    \end{enumerate}
        \end{definicao}
    \end{frame}

    \begin{frame}
        \begin{definicao}
            \begin{enumerate}[label={\roman*})]
                \conti

                \item \textbf{Associatividade}: \pause Para todos $x$, \pause $y$, \pause $z \in A$, \pause vale\pause
                \[
                    (x\otimes y) \pause \otimes z \pause = x \otimes \pause (y\otimes z).\pause
                \]

                \vspace{.7cm}

                \item \textbf{Distributividade}: \pause Para todos $x$, \pause $y$, \pause $z \in A$ \pause vale\pause
                \[
                    (x \oplus y) \pause \otimes z \pause = x \otimes z \pause \oplus \pause y\otimes z.\pause
                \]
                Essa propriedade {\'e} chamada \textbf{distributiva da soma em rela{\c c}{\~a}o ao produto}.\pause

                \vspace{.7cm}

                \seti
            \end{enumerate}
        \end{definicao}
    \end{frame}

    \begin{frame}
        \begin{definicao}
            \begin{enumerate}[label={\roman*})]
                \conti

                \item \textbf{Distributividade}: \pause Para todos $x$, \pause $y$, \pause $z \in A$ \pause vale\pause
                \[
                    x \otimes \pause (y \oplus z) \pause = x \otimes y \pause \oplus \pause x\otimes z.\pause
                \]
                Essa {\'e} a propriedade \textbf{distributiva do produto em rela{\c c}{\~a}o {\`a} soma}.\pause
            \end{enumerate}
        \end{definicao}
    \end{frame}

    \begin{frame}
        \begin{observacoes}
            Seja $(A, \oplus, \otimes)$ \pause um anel.\pause
            \begin{enumerate}[label={\arabic*})]
                \item \textbf{Comutatividade}: Se para todos $x$, \pause $y \in A$ \pause vale
                \[
                    x \otimes y \pause = y \otimes x.\pause
                \]
                Dizemos que $(A, \oplus, \otimes)$ \pause {\'e} um \textbf{anel comutativo}.\pause

                \vspace{.7cm}

                \item \textbf{Unidade}: \pause Se existe em $A$ \pause um elemento denotado por $1$ \pause ou $1_{A}$ \pause tal que\pause
                \[
                    x \otimes 1 \pause = x \pause = 1 \otimes x,\pause
                \]
                para todo $x \in A$, \pause ent{\~a}o dizemos que $(A, \oplus, \otimes)$ \pause \'e um \textbf{anel com unidade} \pause ou um \textbf{anel unit{\'a}rio}. \pause O elemento $1_A$ \pause {\'e} chamado de \textbf{unidade} de $A$ \pause ou \textbf{elemento neutro da multiplica\c{c}\~ao} \pause de $A$.\pause

                \vspace{.7cm}

                \seti
             \end{enumerate}
        \end{observacoes}
    \end{frame}

    \begin{frame}
        \begin{observacoes}
            \begin{enumerate}[label={\arabic*})]
                \conti

                \item Se um anel $(A, \oplus, \otimes)$ \pause satisfaz as duas propriedades anteriores \pause dizemos que $(A, \oplus, \otimes)$ \'e um \textbf{anel comutativo com unidade} \pause ou um \textbf{anel comutativo unit\'ario}.\pause

                \vspace{.5cm}

                \item Seja $(A, \oplus, \otimes)$ um anel. \pause Quando n\~ao houver chance de confus\~ao com rela\c{c}\~ao \`as opera\c{c}\~oes envolvidas diremos simplesmente que \pause $A$ \'e uma anel.\pause
            \end{enumerate}
        \end{observacoes}
    \end{frame}

    \begin{frame}
        \begin{exemplos}
            \begin{enumerate}[label={\arabic*})]
                \item $(\z,+,.)$, \pause $(\rac,+,.)$, \pause $(\real,+,.)$, \pause $(\complex,+,.)$ s{\~a}o an{\'e}is associativos, \pause comutativos \pause e com unidade.\pause

                \seti
            \end{enumerate}
        \end{exemplos}
    \end{frame}

    \begin{frame}
        \begin{exemplos}
            \begin{enumerate}[label={\arabic*})]
                \conti

                \item  Consideremos em $\z \times \z$ \pause as opera\c{c}\~oes $\oplus$ \pause e $\otimes$ \pause definidas por\pause
                \vspace{.3cm}
                \begin{center}
                    \begin{tabular}{l}
                        $(a, b) \oplus (c, d) = (a + c, b + d)$\\
                        \\
                        $(a ,b) \otimes (c, d) = (ac - bd, ad + bc)$.    
                    \end{tabular}
                \end{center}
                \vspace{.3cm}
                Mostre que $(\z \times \z, \oplus, \otimes)$ \'e um anel comutativo e com unidade.
            \end{enumerate}
        \end{exemplos}
    \end{frame}

    \begin{frame}
        \begin{observacao}
            Seja $(A, \oplus, \cdot)$ \pause um anel. \pause Para simplificar a nota\c{c}\~ao \pause vamos denotar a opera\c{c}\~ao $\oplus$ \pause por $+$ \pause e a opera\c{c}\~ao $\otimes$ \pause por $\cdot$ \pause e assim escrever simplesmente \pause que $(A, +, \cdot)$ \pause \'e um anel.\pause
        \end{observacao}
    \end{frame}

    \begin{frame}
        \begin{proposicao}
            Seja $(A, + , \cdot)$ um anel. \pause Ent\~ao:\pause
            \begin{enumerate}[label={\roman*})]
                \item O elemento neutro {\'e} {\'u}nico.\pause

                \vspace{.5cm}

                \item Para cada $x \in A$ \pause existe um {\'u}nico oposto.\pause

                \vspace{.5cm}
                
                \item Para todo $x \in A$, \pause
                \[
                    -(-x) = x.\pause
                \]

                \vspace{.5cm}
                
                \item Dados $x_{1}$, \pause $x_{2}$, \pause \dots, $x_n \in A$, \pause $n \geqslant 2$, \pause ent{\~a}o\pause
                \[
                    -(x_1 + x_2 + \dots + x_n) \pause = (-x_1) \pause + (-x_2) \pause + \dots + (-x_n).\pause
                \]

                \vspace{.2cm}

                \seti
            \end{enumerate}
        \end{proposicao}
    \end{frame}

    \begin{frame}
        \begin{proposicao}
            \begin{enumerate}[label={\roman*})]
                \conti
                
                \item Para todos $\alpha$, \pause $x$, \pause $y \in A$, \pause se
                \[
                    \alpha + x \pause = \alpha + y,\pause
                \]
                ent{\~a}o $x = y$.\pause

                \vspace{.5cm}
                
                \item Para todo $x \in A$, \pause 
                \[
                    x\cdot 0_A \pause = 0_A \pause = 0_A\cdot x.\pause
                \]

                \vspace{.5cm}
            \end{enumerate}
        \end{proposicao}
    \end{frame}

    \begin{frame}
        \begin{proposicao}
            \begin{enumerate}[label={\roman*})]
                \conti

                \item Para todos $x$, \pause $y \in A$, \pause temos\pause
                \[
                    x(-y) \pause = (-x)y \pause = -(xy).\pause
                \]

                \vspace{.5cm}
                
                \item Para todos $x$, \pause $y \in A$, \pause
                \[
                    xy \pause = (-x)(-y).\pause
                \]

                \vspace{.5cm}
            \end{enumerate}
        \end{proposicao}
    \end{frame}

    \begin{frame}
        \noindent \textbf{\textit{Prova:} }\pause
        \begin{enumerate}[label={\roman*})]
            \item Suponha que existam $0_1$, \pause $0_2\in A$ \pause elementos neutros \pause de $A$. \pause Assim\pause
                
            \[
                x + 0_1 \pause = x\pause \quad \mbox{e}\quad x + 0_2 \pause = x\pause
            \]
            para todo $x \in A$. \pause Assim\pause

            \[
                   0_1 \pause = 0_1 +\pause 0_2 \pause = 0_2\pause
            \]
            
            e portanto o elemento neutro \'e \'unico.\pause
                
            \vspace{.5cm}

            \seti
        \end{enumerate}
    \end{frame}

    \begin{frame}
        \begin{enumerate}[label={\roman*})]
            \conti

            \item De fato, \pause dado $x \in A$ \pause suponha que existam $y_1$, \pause $y_2\in A$ \pause tais que\pause
            \[
                x + y_1 \pause = 0_A \pause \quad \mbox{e}\quad x + y_2 \pause = 0_A.\pause
            \]
            Da{\'\i}\pause
            \[
                y_1 \pause = y_2 \pause + 0_A \pause = y_1 \pause + (x + y_2) \pause = (y_1 + x) \pause + y_2 \pause = 0_A\pause  + y_2 \pause= y_2.\pause
            \]
            Logo o oposto de $x$ \'e \'unico \pause e da{\'\i} ser\'a denotado por $-x$.\pause
                
            \vspace{.5cm}

            \item Dado $x \in A$, \pause ent\~ao $-x$ {\'e} oposto de $x$, \pause isto {\'e}, \pause $x \pause + (-x) \pause = 0_A$. \pause Logo o oposto de $(-x)$ \pause {\'e} $x$, \pause ou seja, \pause $-(-x) \pause = x$.\pause
            
            \seti
        \end{enumerate}
    \end{frame}

    \begin{frame}
            \begin{enumerate}[label={\roman*})]
                \conti

                \item Segue usando indu\c{c}\~ao sobre $n$.\pause

                \vspace{.5cm}

                \item Suponha que $\alpha + x \pause = \alpha + y$. \pause Seja $-\alpha$ \pause o oposto de $\alpha$. \pause Da{\'\i}\pause
                \begin{center}
                    $x = 0_A \pause + x = [(-\alpha) + \alpha] \pause + x = \pause (-\alpha) \pause + (\alpha + x) \pause = (-\alpha) + \pause (\alpha + y) \pause = [(-\alpha) + \alpha] \pause + y \pause = 0_A + y \pause = y$\pause
                \end{center}
                como quer{\'\i}amos.\pause
                \seti
            \end{enumerate}
    \end{frame}

    \begin{frame}
            \begin{enumerate}[label={\roman*})]
                \conti
                \item Temos \pause
                \[
                    x\cdot 0_A + \pause 0_A \pause = x\cdot 0_A \pause = x\cdot(0_A + 0_A) \pause = x\cdot 0_A \pause + x\cdot 0_A.\pause
                \]
                Assim do item anterior \pause segue que $x\cdot 0_A \pause = 0_A$.\pause

                \vspace{.5cm}

                \item Provemos que \pause $x(-y) \pause = -(xy)$\pause :
                \[
                    x(-y) \pause + xy \pause = x[(-y) + y] \pause = x\cdot 0_A \pause = 0_A,\pause
                \]
                portanto $-xy \pause = x(-y)$.\pause

                \vspace{.5cm}
                
                \item Basta usar o caso anterior.\pause
            \end{enumerate}
    \end{frame}
\end{document}