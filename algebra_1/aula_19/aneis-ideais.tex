%!TEX program = xelatex
% !TEX encoding = ISO-8859-1
\def\ano{2020}
\def\semestre{1}
\def\disciplina{\'Algebra 1}
\def\turma{C}
\def\autor{Jos\'e Ant\^onio O. Freitas}
\def\instituto{MAT-UnB}

\documentclass{beamer}
\usetheme{Madrid}
\usecolortheme{beaver}
% \mode<presentation>
\usepackage{caption}
\usepackage{textpos}
\usepackage{amssymb}
\usepackage{amsmath,amsfonts,amsthm,amstext}
\usepackage[brazil]{babel}
% \usepackage[latin1]{inputenc}
\usepackage{graphicx}
\graphicspath{{/home/jfreitas/GitHub_Repos/video_aulas/logo/}{D:/Dropbox/imagens-latex/}}
\usepackage{enumitem}
\usepackage{multicol}
\usepackage{answers}
\usepackage{tikz,ifthen}
\usetikzlibrary{lindenmayersystems}
\usetikzlibrary[shadings]
\newtheorem{definicao}{Defini\c{c}\~ao}[section]
\newtheorem{definicoes}{Defini\c{c}\~oes}[section]
\newtheorem{exemplo}{Exemplo}[section]
\newtheorem{exemplos}{Exemplos}[section]
\newtheorem{exercicio}{Exerc{\'\i}cio}
\newtheorem{observacao}{Observa{\c c}{\~a}o:}[section]
\newtheorem{observacoes}{Observa{\c c}{\~o}es:}[section]
\newtheorem*{solucao}{Solu{\c c}{\~a}o:}
\newtheorem{proposicao}{Proposi\c{c}\~ao}
\newtheorem{lema}{Lema}
\newtheorem{teorema}{Teorema}
\newtheorem{corolario}{Corol\'ario}
\newenvironment{prova}[1][Prova]{\noindent\textbf{#1:} }{\qedsymbol}%{\ \rule{0.5em}{0.5em}}
\newcommand{\nsub}{\varsubsetneq}
\newcommand{\vaz}{\emptyset}
\newcommand{\im}{{\rm Im\,}}
\newcommand{\sub}{\subseteq}
\newcommand{\n}{\mathbb{N}}
\newcommand{\z}{\mathbb{Z}}
\newcommand{\rac}{\mathbb{Q}}
\newcommand{\real}{\mathbb{R}}
\newcommand{\complex}{\mathbb{C}}
\newcommand{\cp}[1]{\mathbb{#1}}
\newcommand{\ch}{\mbox{\textrm{car\,}}\nobreak}
\newcommand{\vesp}[1]{\vspace{ #1  cm}}
\newcommand{\compcent}[1]{\vcenter{\hbox{$#1\circ$}}}
\newcommand{\comp}{\mathbin{\mathchoice
{\compcent\scriptstyle}{\compcent\scriptstyle}
{\compcent\scriptscriptstyle}{\compcent\scriptscriptstyle}}}

\title{Anéis - Ideais}
\author[\autor]{\autor}
\institute[\instituto]{\instituto}
\date{\today}

\begin{document}
    \begin{frame}
        \maketitle
    \end{frame}

    \logo{\includegraphics[scale=.1]{logo-MAT.png}\vspace*{8.5cm}}
    \begin{frame}
        \begin{definicao}
            Um anel comutativo $(A, + , \cdot)$ \pause {\'e} dito ser um \textbf{anel de integridade} \pause quando para todos $x$, $y \in A$, \pause se
            \[
                xy = 0_A,\pause
            \]
            ent{\~a}o
            \[
                x = 0_A\pause \mbox{ ou } y = 0_A.\pause
            \]
            Um anel de integridade tamb{\'e}m {\'e} chamado de \textbf{dom{\'\i}nio de integridade} \pause ou simplesmente de \textbf{dom{\'\i}nio}.\pause
        \end{definicao}

        \begin{observacao}
            Se $x$ e $y$ s{\~a}o elementos n{\~a}o nulos \pause de um anel $A$ \pause tais que $xy = 0_A$, \pause ent{\~a}o $x$ e $y$ s{\~a}o chamados de \pause \textbf{divisores pr{\'o}prios de zero}.\pause
        \end{observacao}
    \end{frame}

    \begin{frame}
        \begin{exemplos}
            \begin{enumerate}[label={\arabic*})]
                \item Os an{\'e}is $\z$, \pause $\rac$, \pause $\real$, \pause $\complex$ \pause s{\~a}o an{\'e}is de integridade.\pause
                
                \vspace{.5cm}

                \item Em geral $\z_m$ \pause n{\~a}o {\'e} anel de integridade, \pause por exemplo, em $\z_4$, \pause $\overline{2} \neq \overline{0}$, \pause no entanto
                \[
                    \overline{2}\otimes \overline{2} \pause = \overline{4} \pause = \overline{0}.
                \]
                \vspace{.5cm}

                \seti
            \end{enumerate}
        \end{exemplos}
    \end{frame}

    \begin{frame}
        \begin{exemplos}
            \begin{enumerate}[label={\arabic*})]
                \conti

                \item $M_{n}(\real)$ \pause n{\~a}o {\'e} um anel de integridade, \pause por exemplo, em $M_{2}(\real)$\pause
                \begin{center}
                    $A = \begin{bmatrix}
                        1 & 0\\
                        0 & 0
                    \end{bmatrix} \neq \begin{bmatrix}
                        0 & 0\\
                        0 & 0       
                    \end{bmatrix},\pause \qquad 
                    B = \begin{bmatrix}
                        0 & 0\\
                        1 & 0
                    \end{bmatrix} \neq \begin{bmatrix}
                        0 & 0\\
                        0 & 0
                    \end{bmatrix}\pause$\\
                    $AB  =\begin{bmatrix}
                        0 & 0\\
                        0 & 0
                    \end{bmatrix}\pause$
                \end{center}

                \vspace{.5cm}
                
                \item Suponha que $m = nk$, \pause $m > n > 1$ \pause e $m > k > 1$. \pause Logo, em $\z_m$, \pause $\overline{n} \neq \overline{0}$ \pause e $\overline{k} \neq \overline{0}$ \pause e no entanto $\overline{n} \otimes \overline{k} \pause = \overline{m} \pause = \overline{0}$. \pause Logo, se $m$ n{\~a}o {\'e} primo, \pause ent{\~a}o $\z_m$ n{\~a}o {\'e} um anel de integridade. \pause Agora, suponha que $m = p$ primo. \pause Sejam $\overline{x}$, \pause $\overline{y} \in \z_m$ \pause tais que $\overline{x}\otimes \overline{y} \pause = \overline{0}$, \pause ou seja, $xy \equiv 0 \pmod p$. \pause Da{\'\i} $p\mid xy$. \pause Logo $p\mid x$ \pause ou $p\mid y$. \pause Portanto, $\overline{x} = \overline{0}$ \pause ou $\overline{y} = \bar{0}$. \pause Assim, $\z_m$ {\'e} anel de integridade \pause se, e somente se, \pause $m$ {\'e} primo.\pause
            \end{enumerate}
        \end{exemplos}
    \end{frame}
    
    \begin{frame}
        \begin{definicao}
            Seja $(A, +, \cdot)$ um anel comutativo. \pause Um subconjunto n\~ao-vazio \pause $I \sub A$ \pause {\'e} chamado de \textbf{ideal} \pause de $A$ se:\pause
            \begin{enumerate}[label={\roman*})]
                \item para todos $x$, $y \in I$, \pause temos $x - y \in I$.\pause
                \item Para todo $\alpha \in A$ \pause e todo $x \in I$, \pause temos $\alpha\cdot x \in I$.\pause
            \end{enumerate}
        \end{definicao}

        \begin{observacao}
            Quando $I = A$ \pause ou $I = \{0_A\}$, \pause dizemos que $I$ \pause {\'e} um \textbf{ideal trivial}.\pause
        \end{observacao}
    \end{frame}

    \begin{frame}
        \begin{exemplos}
            \begin{enumerate}[label={\arabic*})]
                \item Considere no anel $\z$ as operações usuais de soma e multiplicação. \pause Seja 
                \[
                    I = m\z \pause = \{mk \mid k \in \z\},\pause
                \]
                com $m > 1$. \pause Então $I$ é um ideal de $\z$.\pause

                \item No anel $\z_p$, \pause onde $p$ {\'e} um n{\'u}mero primo, \pause os {\'u}nicos ideais s{\~a}o os triviais: $\{\overline{0}\}$ \pause e $\z_p$.\pause
            \end{enumerate}
        \end{exemplos}
    \end{frame}

    \begin{frame}
        \begin{proposicao}
            Seja $A$ um anel comutativo \pause e $I$ um ideal de $A$. \pause Ent{\~a}o:\pause
            \begin{enumerate}[label={\roman*})]
                \item $0_{A}\in I$.\pause
                \item $-x \in I$ \pause para todo $x \in I$.\pause
                \item Se $1_A \in I$, \pause ent\~ao $I = A$.\pause
            \end{enumerate}
        \end{proposicao}
        \noindent \textbf{Prova:}
        \vspace{4cm}
    \end{frame}

    \begin{frame}
        \begin{exemplos}
            \begin{enumerate}
                \item[i)] Os {\'u}nicos ideais n{\~a}o triviais de \pause $\z_8 = \{\overline{0}, \overline{1}, \overline{2}, \overline{3}, \overline{4}, \overline{5}, \overline{6}, \overline{7}\}$ s{\~a}o:\pause
                \begin{center}
                    $I_1 = \{\overline{0}, \overline{2}, \overline{4}, \overline{6}\}\pause$\\
                    $I_2 =\{\overline{0}, \overline{4}\}\pause$
                \end{center}
            \end{enumerate}
        \end{exemplos}
    \end{frame}

    \begin{frame}
        \begin{definicao}
            Seja $I$ um ideal \pause de um anel comutativo $(A, +, \cdot)$. \pause Dados $x$, $y \in A$\pause dizemos que $x$ \textbf{\'e congruente a} $y$ \pause \textbf{m\'odulo} $I$ \pause quando $x - y \in I$. \pause Neste caso, escrevemos $x\equiv y \pmod I$.\pause
        \end{definicao}

        \begin{proposicao}
            A congru{\^e}ncia m{\'o}dulo $I$ \pause {\'e} uma rela{\c c}{\~a}o de equival{\^e}ncia em $A \times A$, \pause onde é um $A$ anel comutativo unit{\'a}rio.\pause
        \end{proposicao}
        \noindent \textbf{Prova:}

        \vspace{3cm}
    \end{frame}

    \begin{frame}
        Seja $y \in A$. \pause A classe de equival{\^e}ncia m{\'o}dulo $I$ \pause de $y$ {\'e}\pause 
        \[
            C(y) \pause = \{x \in A \pause \mid x\equiv y \pmod I\} \pause = \{x \in A \pause \mid x - y \in I\}.\pause
        \]

        Agora, $x - y \in I$ \pause significa que existe $t \in I$, \pause tal que $x - y = t$. \pause Logo, $x = y + t$, \pause onde $t \in I$.\pause

        Assim,\pause
        \[
            C(y) = \pause \{y + t\pause \mid t \in I\} \pause = y + I.\pause
        \]

        \begin{observacao}
            Denotamos por $y + I$ \pause (ou $I + y$) \pause a classe de equival{\^e}ncia de $y \in A$ \pause m{\'o}dulo $I$. \pause Denotamos por \pause 
            \[
                \dfrac{A}{I}\pause 
            \]
            o conjunto de todas as classes de equival{\^e}ncia, \pause tal conjunto {\'e} chamado de \textbf{quociente do anel $A$ pelo ideal $I$}.\pause
        \end{observacao}
    \end{frame}

    \begin{frame}
        \begin{exemplos}
            \begin{enumerate}
                \item[1)] Seja $A$ um anel com unidade \pause e $I_{1} = \{0\}$ \pause e $I_{2} = A$ ideais. \pause Ent\~ao:\pause
                \begin{enumerate}
                    \item[i)] Dado $x \in A$:\pause
                    \[
                        C(x) = \pause x + I_{1} \pause = \{x + 0\} = \pause \{x\}.\pause 
                    \]
                    Assim
                    \[
                        \dfrac{A}{I_{1}} \pause = \{x + I \pause \mid x \in A\},\pause
                    \]
                    logo existem tantas classes de equival{\^e}ncia \pause quantos forem os elementos de $A$.\pause
                \end{enumerate}
            \end{enumerate}
        \end{exemplos}
    \end{frame}

    \begin{frame}
        \begin{exemplos}
            \begin{enumerate}
                \item[ii)] Para $I_{2} = A$ \pause temos:
                \[
                    C(0_A) = \pause 0_A + I = \pause \{0_A + t \mid t \in I_{2}\}.\pause 
                \]
                Como $I_2 = A$, \pause para todo $x \in A$ \pause temos $x \in C(0_A)$ \pause logo existe uma \'unica \pause classe de equival\^encia\pause e
                \[
                    \dfrac{A}{I_{2}} = \{0_{A} + I\}.\pause
                \]
            \end{enumerate}
        \end{exemplos}
    \end{frame}

    \begin{frame}
        \begin{exemplos}
            \begin{enumerate}
                \item[2)] Seja $A = \z$. \pause Os ideais de $\z$ \pause s{\~a}o da forma $m\z$, \pause $m > 1$. \pause Seja $I = m\z$ \pause um ideal de $\z$. \pause Assim\pause
                \[
                    x\equiv y \pmod I\pause 
                \]
                se, e s\'o se, 
                \[
                    x - y \in I.\pause
                \]
                Mais isso ocorre \pause se, e somente se, \pause $x - y = mk $, \pause para algum $k \in \z$. \pause Logo $x\equiv y \pmod I$ \pause se, e s\'o se, \pause $m\mid (x - y)$. \pause  Portanto,
                \[
                    \dfrac{\z}{I} = \z_m.\pause
                \]
            \end{enumerate}
        \end{exemplos}
    \end{frame}

    \begin{frame}
        Agora seja $I$ ideal \pause e $A$ um anel. \pause Temos\pause 
        \[
            \dfrac{A}{I} = \pause \{y + I \mid y \in A\}\pause 
        \]
        onde $y + I = \{y + t \mid t \in I\}$ \pause e $y \in A$.\pause 

        Vamos definir uma soma $\oplus$ \pause e um produto $\otimes$ \pause em $\dfrac{A}{I}$ por\pause 
        \begin{center}
            $(x + I) \oplus (y + I) = \pause (x + y) + I$\pause\\
            $(x + I) \otimes (y + I) = \pause (xy) + I$\pause 
        \end{center}
        para $x + I$, \pause $y + I \in \dfrac{A}{I}$.\pause
    \end{frame}

    \begin{frame}
        Verifiquemos que a soma \pause e o produto \pause em $\dfrac{A}{I}$ \pause n{\~a}o dependem do representante da classe de equival{\^e}ncia.\pause

        Para isso, \pause sejam $x_1 + I$, \pause $x_2 + I$, \pause $y_1 + I$, \pause $y_2 + I \in \dfrac{A}{I}$ \pause tais que\pause 
        \begin{center}
            $x_1 + I \pause = x_2 + I$\pause\\
            $y_1 + I \pause = y_2 + I$\pause
        \end{center}

        Ent{\~a}o\pause 
        \begin{center}
            $(x_1 + I) \oplus (y_1 + I) \pause = (x_1 + y_1) + I$\pause\\
            $(x_2 + I) \oplus (y_2 + I) \pause = (x_2 + y_2) + I$\pause
        \end{center}
    \end{frame}

    \begin{frame}
        Como $x_1 + I = x_2 + I$, \pause ent{\~a}o $x_1 - x_2 \in I$ \pause e como $y_1 + I = y_2 + I$, \pause ent{\~a}o $y_1 = y_2 \in I$. \pause Mas $I$ {\'e} ideal, \pause logo $(x_1 - x_2) + (y_1 - y_2) = (x_1 + y_1) - (x_2 + y_2) \in I$, \pause ou seja\pause 
        \[
            (x_1 + I) \oplus (y_1 + I) \pause = (x_2 + I) \oplus (y_2 + I).\pause
        \]
    \end{frame}

    \begin{frame}
        Agora,\pause 
        \begin{center}
            $(x_1 + I) \otimes (y_1 + I) \pause = (x_1y_1) + I$\pause\\
            $(x_2 + I) \otimes (y_2 + I) \pause = (x_2y_2) + I$\pause
        \end{center}

        Como $(x_1 - x_2)y \in I$ \pause e $(y_1 - y_2)x_2 \in I$ \pause ent\~ao\pause
        \begin{center}
            $(x_1 - x_2)y_1 + (y_1 - y_2)x_2 \in I$\pause\\
            $x_1y_2-\underbrace{x_2y_1 + y_1x_2}_{= 0} - y_2x_2 \in I$\pause\\
            $x_1y_1 - x_2y_2\in I,$\pause 
        \end{center}
        ou seja, \pause $xy + I = x_2y_2 + I$. \pause Portanto,\pause 
        \[
            (x_1 + I) \otimes (y + I) \pause = (x_2 + I) \otimes (y_2 + I).\pause
        \]
    \end{frame}

    \begin{frame}
        \begin{teorema}
            Seja $(A, +, \cdot)$ um anel comutativo \pause e com unidade. \pause Se $I$ {\'e} um ideal de $A$, \pause então
            \[
                \left(\dfrac{A}{I}, \oplus, \otimes\right)\pause 
            \]
            {\'e} um anel comutativo \pause e com unidade. \pause O elemento neutro da soma \pause {\'e} a classe $0_{A} + I$ \pause e a unidade do produto \pause {\'e} $1_{A} + I$.\pause
        \end{teorema}
    \end{frame}
\end{document}