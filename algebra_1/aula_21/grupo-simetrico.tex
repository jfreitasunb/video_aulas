%!TEX program = xelatex
\def\ano{2020}
\def\semestre{1}
\def\disciplina{\'Algebra 1}
\def\turma{C}
\def\autor{Jos\'e Ant\^onio O. Freitas}
\def\instituto{MAT-UnB}

\documentclass{beamer}
\usetheme{Madrid}
\usecolortheme{beaver}
% \mode<presentation>
\usepackage{caption}
\usepackage{textpos}
\usepackage{amssymb}
\usepackage{amsmath,amsfonts,amsthm,amstext}
\usepackage[brazil]{babel}
% \usepackage[latin1]{inputenc}
\usepackage{graphicx}
\graphicspath{{/home/jfreitas/GitHub_Repos/video_aulas/logo/}{D:/Dropbox/imagens-latex/}}
\usepackage{enumitem}
\usepackage{multicol}
\usepackage{answers}
\usepackage{tikz,ifthen}
\usetikzlibrary{lindenmayersystems}
\usetikzlibrary[shadings]
\newtheorem{definicao}{Defini\c{c}\~ao}[section]
\newtheorem{definicoes}{Defini\c{c}\~oes}[section]
\newtheorem{exemplo}{Exemplo}[section]
\newtheorem{exemplos}{Exemplos}[section]
\newtheorem{exercicio}{Exerc{\'\i}cio}
\newtheorem{observacao}{Observa{\c c}{\~a}o:}[section]
\newtheorem{observacoes}{Observa{\c c}{\~o}es:}[section]
\newtheorem*{solucao}{Solu{\c c}{\~a}o:}
\newtheorem{proposicao}{Proposi\c{c}\~ao}
\newtheorem{lema}{Lema}
\newtheorem{teorema}{Teorema}
\newtheorem{corolario}{Corol\'ario}
\newenvironment{prova}[1][Prova]{\noindent\textbf{#1:} }{\qedsymbol}%{\ \rule{0.5em}{0.5em}}
\newcommand{\nsub}{\varsubsetneq}
\newcommand{\vaz}{\emptyset}
\newcommand{\im}{{\rm Im\,}}
\newcommand{\sub}{\subseteq}
\newcommand{\n}{\mathbb{N}}
\newcommand{\z}{\mathbb{Z}}
\newcommand{\rac}{\mathbb{Q}}
\newcommand{\real}{\mathbb{R}}
\newcommand{\complex}{\mathbb{C}}
\newcommand{\cp}[1]{\mathbb{#1}}
\newcommand{\ch}{\mbox{\textrm{car\,}}\nobreak}
\newcommand{\vesp}[1]{\vspace{ #1  cm}}
\newcommand{\compcent}[1]{\vcenter{\hbox{$#1\circ$}}}
\newcommand{\comp}{\mathbin{\mathchoice
{\compcent\scriptstyle}{\compcent\scriptstyle}
{\compcent\scriptscriptstyle}{\compcent\scriptscriptstyle}}}

\title{Grupo Sim\'etrico}
\author[\autor]{\autor}
\institute[\instituto]{\instituto}
\date{}

\begin{document}
    \begin{frame}
        \maketitle
    \end{frame}

    \logo{\includegraphics[scale=.1]{logo-MAT.png}\vspace*{8.5cm}}

    \begin{frame}
        Seja $A$ um conjunto n\~ao vazio.\pause

        \vspace{.3cm}

        Dada uma fun\c{c}\~ao $f : A \to A$, sabemos que $f$ possui inversa \pause se, e somente se, $f$ \'e bijetora.\pause

        \vspace{.3cm}

        Assim considere o conjunto\pause
        \[
            \mathcal{S} = \{ f : A \to A \pause \mid f \mbox{ \'e bijetora}\}.\pause
        \]

        Em $\mathcal{S}$ vamos considerar a composi\c{c}\~ao de fun\c{c}\~oes $\circ$.\pause

        \vspace{.3cm}

        Como $id : A \to A$ tal que $id(x) = x$ para todo $x \in A$ \pause \'e uma fun\c{c}\~ao bijetora \pause ent\~ao $id \in \mathcal{S}$ \pause e com isso $\mathcal{S} \ne \emptyset$.
    \end{frame}

    \begin{frame}
        Dadas $f$, $g \in \mathcal{S}$ \pause como $f$ e $g$ s\~ao bijetoras, \pause ent\~ao $f \circ g$ \'e bijetora \pause e da{\'\i} $f \circ g \in \mathcal{S}$. \pause Isto \'e, \pause a composi\c{c}\~ao de fun\c{c}\~oes \pause \'e uma opera\c{c}\~ao bin\'aria em $\mathcal{S}$.

        \vspace{.3cm}

        Agora, sejam $f$, $g$ e $h \in \mathcal{S}$. \pause Para todo $x \in A$ temos\pause
        \begin{center}
            $[(f\circ g) \pause \circ h \pause](x) \pause = (f \circ g) \pause(h(x)) \pause = f(g(h(x))) \pause$\\
            $[f\circ \pause(g\circ h) \pause](x) \pause = f( \pause(g\circ h)(x) \pause) = f(g(h(x))) \pause$
        \end{center}

        Logo $(f\circ g) \pause \circ h  \pause = f\circ \pause (g\circ h)$. \pause

        \vspace{.3cm}

        Agora, para toda $f \in \mathcal{S}$\pause
        \[
            f\circ id  \pause = f  \pause = id\circ f, \pause
        \]
        onde $id : A \to A$ \pause \'e tal que $id(x) = x$, \pause para todo $x \in A$. \pause Logo $id$ \'e o elemento neutro da composi\c{c}\~ao.\pause

    \end{frame}

    \begin{frame}

        Finalmente, \pause para toda $f \in \mathcal{S}$, \pause como $f$ \'e bijetora \pause existe $g \in \mathcal{S}$ \pause tal que \pause
        \[
            f\circ g \pause = id \pause = g \circ f. \pause
        \]
        Logo todo elemento de $\mathcal{S}$ \pause possui inverso. \pause

        \vspace{.3cm}

        Portanto $(\mathcal{S}, \circ)$ \pause \'e um grupo.  \pause Al\'em disso, em geral, esse grupo n\~ao \'e comutativo. \pause

        \vspace{.3cm}

        Vamos considerar agora o caso particular \pause em que $A$ \'e um conjunto finito. \pause

        \vspace{.3cm}

        Nessa situa\c{c}\~ao podemos supor que $A \sub \n$ \pause para simplificar a nota\c{c}\~ao. \pause

        \vspace{.3cm}

        Vamos ver como \'e o conjunto $\mathcal{S}$ com essa hip\'otese. \pause
    \end{frame}

    \begin{frame}
        Se $A = \{1\}$, \pause ent\~ao s\'o existe uma fun\c{c}\~ao $f : A \to A$ \pause que \'e bijetora e essa fun\c{c}\~ao \'e tal que \pause
        \begin{center}
            $f : \{1\} \to \{1\}$ \pause\\
            $f(1) = 1$. \pause
        \end{center}
        Ou seja, $f$ \'e a fun\c{c}\~ao a identidade $id$. \pause Nesse caso $\mathcal{S} \pause = S_1 \pause = \{id\}$ \pause e $(S_1, \circ)$ \'e um grupo, \pause e nesse caso comutativo. \pause
    \end{frame}

    \begin{frame}
        Se $A = \{1, 2\}$ \pause ent\~ao podemos definir as seguintes fun\c{c}\~oes bijetoras em $A$: \pause
        \begin{multicols}{2}
            \begin{enumerate}
                \item[] \begin{center}
                    $id : A \to A$ \pause\\
                    $id(1) = 1$ \pause\\ $id(2) = 2$
                \end{center}\pause
                \item[]  \begin{center}
                    $f : A \to A$ \pause\\ $f(1) = 2 \pause$\\ $f(2) = 1$
                \end{center}\pause
            \end{enumerate}
        \end{multicols}

        Assim $\mathcal{S} \pause = S_2 \pause = \{id, \pause f\}$ \pause e $(S_2, \circ)$ \'e um grupo.\pause

        \begin{table}[!htb]
        \centering
            \begin{tabular}{|c|c|c|}
                \hline
                $\circ$ & $id$ & $f$\T\\
                \hline
                $id$ & \phantom{xyz} & \phantom{xyz}\T\\
                \hline
                $f$ & \phantom{xyz} & \phantom{xyz}\T\\
                \hline
            \end{tabular}
        \end{table}\pause

        Al\'em disso, da tabela acima vemos que esse grupo \'e comutativo.
    \end{frame}

    \begin{frame}
        Agora, seja $A = \{1, 2, 3\}$. \pause Podemos definir ent\~ao as seguintes fun\c{c}\~oes bijetoras em $A$: \pause
        \begin{multicols}{3}
            \begin{enumerate}
                \item[] \begin{center}
                    $id : A \to A$ \pause\\
                    $id(1) = 1$ \pause\\
                    $id(2) = 2$ \pause\\
                    $id(3) = 3$ \pause
                \end{center}\pause

                \vspace{.3cm}

                \item[] \begin{center}
                    $f_1 : A \to A$\pause\\
                    $f_1(1) = 2$\pause\\
                    $f_1(2) = 1$\pause\\
                    $f_1(3) = 3$
                \end{center}\pause

                \vspace{.3cm}

                \item[] \begin{center}
                    $f_2 : A \to A$\pause\\
                    $f_2(1) = 3$\pause\\
                    $f_2(2) = 2$\pause\\
                    $f_2(3) = 1$
                \end{center}\pause

                \vspace{.3cm}

                \item[] \begin{center}
                    $f_3 : A \to A$\pause\\
                    $f_3(1) = 1$\pause\\
                    $f_3(2) = 3$\pause\\
                    $f_3(3) = 2$
                \end{center}\pause

                \vspace{.3cm}

                \item[] \begin{center}
                    $f_4 : A \to A$\pause\\
                    $f_4(1) = 2$\pause\\
                    $f_4(2) = 3$\pause\\
                    $f_4(3) = 1$
                \end{center}\pause

                \vspace{.3cm}

                \item[] \begin{center}
                    $f_5 : A \to A$\pause\\
                    $f_5(1) = 3$\pause\\
                    $f_5(2) = 1$\pause\\
                    $f_5(3) = 2$
                \end{center}
            \end{enumerate}
        \end{multicols}
    \end{frame}

    \begin{frame}
        Logo $\mathcal{S} = S_3 = \{id, f_1, f_2, f_3, f_4, f_5\}$ \pause e $(S_3, \circ)$ \'e um grupo.\pause

        \vspace{.3cm}

        Nesse caso temos\pause
        \begin{center}
            $(f_1 \circ f_4)(1) \pause = f_1(f_4(1)) \pause = f_1(2) \pause = 1$\pause\\
            \vspace{.3cm}
            $(f_4 \circ f_1)(1) \pause = f_4(f_1(1)) \pause = f_4(2) \pause = 3$\pause
        \end{center}
        da{\'\i} $(f_1 \circ f_4)(1) \pause \ne (f_4 \circ f_1)(1)$ \pause, isto \'e, \pause $f_1 \circ f_4 \pause \ne f_4 \circ f_1$. \pause Portanto o grupo $(S_3, \circ)$ \pause n\~ao \'e comutativo.\pause

        \vspace{.3cm}

        Note que em $S_2$ \pause temos $2 = 2!$ elementos \pause e em $S_3$ \pause temos $6 = 3!$ elementos.\pause
    \end{frame}

    \begin{frame}
        De modo geral, \pause se $A = \{1, 2, 3, \dots, n\}$ \pause ent\~ao existem exatamente $n!$ \pause fun\c{c}\~oes $f : A \to A$ bijetoras. \pause

        \vspace{.3cm}

        Assim o grupo $(S_n, \circ)$ \pause possui $n!$ elementos.\pause

        \vspace{.3cm}

        Se $n \geqslant 3$, ent\~ao \pause $S_n$ \'e um grupo n\~ao comutativo.\pause

        \begin{definicao}
            O grupo $S_n$ \'e chamado de \pause \textbf{grupo sim\'etrico} \pause ou \textbf{grupo de permuta\c{c}\~oes} \pause em $A = \{1, 2, 3, \dots, n\}$.
        \end{definicao}
    \end{frame}

    \begin{frame}
        Um modo de representar os elementos de $S_n$ \'e o seguinte: \pause vamos representar as fun\c{c}\~oes $f \in S_n$ \pause na forma de uma matriz contendo 2 linhas \pause e $n$ colunas. \pause A primeira linha \'e o dom{\'\i}nio da fun\c{c}\~ao \pause e a segunda cont\'em suas imagens. \pause Assim se $f \in S_n$ escreveremos\pause
        \[
            f = \pause \begin{pmatrix}
                1 & 2 & 3 & \dots & n\pause \\
                f(1) \pause & f(2) \pause & f(3) \pause & \dots \pause & f(n)\pause
            \end{pmatrix}.
        \]
    \end{frame}

    \begin{frame}
        No caso de $S_3$ vamos escrever\pause
        \begin{multicols}{3}
            \begin{enumerate}
                \item[] $id = \begin{pmatrix}
                    1 & 2 & 3\pause\\
                    1 \pause & 2 \pause & 3\pause
                \end{pmatrix}$\pause

                \vspace{.5cm}

                \item[] $f_1 = \begin{pmatrix}
                    1 & 2 & 3 \pause\\
                    2 \pause & 1 \pause & 3 \pause
                \end{pmatrix}$\pause

                \vspace{.5cm}

                \item[] $f_2 = \begin{pmatrix}
                    1 & 2 & 3 \pause\\
                    3 \pause & 2 \pause & 1 \pause
                \end{pmatrix}$\pause

                \vspace{.5cm}

                \item[] $f_3 = \begin{pmatrix}
                    1 & 2 & 3 \pause\\
                    1 \pause & 3 \pause & 2 \pause
                \end{pmatrix}$\pause

                \vspace{.5cm}

                \item[] $f_4 = \begin{pmatrix}
                    1 & 2 & 3 \pause\\
                    2 \pause & 3 \pause & 1 \pause
                \end{pmatrix}$\pause

                \vspace{.5cm}

                \item[] $f_5 = \begin{pmatrix}
                    1 & 2 & 3 \pause\\
                    3 \pause & 1 \pause & 2 \pause
                \end{pmatrix}$\pause
            \end{enumerate}
        \end{multicols}
    \end{frame}

    \begin{frame}
        Assim a composi\c{c}\~ao $f_3 \circ f_4$ pode ser determinada da seguinte forma:
        \[
            f_3\circ f_4 = \begin{pmatrix}
                    1 & 2 & 3\\
                    1 & 3 & 2
                \end{pmatrix} \pause \circ \begin{pmatrix}
                    1 & 2 & 3\\
                    2 & 3 & 1
                \end{pmatrix}
        \]
    \end{frame}

    \begin{frame}
        A composi\c{c}\~ao $f_4 \circ f_5$ \'e:

        \[
            f_4\circ f_5 = \begin{pmatrix}
                    1 & 2 & 3\\
                    2 & 3 & 1
                \end{pmatrix} \pause \circ \begin{pmatrix}
                    1 & 2 & 3\\
                    3 & 1 & 2
                \end{pmatrix}
        \]
    \end{frame}
\end{document}
