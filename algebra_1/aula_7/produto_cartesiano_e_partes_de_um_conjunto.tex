%!TEX program = xelatex
% !TEX encoding = ISO-8859-1
\def\ano{2020}
\def\semestre{1}
\def\disciplina{\'Algebra 1}
\def\turma{C}
\def\autor{Jos\'e Ant\^onio O. Freitas}
\def\instituto{MAT-UnB}

\documentclass{beamer}
\usetheme{Madrid}
\usecolortheme{beaver}
% \mode<presentation>
\usepackage{caption}
\usepackage{textpos}
\usepackage{amssymb}
\usepackage{amsmath,amsfonts,amsthm,amstext}
\usepackage[brazil]{babel}
% \usepackage[latin1]{inputenc}
\usepackage{graphicx}
\graphicspath{{/home/jfreitas/GitHub_Repos/video_aulas/logo/}{D:/Dropbox/imagens-latex/}}
\usepackage{enumitem}
\usepackage{multicol}
\usepackage{answers}
\usepackage{tikz,ifthen}
\usetikzlibrary{lindenmayersystems}
\usetikzlibrary[shadings]
\newtheorem{definicao}{Defini\c{c}\~ao}[section]
\newtheorem{definicoes}{Defini\c{c}\~oes}[section]
\newtheorem{exemplo}{Exemplo}[section]
\newtheorem{exemplos}{Exemplos}[section]
\newtheorem{exercicio}{Exerc{\'\i}cio}
\newtheorem{observacao}{Observa{\c c}{\~a}o:}[section]
\newtheorem{observacoes}{Observa{\c c}{\~o}es:}[section]
\newtheorem*{solucao}{Solu{\c c}{\~a}o:}
\newtheorem{proposicao}{Proposi\c{c}\~ao}
\newtheorem{lema}{Lema}
\newtheorem{teorema}{Teorema}
\newtheorem{corolario}{Corol\'ario}
\newenvironment{prova}[1][Prova]{\noindent\textbf{#1:} }{\qedsymbol}%{\ \rule{0.5em}{0.5em}}
\newcommand{\nsub}{\varsubsetneq}
\newcommand{\vaz}{\emptyset}
\newcommand{\im}{{\rm Im\,}}
\newcommand{\sub}{\subseteq}
\newcommand{\n}{\mathbb{N}}
\newcommand{\z}{\mathbb{Z}}
\newcommand{\rac}{\mathbb{Q}}
\newcommand{\real}{\mathbb{R}}
\newcommand{\complex}{\mathbb{C}}
\newcommand{\cp}[1]{\mathbb{#1}}
\newcommand{\ch}{\mbox{\textrm{car\,}}\nobreak}
\newcommand{\vesp}[1]{\vspace{ #1  cm}}
\newcommand{\compcent}[1]{\vcenter{\hbox{$#1\circ$}}}
\newcommand{\comp}{\mathbin{\mathchoice
{\compcent\scriptstyle}{\compcent\scriptstyle}
{\compcent\scriptscriptstyle}{\compcent\scriptscriptstyle}}}

\title{Produto cartesiano e Conjuntos das partes}
\author[\autor]{\autor}
\institute[\instituto]{\instituto}
\date{\today}

\begin{document}
    \begin{frame}
        \maketitle
    \end{frame}

    \logo{\includegraphics[scale=.1]{logo-MAT.png}\vspace*{8.5cm}}

    \begin{frame}
        \begin{definicao}
        Dados dois conjuntos $A$ e $B$, \pause definimos o \textbf{produto cartesiano} \pause de $A$ por $B$ como sendo o conjunto\pause
        \[
            A \times B = \pause \{(x,y) \pause \mid x\in A, y\in B\}.\pause
        \]
        \end{definicao}

        Dados $(x,y)$, \pause $(z,t) \in A\times B$, \pause temos
        \begin{center}
            $(x,y) = (z,t)$ \pause \textbf{se, e somente se,} $x = z$ \pause \textbf{e} $y = t$.\pause
        \end{center}

        \begin{exemplo}\label{exemplo_produto_cartesiano}
            Sejam $A = \{1, 2\}$ \pause e $B = \{3, 4, 5\}$. \pause Ent\~ao\pause
            \begin{center}
                $A \times B = \pause \{(1,3), \pause (1,4), \pause  (1,5), \pause (2,3), \pause (2,4), \pause (2,5)\}$\pause\\
                $B \times A = \pause \{(3,1), \pause (3,2), \pause (4,1), \pause (4,2), \pause (5,1), \pause (5,2)\}$\pause
        \end{center}
        \end{exemplo}
    \end{frame}

    \begin{frame}
        \begin{observacoes}
            \begin{enumerate}[label={\arabic*})]
                \item Do Exemplo anterior \pause vemos que em geral $A \times B \neq B\times A$.\pause
                \item No caso em que $A = B$ \pause vamos escrever\pause
                \[
                    A \times A = A^2 = \pause \{ (x, y) \mid x, y \in A\}.\pause
                \]
                De modo geral:\pause
                \[
                    \underbrace{A \times A \times \cdots \times A}_{n\ vezes} = A^n = \pause \{ (x_1, x_2, \dots, x_n) \mid x_1, x_2, \dots, x_n \in A\}\pause
                \]
                para $n \ge 2$.
            \end{enumerate}
        \end{observacoes}
    \end{frame}

    \begin{frame}
        \begin{definicao}
            Para qualquer conjunto $A$, \pause indicamos por $\mathcal{P}(A)$ \pause o conjunto\pause
            \[
                \mathcal{P}(A) = \pause \{ X \mid X\subseteq A\}\pause
            \]
            que \'e chamado de \textbf{conjunto das partes} de $A$.\pause
        \end{definicao}

        % Os elementos desse conjunto s{\~a}o todos os subconjuntos de $A$. Dizer que $Y\in \mathcal{P}(A)$ significa que $Y \subseteq A$. Particularmente, temos $\emptyset\in \mathcal{P}(A)$ e $A\in \mathcal{P}(A)$.

        \begin{exemplos}
            \begin{enumerate}[label={\arabic*})]
                \item $A = \emptyset$, \pause $\mathcal{P}(A) = \{\emptyset\}$;\pause
                \item $B = \{x\}$, \pause $\mathcal{P}(B) = \pause \{\emptyset, \pause \{x\}\}$;\pause
                \item $C = \{\alpha,\beta,\gamma\}$, \pause $\mathcal{P}(C) = \pause \{\emptyset, \pause\{\alpha\}, \pause\{\beta\}, \pause\{\gamma\}, \pause\{\alpha,\beta\}, \pause\{\alpha,\gamma\}, \pause\{\beta,\gamma\}, \pause C\}$;\pause
                \item $D=\real$, \pause $\mathcal{P}(D) = \{X \mid X \subseteq \real\}$, \pause por exemplo $\rac \in \mathcal{P}(D)$.
            \end{enumerate} 
        \end{exemplos}
    \end{frame}
\end{document}