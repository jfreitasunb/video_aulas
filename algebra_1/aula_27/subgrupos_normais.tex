%!TEX program = xelatex
%!TEX encoding = ISO-8859-1
\def\ano{2020}
\def\semestre{1}
\def\disciplina{\'Algebra 1}
\def\turma{C}
\def\autor{Jos\'e Ant\^onio O. Freitas}
\def\instituto{MAT-UnB}

\documentclass{beamer}
\usetheme{Madrid}
\usecolortheme{beaver}
% \mode<presentation>
\usepackage{caption}
\usepackage{textpos}
\usepackage{amssymb}
\usepackage{amsmath,amsfonts,amsthm,amstext}
\usepackage[brazil]{babel}
% \usepackage[latin1]{inputenc}
\usepackage{graphicx}
\graphicspath{{/home/jfreitas/GitHub_Repos/video_aulas/logo/}{D:/Dropbox/imagens-latex/}}
\usepackage{enumitem}
\usepackage{multicol}
\usepackage{answers}
\usepackage{tikz,ifthen}
\usetikzlibrary{lindenmayersystems}
\usetikzlibrary[shadings]
\newtheorem{definicao}{Defini\c{c}\~ao}[section]
\newtheorem{definicoes}{Defini\c{c}\~oes}[section]
\newtheorem{exemplo}{Exemplo}[section]
\newtheorem{exemplos}{Exemplos}[section]
\newtheorem{exercicio}{Exerc{\'\i}cio}
\newtheorem{observacao}{Observa{\c c}{\~a}o:}[section]
\newtheorem{observacoes}{Observa{\c c}{\~o}es:}[section]
\newtheorem*{solucao}{Solu{\c c}{\~a}o:}
\newtheorem{proposicao}{Proposi\c{c}\~ao}
\newtheorem{lema}{Lema}
\newtheorem{teorema}{Teorema}
\newtheorem{corolario}{Corol\'ario}
\newenvironment{prova}[1][Prova]{\noindent\textbf{#1:} }{\qedsymbol}%{\ \rule{0.5em}{0.5em}}
\newcommand{\nsub}{\varsubsetneq}
\newcommand{\vaz}{\emptyset}
\newcommand{\im}{{\rm Im\,}}
\newcommand{\sub}{\subseteq}
\newcommand{\n}{\mathbb{N}}
\newcommand{\z}{\mathbb{Z}}
\newcommand{\rac}{\mathbb{Q}}
\newcommand{\real}{\mathbb{R}}
\newcommand{\complex}{\mathbb{C}}
\newcommand{\cp}[1]{\mathbb{#1}}
\newcommand{\ch}{\mbox{\textrm{car\,}}\nobreak}
\newcommand{\vesp}[1]{\vspace{ #1  cm}}
\newcommand{\compcent}[1]{\vcenter{\hbox{$#1\circ$}}}
\newcommand{\comp}{\mathbin{\mathchoice
{\compcent\scriptstyle}{\compcent\scriptstyle}
{\compcent\scriptscriptstyle}{\compcent\scriptscriptstyle}}}

\title{Subgrupos Normais}
\author[\autor]{\autor}
\institute[\instituto]{\instituto}
\date{\today}

\begin{document}
    \begin{frame}
        \maketitle
    \end{frame}

    \logo{\includegraphics[scale=.1]{logo-MAT.png}\vspace*{8.5cm}}

    \begin{frame}
        Sejam $(G, \cdot)$ um grupo e $A$ e $B$ subconjuntos de $G$. Vamos indicar por
        \[
            AB
        \]
        e chamaremos de \textbf{produto} de $A$ por $B$ o seguinte subconjunto de $G$:
        \begin{center}
            \begin{tabular}{l}
                $AB = \emptyset$, se $A = \emptyset$ ou $B = \emptyset$\\
                \\
                $AB = \{xy \mid x \in A \mbox{ e } y \in B\}$, se  $A \ne \emptyset$  e $B \ne \emptyset$.
            \end{tabular}
        \end{center}

        \vspace{.3cm}

        Assim o \textbf{produto} de $A$ por $B$ é uma operação sobre o subconjuntos das partes de $G$, $\mathcal{P}(G)$, chamada de \textbf{multiplicação de subconjuntos} de $G$.

        \vspace{.3cm}

        Como $G$ é associativo, então a \textbf{multiplicação de subconjuntos} também será associativa. Além disso, caso o grupo $G$ seja comutativo, então \textbf{multiplicação de subconjuntos} também será comutativa.
    \end{frame}

    \begin{frame}
        \begin{exemplos}
            \begin{enumerate}[label=({\arabic*})]
                \item Seja $G = \{e, a, b, c\}$ o grupo tal que
                \begin{table}[h]
                    \begin{tabular}{|c|c|c|c|c|}
                        \hline
                        $\cdot$ & e & a & b & c\\
                        \hline
                        e & e & a & b & c\\
                        \hline
                        a & a & e & c & b\\
                        \hline
                        b & b & c & e & a\\
                        \hline
                        c & c & b & a & e\\
                        \hline
                    \end{tabular}.
                \end{table}
                Esse grupo é chamada de \textbf{grupo de Klein}.

                \vspace{.3cm}

                Se $A = \{e, a\}$ e $B = \{b, c\}$, então:
                \seti
            \end{enumerate}
        \end{exemplos}
    \end{frame}

    \begin{frame}
        \begin{exemplos}
            \begin{enumerate}[label=({\arabic*})]
                \conti
                \item Considere o grupo multiplicativo dos números reais. Se
                \begin{center}
                    $A = \{x \in \real^* \mid x > 0\}$\\
                    $B = \{x \in \real^* \mid x < 0\}$
                \end{center}
                então:
            \end{enumerate}
        \end{exemplos}
    \end{frame}

    \begin{frame}
        \begin{definicao}
            Um subgrupo $N$ de um grupo $G$ é chamado de \textbf{subgrupo normal} (ou \textbf{invariante}) se, para todo $x \in G$, vale
            \[
                xN = Nx.
            \]
            Denotaremos esse fato escrevendo $H \unlhd G$.
        \end{definicao}
    \end{frame}

    \begin{frame}
        \begin{exemplos}
            \begin{enumerate}[label=({\arabic*})]
                \item Seja $G = S_3$. Já vimos que se tomamos
                \[
                    Id = \begin{pmatrix}
                        1 & 2 & 1\\
                        1 & 2 & 3
                    \end{pmatrix},\quad
                    f = \begin{pmatrix}
                        1 & 2 & 1\\
                        2 & 3 & 1
                    \end{pmatrix} \quad \mbox{e}\quad
                    g = \begin{pmatrix}
                        1 & 2 & 1\\
                        1 & 3 & 2
                    \end{pmatrix}
                \]
                então
                \[
                    S_3 = \{Id, f, f^2, g, gf, gf^2\}.
                \]
                Considere o subgrupo $H = [\ f\ ] = \{Id, f, f^2\}$. Então $H$ é um subgrupo normal de $G$.

                \seti
            \end{enumerate}
        \end{exemplos}
    \end{frame}

    \begin{frame}
        \begin{exemplos}
            \begin{enumerate}[label=({\arabic*})]
                \conti

                \item Se $G$ é um grupo abeliano, então todo subgrupo de $G$ é normal.
                
                \seti
            \end{enumerate}
        \end{exemplos}
    \end{frame}

    \begin{frame}
        \begin{exemplos}
            \begin{enumerate}[label=({\arabic*})]
                \conti

                \item Seja $H$ um subgrupo de $G$ tal que $H$ possui somente duas classes laterais. Então $H$ é um subgrupo normal de $G$.
                
                \seti
            \end{enumerate}
        \end{exemplos}
    \end{frame}

    \begin{frame}
        \begin{proposicao}
            Seja $G$ um grupo. Se $H$ e $L$ são subgrupos normais de $G$, então $H \cap L$ é um subgrupo normal de $G$.
        \end{proposicao}
    \end{frame}

    \begin{frame}
        \begin{proposicao}
            Seja $N$ um subgrupo normal do grupo $G$. Então, para quaisquer $a$, $b \in G$ temos
            \[
                (aN)(bN) = (ab)N.
            \]
        \end{proposicao}
    \end{frame}
\end{document}