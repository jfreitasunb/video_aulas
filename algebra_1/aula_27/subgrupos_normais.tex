%!TEX program = xelatex
\def\ano{2020}
\def\semestre{1}
\def\disciplina{\'Algebra 1}
\def\turma{C}
\def\autor{Jos\'e Ant\^onio O. Freitas}
\def\instituto{MAT-UnB}

\documentclass{beamer}
\usetheme{Madrid}
\usecolortheme{beaver}
% \mode<presentation>
\usepackage{caption}
\usepackage{textpos}
\usepackage{amssymb}
\usepackage{amsmath,amsfonts,amsthm,amstext}
\usepackage[brazil]{babel}
% \usepackage[latin1]{inputenc}
\usepackage{graphicx}
\graphicspath{{/home/jfreitas/GitHub_Repos/video_aulas/logo/}{D:/Dropbox/imagens-latex/}}
\usepackage{enumitem}
\usepackage{multicol}
\usepackage{answers}
\usepackage{tikz,ifthen}
\usetikzlibrary{lindenmayersystems}
\usetikzlibrary[shadings]
\newtheorem{definicao}{Defini\c{c}\~ao}[section]
\newtheorem{definicoes}{Defini\c{c}\~oes}[section]
\newtheorem{exemplo}{Exemplo}[section]
\newtheorem{exemplos}{Exemplos}[section]
\newtheorem{exercicio}{Exerc{\'\i}cio}
\newtheorem{observacao}{Observa{\c c}{\~a}o:}[section]
\newtheorem{observacoes}{Observa{\c c}{\~o}es:}[section]
\newtheorem*{solucao}{Solu{\c c}{\~a}o:}
\newtheorem{proposicao}{Proposi\c{c}\~ao}
\newtheorem{lema}{Lema}
\newtheorem{teorema}{Teorema}
\newtheorem{corolario}{Corol\'ario}
\newenvironment{prova}[1][Prova]{\noindent\textbf{#1:} }{\qedsymbol}%{\ \rule{0.5em}{0.5em}}
\newcommand{\nsub}{\varsubsetneq}
\newcommand{\vaz}{\emptyset}
\newcommand{\im}{{\rm Im\,}}
\newcommand{\sub}{\subseteq}
\newcommand{\n}{\mathbb{N}}
\newcommand{\z}{\mathbb{Z}}
\newcommand{\rac}{\mathbb{Q}}
\newcommand{\real}{\mathbb{R}}
\newcommand{\complex}{\mathbb{C}}
\newcommand{\cp}[1]{\mathbb{#1}}
\newcommand{\ch}{\mbox{\textrm{car\,}}\nobreak}
\newcommand{\vesp}[1]{\vspace{ #1  cm}}
\newcommand{\compcent}[1]{\vcenter{\hbox{$#1\circ$}}}
\newcommand{\comp}{\mathbin{\mathchoice
{\compcent\scriptstyle}{\compcent\scriptstyle}
{\compcent\scriptscriptstyle}{\compcent\scriptscriptstyle}}}

\title{Subgrupos Normais}
\author[\autor]{\autor}
\institute[\instituto]{\instituto}
\date{}

\begin{document}
    \begin{frame}
        \maketitle
    \end{frame}

    \logo{\includegraphics[scale=.1]{logo-MAT.png}\vspace*{8.5cm}}

    \begin{frame}
        Sejam $(G, \cdot)$ um grupo \pause e $A$ e $B$ subconjuntos de $G$. \pause Vamos indicar por \pause
        \[
            AB \pause
        \]
        e chamaremos de \textbf{produto} de $A$ por $B$ \pause o seguinte subconjunto de $G$: \pause
        \begin{center}
            \begin{tabular}{l}
                $AB = \emptyset$, \pause se $A = \emptyset$ ou $B = \emptyset$ \pause\\
                \\
                $AB = \{xy \pause \mid x \in A \pause \mbox{ e } y \in B \pause\}$, se  $A \ne \emptyset$ e $B \ne \emptyset$. \pause
            \end{tabular}
        \end{center}

        \vspace{.3cm}

        Assim o \textbf{produto} de $A$ por $B$ \pause \'e uma opera\c{c}\~ao sobre o subconjuntos das partes de $G$, \pause $\mathcal{P}(G)$, chamada de \textbf{multiplica\c{c}\~ao de subconjuntos} de $G$. \pause

        \vspace{.3cm}

        Como $G$ \'e associativo, \pause ent\~ao a \textbf{multiplica\c{c}\~ao de subconjuntos} tamb\'em ser\'a associativa. \pause Al\'em disso, caso o grupo $G$ seja comutativo, \pause ent\~ao \textbf{multiplica\c{c}\~ao de subconjuntos} tamb\'em ser\'a comutativa.
    \end{frame}

    \begin{frame}
        \begin{exemplos}
            \begin{enumerate}[label=({\arabic*})]
                \item Seja $G = \{e, a, b, c\}$ \pause o grupo tal que \pause
                \begin{table}[h]
                    \begin{tabular}{|c|c|c|c|c|}
                        \hline
                        $\cdot$ & e & a & b & c\\
                        \hline
                        e & e & a & b & c\\
                        \hline
                        a & a & e & c & b\\
                        \hline
                        b & b & c & e & a\\
                        \hline
                        c & c & b & a & e\\
                        \hline
                    \end{tabular}.
                \end{table}
                Esse grupo \'e chamada de \textbf{grupo de Klein}. \pause

                \vspace{.3cm}

                Se $A = \{e, a\}$ \pause e $B = \{b, c\}$, \pause ent\~ao:
                \seti
            \end{enumerate}
        \end{exemplos}
    \end{frame}

    \begin{frame}
        \begin{exemplos}
            \begin{enumerate}[label=({\arabic*})]
                \conti
                \item Considere o grupo multiplicativo dos n\'umeros reais. \pause Se
                \begin{center}
                    $A = \{x \in \real^* \mid x > 0\}$ \pause\\
                    $B = \{x \in \real^* \mid x < 0\}$ \pause
                \end{center}
                ent\~ao:
            \end{enumerate}
        \end{exemplos}
    \end{frame}

    \begin{frame}
        \begin{definicao}
            Um subgrupo $N$ \pause de um grupo $G$ \pause \'e chamado de \textbf{subgrupo normal} \pause (ou \textbf{invariante}) \pause se, para todo $x \in G$, \pause vale
            \[
                xN \pause = Nx. \pause
            \]
            Denotaremos esse fato escrevendo $H \unlhd G$.
        \end{definicao}
    \end{frame}

    \begin{frame}
        \begin{exemplos}
            \begin{enumerate}[label=({\arabic*})]
                \item Seja $G = S_3$. \pause J\'a vimos que se tomamos
                \[
                    Id = \begin{pmatrix}
                        1 & 2 & 3\\
                        1 & 2 & 3
                    \end{pmatrix}, \pause \quad
                    f = \begin{pmatrix}
                        1 & 2 & 3\\
                        2 & 3 & 1
                    \end{pmatrix} \pause \quad \mbox{e}\quad
                    g = \begin{pmatrix}
                        1 & 2 & 3\\
                        1 & 3 & 2
                    \end{pmatrix} \pause
                \]
                ent\~ao
                \[
                    S_3 = \{Id, f, f^2, g, gf, gf^2\}. \pause
                \]
                Considere o subgrupo $H = [\ f\ ] \pause = \{Id, f, f^2\}$. \pause Ent\~ao $H$ \'e um subgrupo normal de $G$.

                \seti
            \end{enumerate}
        \end{exemplos}
    \end{frame}

    \begin{frame}
        \begin{exemplos}
            \begin{enumerate}[label=({\arabic*})]
                \conti

                \item Se $G$ \'e um grupo abeliano, \pause ent\~ao todo subgrupo de $G$ \'e normal.

                \seti
            \end{enumerate}
        \end{exemplos}
    \end{frame}

    \begin{frame}
        \begin{exemplos}
            \begin{enumerate}[label=({\arabic*})]
                \conti

                \item Seja $H$ um subgrupo de $G$ \pause tal que $H$ possui somente duas classes laterais. \pause Ent\~ao $H$ \'e um subgrupo normal de $G$.

                \seti
            \end{enumerate}
        \end{exemplos}
    \end{frame}

    \begin{frame}
        \begin{proposicao}
            Seja $G$ um grupo. \pause Se $H$ e $L$ s\~ao subgrupos normais de $G$, \pause ent\~ao $H \cap L$ \pause \'e um subgrupo normal de $G$.
        \end{proposicao}
    \end{frame}

    \begin{frame}
        \begin{proposicao}
            Seja $N$ um subgrupo normal \pause do grupo $G$. \pause Ent\~ao, para quaisquer $a$, $b \in G$ temos \pause
            \[
                (aN)(bN) \pause = (ab)N.
            \]
        \end{proposicao}
    \end{frame}
\end{document}
