%!TEX program = xelatex
% !TEX encoding = ISO-8859-1
\def\ano{2020}
\def\semestre{1}
\def\disciplina{\'Algebra 1}
\def\turma{C}
\def\autor{Jos\'e Ant\^onio O. Freitas}
\def\instituto{MAT-UnB}

\documentclass{beamer}
\usetheme{Madrid}
\usecolortheme{beaver}
% \mode<presentation>
\usepackage{caption}
\usepackage{textpos}
\usepackage{amssymb}
\usepackage{amsmath,amsfonts,amsthm,amstext}
\usepackage[brazil]{babel}
% \usepackage[latin1]{inputenc}
\usepackage{graphicx}
\graphicspath{{/home/jfreitas/GitHub_Repos/video_aulas/logo/}{D:/Dropbox/imagens-latex/}}
\usepackage{enumitem}
\usepackage{multicol}
\usepackage{answers}
\usepackage{tikz,ifthen}
\usetikzlibrary{lindenmayersystems}
\usetikzlibrary[shadings]
\newtheorem{definicao}{Defini\c{c}\~ao}[section]
\newtheorem{definicoes}{Defini\c{c}\~oes}[section]
\newtheorem{exemplo}{Exemplo}[section]
\newtheorem{exemplos}{Exemplos}[section]
\newtheorem{exercicio}{Exerc{\'\i}cio}
\newtheorem{observacao}{Observa{\c c}{\~a}o:}[section]
\newtheorem{observacoes}{Observa{\c c}{\~o}es:}[section]
\newtheorem*{solucao}{Solu{\c c}{\~a}o:}
\newtheorem{proposicao}{Proposi\c{c}\~ao}
\newtheorem{lema}{Lema}
\newtheorem{teorema}{Teorema}
\newtheorem{corolario}{Corol\'ario}
\newenvironment{prova}[1][Prova]{\noindent\textbf{#1:} }{\qedsymbol}%{\ \rule{0.5em}{0.5em}}
\newcommand{\nsub}{\varsubsetneq}
\newcommand{\vaz}{\emptyset}
\newcommand{\im}{{\rm Im\,}}
\newcommand{\sub}{\subseteq}
\newcommand{\n}{\mathbb{N}}
\newcommand{\z}{\mathbb{Z}}
\newcommand{\rac}{\mathbb{Q}}
\newcommand{\real}{\mathbb{R}}
\newcommand{\complex}{\mathbb{C}}
\newcommand{\cp}[1]{\mathbb{#1}}
\newcommand{\ch}{\mbox{\textrm{car\,}}\nobreak}
\newcommand{\vesp}[1]{\vspace{ #1  cm}}
\newcommand{\compcent}[1]{\vcenter{\hbox{$#1\circ$}}}
\newcommand{\comp}{\mathbin{\mathchoice
{\compcent\scriptstyle}{\compcent\scriptstyle}
{\compcent\scriptscriptstyle}{\compcent\scriptscriptstyle}}}

\title{Rela\c{c}\~ao de Equival\^encia}
\author[\autor]{\autor}
\institute[\instituto]{\instituto}
\date{\today}

\begin{document}
    \begin{frame}
        \maketitle
    \end{frame}

    \logo{\includegraphics[scale=.1]{logo-MAT.png}\vspace*{8.5cm}}

    \begin{frame}
        \begin{definicao}
            Seja $A$ um conjunto n{\~a}o vazio \pause e $R\subseteq A \times A$. \pause Dizemos que $R$ \pause {\'e} uma \textbf{rela{\c c}{\~a}o de equival{\^e}ncia} se:\pause
            \begin{enumerate}[label={\roman*})]
                \item Para todo $x \in A$, \pause $(x,x) \in R$. \pause \textit{(Propriedade Reflexiva)}\pause
                \item Se $(x, y) \in R$, \pause ent\~ao $(y, x) \in R$. \pause \textit{(Propriedade Sim\'etrica)}\pause
                \item Se $(x, y) \in R$ \pause e $(y, z) \in R$, \pause ent\~ao $(x, z)\in R$. \pause \textit{(Propriedade Transitiva)}\pause
            \end{enumerate}
        \end{definicao}
    \end{frame}

    \begin{frame}
        \begin{observacoes}
    Seja $R$ uma rela{\c c}{\~a}o de equival{\^e}ncia em $A$, \pause isto \'e, $R \sub A \times A$.\pause
    \begin{enumerate}[label={\arabic*})]
        \item  Para dizermos que $(x, y) \in R$ \pause usaremos a nota{\c c}{\~a}o $x\equiv y\ (R)$, \pause que se l{\^e} ``$x$ \'e equivalente a $y$ m{\'o}dulo $R$", \pause ou ainda a nota{\c c}{\~a}o $xRy$ \pause, com o mesmo significado anterior.\pause
        \item Em alguns casos vamos utilizar a nota\c{c}\~ao $\sim$ \pause para representar a rela\c{c}\~ao $R$. \pause Nesse caso, escrevemos $x \sim y$ \pause para dizer que $(x, y) \in R$, \pause ou que, $xRy$.\pause
    \end{enumerate}
\end{observacoes}

\begin{definicao}
    Seja $A$ um conjunto n{\~a}o vazio \pause e $R\subseteq A \times A$. \pause Dizemos que $R$ {\'e} uma \textbf{rela{\c c}{\~a}o de equival{\^e}ncia} se:\pause
    \begin{enumerate}[label={\roman*})]
        \item Para todo $x \in A$, $xRx$. \textit{(Propriedade Reflexiva)}\pause
        \item Se $xRy$, ent\~ao $yRx$. \textit{(Propriedade Sim\'etrica)}\pause
        \item Se $xRy$ e $yRz$, ent\~ao $xRz$. \textit{(Propriedade Transitiva)}\pause
    \end{enumerate}
\end{definicao}
\end{frame}
\begin{frame}
\begin{definicao}
    Seja $R$ uma rela{\c c}{\~a}o de equival{\^e}ncia sobre um conjunto $A$. \pause Dado $b \in A$, \pause chamamos de \textbf{classe de equival{\^e}ncia \pause determinada por $b$ \pause m{\'o}dulo $R$}\pause, denotada por $\overline{b}$ \pause ou $C(b)$, \pause o subconjunto de $A$ \pause dado por\pause
    \begin{center}
        $\overline{b} =\pause C(b) = \pause \{x \in A \pause \mid (x,b) \in R\}\pause = \{x \in A \pause\mid xRb\}.\pause$
    \end{center}
\end{definicao}
    \end{frame}
\end{document}