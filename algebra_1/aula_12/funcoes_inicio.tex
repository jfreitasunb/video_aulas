%!TEX program = xelatex
% !TEX encoding = ISO-8859-1
\def\ano{2020}
\def\semestre{1}
\def\disciplina{\'Algebra 1}
\def\turma{C}
\def\autor{Jos\'e Ant\^onio O. Freitas}
\def\instituto{MAT-UnB}

\documentclass{beamer}
\usetheme{Madrid}
\usecolortheme{beaver}
% \mode<presentation>
\usepackage{caption}
\usepackage{textpos}
\usepackage{amssymb}
\usepackage{amsmath,amsfonts,amsthm,amstext}
\usepackage[brazil]{babel}
% \usepackage[latin1]{inputenc}
\usepackage{graphicx}
\graphicspath{{/home/jfreitas/GitHub_Repos/video_aulas/logo/}{D:/Dropbox/imagens-latex/}}
\usepackage{enumitem}
\usepackage{multicol}
\usepackage{answers}
\usepackage{tikz,ifthen}
\usetikzlibrary{lindenmayersystems}
\usetikzlibrary[shadings]
\newtheorem{definicao}{Defini\c{c}\~ao}[section]
\newtheorem{definicoes}{Defini\c{c}\~oes}[section]
\newtheorem{exemplo}{Exemplo}[section]
\newtheorem{exemplos}{Exemplos}[section]
\newtheorem{exercicio}{Exerc{\'\i}cio}
\newtheorem{observacao}{Observa{\c c}{\~a}o:}[section]
\newtheorem{observacoes}{Observa{\c c}{\~o}es:}[section]
\newtheorem*{solucao}{Solu{\c c}{\~a}o:}
\newtheorem{proposicao}{Proposi\c{c}\~ao}
\newtheorem{lema}{Lema}
\newtheorem{teorema}{Teorema}
\newtheorem{corolario}{Corol\'ario}
\newenvironment{prova}[1][Prova]{\noindent\textbf{#1:} }{\qedsymbol}%{\ \rule{0.5em}{0.5em}}
\newcommand{\nsub}{\varsubsetneq}
\newcommand{\vaz}{\emptyset}
\newcommand{\im}{{\rm Im\,}}
\newcommand{\sub}{\subseteq}
\newcommand{\n}{\mathbb{N}}
\newcommand{\z}{\mathbb{Z}}
\newcommand{\rac}{\mathbb{Q}}
\newcommand{\real}{\mathbb{R}}
\newcommand{\complex}{\mathbb{C}}
\newcommand{\cp}[1]{\mathbb{#1}}
\newcommand{\ch}{\mbox{\textrm{car\,}}\nobreak}
\newcommand{\vesp}[1]{\vspace{ #1  cm}}
\newcommand{\compcent}[1]{\vcenter{\hbox{$#1\circ$}}}
\newcommand{\comp}{\mathbin{\mathchoice
{\compcent\scriptstyle}{\compcent\scriptstyle}
{\compcent\scriptscriptstyle}{\compcent\scriptscriptstyle}}}

\title{Fun\c{c}\~oes}
\author[\autor]{\autor}
\institute[\instituto]{\instituto}
\date{\today}

\begin{document}
    \begin{frame}
        \maketitle
    \end{frame}

    \logo{\includegraphics[scale=.1]{logo-MAT.png}\vspace*{8.5cm}}

    \begin{frame}
        \begin{definicao}
            Uma \textbf{fun{\c c}{\~a}o} $f : A \to B$, de um conjunto $A$ em um conjunto $B$, {\'e} uma rela{\c c}{\~a}o que associa os elementos de $A$ com os elementos em $B$ satisfazendo as seguintes condi\c{c}\~oes:
            \begin{enumerate}[label={\roman*})]
                \item Para todo $x \in A$, existe $y \in B$ tal que $f(x) = y$.

                \item  Se $x \in A$ \'e tal que $f(x) = y_1$ e $f(x) = y_2$ com $y_1$, $y_2 \in B$, ent\~ao $y_1 = y_2$.
            \end{enumerate}
            Nesse caso $y$ \'e chamado de \textbf{imagem} de $x$ segundo $f$.
        \end{definicao}

        O conjunto $A$ {\'e} chamado de \textbf{dom{\'\i}nio} de $f$ e ser\'a denotado por $\dom(f)$. O conjunto $B$ {\'e} chamado de \textbf{contra-dom{\'\i}nio} de $f$. O conjunto
        \[
            \im(f) = \{f(x) \mid x \in A\} \sub B
        \]
        \'e chamado \textbf{imagem} de $f$.
    \end{frame}

    \begin{frame}
        \begin{exemplos}
            \begin{enumerate}
                \item[1)] Sejam $A = \{0,1,2,3\}$ e $B = \{4,5,6,7,8\}$. Quais das seguintes rela{\c c}{\~o}es s{\~a}o fun{\c c}{\~o}es?
                \begin{enumerate}[label={\alph*})]
                    \item $R_1 = \{(0,5),(1,6),(2,7)\}$
                    \item $R_2 = \{(0,4),(1,5),(1,6),(2,7),(3,8)\}$
                    \item $R_3 = \{(0,4),(1,5),(2,7),(3,8)\}$
                    \item $R_4 = \{(0,5),(1,5),(2,6),(3,7)\}$
                \end{enumerate}
                \begin{solucao}
                    \begin{enumerate}[label={\alph*})]
                        \item N\~ao \'e fun\c{c}\~ao pois $3 \in A$ e $3$ n\~ao est\'a associado {\`a} nenhum elemento de $B$.
                        \item N\~ao \'e fun\c{c}\~ao pois $1 \in A$ est\'a associado a dois elementos diferentes em $B$.
                        \item \'E uma fun\c{c}\~ao.
                        \item \'E uma fun\c{c}\~ao.
                    \end{enumerate}
                \end{solucao}
            \end{enumerate}
        \end{exemplos}
    \end{frame}

    \begin{frame}
        \begin{exemplos}
            \begin{enumerate}
                \item[2)] $R_{5} = \{(x,y) \in \real  \times \real  \mid y^2 = x^2\}$
                \begin{solucao}
                    N\~ao \'e fun\c{c}\~ao pois, por exemplo, para $x = 1$ temos $y = -1$ ou $y = 1$.
                \end{solucao}
                \item[3)] $R_{6} = \{(x,y) \in \real  \times \real  \mid x^2 + y^2 = 1\}$.
                \begin{solucao}
                    N\~ao \'e fun\c{c}\~ao pois, por exemplo, para $x = 0$ temos $y = 1$ ou $y = -1$.
                \end{solucao}
                \item[4)] $R_{7} = \{(x,y) \in \real  \times \real \mid y = x^2\}$
                \begin{solucao}
                    \'E uma fun\c{c}\~ao.
                \end{solucao}
            \end{enumerate} 
        \end{exemplos}
    \end{frame}

    \begin{frame}
        \begin{definicao}
            Seja $f : A \to B$ uma fun\c{c}\~ao.
            \begin{enumerate}[label={\roman*})]
                \item Dizemos que $f$ \'e \textbf{injetora} se dados $x_1$, $x_2 \in A$ tais que $f(x_1) = f(x_2)$, ent\~ao $x_1 = x_2$. De modo equivalente, dizemos que $f$ e \textbf{injetora} se dados $x_1$, $x_2 \in A$ tais que $x_1 \ne x_2$, ent\~ao $f(x_1) \ne f(x_2)$.
                \item Dizemos que $f$ \'e \textbf{sobrejetora} se para todo $y \in B$, existe $x \in A$ tal que $f(x) = y$.
                \item Dizemos que $f$ e \textbf{bijetora} se $f$ for \textbf{injetora} e \textbf{sobrejetora} simultaneamente.
            \end{enumerate}
        \end{definicao}
    \end{frame}

    \begin{frame}
        \begin{exemplos}
            Verifique se as seguintes fun\c{c}\~oes s\~ao injetoras ou sobrejetoras:
            \begin{enumerate}
                \item[1)] $f : \z \to \z$ dada por $f(x) = 3x + 1$
            \end{enumerate}
        \end{exemplos}

        \vspace{5cm}
    \end{frame}

    \begin{frame}
        \vspace{5cm}
    \end{frame}

    \begin{frame}
        \begin{exemplos}
            \begin{enumerate}
                \item[2)] $g : \rac \to \rac$ dada por $f(x) = 3x + 1$
            \end{enumerate}
        \end{exemplos}

        \vspace{5cm}
    \end{frame}

    \begin{frame}
        \vspace{5cm}
    \end{frame}

    \begin{frame}
        \begin{exemplos}
            \begin{enumerate}
                \item[3)] A fun\c{c}\~ao $h :\real \to \real$ dada por $h(x) = x^2$
            \end{enumerate}
        \end{exemplos}

        \vspace{5cm}
    \end{frame}

    \begin{frame}
        \vspace{5cm}
    \end{frame}

    \begin{frame}
        \begin{definicao}
            Sejam $f : A \to B$ e $g : B \to C$ fun\c{c}\~oes. Definimos a \textbf{fun\c{c}\~ao composta} de $g$ com $f$ como sendo a fun\c{c}\~ao denotada por $g \circ f : A \to C$ tal que $(g\circ f)(x) = g(f(x))$ para todo $x \in A$.
        \end{definicao}

        \vspace{5cm}
    \end{frame}

    \begin{frame}
        \begin{exemplos}
            \begin{enumerate}
                \item[1)] Sejam $f : \real \to \real$ e $g : \real \to \real$ dadas por $f(x) = x^2$ e $g(x) = x + 1$. Assim podemos definir $g \circ f$ e $f \circ g$ e:
            \end{enumerate}
        \end{exemplos}

        \vspace{5cm}
    \end{frame}
    
    \begin{frame}
        \vspace{5cm}    
    \end{frame}

    \begin{frame}
        \begin{exemplos}
            \begin{enumerate}
                \item[2)] $f : \real_- \to \real^*_+$ e $g : \real^*_+ \to \real$ dadas por $f(x) = x^2 + 1$ e $g(x) = \ln x$. Nesse caso s\'o podemos definir $g \circ f : \real_- \to \real$ e:
            \end{enumerate}
        \end{exemplos}

        \vspace{5cm}
    \end{frame}

    \begin{frame}
        \vspace{5cm}
    \end{frame}

    \begin{frame}
        \begin{proposicao}
    Se $f : A \to B$ e $g : B \to C$ s{\~a}o fun{\c c}{\~o}es injetoras, ent{\~a}o $g\circ f : A \to C$ {\'e} injetora.
\end{proposicao}
\begin{prova}
    Dados $x_1$, $x_2 \in A$ tais que $(g\circ f)(x_1) = (g\circ f)(x_2)$ queremos mostrar que $x_1 = x_2$. Temos:
    \begin{align*}
        (g\circ f)(x_1) &= (g\circ f)(x_2)\\
        g(f(x_1)) &= g(f(x_2)).
    \end{align*}
    Como por hip\'otese $g$ \'e injetora, dessa \'ultima igualdade segue que $f(x_1) = f(x_2)$. Mas $f$ tamb\'em \'e injetora, por hip\'otese, da{\'\i} $x_1 = x_2$, como quer{\'\i}amos. Portanto $g\circ f$ \'e injetora.
\end{prova}
    \end{frame}
\end{document}