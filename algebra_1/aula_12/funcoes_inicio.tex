%!TEX program = xelatex
\def\ano{2020}
\def\semestre{1}
\def\disciplina{\'Algebra 1}
\def\turma{C}
\def\autor{Jos\'e Ant\^onio O. Freitas}
\def\instituto{MAT-UnB}

\documentclass{beamer}
\usetheme{Madrid}
\usecolortheme{beaver}
% \mode<presentation>
\usepackage{caption}
\usepackage{textpos}
\usepackage{amssymb}
\usepackage{amsmath,amsfonts,amsthm,amstext}
\usepackage[brazil]{babel}
% \usepackage[latin1]{inputenc}
\usepackage{graphicx}
\graphicspath{{/home/jfreitas/GitHub_Repos/video_aulas/logo/}{D:/Dropbox/imagens-latex/}}
\usepackage{enumitem}
\usepackage{multicol}
\usepackage{answers}
\usepackage{tikz,ifthen}
\usetikzlibrary{lindenmayersystems}
\usetikzlibrary[shadings]
\newtheorem{definicao}{Defini\c{c}\~ao}[section]
\newtheorem{definicoes}{Defini\c{c}\~oes}[section]
\newtheorem{exemplo}{Exemplo}[section]
\newtheorem{exemplos}{Exemplos}[section]
\newtheorem{exercicio}{Exerc{\'\i}cio}
\newtheorem{observacao}{Observa{\c c}{\~a}o:}[section]
\newtheorem{observacoes}{Observa{\c c}{\~o}es:}[section]
\newtheorem*{solucao}{Solu{\c c}{\~a}o:}
\newtheorem{proposicao}{Proposi\c{c}\~ao}
\newtheorem{lema}{Lema}
\newtheorem{teorema}{Teorema}
\newtheorem{corolario}{Corol\'ario}
\newenvironment{prova}[1][Prova]{\noindent\textbf{#1:} }{\qedsymbol}%{\ \rule{0.5em}{0.5em}}
\newcommand{\nsub}{\varsubsetneq}
\newcommand{\vaz}{\emptyset}
\newcommand{\im}{{\rm Im\,}}
\newcommand{\sub}{\subseteq}
\newcommand{\n}{\mathbb{N}}
\newcommand{\z}{\mathbb{Z}}
\newcommand{\rac}{\mathbb{Q}}
\newcommand{\real}{\mathbb{R}}
\newcommand{\complex}{\mathbb{C}}
\newcommand{\cp}[1]{\mathbb{#1}}
\newcommand{\ch}{\mbox{\textrm{car\,}}\nobreak}
\newcommand{\vesp}[1]{\vspace{ #1  cm}}
\newcommand{\compcent}[1]{\vcenter{\hbox{$#1\circ$}}}
\newcommand{\comp}{\mathbin{\mathchoice
{\compcent\scriptstyle}{\compcent\scriptstyle}
{\compcent\scriptscriptstyle}{\compcent\scriptscriptstyle}}}

\title{Fun\c{c}\~oes}
\author[\autor]{\autor}
\institute[\instituto]{\instituto}
\date{}

\begin{document}
    \begin{frame}
        \maketitle
    \end{frame}

    \logo{\includegraphics[scale=.1]{logo-MAT.png}\vspace*{8.5cm}}

    \begin{frame}
        \begin{definicao}
            Uma \textbf{fun{\c c}{\~a}o} \pause $f : A \to B$, \pause de um conjunto $A$ \pause em um conjunto $B$, \pause {\'e} uma rela{\c c}{\~a}o que associa os elementos de $A$ \pause com os elementos em $B$ \pause satisfazendo as seguintes condi\c{c}\~oes:\pause
            \begin{enumerate}[label={\roman*})]
                \item Para todo $x \in A$, \pause existe $y \in B$ \pause tal que $f(x) = y$.\pause

                \vspace{.3cm}

                \item  Se $x \in A$ \pause \'e tal que $f(x) = y_1$ \pause e $f(x) = y_2$ \pause com $y_1$, \pause $y_2 \in B$, \pause ent\~ao $y_1 = y_2$.\pause
            \end{enumerate}
            Nesse caso $y$ \'e chamado de \textbf{imagem} \pause de $x$ segundo $f$.\pause
        \end{definicao}

        O conjunto $A$ {\'e} chamado de \textbf{dom{\'\i}nio} \pause de $f$ \pause e ser\'a denotado por $\dom(f)$. \pause O conjunto $B$ {\'e} chamado de \textbf{contra-dom{\'\i}nio} \pause de $f$. \pause O conjunto\pause
        \[
            \im(f) = \pause \{f(x) \pause \mid x \in A\} \pause \sub B\pause
        \]
        \'e chamado \textbf{imagem} de $f$.\pause
    \end{frame}

    \begin{frame}
        \begin{exemplos}
            \begin{enumerate}
                \item[1)] Sejam $A = \{0,1,2,3\}$ \pause e $B = \{4,5,6,7,8\}$. \pause Quais das seguintes rela{\c c}{\~o}es s{\~a}o fun{\c c}{\~o}es?\pause
                \begin{enumerate}[label={\alph*})]
                    \item $R_1 = \{(0,5),(1,6),(2,7)\}$\pause

                    \vspace{1cm}

                    \item $R_2 = \{(0,4),(1,5),(1,6),(2,7),(3,8)\}$\pause

                    \vspace{1cm}

                    \item $R_3 = \{(0,4),(1,5),(2,7),(3,8)\}$\pause

                    \vspace{1cm}

                    \item $R_4 = \{(0,5),(1,5),(2,6),(3,7)\}$\pause

                    \vspace{1cm}
                \end{enumerate}
            \end{enumerate}
        \end{exemplos}
    \end{frame}

    \begin{frame}
        \begin{exemplos}
            \begin{enumerate}
                \item[2)] $R_{5} = \{(x,y) \in \real  \times \real  \mid y^2 = x^2\}$\pause

                \vspace{1cm}

                \item[3)] $R_{6} = \{(x,y) \in \real  \times \real  \mid x^2 + y^2 = 1\}$\pause

                \vspace{1cm}

                \item[4)] $R_{7} = \{(x,y) \in \real  \times \real \mid y = x^2\}$\pause

                \vspace{1.5cm}
            \end{enumerate}
        \end{exemplos}
    \end{frame}

    \begin{frame}
        \begin{definicao}
            Seja $f : A \to B$ uma fun\c{c}\~ao.\pause
            \begin{enumerate}[label={\roman*})]
                \item Dizemos que $f$ \'e \textbf{injetora} \pause se dados $x_1$, \pause $x_2 \in A$ \pause tais que $f(x_1) = \pause f(x_2)$, \pause ent\~ao $x_1 = x_2$. \pause De modo equivalente, \pause dizemos que $f$ \'e \textbf{injetora} \pause se dados $x_1$, \pause $x_2 \in A$ \pause tais que $x_1 \ne x_2$, \pause ent\~ao $f(x_1) \ne f(x_2)$.\pause

                \vspace{.3cm}

                \item Dizemos que $f$ \'e \textbf{sobrejetora} \pause se para todo $y \in B$, \pause existe $x \in A$ \pause tal que $f(x) = y$.\pause

                \vspace{.3cm}

                \item Dizemos que $f$ e \textbf{bijetora} \pause se $f$ for \textbf{injetora} \pause e \textbf{sobrejetora} \pause simultaneamente.\pause

                \vspace{.3cm}
            \end{enumerate}
        \end{definicao}
    \end{frame}

    \begin{frame}
        \begin{exemplos}
            Verifique se as seguintes fun\c{c}\~oes s\~ao injetoras \pause ou sobrejetoras:\pause
            \begin{enumerate}
                \item[1)] $f : \z \to \z$ dada por $f(x) = 3x + 1$
            \end{enumerate}
        \end{exemplos}

        \vspace{5cm}
    \end{frame}

    \begin{frame}
        \vspace{5cm}
    \end{frame}

    \begin{frame}
        \begin{exemplos}
            \begin{enumerate}
                \item[2)] $g : \z_5 \times \z_9 \to \z_5 \times \z_9$ tal que $f(\overline{x},\overline{y}) = (\overline{2} \overline{x} + \overline{3}, \overline{4}\overline{y} + \overline{5})$
            \end{enumerate}
        \end{exemplos}

        \vspace{5cm}
    \end{frame}

    \begin{frame}
        \vspace{5cm}
    \end{frame}

    \begin{frame}
        \begin{exemplos}
            \begin{enumerate}
                \item[3)] $h :\real \to \real$ dada por $h(x) = x^2$
            \end{enumerate}
        \end{exemplos}

        \vspace{5cm}
    \end{frame}

    \begin{frame}
        \vspace{5cm}
    \end{frame}
\end{document}
