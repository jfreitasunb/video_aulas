%!TEX program = xelatex
% !TEX encoding = ISO-8859-1
\def\ano{2020}
\def\semestre{1}
\def\disciplina{\'Algebra 1}
\def\turma{C}
\def\autor{Jos\'e Ant\^onio O. Freitas}
\def\instituto{MAT-UnB}

\documentclass{beamer}
\usetheme{Madrid}
\usecolortheme{beaver}
% \mode<presentation>
\usepackage{caption}
\usepackage{textpos}
\usepackage{amssymb}
\usepackage{amsmath,amsfonts,amsthm,amstext}
\usepackage[brazil]{babel}
% \usepackage[latin1]{inputenc}
\usepackage{graphicx}
\graphicspath{{/home/jfreitas/GitHub_Repos/video_aulas/logo/}{D:/Dropbox/imagens-latex/}}
\usepackage{enumitem}
\usepackage{multicol}
\usepackage{answers}
\usepackage{tikz,ifthen}
\usetikzlibrary{lindenmayersystems}
\usetikzlibrary[shadings]
\newtheorem{definicao}{Defini\c{c}\~ao}[section]
\newtheorem{definicoes}{Defini\c{c}\~oes}[section]
\newtheorem{exemplo}{Exemplo}[section]
\newtheorem{exemplos}{Exemplos}[section]
\newtheorem{exercicio}{Exerc{\'\i}cio}
\newtheorem{observacao}{Observa{\c c}{\~a}o:}[section]
\newtheorem{observacoes}{Observa{\c c}{\~o}es:}[section]
\newtheorem*{solucao}{Solu{\c c}{\~a}o:}
\newtheorem{proposicao}{Proposi\c{c}\~ao}
\newtheorem{lema}{Lema}
\newtheorem{teorema}{Teorema}
\newtheorem{corolario}{Corol\'ario}
\newenvironment{prova}[1][Prova]{\noindent\textbf{#1:} }{\qedsymbol}%{\ \rule{0.5em}{0.5em}}
\newcommand{\nsub}{\varsubsetneq}
\newcommand{\vaz}{\emptyset}
\newcommand{\im}{{\rm Im\,}}
\newcommand{\sub}{\subseteq}
\newcommand{\n}{\mathbb{N}}
\newcommand{\z}{\mathbb{Z}}
\newcommand{\rac}{\mathbb{Q}}
\newcommand{\real}{\mathbb{R}}
\newcommand{\complex}{\mathbb{C}}
\newcommand{\cp}[1]{\mathbb{#1}}
\newcommand{\ch}{\mbox{\textrm{car\,}}\nobreak}
\newcommand{\vesp}[1]{\vspace{ #1  cm}}
\newcommand{\compcent}[1]{\vcenter{\hbox{$#1\circ$}}}
\newcommand{\comp}{\mathbin{\mathchoice
{\compcent\scriptstyle}{\compcent\scriptstyle}
{\compcent\scriptscriptstyle}{\compcent\scriptscriptstyle}}}

\title{Diferen\c{c}a de Conjuntos}
\author[\autor]{\autor}
\institute[\instituto]{\instituto}
\date{\today}

\begin{document}
    \begin{frame}
        \maketitle
    \end{frame}

    \logo{\includegraphics[scale=.1]{logo-MAT.png}\vspace*{8.5cm}}

    \begin{frame}
        \begin{definicao}
            Dados dois conjuntos $A$ e $B$, \pause definimos a \textbf{diferen{\c c}a} \pause dos conjuntos $A$ e $B$, denotada por \pause $A - B$ ou $A \backslash B$ \pause como sendo o conjunto\pause
            \[
                A - B = \pause \{x \mid x \in A \pause \mbox{ e } x \notin B\}.\pause
            \]
        \end{definicao}

        \begin{exemplos}
            \begin{enumerate}[label={\arabic*})]
                \item Se $A=\{1,2,3,5,4\}$, \pause $B=\{2,3,6,8\}$, \pause ent\~ao
                \begin{center}
                    $A - B = \pause \{1,4,5\}$\pause\\
                    $B - A = \pause \{6,8\}$.\pause
                \end{center}
                \item Se $A=\{2,4,6,8,10,...\}$, \pause $B=\{3,6,9,12,15,...\}$ \pause, ent\~ao\pause
                \begin{center}
                    $A - B = \pause \{2,4,8,10,14,16,...\}\pause$\\
                    $B - A = \pause \{3,9,15,21,...\}$
                 \end{center}
            \end{enumerate}
        \end{exemplos}
    \end{frame}

    \begin{frame}
        \begin{proposicao}
            Sejam $A$, $B$ e $C$ \pause conjuntos n\~ao vazios. Ent\~ao\pause
            \[
                (A \cup B) - C = (A - C) \cup (B - C).\pause
            \]
        \end{proposicao}
        \textit{Prova: }\pause
        Precisamos mostrar que\pause
        \begin{enumerate}[label={\arabic*})]
            \item $(A \cup B) - C \sub (A - C) \cup (B - C)$\pause
            \item $(A - C) \cup (B - C) \sub  (A \cup B) - C$\pause
        \end{enumerate}
        Para a primeira inclus\~ao \pause seja $x \in (A \cup B) - C$. \pause Assim por defini\c{c}\~ao, \pause $x \in A \cup B$ \pause e $x \notin C$. \pause De $x \in A \cup B$, \pause ent\~ao $x \in A$ ou $x \in B$.\pause
        
        Se $x \in A$, \pause como $x \notin C$ \pause segue ent\~ao que $x \in A - C$. \pause Logo $x \in (A - C) \cup (B - C)$.\pause

        Se $x \in B$, \pause como $x \notin C$ \pause segue ent\~ao que $x \in B - C$. \pause Logo $x \in (A - C) \cup (B - C)$.\pause
    \end{frame}

    \begin{frame}
        Assim $(A \cup B) - C \sub (A - C) \cup (B - C)$.\pause
        
        Agora, para a segunda inclus\~ao, \pause seja $y \in (A - C) \cup (B - C)$. \pause Por defini\c{c}\~ao, \pause $y \in A - C$ ou $y \in B - C$.\pause

        Se $y \in A - C$, \pause ent\~ao $y \in A$ e $y \notin C$. \pause Como $y \in A$, \pause segue que $y \in A \cup B$. \pause Mas $y \notin C$, \pause com isso, $y \in (A \cup B) - C$.\pause

        Se $y \in B - C$, \pause ent\~ao $y \in B$ \pause e $y \notin C$. \pause Como $y \in B$, \pause segue que $y \in A \cup B$. \pause Mas $y \notin C$, \pause com isso, $y \in (A \cup B) - C$.\pause
        
        Assim $(A - C) \cup (B - C) \sub (A \cup B) - C$.\pause

        Portanto, \pause $(A \cup B) - C = (A - C) \cup (B - C)$, \pause como quer{\'\i}amos.\qedsymbol
    \end{frame}
\end{document}