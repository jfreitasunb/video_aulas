%!TEX program = xelatex
\def\ano{2020}
\def\semestre{1}
\def\disciplina{\'Algebra 1}
\def\turma{C}
\def\autor{Jos\'e Ant\^onio O. Freitas}
\def\instituto{MAT-UnB}

\documentclass{beamer}
\usetheme{Madrid}
\usecolortheme{beaver}
% \mode<presentation>
\usepackage{caption}
\usepackage{textpos}
\usepackage{amssymb}
\usepackage{amsmath,amsfonts,amsthm,amstext}
\usepackage[brazil]{babel}
% \usepackage[latin1]{inputenc}
\usepackage{graphicx}
\graphicspath{{/home/jfreitas/GitHub_Repos/video_aulas/logo/}{D:/Dropbox/imagens-latex/}}
\usepackage{enumitem}
\usepackage{multicol}
\usepackage{answers}
\usepackage{tikz,ifthen}
\usetikzlibrary{lindenmayersystems}
\usetikzlibrary[shadings]
\newtheorem{definicao}{Defini\c{c}\~ao}[section]
\newtheorem{definicoes}{Defini\c{c}\~oes}[section]
\newtheorem{exemplo}{Exemplo}[section]
\newtheorem{exemplos}{Exemplos}[section]
\newtheorem{exercicio}{Exerc{\'\i}cio}
\newtheorem{observacao}{Observa{\c c}{\~a}o:}[section]
\newtheorem{observacoes}{Observa{\c c}{\~o}es:}[section]
\newtheorem*{solucao}{Solu{\c c}{\~a}o:}
\newtheorem{proposicao}{Proposi\c{c}\~ao}
\newtheorem{lema}{Lema}
\newtheorem{teorema}{Teorema}
\newtheorem{corolario}{Corol\'ario}
\newenvironment{prova}[1][Prova]{\noindent\textbf{#1:} }{\qedsymbol}%{\ \rule{0.5em}{0.5em}}
\newcommand{\nsub}{\varsubsetneq}
\newcommand{\vaz}{\emptyset}
\newcommand{\im}{{\rm Im\,}}
\newcommand{\sub}{\subseteq}
\newcommand{\n}{\mathbb{N}}
\newcommand{\z}{\mathbb{Z}}
\newcommand{\rac}{\mathbb{Q}}
\newcommand{\real}{\mathbb{R}}
\newcommand{\complex}{\mathbb{C}}
\newcommand{\cp}[1]{\mathbb{#1}}
\newcommand{\ch}{\mbox{\textrm{car\,}}\nobreak}
\newcommand{\vesp}[1]{\vspace{ #1  cm}}
\newcommand{\compcent}[1]{\vcenter{\hbox{$#1\circ$}}}
\newcommand{\comp}{\mathbin{\mathchoice
{\compcent\scriptstyle}{\compcent\scriptstyle}
{\compcent\scriptscriptstyle}{\compcent\scriptscriptstyle}}}

\title{Uni\~ao e Interse\c{c}\~ao de Conjuntos}
\author[\autor]{\autor}
\institute[\instituto]{\instituto}
\date{}

\begin{document}
    \begin{frame}
        \maketitle
    \end{frame}

    \logo{\includegraphics[scale=.1]{logo-MAT.png}\vspace*{8.5cm}}

    \begin{frame}
        \begin{definicao}
            Sejam $A$ e $B$ dois conjuntos. Definimos a \textbf{interse{\c c}{\~a}o} de $A$ e $B$ \pause como sendo o conjunto $A \cap B$ \pause cujos elementos pertencem aos conjuntos $A$ e $B$ simultaneamente. \pause Assim,
            \[
                A \cap B = \{x \mid x \in A\mbox{ e }  x \in B\}.\pause
            \]
        \end{definicao}

        \begin{exemplo}
            Sejam $A = \{1, 2, 3\}$, \pause $B = \{2, 3, 4\}$ \pause e $C = \{r, s, t\}$. \pause Ent\~ao\pause
            \begin{center}
                $A \cap B \pause = \{2, 3\}$\pause\\
                $A \cap C \pause = \emptyset.$\pause
            \end{center}
        \end{exemplo}

    \end{frame}

    \begin{frame}
        \begin{definicao}
            Sejam $A$ e $B$ dois conjuntos. \pause Definimos a \textbf{uni{\~a}o} de $A$ com $B$ \pause como sendo o conjunto $A \cup B$, \pause cujos elementos pertencem ao conjunto $A$ ou ao conjunto $B$. \pause Assim,\pause
            \[
                A \cup B = \{x \mid x \in A \mbox{ ou } x \in B\}.\pause
            \]
        \end{definicao}

        \begin{exemplo}
            Sejam $A = \{1, 2, 3\}$, \pause $B = \{2, 3, 4\}$ \pause e $C = \{r, s, t\}$. \pause Ent\~ao\pause
            \begin{center}
                $A \cup B \pause = \{1,2,3,4\}$\pause\\
                $A \cup C \pause = \{1,2,3,r,s,t\}.$\pause
            \end{center}
        \end{exemplo}
    \end{frame}

    \begin{frame}
        \begin{proposicao} Sejam $A$ e $B$ dois conjuntos. \pause Ent{\~a}o:\pause
            \begin{enumerate}[label={\roman*})]
                \item $(A \cap B) \subseteq A$;\pause
                \item $(A \cap B) \subseteq B$;\pause
                \item $A \subseteq A \cup B$;\pause
                \item $B \subseteq A \cup B$.\pause
            \end{enumerate}
        \end{proposicao}
        \begin{prova}
            Para provar a primeira afirma\c{c}\~ao seja $x \in A \cap B$ um elemento qualquer. \pause Da defini\c{c}\~ao de interse\c{c}\~ao de conjuntos \pause temos $x \in A$ e $x \in B$. \pause Assim podemos afirmar com certeza que $x \in A$. \pause Logo todo elemento de $A \cap B$ tamb\'em est\'a em $A$, \pause ou seja, $A \cap B \subseteq A$. \pause De modo an\'alogo prova-se a segunda afirma\c{c}\~ao sobre a interse\c{c}\~ao.\pause

            Para a terceira afirma\c{c}\~ao, seja $x \in A$. \pause Da defini\c{c}\~ao de uni\~ao de conjuntos \pause segue que $x \in A \cup B$. \pause Logo todo elemento de $A$ tamb\'em est\'a em $A \cup B$, \pause ou seja, $A \subseteq (A \cup B)$. \pause De modo an\'alogo prova-se a quarta afirma\c{c}\~ao.\pause
        \end{prova}
    \end{frame}
\end{document}
