%!TEX program = xelatex
% !TEX encoding = ISO-8859-1
\def\ano{2020}
\def\semestre{1}
\def\disciplina{\'Algebra 1}
\def\turma{C}
\def\autor{Jos\'e Ant\^onio O. Freitas}
\def\instituto{MAT-UnB}

\documentclass{beamer}
\usetheme{Madrid}
\usecolortheme{beaver}
% \mode<presentation>
\usepackage{caption}
\usepackage{textpos}
\usepackage{amssymb}
\usepackage{amsmath,amsfonts,amsthm,amstext}
\usepackage[brazil]{babel}
% \usepackage[latin1]{inputenc}
\usepackage{graphicx}
\graphicspath{{/home/jfreitas/GitHub_Repos/video_aulas/logo/}{D:/Dropbox/imagens-latex/}}
\usepackage{enumitem}
\usepackage{multicol}
\usepackage{answers}
\usepackage{tikz,ifthen}
\usetikzlibrary{lindenmayersystems}
\usetikzlibrary[shadings]
\newtheorem{definicao}{Defini\c{c}\~ao}[section]
\newtheorem{definicoes}{Defini\c{c}\~oes}[section]
\newtheorem{exemplo}{Exemplo}[section]
\newtheorem{exemplos}{Exemplos}[section]
\newtheorem{exercicio}{Exerc{\'\i}cio}
\newtheorem{observacao}{Observa{\c c}{\~a}o:}[section]
\newtheorem{observacoes}{Observa{\c c}{\~o}es:}[section]
\newtheorem*{solucao}{Solu{\c c}{\~a}o:}
\newtheorem{proposicao}{Proposi\c{c}\~ao}
\newtheorem{lema}{Lema}
\newtheorem{teorema}{Teorema}
\newtheorem{corolario}{Corol\'ario}
\newenvironment{prova}[1][Prova]{\noindent\textbf{#1:} }{\qedsymbol}%{\ \rule{0.5em}{0.5em}}
\newcommand{\nsub}{\varsubsetneq}
\newcommand{\vaz}{\emptyset}
\newcommand{\im}{{\rm Im\,}}
\newcommand{\sub}{\subseteq}
\newcommand{\n}{\mathbb{N}}
\newcommand{\z}{\mathbb{Z}}
\newcommand{\rac}{\mathbb{Q}}
\newcommand{\real}{\mathbb{R}}
\newcommand{\complex}{\mathbb{C}}
\newcommand{\cp}[1]{\mathbb{#1}}
\newcommand{\ch}{\mbox{\textrm{car\,}}\nobreak}
\newcommand{\vesp}[1]{\vspace{ #1  cm}}
\newcommand{\compcent}[1]{\vcenter{\hbox{$#1\circ$}}}
\newcommand{\comp}{\mathbin{\mathchoice
{\compcent\scriptstyle}{\compcent\scriptstyle}
{\compcent\scriptscriptstyle}{\compcent\scriptscriptstyle}}}

\title{Teoria de Conjuntos}
\author[\autor]{\autor}
\institute[\instituto]{\instituto}
\date{\today}

\begin{document}
    \begin{frame}
        \maketitle
    \end{frame}

    \logo{\includegraphics[scale=.1]{logo-MAT.png}\vspace*{8.5cm}}

    \begin{frame}
        Um conjunto {\'e} uma ``cole{\c c}{\~a}o'' ou ``fam{\'\i}lia'' de elementos.\pause

        Denotaremos os conjuntos por letras mai\'uscula e os elementos de um dado conjunto por letras min{\'u}sculas.\pause

        Seja $A$ um conjunto, para indicar que $x$ {\'e} um elemento de $A$, escrevemos:
        \[
            x \in A.\pause
        \]

        Para dizer que um elemento $x$ n{\~a}o pertence ao conjunto $A$, escrevemos:
        \[
            x \notin A.\pause
        \]

        Um conjunto sem elementos {\'e} chamado de \textbf{conjunto vazio} e {\'e} denotado por $\emptyset$.\pause

        Dado um conjunto $A$ e $x$ um elemento, temos:\pause
        \[
            x \in A\pause \mbox{ ou } x \notin A.\pause
        \]

        Al{\'e}m disso, para dois elementos $x$, $y \in A$, sempre ocorre:\pause
        \[
            x = y\pause \mbox{ ou } x \neq y
        \]
    \end{frame}

    \begin{frame}
        Um conjunto $A$ pode ser dado pela simples listagem dos seus elementos, entre chaves:\pause
            \begin{center}
                $A = \{1,2,3,4,5\}$\pause\\
                $B = \{verdade, falso\}.$\pause
            \end{center}
            

        Ou pela descri{\c c}{\~a}o das propriedades dos seus elementos, tamb\'em  entre chaves:\pause
        \begin{center}
            $A = \{n \mid n \mbox{ \'e m{\'u}ltiplo de } 2\} = \{2,4,6,...\}.$
        \end{center}

        \begin{enumerate}[label={\arabic*})]
            \item $\n = \{0,1,2,3,...\}$ o conjunto do n{\'u}meros naturais.\pause
            \item $\n_0 = \{0,1,2,3,...\}$ o conjunto dos n{\'u}meros inteiros n{\~a}o negativos.\pause
            \item $\z = \{...,-2,-1,0,1,2,...\}$ o conjunto dos n{\'u}meros inteiros.\pause
            \item $\rac = \left\{\dfrac{p}{q} \mid p,q \in \z, q \neq 0 \right\}$ o conjunto dos n{\'u}meros racionais.\pause
            \item $\real $ o conjunto dos n{\'u}meros reais.\pause
            \item $\complex = \{a + bi \mid a,b \in \real,\ i^2 = -1\}$ o conjunto dos n\'umeros complexos.
        \end{enumerate}
    \end{frame}

    \begin{frame}
        \begin{definicao}
            Dados dois conjuntos $A$ e $B$, \pause dizemos que $A$ e $B$ s{\~a}o \textbf{iguais} \pause se, e somente se, eles t{\^e}m os mesmos elementos. \pause Ou seja, para todo $x \in A$ tamb\'em vale que $x \in B$ \pause e para todo $y \in B$ tamb\'em vale que $y \in A$. \pause Se $A$ e $B$ s{\~a}o iguais, \pause escrevemos $A = B$.\pause
        \end{definicao}

        \begin{exemplo}
            Sejam $A = \{1,1,2,3,4,4\}$, \pause $B = \{3,2,1,4\}$, \pause $C = \{1,2,3\}$ \pause e $D = \{2,3\}$. \pause Ent\~ao temos $A = B$. \pause Agora como $1 \in C$ e $1 \notin D$ ent\~ao $C \ne D$.\pause
        \end{exemplo}
            
        \begin{definicao}
            Se $A$ e $B$ s{\~a}o dois conjuntos, \pause dizemos que $A$ {\'e} um \textbf{subconjunto} de $B$ \pause ou que $A$ \textbf{est\'a contido} em $B$ \pause ou que $B$ \textbf{cont\'em} $A$ \pause se todo elemento de $A$ for elemento de $B$. \pause Ou seja, se para todo elemento $x \in A$, \pause temos $x \in B$. \pause Nesse caso, escrevemos $A \subseteq B$ (ou $A \subset B$) \pause ou $B \supseteq A$ (ou $B \supset A$).\pause
        \end{definicao}
    \end{frame}

    \begin{frame}
        \begin{exemplos}
            Sejam $A = \{1,2,3,x,y,z\}$, \pause $B = \{x, y\}$ \pause e $C = \{x, y , z\}$.\pause
            \begin{enumerate}[label={\arabic*})]
                \item $A \nsubseteq B$ \pause pois $1 \in A$ e $1 \notin B$.\pause
                \item $B \subsetneq A$\pause
                \item $B \subseteq C$\pause
                \item $C \subseteq A$\pause
            \end{enumerate}
        \end{exemplos}

        \begin{observacao}
            Dados dois conjuntos $A$ e $B$ \pause para que $A$ \textbf{n\~ao esteja contido em} $B$ basta \pause que exista $x \in A$ tal que $x \notin B$. \pause Nesse caso escrevemos $A \nsubseteq B$.
        \end{observacao}
    \end{frame}
    \begin{frame}
        Pela defini\c{c}\~ao de contin\^encia de conjuntos, podemos reescrever a igualdade de conjuntos, da seguinte forma: \pause dados dois conjuntos $A$ e $B$\pause
        \begin{center}
            $A = B$\quad \textbf{se, e somente se,} \pause\quad $A \subseteq B$ \quad\pause \textbf{e}\quad $B \subseteq A$.\pause
        \end{center}

        Ou seja,
        \begin{center}
            \textbf{se} $A = B$ \textbf{ent{\~a}o} $A \subseteq B$ \textbf{e} $B \subseteq A$.\pause
        \end{center}

        Al\'em disso,
        \begin{center}
            \textbf{se} $A \subseteq B$ \textbf{e} $B \subseteq A$, \textbf{ent{\~a}o} $A = B$.\pause
        \end{center}

        Quando $A$ e $B$ n{\~a}o s{\~a}o iguais, escrevemos $A \neq B$.\pause

        \begin{proposicao}
            Dados tr\^es conjuntos $A$, $B$ e $C$ temos:\pause
            \begin{enumerate}[label={\roman*})]
                \item $A\subseteq A$ (Reflexividade)\pause
                \item Se $A\subseteq B \mbox{ e } B\subseteq A$, ent{\~a}o $A=B$. (Antissimetria)\pause
                \item Se $A\subseteq B$ e $B\subseteq C$, ent{\~a}o $A\subseteq C$. (Transitividade)\pause
            \end{enumerate}
        \end{proposicao}
    \end{frame}

    \begin{frame}
        Considere os seguintes conjuntos:\pause
\begin{center}
    $A = \{ n \in \n \mid n \mbox{ {\'e} m{\'u}ltiplo de } 2\} = \{2,4,6,...\}$\pause\\
    $B = \{n \in \n \mid n \mbox{ {\'e} m{\'u}ltiplo de } 3\} = \{3,6,9,...\}.$\pause
\end{center}

Neste caso, $A \nsubseteq B$ \pause e $B \nsubseteq A$. \pause Portanto, dados dois conjuntos $A$ e $B$, nem sempre 
temos $A \subseteq B$ \pause ou $B \subseteq A$.\pause

\begin{proposicao} 
    Seja $A$ um conjunto. Ent{\~a}o $ \emptyset \subseteq A$.\pause
\end{proposicao}
\begin{prova}
    Suponha que $\emptyset \nsubseteq A$. \pause Logo existe $x \in \emptyset$ tal que $x \notin A$. \pause Mas por defini{\c c}{\~a}o, o conjunto vazio n{\~a}o cont{\'e}m elementos. \pause Logo a exist\^encia de $x \in \emptyset$ {\'e} uma contradi{\c c}{\~a}o. \pause Tal contradi\c{c}\~ao surgiu por termos suposto que $\emptyset \nsubseteq A$. \pause Portanto, $\emptyset \subseteq A$, como quer{\'\i}amos demonstrar.
\end{prova}
    \end{frame}
\end{document}