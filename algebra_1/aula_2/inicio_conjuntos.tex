%!TEX program = xelatex
% !TEX encoding = ISO-8859-1
\def\ano{2020}
\def\semestre{1}
\def\disciplina{\'Algebra 1}
\def\turma{C}
\def\autor{Jos\'e Ant\^onio O. Freitas}
\def\instituto{MAT-UnB}

\documentclass{beamer}
\usetheme{Madrid}
\usecolortheme{beaver}
% \mode<presentation>
\usepackage{caption}
\usepackage{textpos}
\usepackage{amssymb}
\usepackage{amsmath,amsfonts,amsthm,amstext}
\usepackage[brazil]{babel}
% \usepackage[latin1]{inputenc}
\usepackage{graphicx}
\graphicspath{{/home/jfreitas/GitHub_Repos/video_aulas/logo/}{D:/Dropbox/imagens-latex/}}
\usepackage{enumitem}
\usepackage{multicol}
\usepackage{answers}
\usepackage{tikz,ifthen}
\usetikzlibrary{lindenmayersystems}
\usetikzlibrary[shadings]
\newtheorem{definicao}{Defini\c{c}\~ao}[section]
\newtheorem{definicoes}{Defini\c{c}\~oes}[section]
\newtheorem{exemplo}{Exemplo}[section]
\newtheorem{exemplos}{Exemplos}[section]
\newtheorem{exercicio}{Exerc{\'\i}cio}
\newtheorem{observacao}{Observa{\c c}{\~a}o:}[section]
\newtheorem{observacoes}{Observa{\c c}{\~o}es:}[section]
\newtheorem*{solucao}{Solu{\c c}{\~a}o:}
\newtheorem{proposicao}{Proposi\c{c}\~ao}
\newtheorem{lema}{Lema}
\newtheorem{teorema}{Teorema}
\newtheorem{corolario}{Corol\'ario}
\newenvironment{prova}[1][Prova]{\noindent\textbf{#1:} }{\qedsymbol}%{\ \rule{0.5em}{0.5em}}
\newcommand{\nsub}{\varsubsetneq}
\newcommand{\vaz}{\emptyset}
\newcommand{\im}{{\rm Im\,}}
\newcommand{\sub}{\subseteq}
\newcommand{\n}{\mathbb{N}}
\newcommand{\z}{\mathbb{Z}}
\newcommand{\rac}{\mathbb{Q}}
\newcommand{\real}{\mathbb{R}}
\newcommand{\complex}{\mathbb{C}}
\newcommand{\cp}[1]{\mathbb{#1}}
\newcommand{\ch}{\mbox{\textrm{car\,}}\nobreak}
\newcommand{\vesp}[1]{\vspace{ #1  cm}}
\newcommand{\compcent}[1]{\vcenter{\hbox{$#1\circ$}}}
\newcommand{\comp}{\mathbin{\mathchoice
{\compcent\scriptstyle}{\compcent\scriptstyle}
{\compcent\scriptscriptstyle}{\compcent\scriptscriptstyle}}}

\title{Teoria de Conjuntos}
\author[\autor]{\autor}
\institute[\instituto]{\instituto}
\date{\today}

\begin{document}
    \begin{frame}
        \maketitle
    \end{frame}

    \logo{\includegraphics[scale=.1]{logo-MAT.png}\vspace*{8.5cm}}

    \begin{frame}
        Um conjunto {\'e} uma ``cole{\c c}{\~a}o'' ou ``fam{\'\i}lia'' de elementos.\pause

        Denotaremos os conjuntos por letras maiúscula e os elementos de um dado conjunto por letras min{\'u}sculas.\pause

        Seja $A$ um conjunto, para indicar que $x$ {\'e} um elemento de $A$, escrevemos:
        \[
            x \in A.\pause
        \]

        Para dizer que um elemento $x$ n{\~a}o pertence ao conjunto $A$, escrevemos:
        \[
            x \notin A.\pause
        \]

        Um conjunto sem elementos {\'e} chamado de \textbf{conjunto vazio} e {\'e} denotado por $\emptyset$.\pause

        Dado um conjunto $A$ e $x$ um elemento, temos:\pause
        \[
            x \in A\pause \mbox{ ou } x \notin A.\pause
        \]

        Al{\'e}m disso, para dois elementos $x$, $y \in A$, sempre ocorre:\pause
        \[
            x = y\pause \mbox{ ou } x \neq y
        \]
    \end{frame}

    \begin{frame}
        Um conjunto $A$ pode ser dado pela simples listagem dos seus elementos, entre chaves:\pause
            \begin{center}
                $A = \{1,2,3,4,5\}$\pause\\
                $B = \{verdade, falso\}.$\pause
            \end{center}
            

        Ou pela descri{\c c}{\~a}o das propriedades dos seus elementos, também  entre chaves:\pause
        \begin{center}
            $A = \{n \mid n \mbox{ \'e m{\'u}ltiplo de } 2\} = \{2,4,6,...\}.$
        \end{center}

        \begin{itemize}
            \item $\n = \{0,1,2,3,...\}$ o conjunto do n{\'u}meros naturais.\pause
            \item $\n_0 = \{0,1,2,3,...\}$ o conjunto dos n{\'u}meros inteiros n{\~a}o negativos.\pause
            \item $\z = \{...,-2,-1,0,1,2,...\}$ o conjunto dos n{\'u}meros inteiros.\pause
            \item $\rac = \left\{\dfrac{p}{q} \mid p,q \in \z, q \neq 0 \right\}$ o conjunto dos n{\'u}meros racionais.\pause
            \item $\real $ o conjunto dos n{\'u}meros reais.\pause
            \item $\complex = \{a + bi \mid a,b \in \real,\ i^2 = -1\}$ o conjunto dos n\'umeros complexos.
        \end{itemize}
    \end{frame}

    \begin{frame}
        \begin{definicao}\label{igualdade_conjuntos}
            Dados dois conjuntos $A$ e $B$,\pause dizemos que $A$ e $B$ s{\~a}o \textbf{iguais}\quando se, e somente se, eles t{\^e}m os mesmos elementos.\pause Ou seja, para todo $x \in A$ também vale que $x \in B$\pause e para todo $y \in B$ também vale que $y \in A$.\pause Se $A$ e $B$ s{\~a}o iguais,\pause escrevemos $A = B$.\pause
        \end{definicao}

        \begin{exemplo}
            Sejam $A = \{1,1,2,3,4,4\}$, $B = \{3,2,1,4\}$, $C = \{1,2,3\}$ e $D = \{2,3\}$. Ent\~ao temos $A = B$. Agora como $1 \in C$ e $1 \notin D$ ent\~ao $C \ne D$.
        \end{exemplo}
            
        
    \end{frame}
\end{document}