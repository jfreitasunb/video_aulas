%!TEX program = xelatex
%!TEX encoding = ISO-8859-1
\def\ano{2020}
\def\semestre{1}
\def\disciplina{\'Algebra 1}
\def\turma{C}
\def\autor{Jos\'e Ant\^onio O. Freitas}
\def\instituto{MAT-UnB}

\documentclass{beamer}
\usetheme{Madrid}
\usecolortheme{beaver}
% \mode<presentation>
\usepackage{caption}
\usepackage{textpos}
\usepackage{amssymb}
\usepackage{amsmath,amsfonts,amsthm,amstext}
\usepackage[brazil]{babel}
% \usepackage[latin1]{inputenc}
\usepackage{graphicx}
\graphicspath{{/home/jfreitas/GitHub_Repos/video_aulas/logo/}{D:/Dropbox/imagens-latex/}}
\usepackage{enumitem}
\usepackage{multicol}
\usepackage{answers}
\usepackage{tikz,ifthen}
\usetikzlibrary{lindenmayersystems}
\usetikzlibrary[shadings]
\newtheorem{definicao}{Defini\c{c}\~ao}[section]
\newtheorem{definicoes}{Defini\c{c}\~oes}[section]
\newtheorem{exemplo}{Exemplo}[section]
\newtheorem{exemplos}{Exemplos}[section]
\newtheorem{exercicio}{Exerc{\'\i}cio}
\newtheorem{observacao}{Observa{\c c}{\~a}o:}[section]
\newtheorem{observacoes}{Observa{\c c}{\~o}es:}[section]
\newtheorem*{solucao}{Solu{\c c}{\~a}o:}
\newtheorem{proposicao}{Proposi\c{c}\~ao}
\newtheorem{lema}{Lema}
\newtheorem{teorema}{Teorema}
\newtheorem{corolario}{Corol\'ario}
\newenvironment{prova}[1][Prova]{\noindent\textbf{#1:} }{\qedsymbol}%{\ \rule{0.5em}{0.5em}}
\newcommand{\nsub}{\varsubsetneq}
\newcommand{\vaz}{\emptyset}
\newcommand{\im}{{\rm Im\,}}
\newcommand{\sub}{\subseteq}
\newcommand{\n}{\mathbb{N}}
\newcommand{\z}{\mathbb{Z}}
\newcommand{\rac}{\mathbb{Q}}
\newcommand{\real}{\mathbb{R}}
\newcommand{\complex}{\mathbb{C}}
\newcommand{\cp}[1]{\mathbb{#1}}
\newcommand{\ch}{\mbox{\textrm{car\,}}\nobreak}
\newcommand{\vesp}[1]{\vspace{ #1  cm}}
\newcommand{\compcent}[1]{\vcenter{\hbox{$#1\circ$}}}
\newcommand{\comp}{\mathbin{\mathchoice
{\compcent\scriptstyle}{\compcent\scriptstyle}
{\compcent\scriptscriptstyle}{\compcent\scriptscriptstyle}}}

\title{Grupo Quociente}
\author[\autor]{\autor}
\institute[\instituto]{\instituto}
\date{\today}

\begin{document}
    \begin{frame}
        \maketitle
    \end{frame}

    \logo{\includegraphics[scale=.1]{logo-MAT.png}\vspace*{8.5cm}}

    \begin{frame}
        Seja $N$ um subgrupo normal de um grupo $G$, onde $e$ denota o elemento neutro de $G$. Denote por
        \[
            G/N = \{aN \mid a \in G\}
        \]
        o conjunto das classes de equivalência determinadas por $N$.
        
        \vspace{.3cm}

        Defina em $G/N$ a operação
        \[
            (aN)(bN) = (ab)N
        \]
        para todos $a$, $b \in N$.
    \end{frame}

    \begin{frame}
        Temos:

        \vspace{.3cm}
        
        \begin{enumerate}[label=({\roman*})]
            \item $[(aN)(bN)](cN) = (an)[(bN)(cN)]$ para todos $aN$, $bN$, $cN \in G/N$;

            \vspace{.3cm}
        
            \item $(aN)(eN) = (ae)N = aN = (ea)N = (eN)(aN)$ para todo $aN \in G/N$;

            \vspace{.3cm}
        

            \item $(aN)(a^{-1}N) = (aa^{-1})N = eN = (a^{-1}a)N = (a^{-1}N)(aN)$ para todo $aN \in G/N$.

            \vspace{.3cm}
        \end{enumerate}

        Assim, o conjunto $G/N$ é um grupo com a multiplicação de conjuntos.

        \vspace{.3cm}
        
        Nesse grupo o elemento neutro é $eN$ e $(aN)^{-1} = (a^{-1})N$.
    \end{frame}

    \begin{frame}
        \begin{definicao}
            Sejam $G$ um grupo e $N$ um subgrupo normal de $G$. Nessas condições, o \textbf{grupo quociente} 
            de $G$ por $N$ é o par formado pelo conjunto quociente $G/N$ e da operação de multiplicação de conjuntos 
            aplicadas aos elementos desse conjunto.
        \end{definicao}    
    \end{frame}

    \begin{frame}
        \begin{exemplos}
            \begin{enumerate}[label=({\arabic*})]
                \item Seja $G = \{1, -1, i, -i\}$ um grupo e $N = \{1, -1\}$. 

                \seti
            \end{enumerate}
        \end{exemplos}
    \end{frame}

    \begin{frame}
        \begin{exemplos}
            \begin{enumerate}[label=({\arabic*})]
                \conti

                \item Seja $G = \z_6 = \{\overline{0}, \overline{1}, \overline{2}, \overline{3}, \overline{4}, \overline{5}\}$ e $H = \{\overline{0}, \overline{3}\}$.

                \seti
            \end{enumerate}
        \end{exemplos}
    \end{frame}

    \begin{frame}
        \begin{exemplos}
            \begin{enumerate}[label=({\arabic*})]
                \conti
                
                \item Seja $G = S_3$. Já vimos que se tomamos
                \[
                    Id = \begin{pmatrix}
                        1 & 2 & 1\\
                        1 & 2 & 3
                    \end{pmatrix},\quad
                    f = \begin{pmatrix}
                        1 & 2 & 1\\
                        2 & 3 & 1
                    \end{pmatrix} \quad \mbox{e}\quad
                    g = \begin{pmatrix}
                        1 & 2 & 1\\
                        1 & 3 & 2
                    \end{pmatrix}
                \]
                então
                \[
                    S_3 = \{Id, f, f^2, g, gf, gf^2\}.
                \]
                Considere o subgrupo $H = [\ f\ ] = \{Id, f, f^2\}$.

                \seti
            \end{enumerate}
        \end{exemplos}
    \end{frame}

    \begin{frame}
        \begin{proposicao}
            Se $N$ é um subgrupo normal de $G$, então a função $\mu : G \to G/N$ definida por $\mu(a) = aN$ é um homomorfismo sobrejetor de grupos tal que
            \[
                \ker(\mu) = N.
            \]
        \end{proposicao}
    \end{frame}

    \begin{frame}
        \begin{definicao}
            Se $N$ é um subgrupo normal de $G$, então o homomorfismo $\mu : G \to G/N$ definido por $\mu(a) = aN$ é chamado de \textbf{homomorfismo canônico} de $G$ sobre $G/N$.
        \end{definicao}
    \end{frame}

    \begin{frame}
        \begin{lema}
            Se $f : G \to L$ é um homomorfismo de grupos, então $N = \ker(f)$ é um subgrupo normal de $G$ e, portanto, $G/N$ é um grupo.
        \end{lema}
    \end{frame}

    \begin{frame}
        \begin{teorema}[Teorema do Homomorfismo para Grupos]
            Seja $f : G \to L$ um homomorfismo sobrejetor de grupos. Se $N = \ker(f)$, então o grupo quociente $G/N$ é isomorfo ao grupo $L$.
        \end{teorema}
    \end{frame}

    \begin{frame}
        \begin{exemplo}
            Dado um inteiro $m > 1$, considere o homomorfismo $\rho_m : \z \to \z_m$ definido por $\rho_m(x) = \overline{x}$.
        \end{exemplo}
    \end{frame}
    
\end{document}