%!TEX program = xelatex
\def\ano{2020}
\def\semestre{1}
\def\disciplina{\'Algebra 1}
\def\turma{C}
\def\autor{Jos\'e Ant\^onio O. Freitas}
\def\instituto{MAT-UnB}

\documentclass{beamer}
\usetheme{Madrid}
\usecolortheme{beaver}
% \mode<presentation>
\usepackage{caption}
\usepackage{textpos}
\usepackage{amssymb}
\usepackage{amsmath,amsfonts,amsthm,amstext}
\usepackage[brazil]{babel}
% \usepackage[latin1]{inputenc}
\usepackage{graphicx}
\graphicspath{{/home/jfreitas/GitHub_Repos/video_aulas/logo/}{D:/Dropbox/imagens-latex/}}
\usepackage{enumitem}
\usepackage{multicol}
\usepackage{answers}
\usepackage{tikz,ifthen}
\usetikzlibrary{lindenmayersystems}
\usetikzlibrary[shadings]
\newtheorem{definicao}{Defini\c{c}\~ao}[section]
\newtheorem{definicoes}{Defini\c{c}\~oes}[section]
\newtheorem{exemplo}{Exemplo}[section]
\newtheorem{exemplos}{Exemplos}[section]
\newtheorem{exercicio}{Exerc{\'\i}cio}
\newtheorem{observacao}{Observa{\c c}{\~a}o:}[section]
\newtheorem{observacoes}{Observa{\c c}{\~o}es:}[section]
\newtheorem*{solucao}{Solu{\c c}{\~a}o:}
\newtheorem{proposicao}{Proposi\c{c}\~ao}
\newtheorem{lema}{Lema}
\newtheorem{teorema}{Teorema}
\newtheorem{corolario}{Corol\'ario}
\newenvironment{prova}[1][Prova]{\noindent\textbf{#1:} }{\qedsymbol}%{\ \rule{0.5em}{0.5em}}
\newcommand{\nsub}{\varsubsetneq}
\newcommand{\vaz}{\emptyset}
\newcommand{\im}{{\rm Im\,}}
\newcommand{\sub}{\subseteq}
\newcommand{\n}{\mathbb{N}}
\newcommand{\z}{\mathbb{Z}}
\newcommand{\rac}{\mathbb{Q}}
\newcommand{\real}{\mathbb{R}}
\newcommand{\complex}{\mathbb{C}}
\newcommand{\cp}[1]{\mathbb{#1}}
\newcommand{\ch}{\mbox{\textrm{car\,}}\nobreak}
\newcommand{\vesp}[1]{\vspace{ #1  cm}}
\newcommand{\compcent}[1]{\vcenter{\hbox{$#1\circ$}}}
\newcommand{\comp}{\mathbin{\mathchoice
{\compcent\scriptstyle}{\compcent\scriptstyle}
{\compcent\scriptscriptstyle}{\compcent\scriptscriptstyle}}}

\title{Grupo Quociente}
\author[\autor]{\autor}
\institute[\instituto]{\instituto}
\date{}

\begin{document}
    \begin{frame}
        \maketitle
    \end{frame}

    \logo{\includegraphics[scale=.1]{logo-MAT.png}\vspace*{8.5cm}}

    \begin{frame}
        Seja $N$ um subgrupo normal \pause de um grupo $G$, onde $e$ denota o elemento neutro de $G$. \pause Denote por \pause
        \[
            G/N = \{aN \mid a \in G\} \pause
        \]
        o conjunto das classes de equival\^encia determinadas por $N$. \pause

        \vspace{.5cm}

        Defina em $G/N$ a opera\c{c}\~ao \pause
        \[
            (aN)(bN)  \pause = (ab)N \pause
        \]
        para todos $aN$, $bN \in G/N$.
    \end{frame}

    \begin{frame}
        Temos: \pause

        \vspace{.3cm}

        \begin{enumerate}[label={\roman*})]
            \item $[(aN)(bN)](cN) \pause = (an)[(bN)(cN)]$ \pause para todos $aN$, $bN$, $cN \in G/N$; \pause

            \vspace{.3cm}

            \item $(aN)(eN) \pause = (ae)N \pause = aN \pause = (ea)N \pause = (eN)(aN)$ \pause para todo $aN \in G/N$; \pause

            \vspace{.3cm}

            \item $(aN)(a^{-1}N) \pause = (aa^{-1})N \pause = eN \pause = (a^{-1}a)N \pause = (a^{-1}N)(aN)$ \pause para todo $aN \in G/N$. \pause

            \vspace{.3cm}
        \end{enumerate}

        Assim, o conjunto $G/N$ \'e um grupo com a multiplica\c{c}\~ao de conjuntos. \pause

        \vspace{.3cm}

        Nesse grupo o elemento neutro \'e $eN$ \pause e $(aN)^{-1} = (a^{-1})N$.
    \end{frame}

    \begin{frame}
        \begin{definicao}
            Sejam $G$ um grupo e $N$ um subgrupo normal de $G$. \pause Nessas condi\c{c}\~oes, \pause o \textbf{grupo quociente} \pause
            de $G$ por $N$ \pause \'e o par formado pelo conjunto quociente $G/N$ \pause e da opera\c{c}\~ao de multiplica\c{c}\~ao de conjuntos
            aplicadas aos elementos desse conjunto.
        \end{definicao}
    \end{frame}

    \begin{frame}
        \begin{exemplos}
            \begin{enumerate}[label=({\arabic*})]
                \item Seja $G = \{1, -1, i, -i\}$ um grupo \pause e $N = \{1, -1\}$.

                \seti
            \end{enumerate}
        \end{exemplos}
    \end{frame}

    \begin{frame}
        \begin{exemplos}
            \begin{enumerate}[label=({\arabic*})]
                \conti

                \item Seja $G = \z_6 = \{\overline{0}, \overline{1}, \overline{2}, \overline{3}, \overline{4}, \overline{5}\}$ \pause e $H = \{\overline{0}, \overline{3}\}$.

                \seti
            \end{enumerate}
        \end{exemplos}
    \end{frame}

    \begin{frame}
        \begin{exemplos}
            \begin{enumerate}[label=({\arabic*})]
                \conti

                \item Seja $G = S_3$. \pause J\'a vimos que se tomamos
                \[
                    Id = \begin{pmatrix}
                        1 & 2 & 3\\
                        1 & 2 & 3
                    \end{pmatrix}, \pause\quad
                    f = \begin{pmatrix}
                        1 & 2 & 3\\
                        2 & 3 & 1
                    \end{pmatrix} \pause \quad \mbox{e}\quad
                    g = \begin{pmatrix}
                        1 & 2 & 3\\
                        1 & 3 & 2
                    \end{pmatrix}
                \]
                ent\~ao
                \[
                    S_3 = \{Id, f, f^2, g, gf, gf^2\}. \pause
                \]
                Considere o subgrupo $H = [\ f\ ] \pause = \{Id, f, f^2\}$.

                \seti
            \end{enumerate}
        \end{exemplos}
    \end{frame}

    \begin{frame}
        \begin{proposicao}
            Se $N$ \'e um subgrupo normal de $G$, \pause ent\~ao a fun\c{c}\~ao $\mu : G \to G/N$ \pause definida por $\mu(a) = aN$ \pause \'e um homomorfismo sobrejetor \pause de grupos tal que \pause
            \[
                \ker(\mu) = N.
            \]
        \end{proposicao}
    \end{frame}

    \begin{frame}
        \begin{definicao}
            Se $N$ \'e um subgrupo normal de $G$, \pause ent\~ao o homomorfismo $\mu : G \to G/N$ \pause definido por $\mu(a) = aN$ \pause \'e chamado de \textbf{homomorfismo can\^onico} \pause de $G$ sobre $G/N$.
        \end{definicao}
    \end{frame}

    \begin{frame}
        \begin{lema}
            Se $f : G \to L$ \'e um homomorfismo de grupos, \pause ent\~ao $N = \ker(f)$ \pause \'e um subgrupo normal de $G$ \pause e, portanto, $G/N$ \'e um grupo.
        \end{lema}
    \end{frame}

    \begin{frame}
        \begin{teorema}[Teorema do Homomorfismo para Grupos]
            Seja $f : G \to L$ um homomorfismo sobrejetor \pause de grupos. \pause Se $N = \ker(f)$, \pause ent\~ao o grupo quociente $G/N$ \'e isomorfo ao grupo $L$.
        \end{teorema}
    \end{frame}

    \begin{frame}
        \begin{exemplo}
            Dado um inteiro $m > 1$, \pause considere o homomorfismo $\rho_m : \z \to \z_m$ \pause definido por $\rho_m(x) = \overline{x}$.
        \end{exemplo}
    \end{frame}

\end{document}
