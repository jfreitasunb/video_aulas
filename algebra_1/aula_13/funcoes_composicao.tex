%!TEX program = xelatex
\def\ano{2020}
\def\semestre{1}
\def\disciplina{\'Algebra 1}
\def\turma{C}
\def\autor{Jos\'e Ant\^onio O. Freitas}
\def\instituto{MAT-UnB}

\documentclass{beamer}
\usetheme{Madrid}
\usecolortheme{beaver}
% \mode<presentation>
\usepackage{caption}
\usepackage{textpos}
\usepackage{amssymb}
\usepackage{amsmath,amsfonts,amsthm,amstext}
\usepackage[brazil]{babel}
% \usepackage[latin1]{inputenc}
\usepackage{graphicx}
\graphicspath{{/home/jfreitas/GitHub_Repos/video_aulas/logo/}{D:/Dropbox/imagens-latex/}}
\usepackage{enumitem}
\usepackage{multicol}
\usepackage{answers}
\usepackage{tikz,ifthen}
\usetikzlibrary{lindenmayersystems}
\usetikzlibrary[shadings]
\newtheorem{definicao}{Defini\c{c}\~ao}[section]
\newtheorem{definicoes}{Defini\c{c}\~oes}[section]
\newtheorem{exemplo}{Exemplo}[section]
\newtheorem{exemplos}{Exemplos}[section]
\newtheorem{exercicio}{Exerc{\'\i}cio}
\newtheorem{observacao}{Observa{\c c}{\~a}o:}[section]
\newtheorem{observacoes}{Observa{\c c}{\~o}es:}[section]
\newtheorem*{solucao}{Solu{\c c}{\~a}o:}
\newtheorem{proposicao}{Proposi\c{c}\~ao}
\newtheorem{lema}{Lema}
\newtheorem{teorema}{Teorema}
\newtheorem{corolario}{Corol\'ario}
\newenvironment{prova}[1][Prova]{\noindent\textbf{#1:} }{\qedsymbol}%{\ \rule{0.5em}{0.5em}}
\newcommand{\nsub}{\varsubsetneq}
\newcommand{\vaz}{\emptyset}
\newcommand{\im}{{\rm Im\,}}
\newcommand{\sub}{\subseteq}
\newcommand{\n}{\mathbb{N}}
\newcommand{\z}{\mathbb{Z}}
\newcommand{\rac}{\mathbb{Q}}
\newcommand{\real}{\mathbb{R}}
\newcommand{\complex}{\mathbb{C}}
\newcommand{\cp}[1]{\mathbb{#1}}
\newcommand{\ch}{\mbox{\textrm{car\,}}\nobreak}
\newcommand{\vesp}[1]{\vspace{ #1  cm}}
\newcommand{\compcent}[1]{\vcenter{\hbox{$#1\circ$}}}
\newcommand{\comp}{\mathbin{\mathchoice
{\compcent\scriptstyle}{\compcent\scriptstyle}
{\compcent\scriptscriptstyle}{\compcent\scriptscriptstyle}}}

\title{Fun\c{c}\~oes - Composição}
\author[\autor]{\autor}
\institute[\instituto]{\instituto}
\date{}

\begin{document}
    \begin{frame}
        \maketitle
    \end{frame}

    \logo{\includegraphics[scale=.1]{logo-MAT.png}\vspace*{8.5cm}}

    \begin{frame}
        \begin{definicao}
            Sejam $f : A \to B$ \pause e $g : B \to C$ \pause fun\c{c}\~oes. \pause Definimos a \textbf{fun\c{c}\~ao composta} \pause de $g$ com $f$ \pause como sendo a fun\c{c}\~ao denotada por $g \circ f \pause : A \to C$ \pause tal que \pause $(g\circ f)(x) \pause = g(f(x))$ \pause para todo $x \in A$.\pause
        \end{definicao}

        \vspace{5cm}
    \end{frame}

    \begin{frame}
        \begin{exemplos}
            \begin{enumerate}
                \item[1)] Sejam $f : \real \to \real$ \pause e $g : \real \to \real$ \pause dadas por $f(x) = x^2$ \pause e $g(x) = x + 1$. \pause Assim podemos definir $g \circ f$ \pause e $f \circ g$ e:\pause
            \end{enumerate}
        \end{exemplos}

        \vspace{5cm}
    \end{frame}

    \begin{frame}
        \begin{exemplos}
            \begin{enumerate}
                \item[2)] $f : \real_- \to \real^*_+$ \pause e $g : \real^*_+ \to \real$ \pause dadas por $f(x) = x^2 + 1$ \pause e $g(x) = \ln x$. \pause Nesse caso s\'o podemos definir $g \circ f : \real_- \to \real$ e:\pause
            \end{enumerate}
        \end{exemplos}

        \vspace{5cm}
    \end{frame}

    \begin{frame}
        \begin{proposicao}
            Se $f : A \to B$ \pause e $g : B \to C$ \pause s{\~a}o fun{\c c}{\~o}es injetoras, \pause ent{\~a}o $g\circ f : \pause A \to C$ \pause {\'e} injetora.\pause
        \end{proposicao}
    \end{frame}

    \begin{frame}
        \begin{proposicao}
            Se $f : A \to B$ \pause e $g : B \to C$ \pause s{\~a}o fun\c{c}\~oes sobrejetoras, \pause ent{\~a}o $g\circ f : A \to C$ \pause {\'e} sobrejetora.\pause
        \end{proposicao}
    \end{frame}
\end{document}
