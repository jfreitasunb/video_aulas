%!TEX program = xelatex
% !TEX encoding = ISO-8859-1
\def\ano{2020}
\def\semestre{1}
\def\disciplina{\'Algebra 1}
\def\turma{C}
\def\autor{Jos\'e Ant\^onio O. Freitas}
\def\instituto{MAT-UnB}

\documentclass{beamer}
\usetheme{Madrid}
\usecolortheme{beaver}
% \mode<presentation>
\usepackage{caption}
\usepackage{textpos}
\usepackage{amssymb}
\usepackage{amsmath,amsfonts,amsthm,amstext}
\usepackage[brazil]{babel}
% \usepackage[latin1]{inputenc}
\usepackage{graphicx}
\graphicspath{{/home/jfreitas/GitHub_Repos/video_aulas/logo/}{D:/Dropbox/imagens-latex/}}
\usepackage{enumitem}
\usepackage{multicol}
\usepackage{answers}
\usepackage{tikz,ifthen}
\usetikzlibrary{lindenmayersystems}
\usetikzlibrary[shadings]
\newtheorem{definicao}{Defini\c{c}\~ao}[section]
\newtheorem{definicoes}{Defini\c{c}\~oes}[section]
\newtheorem{exemplo}{Exemplo}[section]
\newtheorem{exemplos}{Exemplos}[section]
\newtheorem{exercicio}{Exerc{\'\i}cio}
\newtheorem{observacao}{Observa{\c c}{\~a}o:}[section]
\newtheorem{observacoes}{Observa{\c c}{\~o}es:}[section]
\newtheorem*{solucao}{Solu{\c c}{\~a}o:}
\newtheorem{proposicao}{Proposi\c{c}\~ao}
\newtheorem{lema}{Lema}
\newtheorem{teorema}{Teorema}
\newtheorem{corolario}{Corol\'ario}
\newenvironment{prova}[1][Prova]{\noindent\textbf{#1:} }{\qedsymbol}%{\ \rule{0.5em}{0.5em}}
\newcommand{\nsub}{\varsubsetneq}
\newcommand{\vaz}{\emptyset}
\newcommand{\im}{{\rm Im\,}}
\newcommand{\sub}{\subseteq}
\newcommand{\n}{\mathbb{N}}
\newcommand{\z}{\mathbb{Z}}
\newcommand{\rac}{\mathbb{Q}}
\newcommand{\real}{\mathbb{R}}
\newcommand{\complex}{\mathbb{C}}
\newcommand{\cp}[1]{\mathbb{#1}}
\newcommand{\ch}{\mbox{\textrm{car\,}}\nobreak}
\newcommand{\vesp}[1]{\vspace{ #1  cm}}
\newcommand{\compcent}[1]{\vcenter{\hbox{$#1\circ$}}}
\newcommand{\comp}{\mathbin{\mathchoice
{\compcent\scriptstyle}{\compcent\scriptstyle}
{\compcent\scriptscriptstyle}{\compcent\scriptscriptstyle}}}

\title{Fun\c{c}\~oes - Continuação}
\author[\autor]{\autor}
\institute[\instituto]{\instituto}
\date{\today}

\begin{document}
    \begin{frame}
        \maketitle
    \end{frame}

    \logo{\includegraphics[scale=.1]{logo-MAT.png}\vspace*{8.5cm}}

    \begin{frame}
\begin{proposicao}
    Se $f : A \to B$ e $g : B \to C$ s{\~a}o fun\c{c}\~oes sobrejetoras, ent{\~a}o $g\circ f : A \to C$ {\'e} sobrejetora.
\end{proposicao}
\begin{prova}
    Para mostrar que $g \circ f : A \to C$ \'e sobrejetora, precisamos mostrar que para todo $y \in C$, existe $x \in A$ tal que $(g\circ f)(x) = y$.

    Assim seja $y \in C$. Como $g : B \to C$ \'e sobrejetora, existe $z \in A$ tal que $g(z) = y$. Mas $z \in B$ e $f : A \to B$ \'e sobrejetora e assim existe $x \in A$ tal que $f(x) = z$. Logo
    \[
        (g\circ f)(x) = g(f(x)) = g(z) = y.
    \]
    Portanto $g \circ f$ \'e sobrejetora.
\end{prova}
    \end{frame}

    \begin{frame}
\begin{definicao}
    Seja $f : A \to B$ uma fun{\c c}{\~a}o.
    \begin{enumerate}[label={\roman*})]
        \item Dado $P \sub A$, chama-se \textbf{imagem direta} de $P$  \textbf{segundo} $f$ e indica-se por $f(P)$ o subconjunto de $B$ dado por
        \[
            f(P) = \{f(x) \mid x \in P\},
        \]
        isto {\'e}, $f(P)$ {\'e} o conjunto das imagens por $f$ dos elementos de $P$.

        \item Dado $Q \sub B$, chama-se \textbf{imagem inversa} de $Q$ \textbf{segundo} $f$ e indica-se por $f^{-1}(Q)$ o subconjunto de $A$ dado por
        \[
            f^{-1}(Q) = \{x \in A \mid f(x) \in Q\},
        \]
        isto {\'e}, $f^{-1}(Q)$ {\'e} o conjunto dos elementos de $A$ que tem imagem em $Q$ atrav{\'e}s de $f$.
    \end{enumerate}
\end{definicao}
    \end{frame}

    \begin{frame}
\begin{exemplos}
    \begin{enumerate}[label={\arabic*})]
        \item Seja $A = \{1, 3, 5, 7, 9 \}$ e $B = \{0, 1, 2, 3, \dots, 10\}$ e $f : A \to B$ dada por $f(x) = x + 1$. Temos:
        \begin{itemize}
            \item $f(\{1\}) = \{f(1)\} = \{2\}$

            \item $f(\{3, 5, 7\}) = \{f(3), f(5), f(7)\} = \{4, 6, 8\}$

            \item $f(A) = \{f(1), f(3), f(5), f(7), f(9)\} = \{2, 4, 6, 8, 10\}$

            \item $f(\emptyset) = \emptyset$

            \item $f^{-1}(\{2, 4, 10\}) = \{x \in A \mid f(x) \in \{2, 4, 10\}\} = \{1, 3, 9\}$

            \item $f^{-1}(\{0, 1, 3, 5, 7, 9\}) = \{x \in A \mid f(x) \in \{0, 1, 3, 5, 7, 9\}\} = \emptyset$
        \end{itemize}

        \item Sejam $A = B = \real$ e $f : \real \to \real$ dada por $f(x) = x^2$. Temos:
        \begin{itemize}
            \item $f(\{1, 2, 3\}) = \{1, 4, 9\}$

            \item $f([0,2]) = \{f(x) \in \real \mid 0 \le x \le 2 \} = \{x^2 \mid 0 \le x \le 2\} = [0, 4]$

            \item $f^{-1}([1, 9]) = \{x \in \real \mid f(x) \in [1, 9]\} = \{ x \in \real \mid 1 \le f(x) \le 9\} = \{x \in \real \mid 1 \le x^2 \le 9\} = [-1, -3] \cup [1, 3]$
        \end{itemize}
    \end{enumerate}
\end{exemplos}
    \end{frame}

    \begin{frame}
\begin{proposicao}
    Seja $f : A \to B$ uma fun{\c c}{\~a}o e sejam $P$, $Q \sub A$, $X$, $Y \sub B$.
    \begin{enumerate}[label={\roman*})]
        \item Se $P \sub Q$, ent{\~a}o $f(P) \sub f(Q)$.

        \item $f^{-1}(X \cup Y) = f^{-1}(X) \cup f^{-1}(Y)$.
    \end{enumerate}
\end{proposicao}
\begin{prova}
    \begin{enumerate}[label={\roman*})]
        \item Se $y \in f(P)$, ent{\~a}o existe $x \in P$ tal que $f(x) = y$. Mas como $P \sub Q$, ent{\~a}o $x \in Q$ e da{\'\i} $y \in f(Q)$. Logo $f(P) \sub f(Q)$.

        \item Seja $z \in f^{-1}(X \cup Y)$. Ent{\~a}o $f(z) \in X \cup Y$. Se $f(z) \in X$, ent\~ao $z \in f^{-1}(X)$ e da{\'\i} $z \in f^{-1}(X) \cup f^{-1}(Y)$. Se $f(z) \in Y$, ent{\~a}o $z \in f^{-1}(Y)$ e assim $z \in f^{-1}(X) \cup f^{-1}(Y)$. Logo, $f^{-1}(X \cup Y) \sub f^{-1}(X) \cup f^{-1}(Y)$.

        Agora, seja $z \in f^{-1}(X) \cup f^{-1}(Y)$. Se $z \in f^{-1}(X)$, ent{\~a}o $f(z) \in X$, da{\'\i} $f(z) \in X \cup Y$, isto {\'e}, $z \in f^{-1}(X \cup Y)$. Se $z \in f^{-1}(Y)$, ent{\~a}o $f(z) \in Y$ e assim $f(z) \in X \cup Y$, isto {\'e}, $z \in f^{-1}(X \cup Y)$. Logo $f^{-1}(X) \cup f^{-1}(Y) \sub f^{-1}(X \cup Y)$.

        Portanto, $f^{-1}(X \cup Y) = f^{-1}(X) \cup f^{-1}(Y)$.
    \end{enumerate}
\end{prova}
    \end{frame}

    \begin{frame}

Dado $f : A \to B$ uma fun\c{c}{\~a}o, queremos construir uma fun\c{c}\~ao $g : B \to A$ de modo que
\[
    g(f(x)) = x
\]
para todo $x \in A$. Mas $f(x) = y$ com $y \in B$. Assim podemos tentar definir $g$ como
\begin{align}\label{condicao_funcao_inversa}
    g(y) = x,\ y \in B \mbox{ se, e somente se, } f(x) = y.
\end{align}
Com essa defini\c{c}\~ao $g$ \'e uma fun\c{c}\~ao? Vejamos um exemplo: definia $f : \{0,1,2,3\} \to \{4,5,6,7,8\}$ por:
\begin{align*}
    f(0) &= 5\\
    f(1) &= 5\\
    f(2) &= 6\\
    f(3) &= 7.  
\end{align*}
    \end{frame}

    \begin{frame}
A partir da defini\c{c}\~ao \eqref{condicao_funcao_inversa} temos
\begin{align*}
    g(5) &= 0\\
    g(5) &= 1\\
    g(6) &= 2\\
    g(7) &= 3.
\end{align*}

Assim $g$ definida pela condi\c{c}\~ao \eqref{condicao_funcao_inversa} n\~ao \'e uma fun\c{c}\~ao pois $g$ atribui ao n\'umero 5 dois poss{\'\i}veis valores: 0 e 1. Isso ocorre pois $f$ n\~ao \'e injetora. Vamos ent\~ao redefinir $f$ de modo a torn\'a-la injetora:
\begin{align*}
    f(0) &= 5\\
    f(1) &= 4\\
    f(2) &= 6\\
    f(3) &= 7.  
\end{align*}
    \end{frame}

    \begin{frame}
Agora $g$ torna-se:
\begin{align*}
    g(5) &= 0\\
    g(4) &= 1\\
    g(6) &= 2\\
    g(7) &= 3.
\end{align*}

Ainda assim $g$ n\~ao \'e fun\c{c}\~ao pois $g$ n\~ao associa $8 \in B$ com nenhum elemento em $A$. Isso ocorre pois $f$ n\~ao \'e sobrejetora.

Portanto para que a condi\c{c}\~ao \eqref{condicao_funcao_inversa} defina uma fun\c{c}\~ao \'e necess\'ario que $f$ seja bijetora. Temos ent\~ao o seguinte teorema:
    \end{frame}

    \begin{frame}
\begin{teorema}\label{teorema_funcao_inversa}
    Seja $f: A \to B$ fun{\c c}{\~a}o. Defina $g : B \to A$ por
    \begin{align}\label{funcao_inversa}
        g(y) = x,\ y \in B \mbox{ se, e somente se, } f(x) = y.
    \end{align}
    Ent{\~a}o $g$ {\'e} uma fun{\c c}{\~a}o se, e somente se, $f$ {\'e} bijetora.
\end{teorema}
    \noindent \textbf{\textit{Prova:}}
    Precisamos mostrar que:
    \begin{enumerate}[label={\roman*})]
        \item Se $g$ definida por \eqref{funcao_inversa} \'e uma fun\c{c}\~ao, ent\~ao $f$ \'e bijetora.
        \item Se $f$ \'e bijetora, ent\~ao $g$ definida por \eqref{funcao_inversa} \'e uma fun\c{c}\~ao.
    \end{enumerate}

    Provemos a primeira afirma\c{c}\~ao: suponha que $g$ \'e uma fun\c{c}\~ao. Precisamos provar que $f$ {\'e} injetora e sobrejetora.

    Sejam $x_1$, $x_2 \in A$ tais que $f(x_1) = y = f(x_2)$. Como $f(x_1) = y$ temos $g(y) = x_1$, al{\'e}m disso, $g(y) = x_2$. Mas $g$ {\'e} uma fun{\c c}{\~a}o, da{\'\i} $x_1 = x_2$, ou seja, $f$ {\'e} injetora.
    \end{frame}

    \begin{frame}

    Dado $y \in B$, como $g$ {\'e} uma fun{\c c}{\~a}o, existe $x \in A$, tal que $g(y) = x$, logo $f(x) = y$ e assim $f$ {\'e} sobrejetora.

    Portanto $f$ {\'e} bijetora.

    Agora vamos provar a segunda afirma\c{c}\~ao: suponha que $f$ \'e bijetora. Precisamos mostrar que $g$ \'e uma fun\c{c}\~ao. Primeiramente, dado $y \in B$, como $f$ {\'e} sobrejetora, existe $x \in A$ tal que $f(x) = y$. Logo por \eqref{funcao_inversa} segue que $g(y) = x \in A$. Logo $g$ associa cada elemento de $B$ com algum elemento em $A$.

    Suponha que $g(y) = x_1$ e que $g(y) = x_2$. Da{\'\i}, de \eqref{funcao_inversa} temos $f(x_1) = y$ e $f(x_2) = y$. Mas $f$ {\'e} injetora, logo $x_1 = x_2$ e ent{\~a}o $g(y) = x_1 = x_2$. Assim $g$ associa cada elemento de $B$ com somente um elemento em $A$.

    Portanto $g$ {\'e} fun{\c c}{\~a}o.
    \end{frame}

    \begin{frame}
\begin{definicao}
    A fun\c{c}\~ao $g : B \to A$ do teorema \ref{teorema_funcao_inversa} \'e chamada de \textbf{fun\c{c}\~ao inversa} de $f : A \to B$ e ser\'a denotada por $g = f^{-1}$.
\end{definicao}


\begin{definicao}
    Dado um conjunto $A \ne \emptyset$, a fun{\c c}{\~a}o $i_{A}: A \to A$ dada por $i_{A}(x)=(x)$ {\'e} chamada de \textbf{fun{\c c}{\~a}o identidade}.
\end{definicao}
\end{frame}

    \begin{frame}

\begin{proposicao}
    Se $f : A \to B$ {\'e} bijetora, ent{\~a}o $f\circ f^{-1} = i_{B}$ e $f^{-1}\circ f = i_{A}$.
\end{proposicao}
\begin{prova}
    Temos $i_{B} : B \to B$ e $i_{A} : A \to A$. Al{\'e}m disso, $f\circ f^{-1} : B \to B$ e $f^{-1}\circ f : A \to A$, da{\'\i} $\dom(f\circ f^{-1}) = \dom(i_{B})$ e $\dom(f^{-1}\circ f) = \dom(i_{A})$. Agora, $y \in B$, $(f\circ f^{-1})(y) = f(f^{-1}(y)) = y = i_{B}(y)$. E se $x \in A$, $(f^{-1}\circ f)(x) = f^{-1}(f(x)) = x = i_{A}(x)$. Portanto $f\circ f^{-1} = i_{B}$ e $f^{-1}\circ f = i_{A}$ como quer{\'\i}amos.
\end{prova}
\end{frame}

    \begin{frame}

\begin{proposicao}\label{propriedades_identidade}
    Se $f : A \to B$ e $g : B \to A$ s{\~a}o fun{\c c}{\~o}es, ent{\~a}o:
    \begin{enumerate}[label={\roman*})]
        \item $f\circ i_{A} = f$
        \item $i_{B}\circ f = f$
        \item $g\circ i_{B} = g$
        \item $i_{A}\circ g = g$
        \item Se $g\circ f = i_{A}$ e $f\circ g = i_{B}$, ent{\~a}o $f$ e $g$ s{\~a}o bijetoras e $g=f^{-1}$
    \end{enumerate}
\end{proposicao}
    \noindent \textbf{\textit{Prova:}}
    \begin{enumerate}
        \item[i)] Primeiro temos $f: A \to B$ e $i_{A} : A \to A$ e $f\circ i_{A} : A \to B$. Assim $\dom(f\circ i_{A}) = \dom(f)$. Agora dado $x \in A$, temos $(f\circ i_{A})(x) = f(i_{A}(x)) = f(x)$. Portanto, $f\circ i_{A} = f$.
        \item[ii)] Segue de forma semelhante ao caso anteiror.
        \item[iii)] Segue de forma semelhante ao primeiro caso.
        \item[iv)] Segue de forma semelhante ao primeiro caso.
    \end{enumerate}

    \end{frame}

    \begin{frame}
        \begin{enumerate}
        \item[v)] Provemos que $f$ \'e bijetora: sejam $x_1$, $x_2 \in B$ tais que $f(x_1) = f(x_2)$. Como $f : A \to B$ e $g : B \to A$, ent{\~a}o $g(f(x_1)) = g(f(x_2))$, ou seja, $(g\circ f)(x_1) = (g\circ f)(x_2)$. Da{\'\i}, $i_{A}(x_1) = i_{A}(x_2)$. Logo, $x_1 = x_2$. Logo $f$ {\'e} injetora.

        Agora, dado $y \in B$, segue que $y = i_{B}(y)$. Mas $i_{B} = f\circ g$. Da{\'\i}, $y = i_{B}(y) = (f\circ g)(y) = f(g(y))$. Assim, $x = g(y)\in A$ e $f(x) = y$. Logo $f$ {\'e} sobrejetora.

        Portanto $f$ {\'e} bijetora. Analogamente, prova-se que g {\'e} bijetora. 

        Provemos agora que $g = f^{-1}$. Para isso, primeiro temos  $f^{-1} : B \to A$ e ent\~ao $\dom(g) = B = \dom(f^{-1})$. Agora, $f\circ g = i_{B} = f\circ f^{-1}$. Assim, para todo $x \in B$, $(f\circ g)(x) = (f\circ f^{-1})(x)$. Isto {\'e}, $f(g(x)) = f(f^{-1}(x))$. Portanto como $f$ \'e injetora, $g(x) = f^{-1}(x)$ para todo $x\in B$. Logo $g = f^{-1}$ como quer{\'\i}amos.
    \end{enumerate}

\end{frame}

    \begin{frame}
        \vspace{5cm}
    \end{frame}
\end{document}