%!TEX program = xelatex
% !TEX encoding = ISO-8859-1
\def\ano{2020}
\def\semestre{1}
\def\disciplina{\'Algebra 1}
\def\turma{C}
\def\autor{Jos\'e Ant\^onio O. Freitas}
\def\instituto{MAT-UnB}

\documentclass{beamer}
\usetheme{Madrid}
\usecolortheme{beaver}
% \mode<presentation>
\usepackage{caption}
\usepackage{textpos}
\usepackage{amssymb}
\usepackage{amsmath,amsfonts,amsthm,amstext}
\usepackage[brazil]{babel}
% \usepackage[latin1]{inputenc}
\usepackage{graphicx}
\graphicspath{{/home/jfreitas/GitHub_Repos/video_aulas/logo/}{D:/Dropbox/imagens-latex/}}
\usepackage{enumitem}
\usepackage{multicol}
\usepackage{answers}
\usepackage{tikz,ifthen}
\usetikzlibrary{lindenmayersystems}
\usetikzlibrary[shadings]
\newtheorem{definicao}{Defini\c{c}\~ao}[section]
\newtheorem{definicoes}{Defini\c{c}\~oes}[section]
\newtheorem{exemplo}{Exemplo}[section]
\newtheorem{exemplos}{Exemplos}[section]
\newtheorem{exercicio}{Exerc{\'\i}cio}
\newtheorem{observacao}{Observa{\c c}{\~a}o:}[section]
\newtheorem{observacoes}{Observa{\c c}{\~o}es:}[section]
\newtheorem*{solucao}{Solu{\c c}{\~a}o:}
\newtheorem{proposicao}{Proposi\c{c}\~ao}
\newtheorem{lema}{Lema}
\newtheorem{teorema}{Teorema}
\newtheorem{corolario}{Corol\'ario}
\newenvironment{prova}[1][Prova]{\noindent\textbf{#1:} }{\qedsymbol}%{\ \rule{0.5em}{0.5em}}
\newcommand{\nsub}{\varsubsetneq}
\newcommand{\vaz}{\emptyset}
\newcommand{\im}{{\rm Im\,}}
\newcommand{\sub}{\subseteq}
\newcommand{\n}{\mathbb{N}}
\newcommand{\z}{\mathbb{Z}}
\newcommand{\rac}{\mathbb{Q}}
\newcommand{\real}{\mathbb{R}}
\newcommand{\complex}{\mathbb{C}}
\newcommand{\cp}[1]{\mathbb{#1}}
\newcommand{\ch}{\mbox{\textrm{car\,}}\nobreak}
\newcommand{\vesp}[1]{\vspace{ #1  cm}}
\newcommand{\compcent}[1]{\vcenter{\hbox{$#1\circ$}}}
\newcommand{\comp}{\mathbin{\mathchoice
{\compcent\scriptstyle}{\compcent\scriptstyle}
{\compcent\scriptscriptstyle}{\compcent\scriptscriptstyle}}}

\title{Fun\c{c}\~oes - Continuação}
\author[\autor]{\autor}
\institute[\instituto]{\instituto}
\date{\today}

\begin{document}
    \begin{frame}
        \maketitle
    \end{frame}

    \logo{\includegraphics[scale=.1]{logo-MAT.png}\vspace*{8.5cm}}

    \begin{frame}
        \begin{proposicao}
            Se $f : A \to B$ \pause e $g : B \to C$ \pause s{\~a}o fun\c{c}\~oes sobrejetoras, \pause ent{\~a}o $g\circ f : A \to C$ \pause {\'e} sobrejetora.\pause
        \end{proposicao}
        \noindent \textbf{\textit{Prova: }}\pause
        Para mostrar que $g \circ f : A \to C$ \pause \'e sobrejetora, \pause precisamos mostrar que para todo $y \in C$, \pause existe $x \in A$ \pause tal que $(g\circ f)(x) = y$.\pause

        Assim seja $y \in C$. \pause Como $g : B \to C$ \'e sobrejetora, \pause existe $z \in A$ \pause tal que $g(z) = y$. \pause Mas $z \in B$ \pause e $f : A \to B$ \pause \'e sobrejetora \pause. Assim existe $x \in A$ \pause tal que $f(x) = z$. \pause Logo\pause
        \[
            (g\circ f)(x) \pause = g(f(x)) \pause = g(z) \pause = y.\pause
        \]
        Portanto $g \circ f$ \pause \'e sobrejetora.\pause \hspace{.5cm} \qedsymbol
    \end{frame}

    \begin{frame}
        \begin{definicao}
            Seja $f : A \to B$ \pause uma fun{\c c}{\~a}o.\pause
            \begin{enumerate}[label={\roman*})]
                \item Dado $P \sub A$, \pause chama-se \textbf{imagem direta} \pause de $P$ \pause \textbf{segundo} $f$ \pause e indica-se por $f(P)$ \pause o subconjunto de $B$ \pause dado por\pause
                \[
                    f(P) = \pause \{f(x) \pause \mid x \in P\},\pause
                \]
                isto {\'e}, \pause $f(P)$ \pause {\'e} o conjunto das imagens por $f$ \pause dos elementos de $P$.\pause
                
                \vspace{.5cm}

                \item Dado $Q \sub B$, \pause chama-se \textbf{imagem inversa} \pause de $Q$ \textbf{segundo} $f$ \pause e indica-se por \pause $f^{-1}(Q)$ \pause o subconjunto de $A$ \pause dado por\pause
                \[
                    f^{-1}(Q) \pause = \{x \in A \pause \mid f(x) \in Q\},\pause
                \]
                isto {\'e}, \pause $f^{-1}(Q)$ \pause {\'e} o conjunto dos elementos de $A$ \pause que tem imagem em $Q$ \pause atrav{\'e}s de $f$.\pause
            \end{enumerate}
        \end{definicao}
    \end{frame}

    \begin{frame}
        \vspace{5cm}
    \end{frame}

    \begin{frame}
        \vspace{5cm}
    \end{frame}

    \begin{frame}
        \begin{exemplos}
            \begin{enumerate}
                \item[1)] Seja $A = \{1, 3, 5, 7, 9 \}$ \pause e $B = \{0, 1, 2, 3, \dots, 10\}$ \pause e $f : A \to B$ \pause dada por $f(x) = x + 1$. \pause Temos:\pause

                \vspace{.5cm}

                \begin{itemize}
                    \item $f(\{1\}) = \pause \{f(1)\} \pause= \{2\}$\pause

                    \vspace{.5cm}

                    \item $f(\{3, 5, 7\}) \pause= \{f(3), \pause f(5), \pause f(7)\} \pause = \{4, 6, 8\}$\pause

                    \vspace{.5cm}

                    \item $f(A) \pause = \{f(1), \pause f(3), \pause f(5), \pause f(7), \pause f(9)\} = \pause \{2, 4, 6, 8, 10\}$\pause

                    \vspace{.5cm}

                    \item $f(\emptyset) \pause = \emptyset$\pause

                    \vspace{.5cm}

                    \item $f^{-1}(\{2, 4, 10\}) = \pause \{x \in A \pause \mid f(x) \pause \in \{2, 4, 10\}\pause\} = \{1, 3, 9\}$\pause

                    \vspace{.5cm}

                    \item $f^{-1}(\{0, 1, 3, 5, 7, 9\}) \pause = \{x \in A \pause \mid f(x) \pause \in \{0, 1, 3, 5, 7, 9\}\pause\} = \emptyset$\pause

                    \vspace{.5cm}
                \end{itemize}
            \end{enumerate}
        \end{exemplos}
    \end{frame}

    \begin{frame}
        \begin{exemplos}
            \begin{enumerate}
                \item[2)] Sejam $A = B = \real$ \pause e $f : \real \to \real$ \pause dada por $f(x) = x^2$. \pause Temos:\pause

                \vspace{.5cm}

                \begin{itemize}
                    \item $f(\{1, 2, 3\}) \pause = \{1, 4, 9\}$\pause

                    \vspace{.5cm}

                    \item $f([0,2]) \pause = \{f(x) \pause \in \real \pause \mid 0 \le x \pause \le 2 \pause \} = \{x^2 \pause \mid 0 \le x \le 2\} \pause = [0, 4]$\pause

                    \vspace{.5cm}

                    \item $f^{-1}([1, 9]) \pause = \{x \in \real \pause \mid f(x) \in [1, 9]\pause \} = \{ x \in \real \pause \mid 1 \le f(x) \le 9\} \pause = \{x \in \real \pause \mid 1 \le x^2 \pause \le 9\} \pause = [-1, -3] \pause \cup [1, 3]$\pause

                    \vspace{.5cm}
                \end{itemize}
            \end{enumerate}
        \end{exemplos}
    \end{frame}

    \begin{frame}
        \begin{proposicao}
            Seja $f : A \to B$ uma fun{\c c}{\~a}o \pause e sejam $P$, \pause $Q \sub A$, \pause $X$, \pause $Y \sub B$.\pause
            \begin{enumerate}[label={\roman*})]
                \item Se $P \sub Q$, \pause ent{\~a}o $f(P) \sub f(Q)$.\pause

                \vspace{.5cm}

                \item $f^{-1}(X \cup Y) \pause = f^{-1}(X) \pause \cup f^{-1}(Y)$.\pause
            \end{enumerate}
        \end{proposicao}
            \noindent \textbf{\textit{Prova: }}\pause
            \begin{enumerate}
                \item[i)] Se $y \in f(P)$, \pause ent{\~a}o existe $x \in P$ \pause tal que $f(x) = y$. \pause Mas como $P \sub Q$, \pause ent{\~a}o $x \in Q$ \pause e da{\'\i} $y \in f(Q)$. \pause Logo $f(P) \sub f(Q)$.\pause
            \end{enumerate}
        \end{frame}

        \begin{frame}
            \begin{enumerate}
                \item[ii)] Seja $z \in f^{-1}(X \cup Y)$. \pause Ent{\~a}o $f(z) \in X \cup Y$. \pause Se $f(z) \in X$, \pause ent\~ao $z \in f^{-1}(X)$ \pause e da{\'\i} $z \in f^{-1}(X) \cup \pause f^{-1}(Y)$. \pause Se $f(z) \in Y$, \pause ent{\~a}o $z \in f^{-1}(Y)$ \pause e assim $z \in f^{-1}(X) \cup \pause f^{-1}(Y)$. \pause Logo, $f^{-1}(X \cup Y) \sub f^{-1}(X) \cup f^{-1}(Y)$.\pause

                \vspace{.3cm}

                Agora, seja $z \in f^{-1}(X) \cup f^{-1}(Y)$. \pause Se $z \in f^{-1}(X)$, \pause ent{\~a}o $f(z) \in X$, \pause da{\'\i} $f(z) \in X \cup Y$, \pause isto {\'e}, \pause $z \in f^{-1}(X \cup Y)$. \pause Se $z \in f^{-1}(Y)$, \pause ent{\~a}o $f(z) \in Y$ \pause e assim $f(z) \in X \cup Y$, \pause isto {\'e}, \pause $z \in f^{-1}(X \cup Y)$. \pause Logo $f^{-1}(X) \cup f^{-1}(Y) \sub f^{-1}(X \cup Y)$.\pause

                \vspace{.3cm}

                Portanto, \pause $f^{-1}(X \cup Y) = \pause f^{-1}(X) \cup f^{-1}(Y)$.\pause \hspace{.5cm} \qedsymbol
            \end{enumerate}
    \end{frame}

    \begin{frame}
        Dado $f : A \to B$ \pause uma fun\c{c}{\~a}o, \pause queremos construir uma fun\c{c}\~ao $g : B \to A$ \pause de modo que
        \[
            g(f(x)) = x,\pause
        \]
        para todo $x \in A$. \pause Mas $f(x) = y$ \pause com $y \in B$. \pause Assim podemos tentar definir $g$ \pause como
        \[
            g(y) = x,\ y \in B \pause \quad \mbox{ se, e somente se, } \pause f(x) = y.\pause
        \]
        Com essa defini\c{c}\~ao \pause $g$ \'e uma fun\c{c}\~ao? \pause Vejamos um exemplo: \pause definia $f : \{0,1,2,3\} \to \{4,5,6,7,8\}$ por:\pause
        \begin{center}
            $f(0) = 5$ \pause\\
            \vspace{.3cm}
            $f(1) = 5$\pause\\
            \vspace{.3cm}
            $f(2) = 6$\pause\\
            \vspace{.3cm}
            $f(3) = 7$.\pause
        \end{center}
    \end{frame}

    \begin{frame}
        A partir da defini\c{c}\~ao acimas temos\pause
        \begin{center}
            $g(5) = 0 $\pause\\

            \vspace{.3cm}
            
            $g(5) = 1$\pause\\
            
            \vspace{.3cm}

            $g(6) = 2$\pause\\

            \vspace{.3cm}
            
            $g(7) = 3$.\pause
        \end{center}

        Assim $g$ definida dessa forma \pause n\~ao \'e uma fun\c{c}\~ao \pause pois $g$ atribui ao n\'umero 5 \pause dois poss{\'\i}veis valores: \pause 0 e 1. \pause Isso ocorre pois $f$ n\~ao \'e injetora. \pause Vamos ent\~ao redefinir $f$ de modo a torn\'a-la injetora:\pause
        \begin{center}
            $f(0) = 5$\pause\\

            \vspace{.3cm}

            $f(1) = 4$\pause\\

            \vspace{.3cm}

            $f(2) = 6$\pause\\

            \vspace{.3cm}

            $f(3) = 7$.\pause
        \end{center}
    \end{frame}

    \begin{frame}
        Agora $g$ torna-se:\pause
        \begin{center}
            $g(5) = 0$\pause\\

            \vspace{.3cm}

            $g(4) = 1$\pause\\

            \vspace{.3cm}

            $g(6) = 2$\\

            \vspace{.3cm}

            $g(7) = 3.$\pause
        \end{center}

        Ainda assim $g$ n\~ao \'e fun\c{c}\~ao \pause pois $g$ n\~ao associa $8 \in B$ \pause com nenhum elemento em $A$.  Isso ocorre pois \pause $f$ n\~ao \'e sobrejetora.\pause

        \vspace{.3cm}

        Portanto para que a condi\c{c}\~ao dada \pause defina uma fun\c{c}\~ao \pause \'e necess\'ario que $f$ seja bijetora. \pause Temos ent\~ao o seguinte teorema:\pause
    \end{frame}

    \begin{frame}
        \begin{teorema}
            Seja $f: A \to B$ uma fun{\c c}{\~a}o. \pause Defina $g : B \to A$ \pause por\pause
            \[
                g(y) = x,\ y \in B \pause \quad \mbox{ se, e somente se, } \pause f(x) = y.\pause
            \]
            Ent{\~a}o $g$ \pause {\'e} uma fun{\c c}{\~a}o \pause se, e somente se, \pause $f$ {\'e} bijetora.\pause
        \end{teorema}
        \noindent \textbf{\textit{Prova:}}
        Precisamos mostrar que:\pause
        \begin{enumerate}[label={\roman*})]
            \item Se $g$ definida como acima \'e uma fun\c{c}\~ao, \pause ent\~ao $f$ \'e bijetora.\pause

            \vspace{.3cm}

            \item Se $f$ \'e bijetora, \pause ent\~ao $g$ definida como acima \'e uma fun\c{c}\~ao.\pause
        \end{enumerate}
    \end{frame}

    \begin{frame}
        Provemos a primeira afirma\c{c}\~ao: \pause suponha que $g$ \'e uma fun\c{c}\~ao. \pause Precisamos provar que $f$ {\'e} injetora \pause e sobrejetora.\pause

        \vspace{.5cm}

        Sejam $x_1$, \pause $x_2 \in A$\pause  tais que $f(x_1) = y \pause = f(x_2)$. \pause Como $f(x_1) = y$ \pause temos $g(y) = x_1$, \pause al{\'e}m disso, \pause $g(y) = x_2$. \pause Mas $g$ {\'e} uma fun{\c c}{\~a}o, \pause da{\'\i} $x_1 = x_2$, \pause ou seja, \pause $f$ {\'e} injetora.\pause
        
        \vspace{.5cm}

        Dado $y \in B$, \pause como $g$ {\'e} uma fun{\c c}{\~a}o, \pause existe $x \in A$, \pause tal que $g(y) = x$, \pause logo $f(x) = y$ \pause e assim $f$ {\'e} sobrejetora.\pause

        \vspace{.5cm}

        Portanto $f$ {\'e} bijetora.\pause
    \end{frame}

    \begin{frame}

        Agora vamos provar a segunda afirma\c{c}\~ao. \pause Para isso suponha que $f$ \'e bijetora. \pause Precisamos mostrar que $g$ \'e uma fun\c{c}\~ao. \pause

        \vspace{.5cm}

        Primeiramente, \pause dado $y \in B$, \pause como $f$ {\'e} sobrejetora, \pause existe $x \in A$ \pause tal que $f(x) = y$. \pause Logo pela definição de $g$ \pause segue que $g(y) = x \in A$. \pause Logo $g$ associa cada elemento de $B$ \pause com algum elemento em $A$.\pause

        \vspace{.5cm}

        Agora, suponha que $g(y) = x_1$ \pause e que $g(y) = x_2$. \pause Da{\'\i}, da definição de $g$ \pause  temos $f(x_1) = y$ \pause e $f(x_2) = y$. \pause Mas $f$ {\'e} injetora, \pause logo $x_1 = x_2$ \pause e ent{\~a}o $g(y) = \pause x_1 \pause = x_2$. \pause Assim $g$ associa cada elemento de $B$ \pause com somente um elemento em $A$.\pause

        \vspace{.5cm}

        Portanto $g$ {\'e} fun{\c c}{\~a}o.\pause
    \end{frame}

    \begin{frame}
        \begin{definicao}
            A fun\c{c}\~ao $g : B \to A$ \pause do teorema anterior \pause \'e chamada de \textbf{fun\c{c}\~ao inversa} \pause de $f : A \to B$ \pause e ser\'a denotada por $g = f^{-1}$.\pause
        \end{definicao}


        \begin{definicao}
            Dado um conjunto $A \ne \emptyset$, \pause a fun{\c c}{\~a}o $i_{A}: A \to A$ \pause dada por $i_{A}(x) \pause = x$ \pause {\'e} chamada de \pause \textbf{fun{\c c}{\~a}o identidade}.\pause
        \end{definicao}
    \end{frame}

    \begin{frame}
        \begin{proposicao}
            Se $f : A \to B$ \pause {\'e} bijetora, \pause ent{\~a}o $f\circ f^{-1} \pause = i_{B}$ \pause e $f^{-1}\circ f \pause = i_{A}$.\pause
        \end{proposicao}
        \textbf{\textit{Prova: }}\pause
        Temos $i_{B} : \pause B \to B$ \pause e $i_{A} : \pause A \to A$. \pause Al{\'e}m disso, \pause $f\circ f^{-1} : \pause B \to B$ \pause e $f^{-1}\circ f : \pause A \to A$, \pause da{\'\i} $\dom(f\circ f^{-1}) \pause = \dom(i_{B})$ \pause e $\dom(f^{-1}\circ f) = \pause \dom(i_{A})$. \pause 

        \vspace{.5cm}

        Agora, $y \in B$, \pause $(f\circ f^{-1})(y) \pause = f(f^{-1}(y)) \pause = y \pause = i_{B}(y)$. \pause E se $x \in A$, \pause $(f^{-1}\circ f)(x) = \pause f^{-1}(f(x)) \pause = x \pause = i_{A}(x)$. \pause 

        \vspace{.5cm}

        Portanto $f\circ f^{-1} = \pause i_{B}$ \pause e $f^{-1}\circ f = \pause i_{A}$ como quer{\'\i}amos.\pause \hspace{.5cm} \qedsymbol
    \end{frame}

    \begin{frame}
        \begin{proposicao}
            Se $f : A \to B$ \pause e $g : B \to A$ \pause s{\~a}o fun{\c c}{\~o}es, \pause ent{\~a}o:\pause
            \begin{enumerate}[label={\roman*})]
                \item $f\circ i_{A} = f$\pause

                \vspace{.3cm}

                \item $i_{B}\circ f = f$\pause
                
                \vspace{.3cm}

                \item $g\circ i_{B} = g$\pause

                \vspace{.3cm}

                \item $i_{A}\circ g = g$\pause

                \vspace{.3cm}

                \item Se $g\circ f = i_{A}$ \pause e $f\circ g = i_{B}$, \pause ent{\~a}o \pause $f$ e $g$ s{\~a}o bijetoras \pause e $g=f^{-1}$.\pause
            \end{enumerate}
        \end{proposicao}
    \end{frame}

    \begin{frame}
        \noindent \textbf{\textit{Prova:}}
        \begin{enumerate}
            \item[i)] Primeiro temos $f: A \to B$ \pause, $i_{A} : A \to A$ \pause e $f\circ i_{A} : A \to B$. \pause Assim \pause $\dom(f\circ i_{A}) = \pause \dom(f)$. \pause Agora dado $x \in A$, \pause temos $(f\circ i_{A})(x) = \pause f(i_{A}(x)) = \pause f(x)$. Portanto, \pause $f\circ i_{A} = f$.\pause

            \vspace{.3cm}

            \item[ii)] Segue de forma semelhante ao caso anteiror.\pause

            \vspace{.3cm}

            \item[iii)] Segue de forma semelhante ao primeiro caso.\pause

            \vspace{.3cm}

            \item[iv)] Segue de forma semelhante ao primeiro caso.\pause
        \end{enumerate}
    \end{frame}

    \begin{frame}
        \begin{enumerate}
            \item[v)] Provemos que $f$ \'e bijetora: \pause sejam $x_1$, \pause $x_2 \in B$ \pause tais que $f(x_1) = \pause f(x_2)$. \pause Como $f : A \to B$ \pause e $g : B \to A$, \pause ent{\~a}o $g(f(x_1)) = \pause g(f(x_2))$, \pause ou seja, \pause $(g\circ f)(x_1) \pause = (g\circ f)(x_2)$. \pause Da{\'\i}, $i_{A}(x_1) = \pause i_{A}(x_2)$. \pause Logo, \pause $x_1 = x_2$ \pause \linebreak e então $f$ {\'e} injetora.\pause

            \vspace{.3cm}

            Agora, dado $y \in B$, \pause segue que $y = \pause i_{B}(y)$. \pause Mas $i_{B} = \pause f\circ g$. \pause Da{\'\i}, \pause $y = \pause i_{B}(y) \pause = (f\circ g)(y) \pause = f(g(y))$. \pause Assim, \pause $x = g(y) \in A$ \pause e $f(x) = y$. \pause Logo $f$ {\'e} sobrejetora.\pause

            \vspace{.3cm}

            Portanto $f$ {\'e} bijetora. \pause Analogamente, prova-se que g {\'e} bijetora.\pause

            \vspace{.3cm}

            Provemos agora que \pause $g = f^{-1}$. \pause Para isso, \pause primeiro temos \linebreak $f^{-1} : B \to A$ \pause e ent\~ao $\dom(g) = B = \pause \dom(f^{-1})$. \pause Agora, $f\circ g = \pause i_{B} = \pause f\circ f^{-1}$. \pause Assim, para todo $x \in B$, \pause $(f\circ g)(x) = \pause (f\circ f^{-1})(x)$. \pause Isto {\'e}, \pause $f(g(x)) = \pause f(f^{-1}(x))$. \pause Portanto como $f$ \'e injetora, \pause $g(x) = f^{-1}(x)$ \pause para todo $x \in B$. \pause Logo $g = f^{-1}$ \pause como quer{\'\i}amos.\pause \hspace{.5cm} \qedsymbol
        \end{enumerate}
    \end{frame}
\end{document}