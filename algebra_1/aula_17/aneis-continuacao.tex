%!TEX program = xelatex
% !TEX encoding = ISO-8859-1
\def\ano{2020}
\def\semestre{1}
\def\disciplina{\'Algebra 1}
\def\turma{C}
\def\autor{Jos\'e Ant\^onio O. Freitas}
\def\instituto{MAT-UnB}

\documentclass{beamer}
\usetheme{Madrid}
\usecolortheme{beaver}
% \mode<presentation>
\usepackage{caption}
\usepackage{textpos}
\usepackage{amssymb}
\usepackage{amsmath,amsfonts,amsthm,amstext}
\usepackage[brazil]{babel}
% \usepackage[latin1]{inputenc}
\usepackage{graphicx}
\graphicspath{{/home/jfreitas/GitHub_Repos/video_aulas/logo/}{D:/Dropbox/imagens-latex/}}
\usepackage{enumitem}
\usepackage{multicol}
\usepackage{answers}
\usepackage{tikz,ifthen}
\usetikzlibrary{lindenmayersystems}
\usetikzlibrary[shadings]
\newtheorem{definicao}{Defini\c{c}\~ao}[section]
\newtheorem{definicoes}{Defini\c{c}\~oes}[section]
\newtheorem{exemplo}{Exemplo}[section]
\newtheorem{exemplos}{Exemplos}[section]
\newtheorem{exercicio}{Exerc{\'\i}cio}
\newtheorem{observacao}{Observa{\c c}{\~a}o:}[section]
\newtheorem{observacoes}{Observa{\c c}{\~o}es:}[section]
\newtheorem*{solucao}{Solu{\c c}{\~a}o:}
\newtheorem{proposicao}{Proposi\c{c}\~ao}
\newtheorem{lema}{Lema}
\newtheorem{teorema}{Teorema}
\newtheorem{corolario}{Corol\'ario}
\newenvironment{prova}[1][Prova]{\noindent\textbf{#1:} }{\qedsymbol}%{\ \rule{0.5em}{0.5em}}
\newcommand{\nsub}{\varsubsetneq}
\newcommand{\vaz}{\emptyset}
\newcommand{\im}{{\rm Im\,}}
\newcommand{\sub}{\subseteq}
\newcommand{\n}{\mathbb{N}}
\newcommand{\z}{\mathbb{Z}}
\newcommand{\rac}{\mathbb{Q}}
\newcommand{\real}{\mathbb{R}}
\newcommand{\complex}{\mathbb{C}}
\newcommand{\cp}[1]{\mathbb{#1}}
\newcommand{\ch}{\mbox{\textrm{car\,}}\nobreak}
\newcommand{\vesp}[1]{\vspace{ #1  cm}}
\newcommand{\compcent}[1]{\vcenter{\hbox{$#1\circ$}}}
\newcommand{\comp}{\mathbin{\mathchoice
{\compcent\scriptstyle}{\compcent\scriptstyle}
{\compcent\scriptscriptstyle}{\compcent\scriptscriptstyle}}}

\title{Anéis - Continuação}
\author[\autor]{\autor}
\institute[\instituto]{\instituto}
\date{\today}

\begin{document}
    \begin{frame}
        \maketitle
    \end{frame}

    \logo{\includegraphics[scale=.1]{logo-MAT.png}\vspace*{8.5cm}}
    
    \begin{frame}
        \begin{definicao}
            Seja $(A, +, \cdot)$ um anel. Dizemos que um subconjunto n{\~a}o vazio $B\subseteq A$ {\'e} um \textbf{subanel} de $A$ quando $(B, +, \cdot)$ \'e um anel.
        \end{definicao}

        \begin{exemplos}
            \begin{enumerate}[label={\arabic*})]
                \item Todo anel $A$ sempre tem dois suban{\'e}is: $\{0_{A}\}$ e $A$, que s{\~a}o chamados de \textbf{suban{\'e}is triviais}.

                \vspace{.5cm}

                \item Em $(\z_4,\oplus,\otimes)$ o conjunto $B = \{\overline{0}, \overline{2}\}$ \'e um subanel.

                \vspace{.5cm}
                
                \item No anel $\z$, o conjunto $m\z$, $m > 1$ {\'e} um subanel de $\z$.

                \vspace{.5cm}
            \end{enumerate} 
        \end{exemplos}
    \end{frame}

    \begin{frame}
        \begin{proposicao}
            Seja $(A, +,\cdot)$ um anel. Um subconjunto n{\~a}o vazio $B\subseteq A$ {\'e} um subanel de $A$ se, e somente se, $x - y \in B$ e $x\cdot y \in B$ para todos $x$, $y \in B$.
        \end{proposicao}

        \noindent \textbf{\textit{Prova: }}FAZER!!!!!

        \begin{exemplos}
            COLOCAR EXEMPLOS
        \end{exemplos}
    \end{frame}
    
    \begin{frame}
        \begin{definicao}
            Um homomorfismo do anel $(A, +, \cdot)$ no anel $(B, \oplus, \otimes)$ {\'e} uma fun{\c c}{\~a}o $f : A \to B$ que satisfaz:
            \begin{enumerate}[label={\roman*})]
                \item $f(x + y) = f(x) \oplus f(y)$, para todos $x$, $y \in A$;

                \vspace{.5cm}
                
                \item $f(x \cdot y) = f(x)\otimes f(y)$, para todos $x$, $y \in A$.
                
                \vspace{.5cm}
            \end{enumerate}
        \end{definicao}
    \end{frame}

    \begin{frame}
        \begin{proposicao}
            Sejam $(A, +, \cdot)$ e $(B, \oplus, \otimes)$ an\'eis e seja $f : A \to B$ um homomorfismo. Ent{\~a}o:
            \begin{enumerate}[label={\roman*})]
                \item $f(0_{A}) = 0_{B}$

                \vspace{.5cm}

                \item $f(-x) = -f(x)$, para todo $x \in A$.
            \end{enumerate}
        \end{proposicao}

        \noindent \textbf{\textit{Prova: }}
        \begin{enumerate}[label={\roman*})]
            \item Fazendo $x = y = 0_{A}$, temos
            \[
                f(0_A) = f(0_A + 0_A) = f(0_A) \oplus f(0_A)
             \]
            Somando $-f(0_A)$ em ambos os lados obtemos
            \begin{align*}
                f(0_A) \oplus (-f(0_A)) &= (f(0_A)\oplus f(0_A)) \oplus (-f(0_A))\\
                0_B &= f(0_A) \oplus 0_B\\
                f(0_A) &= 0_B
            \end{align*}
        \seti
        \end{enumerate}
    \end{frame}

    \begin{frame}
        \begin{enumerate}[label={\roman*})]
            \conti
            \item Temos $0_B = f(0_A) = f(x + (-x)) = f(x)\oplus f(-x)$. Assim somando $-f(x)$ em ambos os lados obtemos
            \begin{align*}
                0_B\oplus(-f(x)) &= [f(x)\oplus f(-x)] + (-f(x))\\
                -f(x) &= f(-x) \oplus (f(x) \oplus (-f(x)))\\
                f(-x) &= -f(x)
            \end{align*}
        \end{enumerate}
        como queríamos. \hspace{.5cm} \qedsymbol
    \end{frame}

    \begin{frame}
        \begin{definicao}Seja $f:A\rightarrow B$ um homomorfismo, onde $A$ e $B$ s{\~a}o an{\'e}is. Dizemos que
            \begin{enumerate}[label={\roman*})]
                \item $f$ {\'e} um epimorfismo se $f$ for sobrejetora.

                \vspace{.5cm}

                \item $f$ {\'e} um monomorfismo se $f$ for injetora.

                \vspace{.5cm}
                
                \item $f$ {\'e} um isomorfismo se $f$ for bijetora.

                \vspace{.5cm}
                
                \item Quando $A=B$ e $f$ {\'e} um isomorfismo, ent{\~a}o $f$ {\'e} um automorfismo.

                \vspace{.5cm}
                
            \end{enumerate}
        \end{definicao}
    \end{frame}

    \begin{frame}
        \begin{definicao}
            Sejam $(A, +, \cdot)$ e $(B, \oplus, \otimes)$ an\'eis e $f : A \to B$ um homomorfismo de an\'eis. Ent\~ao o subconjunto de $A$ definido por
            \[
                \ker(f) = \{ x \in A \mid f(x) = 0_B\}
            \]
            \'e chamado de \textbf{kernel} ou \textbf{n\'ucleo} de $f$.
        \end{definicao}
    \end{frame}

    \begin{frame}
        \begin{proposicao}
            Sejam $(A, +, \cdot)$ e $(B, \oplus, \otimes)$ an\'eis e $f : A \to B$ um homomorfismo de an\'eis. Ent\~ao:
            \begin{enumerate}[label={\roman*})]
                \item $\ker(f)$ \'e um subanel de $A$.

                \vspace{.5cm}

                \item $f$ \'e injetora se, e somente se, $\ker(f) = \{0_A\}$.

                \vspace{.5cm}
            \end{enumerate}
        \end{proposicao}
    \end{frame}

    \begin{frame}
        \noindent \textbf{\textit{Prova: }}
        \begin{enumerate}[label={\roman*})]
            \item Primeiro note que sendo $f$ \'e um homomorfismo ent\~ao $f(0_A) = 0_B$. Logo $0_A \in \ker(f)$, isto \'e, $\ker(f) \ne \emptyset$.

            Agora dados $x$, $y \in \ker(f)$ precisamos mostrar que $x - y \in \ker(f)$ e $xy \in \ker(f)$, e para mostrar isso basta mostrar que $f(x - y) = 0_B$ e$f(xy) = 0_B$. Inicialmente como $x$, $y \in \ker(f)$ da{\'\i} $f(x) = f(y) = 0_B$. Assim
            \begin{align*}
                f(x - y) &= f(x + (-y)) = f(x) \oplus f(-y) = f(x) \oplus (-f(y)) = 0_B \oplus 0_B = 0_B\\
                f(xy) &= f(x)\otimes f(y) = 0_B \otimes 0_B = 0_B
            \end{align*}
            Logo $x - y \in \ker(f)$ e $xy \in \ker(f)$. Portanto $\ker(f)$ \'e um subanel de $A$.

            \seti
        \end{enumerate}
    \end{frame}

    \begin{frame}
        \begin{enumerate}[label={\roman*})]
            \conti
            \item Primeiro suponha que $f$ \'e injetora e vamos mostrar que $\ker(f) = \{0_A\}$. Para isso seja $x \in \ker(f)$. Ent\~ao
            \[
                f(x) = 0_B,
            \]
            mas $f$ sendo um homomorfismo temos $f(0_A) = 0_B$. Da{\'\i}
            \[
                f(x) = 0_B = f(0_A).
            \]
            E como $f$ \'e injetora, por hip\'otese, segue que $x = 0_A$. Logo $\ker(f) = \{0_A\}$.
        \end{enumerate}
    \end{frame}

    \begin{frame}
        \begin{enumerate}
            \item[] Agora suponha que $\ker(f) = \{0_A\}$ e vamos mostrar que $f$ \'e injetora. Para isso sejam $x_1$, $x_2 \in A$ tais que $f(x_1) = f(x_2)$. Da{\'\i}
            \begin{align*}
                &f(x_1) = f(x_2)\\
                &f(x_1) \oplus (-f(x_2)) = 0_B\\
                &f(x_1) \oplus f(-x_2) = 0_B\\
                &f(x_1 - x_2) = 0_B
            \end{align*}
            Logo $x_1 - x_2 \in \ker(f) = \{0_A\}$. Com isso $x_1 - x_2 = 0_A$, isto \'e, $x_1 = x_2$. Portanto $f$ \'e injetora. \hspace{.5cm} \qedsymbol
        \end{enumerate}
    \end{frame}

    \begin{frame}
        \begin{proposicao}
            Sejam $(A, +, \cdot)$ e $(B, \oplus, \otimes)$ an\'eis e seja $f : A \to B$ um homomorfismo sobrejetor de an\'eis.
            \begin{enumerate}[label={\roman*})]
                \item Se $A$ tem unidade, ent\~ao $B$ tem unidade e
                \[
                    f(1_A) = 1_B.
                \]

                \vspace{.5cm}

                \item Se $A$ tem unidade e $x \in A$ possui inverso multiplicativo, ent\~ao $f(x)$ tem inverso e
                \[
                    [f(x)]^{-1} = f(x^{-1}).
                \]

                \vspace{.5cm}
            \end{enumerate}
        \end{proposicao}
    \end{frame}

    \begin{frame}
        \noindent \textbf{\textit{Prova: }}
        \begin{enumerate}[label={\roman*})]
            \item Incialmente como num anel a unidade \'e \'unica, para mostrar que $B$ possui unidade basta mostrar que
            \[
                y\otimes f(1_A) = y = f(1_A)\otimes y
            \]
            para todo $y \in B$. Sendo assim, seja $y \in B$. Como $f$ \'e sobrejetor ent\~ao existe $x \in A$ tal que $f(x) = y$. Assim
            \begin{align*}
                y\otimes f(1_A) &= f(x) \otimes f(1_A) = f(x\cdot 1_A) = f(x) = y\\
                f(1_A)\otimes y &= f(1_A) \otimes f(x) = f(1_A\cdot x) = f(x) = y
            \end{align*}
            para todo $y \in B$. Portanto $B$ possui unidade e
            \[
                1_B = f(1_A).
            \]

            \seti
        \end{enumerate}
    \end{frame}

    \begin{frame}
        \begin{enumerate}[label={\roman*})]
            \conti

            \item Novamente, devido \'a unicidade do inverso em um anel, para mostrar que $f(x)$ possui inverso basta mostrar que
            \[
                f(x)\otimes f(x^{-1}) = 1_B = f(x^{-1})\otimes f(x)
            \]
            desde que $x \in A$ possua inverso multiplicativo. Sendo assim suponha que $x \in A$ possui inverso multiplicativo. Seja $x^{-1}$ o inverso multiplicativo de $x$ em $A$. Temos
            \begin{align*}
                f(x)\otimes f(x^{-1}) &= f(x\cdot x^{-1}) = f(1_A) = 1_B\\
                f(x^{-1})\otimes f(x) &= f(x^{-1}\cdot x) = f(1_A) = 1_B
            \end{align*}
            Portanto $f(x)$ possui inverso multiplicativo e
            \[
                [f(x)]^{-1} = f(x^{-1}),
            \]
            como quer{\'\i}amos. \hspace{.5cm} \qedsymbol
        \end{enumerate}
    \end{frame}

    \begin{frame}
        \begin{definicao}
            Seja $(A, +, \cdot)$ um anel comutativo. Um subconjunto n\~ao-vazio $I \sub A$ {\'e} chamado de \textbf{ideal} de $A$ se:
            \begin{enumerate}[label={\roman*})]
                \item para todos $x$, $y \in I$, temos $x - y \in I$.
                \item Para todo $\alpha \in A$ e todo $x \in I$, temos $\alpha\cdot x \in I$.
            \end{enumerate}
        \end{definicao}

        \begin{observacao}
            Quando $I = A$ ou $I = \{0_A\}$, dizemos que $I$ {\'e} um \textbf{ideal trivial}.
        \end{observacao}
    \end{frame}

    \begin{frame}
        \begin{proposicao}
            Seja $A$ um anel comutativo e $I$ um ideal de $A$. Ent{\~a}o:
            \begin{enumerate}[label={\roman*})]
                \item $0_{A}\in I$.
                \item $-x \in I$ para todo $x \in I$.
                \item Se $1_A \in I$, ent\~ao $I = A$.
            \end{enumerate}
        \end{proposicao}
        \begin{prova}
            \begin{enumerate}[label={\roman*})]
                \item Da defini\c{c}\~ao de ideal temos $\alpha \cdot x \in I$ para todo $x \in I$ e todo $\alpha \in A$.
                Assim dado $x \in I$ $0_A = 0_A \cdot x \in I$.

                \item Como $0_A \in I$, dado $x \in I$ da defini\c{c}\~ao de ideal segue que $0_A - x \in I$, isto \'e, $-x \in I$.

                \item Suponha que $1_A \in I$. Como $I$ {\'e} ideal, para todo $\alpha \in A$ e todo $x \in I$ devemos ter $\alpha\cdot x \in I$. Assim, em particular, $1_A \cdot x \in I$ para todo $x \in A$. Logo, $A\sub I$ e como $I\sub A$, ent{\~a}o $I = A$.
            \end{enumerate}
        \end{prova}
    \end{frame}

    \begin{frame}
        \begin{exemplos}
            \begin{enumerate}[label={\arabic*})]
                \item Em $\z$ todos os ideais n{\~a}o triviais s{\~a}o da forma $m\z$, $m > 1$.
                \item No anel $\z_p$, onde $p$ {\'e} um n{\'u}mero primo, os {\'u}nicos ideais  s{\~a}o os triviais $\{\overline{0}\}$ e $\z_p$.
                
                De fato, seja $I \sub \z_p$ um ideal, $I \neq \{\overline{0}\}$. Provemos que $I = \z_p$. Para isso,
                vamos provar que $\overline{1} \in I$. Seja $\overline{a} \in I$, $\overline{a} \neq \overline{0}$, pois $I \neq \{\overline{0}\}$. Como $p$ {\'e} primo, $mdc(a,p) = 1$, da{\'\i} existe $\overline{b} \in \z_p$, $\overline{b} \neq \overline{0}$, tal que $\overline{1} = \overline{a} \otimes \overline{b}$. Mas $I$ {\'e} ideal e $\overline{a} \in I$, logo $\overline{1} = \overline{a} \otimes \overline{b} \in I$.

                Portanto $I = \z_p$.

                \item Os {\'u}nicos ideais n{\~a}o triviais de $\z_8 = \{\overline{0}, \overline{1}, \overline{2}, \overline{3}, \overline{4}, \overline{5}, \overline{6}, \overline{7}\}$ s{\~a}o:
                \begin{align*}
                    I_1 &= \{\overline{0}, \overline{2}, \overline{4}, \overline{6}\}\\
                    I_2 &=\{\overline{0}, \overline{4}\}
                \end{align*}
            \end{enumerate}
        \end{exemplos}
    \end{frame}

    \begin{frame}
        \begin{definicao}
            Seja $I$ um ideal de um anel $(A, +, \cdot)$. Dados $x$, $y \in A$ dizemos que $x$ \textbf{\'e congruente a} $y$ \textbf{m\'odulo} $I$ quando $x-y \in I$. Neste caso, escrevemos $x\equiv y \pmod I$.
        \end{definicao}

        \begin{proposicao}
            A congru{\^e}ncia m{\'o}dulo $I$ {\'e} uma rela{\c c}{\~a}o de equival{\^e}ncia em $A \times A$, onde $A$ anel unit{\'a}rio.
        \end{proposicao}
        \begin{prova}
            Como $0 = 0_{A} \in I$ e para todo $x \in I$, $x - x = 0 \in I$, ent{\~a}o $x \equiv x \pmod I$.

            Suponha que $x\equiv y \pmod I$. Ent{\~a}o $x - y \in I$. Como $-1 \in A$, $y - x = -(x - y) = -[(x - y)1] = (x - y)(-1) \in I$, ou seja, $y\equiv x \pmod I$.

            Agora, se $x\equiv y \pmod I$ e $y\equiv z \pmod I$, ent{\~a}o $x - y \in I$ e $y - z \in I$. Da{\'\i}, $x - z = (x - z) + (y - z)\in I$, ou seja, $x\equiv z \pmod I$.

            Logo, {\'e} uma rela{\c c}{\~a}o de equival{\^e}ncia.
        \end{prova}
    \end{frame}

    \begin{frame}
        Seja $y \in A$. A classe de equival{\^e}ncia m{\'o}dulo $I$ de $y$ {\'e}
        \[
            C(y) = \{x \in A \mid x\equiv y \pmod I\} = \{x \in A \mid x - y \in I\}.
        \]

        Agora, $x - y \in I$ significa que existe $t \in I$, tal que $x - y = t$. Logo, $x = y + t$, onde $t \in I$.

        Assim,
        \[
            C(y) = \{y + t\mid t \in I\} = y + I.
        \]

        \begin{observacao}
            Denotamos por $y + I$ (ou $I + y$) a classe de equival{\^e}ncia m{\'o}dulo $I$ de $y \in A$. Denotamos por $\dfrac{A}{I}$ o conjunto de todas as classes de equival{\^e}ncia, tal conjunto {\'e} chamado de \textbf{quociente do anel $A$ pelo ideal $I$}.
        \end{observacao}
    \end{frame}

    \begin{frame}
        \begin{exemplos}
            \begin{enumerate}[label={\arabic*})]
                \item Seja $A$ um anel com unidade e $I_{1} = \{0\}$ e $I_{2} = A$ ideais. Ent\~ao:
                \begin{enumerate}[label={\roman*})]
                    \item Dado $x \in A$:
                    \[
                        C(x) = x + I_{1} = \{x + 0\} = \{x\}.
                    \]
                    Assim $\dfrac{A}{I_{1}} = \{x + I \mid x \in A\}$, logo existem tantas classes de equival{\^e}ncia quantos forem os elementos de $A$.

                    \item Para $I_{2} = A$ temos:
                    \[
                        C(0_A) = 0_A + I = \{0_A + t \mid t \in I_{2}\}.
                    \]
                    Como $I_2 = A$, para todo $x \in A$ temos $x \in C(0_A)$ logo existem uma \'unica classe de equival\^encia
                    e $\dfrac{A}{I_{2}} = \{0_{A} + I\}$.
                \end{enumerate}
                \seti
            \end{enumerate}
        \end{exemplos}
    \end{frame}

    \begin{frame}
        \begin{exemplos}
            \begin{enumerate}[label={\arabic*})]
                \conti

                \item Seja $A = \z$. Sabemos que os ideais de $\z$ s{\~a}o da forma $m\z$, $m > 1$. Seja $I = m\z$ um ideal de $\z$. Assim $x\equiv y \pmod I$ se, e s\'o se, $x - y \in I$. Mais isso ocorre se, e somente se, $x - y = mk $, para algum $k \in \z$. Logo $x\equiv y \pmod I$ se, e s\'o se, $m\mid (x - y)$. Portanto, $\dfrac{\z}{I} = \z_m$.
            \end{enumerate}
        \end{exemplos}
    \end{frame}

    \begin{frame}
        Agora seja $I$ ideal e $A$ um anel. Temos
        \[
            \dfrac{A}{I} = \{y + I \mid y \in A\}
        \]
        onde $y + I = \{y + t \mid t \in I\}$ e $y \in A$.

        Vamos definir uma soma $\oplus$ e um produto $\otimes$ em $\dfrac{A}{I}$ por
        \begin{align*}
            (x + I)\oplus(y + I) &= (x + y) + I\\
            (x + I)\otimes(y + I) &= (xy) + I
        \end{align*}
        para $x + I$, $y + I \in \dfrac{A}{I}$.
    \end{frame}

    \begin{frame}
        Verifiquemos que a soma e o produto em $\dfrac{A}{I}$ n{\~a}o dependem do representante da classe de equival{\^e}ncia.
        Para isso sejam $x_1 + I$, $x_2 + I$, $y_1 + I$, $y_2 + I \in \dfrac{A}{I}$ tais que
        \begin{align*}
            x_1 + I &= x_2 + I\\
            y_1 + I &= y_2 + I  
        \end{align*}

        Ent{\~a}o
        \begin{align*}
            (x_1 + I) \oplus (y_1 + I) &= (x_1 + y_1) + I\\
            (x_2 + I) \oplus (y_2 + I) &= (x_2 + y_2) + I
        \end{align*}
    \end{frame}

    \begin{frame}
        Como $x_1 + I = x_2 + I$, ent{\~a}o $x_1 - x_2 \in I$ e como $y_1 + I = y_2 + I$, ent{\~a}o $y_1 = y_2 \in I$. Mas $I$ {\'e} ideal, logo $(x_1 - x_2) + (y_1 - y_2) = (x_1 + y_1) - (x_2 + y_2) \in I$, ou seja
        \[
            (x_1 + I) \oplus (y_1 + I) = (x_2 + I) \oplus (y_2 + I).
        \]
    \end{frame}

    \begin{frame}
        Agora,
        \begin{align*}
            (x_1 + I) \otimes (y_1 + I) &= (x_1y_1) + I\\
            (x_2 + I) \otimes (y_2 + I) &= (x_2y_2) + I
        \end{align*}

        Como $(x_1 - x_2)y \in I$ e $(y_1 - y_2)x_2 \in I$ ent\~ao
        \begin{align*}
            &(x_1 - x_2)y_1 + (y_1 - y_2)x_2 \in I\\
            &x_1y_2-\underbrace{x_2y_1 + y_1x_2}_{= 0} - y_2x_2 \in I\\
            &x_1y_1 - x_2y_2\in I,
        \end{align*}
        ou seja, $xy + I = x_2y_2 + I$. Portanto,
        \[
            (x_1 + I) \otimes (y + I) = (x_2 + I) \otimes (y_2 + I).
        \]
    \end{frame}

    \begin{frame}
        \begin{teorema}
            Seja $(A, +, \cdot)$ um anel associativo, comutativo e com unidade. Ent{\~a}o, se $I$ {\'e} um ideal de $A$,
            o quociente $\dfrac{A}{I}$ com as opera{\c c}{\~o}es $\oplus$ e $\otimes$ {\'e} um anel associativo,
            comutativo e com unidade. O elemento neutro da soma {\'e} a classe $0_{A} + I$ e unidade do produto {\'e} $1_{A} + I$.
        \end{teorema}
    \end{frame}







\end{document}