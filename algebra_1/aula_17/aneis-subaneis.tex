%!TEX program = xelatex
% !TEX encoding = ISO-8859-1
\def\ano{2020}
\def\semestre{1}
\def\disciplina{\'Algebra 1}
\def\turma{C}
\def\autor{Jos\'e Ant\^onio O. Freitas}
\def\instituto{MAT-UnB}

\documentclass{beamer}
\usetheme{Madrid}
\usecolortheme{beaver}
% \mode<presentation>
\usepackage{caption}
\usepackage{textpos}
\usepackage{amssymb}
\usepackage{amsmath,amsfonts,amsthm,amstext}
\usepackage[brazil]{babel}
% \usepackage[latin1]{inputenc}
\usepackage{graphicx}
\graphicspath{{/home/jfreitas/GitHub_Repos/video_aulas/logo/}{D:/Dropbox/imagens-latex/}}
\usepackage{enumitem}
\usepackage{multicol}
\usepackage{answers}
\usepackage{tikz,ifthen}
\usetikzlibrary{lindenmayersystems}
\usetikzlibrary[shadings]
\newtheorem{definicao}{Defini\c{c}\~ao}[section]
\newtheorem{definicoes}{Defini\c{c}\~oes}[section]
\newtheorem{exemplo}{Exemplo}[section]
\newtheorem{exemplos}{Exemplos}[section]
\newtheorem{exercicio}{Exerc{\'\i}cio}
\newtheorem{observacao}{Observa{\c c}{\~a}o:}[section]
\newtheorem{observacoes}{Observa{\c c}{\~o}es:}[section]
\newtheorem*{solucao}{Solu{\c c}{\~a}o:}
\newtheorem{proposicao}{Proposi\c{c}\~ao}
\newtheorem{lema}{Lema}
\newtheorem{teorema}{Teorema}
\newtheorem{corolario}{Corol\'ario}
\newenvironment{prova}[1][Prova]{\noindent\textbf{#1:} }{\qedsymbol}%{\ \rule{0.5em}{0.5em}}
\newcommand{\nsub}{\varsubsetneq}
\newcommand{\vaz}{\emptyset}
\newcommand{\im}{{\rm Im\,}}
\newcommand{\sub}{\subseteq}
\newcommand{\n}{\mathbb{N}}
\newcommand{\z}{\mathbb{Z}}
\newcommand{\rac}{\mathbb{Q}}
\newcommand{\real}{\mathbb{R}}
\newcommand{\complex}{\mathbb{C}}
\newcommand{\cp}[1]{\mathbb{#1}}
\newcommand{\ch}{\mbox{\textrm{car\,}}\nobreak}
\newcommand{\vesp}[1]{\vspace{ #1  cm}}
\newcommand{\compcent}[1]{\vcenter{\hbox{$#1\circ$}}}
\newcommand{\comp}{\mathbin{\mathchoice
{\compcent\scriptstyle}{\compcent\scriptstyle}
{\compcent\scriptscriptstyle}{\compcent\scriptscriptstyle}}}

\title{Anéis - Subanéis}
\author[\autor]{\autor}
\institute[\instituto]{\instituto}
\date{\today}

\begin{document}
    \begin{frame}
        \maketitle
    \end{frame}

    \logo{\includegraphics[scale=.1]{logo-MAT.png}\vspace*{8.5cm}}
    
    \begin{frame}
        \begin{definicao}
            Um anel comutativo $(A, + , \cdot)$ {\'e} dito ser um \textbf{anel de integridade} quando para todos 
            $x$, $y \in A$, se $xy = 0_A$, ent{\~a}o $x = 0_A$ ou $y = 0_a$. Um anel de integridade tamb{\'e}m {\'e} chamado de \textbf{dom{\'\i}nio de integridade} ou simplesmente de \textbf{dom{\'\i}nio}.
        \end{definicao}

        \begin{observacao}
            Se $x$ e $y$ s{\~a}o elementos n{\~a}o nulos de um anel $A$ tais que $xy = 0_A$, ent{\~a}o $x$ e $y$ s{\~a}o chamados de \textbf{divisores pr{\'o}prios de zero}.
        \end{observacao}
    \end{frame}

    \begin{frame}
        \begin{exemplos}
            \begin{enumerate}[label={\arabic*})]
                \item Os an{\'e}is $\z$, $\rac$, $\real$, $\complex$ s{\~a}o an{\'e}is de integridade.
                
                \vspace{.5cm}

                \item Em geral $\z_m$ n{\~a}o {\'e} anel de integridade, por exemplo, em $\z_4$, $\overline{2} \neq \overline{0}$, no entanto $\overline{2}\otimes \overline{2} = \overline{4} = \overline{0}$.

                \vspace{.5cm}

                \seti
            \end{enumerate}
        \end{exemplos}
    \end{frame}

    \begin{frame}
        \begin{exemplos}
            \begin{enumerate}[label={\arabic*})]
                \conti

                \item $M_{n}(\real)$ n{\~a}o {\'e} um anel de integridade, por exemplo, em $M_{2}(\real)$
                \begin{align*}
                    A &= \begin{bmatrix}
                        1 & 0\\
                        0 & 0
                    \end{bmatrix} \neq \begin{bmatrix}
                        0 & 0\\
                        0 & 0       
                    \end{bmatrix},\qquad 
                    B = \begin{bmatrix}
                        0 & 0\\
                        1 & 0
                    \end{bmatrix} \neq \begin{bmatrix}
                        0 & 0\\
                        0 & 0
                    \end{bmatrix}\\
                    AB & =\begin{bmatrix}
                        0 & 0\\
                        0 & 0
                    \end{bmatrix}
                \end{align*}

                \vspace{.5cm}
                
                \item Suponha que $m = nk$, $m > n > 1$ e $m > k > 1$. Logo, em $\z_m$, $\overline{n} \neq \overline{0}$ e $\overline{k} \neq \overline{0}$ e no entanto $\overline{n} \otimes \overline{k} = \overline{m} = \overline{0}$. Logo, se $m$ n{\~a}o {\'e} primo, ent{\~a}o $\z_m$ n{\~a}o {\'e} um anel de integridade. Agora, suponha que $m = p$ primo. Sejam $\overline{x}$, $\overline{y} \in \z_m$ tais que $\overline{x}\otimes \overline{y} = \overline{0}$, ou seja, $xy \equiv 0 \pmod p$. Da{\'\i} $p\mid xy$. Logo $p\mid x$ ou $p\mid y$. Portanto, $\overline{x} = \overline{0}$ ou $\overline{y} = \bar{0}$. Assim, $\z_m$ {\'e} anel de integridade se, e somente se, $m$ {\'e} primo.
            \end{enumerate}
        \end{exemplos}
    \end{frame}

    \begin{frame}
        \begin{definicao}
            Seja $(A, +, \cdot)$ um anel. Dizemos que um subconjunto n{\~a}o vazio $B\subseteq A$ {\'e} um \textbf{subanel} de $A$ quando $(B, +, \cdot)$ \'e um anel.
        \end{definicao}

        \begin{exemplos}
            \begin{enumerate}[label={\arabic*})]
                \item Todo anel $A$ sempre tem dois suban{\'e}is: $\{0_{A}\}$ e $A$, que s{\~a}o chamados de \textbf{suban{\'e}is triviais}.

                \vspace{.5cm}

                \item Em $(\z_4,\oplus,\otimes)$ o conjunto $B = \{\overline{0}, \overline{2}\}$ \'e um subanel.

                \vspace{.5cm}
                
                \item No anel $\z$, o conjunto $m\z$, $m > 1$ {\'e} um subanel de $\z$.

                \vspace{.5cm}
            \end{enumerate} 
        \end{exemplos}
    \end{frame}

    \begin{frame}
        \begin{proposicao}
            Seja $(A, +,\cdot)$ um anel. Um subconjunto n{\~a}o vazio $B\subseteq A$ {\'e} um subanel de $A$ se, e somente se, $x - y \in B$ e $x\cdot y \in B$ para todos $x$, $y \in B$.
        \end{proposicao}

        \noindent \textbf{\textit{Prova: }}FAZER!!!!!

        \begin{exemplos}
            COLOCAR EXEMPLOS
        \end{exemplos}
    \end{frame}
\end{document}