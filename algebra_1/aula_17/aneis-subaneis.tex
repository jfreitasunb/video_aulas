%!TEX program = xelatex
% !TEX encoding = ISO-8859-1
\def\ano{2020}
\def\semestre{1}
\def\disciplina{\'Algebra 1}
\def\turma{C}
\def\autor{Jos\'e Ant\^onio O. Freitas}
\def\instituto{MAT-UnB}

\documentclass{beamer}
\usetheme{Madrid}
\usecolortheme{beaver}
% \mode<presentation>
\usepackage{caption}
\usepackage{textpos}
\usepackage{amssymb}
\usepackage{amsmath,amsfonts,amsthm,amstext}
\usepackage[brazil]{babel}
% \usepackage[latin1]{inputenc}
\usepackage{graphicx}
\graphicspath{{/home/jfreitas/GitHub_Repos/video_aulas/logo/}{D:/Dropbox/imagens-latex/}}
\usepackage{enumitem}
\usepackage{multicol}
\usepackage{answers}
\usepackage{tikz,ifthen}
\usetikzlibrary{lindenmayersystems}
\usetikzlibrary[shadings]
\newtheorem{definicao}{Defini\c{c}\~ao}[section]
\newtheorem{definicoes}{Defini\c{c}\~oes}[section]
\newtheorem{exemplo}{Exemplo}[section]
\newtheorem{exemplos}{Exemplos}[section]
\newtheorem{exercicio}{Exerc{\'\i}cio}
\newtheorem{observacao}{Observa{\c c}{\~a}o:}[section]
\newtheorem{observacoes}{Observa{\c c}{\~o}es:}[section]
\newtheorem*{solucao}{Solu{\c c}{\~a}o:}
\newtheorem{proposicao}{Proposi\c{c}\~ao}
\newtheorem{lema}{Lema}
\newtheorem{teorema}{Teorema}
\newtheorem{corolario}{Corol\'ario}
\newenvironment{prova}[1][Prova]{\noindent\textbf{#1:} }{\qedsymbol}%{\ \rule{0.5em}{0.5em}}
\newcommand{\nsub}{\varsubsetneq}
\newcommand{\vaz}{\emptyset}
\newcommand{\im}{{\rm Im\,}}
\newcommand{\sub}{\subseteq}
\newcommand{\n}{\mathbb{N}}
\newcommand{\z}{\mathbb{Z}}
\newcommand{\rac}{\mathbb{Q}}
\newcommand{\real}{\mathbb{R}}
\newcommand{\complex}{\mathbb{C}}
\newcommand{\cp}[1]{\mathbb{#1}}
\newcommand{\ch}{\mbox{\textrm{car\,}}\nobreak}
\newcommand{\vesp}[1]{\vspace{ #1  cm}}
\newcommand{\compcent}[1]{\vcenter{\hbox{$#1\circ$}}}
\newcommand{\comp}{\mathbin{\mathchoice
{\compcent\scriptstyle}{\compcent\scriptstyle}
{\compcent\scriptscriptstyle}{\compcent\scriptscriptstyle}}}

\title{Anéis - Subanéis}
\author[\autor]{\autor}
\institute[\instituto]{\instituto}
\date{\today}

\begin{document}
    \begin{frame}
        \maketitle
    \end{frame}

    \logo{\includegraphics[scale=.1]{logo-MAT.png}\vspace*{8.5cm}}
    
    \begin{frame}
        \begin{observacao}
            Seja $(A, \oplus, \cdot)$ \pause um anel. \pause Para simplificar a nota\c{c}\~ao \pause vamos denotar a opera\c{c}\~ao $\oplus$ \pause por $+$ \pause e a opera\c{c}\~ao $\otimes$ \pause por $\cdot$ \pause e assim escrever simplesmente \pause que $(A, +, \cdot)$ \pause \'e um anel.\pause
        \end{observacao}
    \end{frame}

    \begin{frame}
        \begin{proposicao}
            Seja $(A, + , \cdot)$ um anel. \pause Ent\~ao:\pause
            \begin{enumerate}[label={\roman*})]
                \item O elemento neutro {\'e} {\'u}nico.\pause

                \vspace{.5cm}

                \item Para cada $x \in A$ \pause existe um {\'u}nico oposto.\pause

                \vspace{.5cm}
                
                \item Para todo $x \in A$, \pause
                \[
                    -(-x) = x.\pause
                \]

                \vspace{.5cm}
                
                \item Dados $x_{1}$, \pause $x_{2}$, \pause \dots, $x_n \in A$, \pause $n \geqslant 2$, \pause ent{\~a}o\pause
                \[
                    -(x_1 + x_2 + \dots + x_n) \pause = (-x_1) \pause + (-x_2) \pause + \dots + (-x_n).\pause
                \]

                \vspace{.2cm}

                \seti
            \end{enumerate}
        \end{proposicao}
    \end{frame}

    \begin{frame}
        \begin{proposicao}
            \begin{enumerate}[label={\roman*})]
                \conti
                
                \item Para todos $\alpha$, \pause $x$, \pause $y \in A$, \pause se
                \[
                    \alpha + x \pause = \alpha + y,\pause
                \]
                ent{\~a}o $x = y$.\pause

                \vspace{.5cm}
                
                \item Para todo $x \in A$, \pause 
                \[
                    x\cdot 0_A \pause = 0_A \pause = 0_A\cdot x.\pause
                \]

                \vspace{.5cm}
            \end{enumerate}
        \end{proposicao}
    \end{frame}

    \begin{frame}
        \begin{proposicao}
            \begin{enumerate}[label={\roman*})]
                \conti

                \item Para todos $x$, \pause $y \in A$, \pause temos\pause
                \[
                    x(-y) \pause = (-x)y \pause = -(xy).\pause
                \]

                \vspace{.5cm}
                
                \item Para todos $x$, \pause $y \in A$, \pause
                \[
                    xy \pause = (-x)(-y).\pause
                \]

                \vspace{.5cm}
            \end{enumerate}
        \end{proposicao}
    \end{frame}

    \begin{frame}
        \noindent \textbf{\textit{Prova:} }\pause
        \begin{enumerate}[label={\roman*})]
            \item Suponha que existam $0_1$, \pause $0_2\in A$ \pause elementos neutros \pause de $A$. \pause Assim\pause
                
            \[
                x + 0_1 \pause = x\pause \quad \mbox{e}\quad x + 0_2 \pause = x\pause
            \]
            para todo $x \in A$. \pause Assim\pause

            \[
                   0_1 \pause = 0_1 +\pause 0_2 \pause = 0_2\pause
            \]
            
            e portanto o elemento neutro \'e \'unico.\pause
                
            \vspace{.5cm}

            \seti
        \end{enumerate}
    \end{frame}

    \begin{frame}
        \begin{enumerate}[label={\roman*})]
            \conti

            \item De fato, \pause dado $x \in A$ \pause suponha que existam $y_1$, \pause $y_2\in A$ \pause tais que\pause
            \[
                x + y_1 \pause = 0_A \pause \quad \mbox{e}\quad x + y_2 \pause = 0_A.\pause
            \]
            Da{\'\i}\pause
            \[
                y_1 \pause = y_2 \pause + 0_A \pause = y_1 \pause + (x + y_2) \pause = (y_1 + x) \pause + y_2 \pause = 0_A\pause  + y_2 \pause= y_2.\pause
            \]
            Logo o oposto de $x$ \'e \'unico \pause e da{\'\i} ser\'a denotado por $-x$.\pause
                
            \vspace{.5cm}

            \item Dado $x \in A$, \pause ent\~ao $-x$ {\'e} oposto de $x$, \pause isto {\'e}, \pause $x \pause + (-x) \pause = 0_A$. \pause Logo o oposto de $(-x)$ \pause {\'e} $x$, \pause ou seja, \pause $-(-x) \pause = x$.\pause
            
            \seti
        \end{enumerate}
    \end{frame}

    \begin{frame}
            \begin{enumerate}[label={\roman*})]
                \conti

                \item Segue usando indu\c{c}\~ao sobre $n$.\pause

                \vspace{.5cm}

                \item Suponha que $\alpha + x \pause = \alpha + y$. \pause Seja $-\alpha$ \pause o oposto de $\alpha$. \pause Da{\'\i}\pause
                \begin{center}
                    $x = 0_A \pause + x = [(-\alpha) + \alpha] \pause + x = \pause (-\alpha) \pause + (\alpha + x) \pause = (-\alpha) + \pause (\alpha + y) \pause = [(-\alpha) + \alpha] \pause + y \pause = 0_A + y \pause = y$\pause
                \end{center}
                como quer{\'\i}amos.\pause
                \seti
            \end{enumerate}
    \end{frame}

    \begin{frame}
            \begin{enumerate}[label={\roman*})]
                \conti
                \item Temos \pause
                \[
                    x\cdot 0_A + \pause 0_A \pause = x\cdot 0_A \pause = x\cdot(0_A + 0_A) \pause = x\cdot 0_A \pause + x\cdot 0_A.\pause
                \]
                Assim do item anterior \pause segue que $x\cdot 0_A \pause = 0_A$.\pause

                \vspace{.5cm}

                \item Provemos que \pause $x(-y) \pause = -(xy)$\pause :
                \[
                    x(-y) \pause + xy \pause = x[(-y) + y] \pause = x\cdot 0_A \pause = 0_A,\pause
                \]
                portanto $-xy \pause = x(-y)$.\pause

                \vspace{.5cm}
                
                \item Basta usar o caso anterior.\pause
            \end{enumerate}
    \end{frame}

    \begin{frame}
        \begin{definicao}
            Seja $(A, +, \cdot)$ um anel. \pause Dizemos que um subconjunto n{\~a}o vazio \pause $B\subseteq A$ \pause {\'e} um \textbf{subanel} de $A$ \pause quando $(B, +, \cdot)$ \'e um anel.\pause
        \end{definicao}

        \begin{exemplos}
            \begin{enumerate}[label={\arabic*})]
                \item Todo anel $A$ sempre tem dois suban{\'e}is: \pause $\{0_{A}\}$ \pause e $A$, \pause que s{\~a}o chamados de \textbf{suban{\'e}is triviais}.\pause

                \vspace{.5cm}

                \item Em $(\z_4,\oplus,\otimes)$ \pause o conjunto $B = \{\overline{0}, \overline{2}\}$ \'e um subanel.\pause

                \vspace{.5cm}
                
                \item No anel $\z$, \pause o conjunto $m\z$, $m > 1$ \pause {\'e} um subanel de $\z$.\pause

                \vspace{.5cm}
            \end{enumerate} 
        \end{exemplos}
    \end{frame}

    \begin{frame}
        \begin{proposicao}
            Seja $(A, +,\cdot)$ um anel. \pause Um subconjunto n{\~a}o vazio \pause $B\subseteq A$ \pause {\'e} um subanel de $A$ \pause se, e somente se, \pause $x - y = x + (-y) \in B$ \pause e $x\cdot y \in B$ \pause para todos $x$, $y \in B$.\pause
        \end{proposicao}

        \noindent \textbf{\textit{Prova: }}\pause
    \end{frame}

    \begin{frame}
        \begin{exemplos}
            \begin{enumerate}[label={\arabic*})]
                \item Em $(\z_4,\oplus,\otimes)$ \pause o conjunto $B = \{\overline{0}, \overline{2}\}$ \'e um subanel.\pause

                \vspace{5cm}
                
                \seti
            \end{enumerate}
        \end{exemplos}
    \end{frame}

    \begin{frame}
        \begin{exemplos}
            \begin{enumerate}[label={\arabic*})]
                \conti

                \item No anel $\z$, \pause o conjunto $m\z$, $m > 1$ \pause {\'e} um subanel de $\z$.\pause

                \vspace{5cm}
            \end{enumerate}
        \end{exemplos}
    \end{frame}

    \begin{frame}
        \begin{exemplos}
            \begin{enumerate}[label={\arabic*})]
                \conti

                \item No anel $(\rac, \star, \odot)$ \pause onde as opera\c{c}\~oes $\star$ e $\odot$ em $\rac$ definidas por\pause
                \begin{center}
                    $x \star y = x + y - 8$\\\pause
                    $x \odot y = x + y - \dfrac{xy}{8}.$\pause
                \end{center}
                Quais dos seguintes subconjuntos s\~ao suban\'eis?\pause
                \begin{enumerate}
                    \item[(a)] $B = \{2k \mid k \in \z\}$\pause

                    \vspace{.5cm}

                    \item[(b)] $C = \{8k \mid k \in \z\}$\pause
                    \vspace{.5cm}
                \end{enumerate}
            \end{enumerate}
        \end{exemplos}
    \end{frame}
\end{document}