%!TEX program = xelatex
% !TEX encoding = ISO-8859-1
\def\ano{2020}
\def\semestre{1}
\def\disciplina{\'Algebra 1}
\def\turma{C}
\def\autor{Jos\'e Ant\^onio O. Freitas}
\def\instituto{MAT-UnB}

\documentclass{beamer}
\usetheme{Madrid}
\usecolortheme{beaver}
% \mode<presentation>
\usepackage{caption}
\usepackage{textpos}
\usepackage{amssymb}
\usepackage{amsmath,amsfonts,amsthm,amstext}
\usepackage[brazil]{babel}
% \usepackage[latin1]{inputenc}
\usepackage{graphicx}
\graphicspath{{/home/jfreitas/GitHub_Repos/video_aulas/logo/}{D:/Dropbox/imagens-latex/}}
\usepackage{enumitem}
\usepackage{multicol}
\usepackage{answers}
\usepackage{tikz,ifthen}
\usetikzlibrary{lindenmayersystems}
\usetikzlibrary[shadings]
\newtheorem{definicao}{Defini\c{c}\~ao}[section]
\newtheorem{definicoes}{Defini\c{c}\~oes}[section]
\newtheorem{exemplo}{Exemplo}[section]
\newtheorem{exemplos}{Exemplos}[section]
\newtheorem{exercicio}{Exerc{\'\i}cio}
\newtheorem{observacao}{Observa{\c c}{\~a}o:}[section]
\newtheorem{observacoes}{Observa{\c c}{\~o}es:}[section]
\newtheorem*{solucao}{Solu{\c c}{\~a}o:}
\newtheorem{proposicao}{Proposi\c{c}\~ao}
\newtheorem{lema}{Lema}
\newtheorem{teorema}{Teorema}
\newtheorem{corolario}{Corol\'ario}
\newenvironment{prova}[1][Prova]{\noindent\textbf{#1:} }{\qedsymbol}%{\ \rule{0.5em}{0.5em}}
\newcommand{\nsub}{\varsubsetneq}
\newcommand{\vaz}{\emptyset}
\newcommand{\im}{{\rm Im\,}}
\newcommand{\sub}{\subseteq}
\newcommand{\n}{\mathbb{N}}
\newcommand{\z}{\mathbb{Z}}
\newcommand{\rac}{\mathbb{Q}}
\newcommand{\real}{\mathbb{R}}
\newcommand{\complex}{\mathbb{C}}
\newcommand{\cp}[1]{\mathbb{#1}}
\newcommand{\ch}{\mbox{\textrm{car\,}}\nobreak}
\newcommand{\vesp}[1]{\vspace{ #1  cm}}
\newcommand{\compcent}[1]{\vcenter{\hbox{$#1\circ$}}}
\newcommand{\comp}{\mathbin{\mathchoice
{\compcent\scriptstyle}{\compcent\scriptstyle}
{\compcent\scriptscriptstyle}{\compcent\scriptscriptstyle}}}

\title{Homomorfismo de Grupos}
\author[\autor]{\autor}
\institute[\instituto]{\instituto}
\date{\today}

\begin{document}
    \begin{frame}
        \maketitle
    \end{frame}

    \logo{\includegraphics[scale=.1]{logo-MAT.png}\vspace*{8.5cm}}

    \begin{frame}
        \begin{definicao}
            Dados dois grupos $(G, *)$ \pause e $(H,\triangle)$ \pause dizemos que uma fun\c{c}\~ao $f : G \to H$ \pause \'e um \textbf{homomorfismo de grupos} se \pause
            \[
                f(x * y) = \pause f(x)\triangle f(y) \pause
            \]
            para todos $x$, $y \in G$.
        \end{definicao}
    \end{frame}

    \begin{frame}
        \begin{observacao}
            Sejam $(G, *)$ \pause, $(H, \triangle)$ grupos \pause e $f : G \to H$ um homomorfismo. \pause
            \begin{enumerate}[label={\arabic*})]
                \item Se $G = H$, \pause neste caso $f : G \to G$ \pause \'e chamado de um \textbf{endomorfimos} de grupos.\pause
                \item Se $f : G \to H$ \'e uma fun\c{c}\~ao injetora, \pause ent\~ao dizemos que $f$ \'e um \textbf{monomorfismo} de grupos.\pause
                \item Se $f : G \to H$ \'e uma fun\c{c}\~ao sobrejetora, \pause ent\~ao dizemos que $f$ \'e um \textbf{epimorfismo} de grupos.\pause
                \item Se $f : G \to H$ \'e uma fun\c{c}\~ao bijetora, \pause ent\~ao dizemos que $f$ \'e um \textbf{isomorfismo} de grupos.\pause
                \item Se $f : G \to G$ \'e uma fun\c{c}\~ao bijetora, \pause ent\~ao dizemos que $f$ \'e um \textbf{automorfismo} de grupos.
            \end{enumerate}
        \end{observacao}
    \end{frame}

    \begin{frame}
        \begin{exemplos}
            \begin{enumerate}[label={\arabic*})]
                \item A fun\c{c}\~ao $f : \z \to \complex$ dada por $f(x) = i^x$ \'e um homomorfismo de $(\z, +)$ em $(\complex, \cdot)$.

                \item A fun\c{c}\~ao $f : \real^*_+ \to \real$ dada por $f(x) = \ln(x)$ \'e um homomorfismo de $(\real^*_+, \cdot)$ em $(\real, +)$.

                \item Sejam $m$ um inteiro positivo fixo. A fun\c{c}\~ao $f: \z \to \z_m$ definida por $f(x) = \overline{x}$ \'e um homomorfimos de $(\z, +)$ em $(\z_m, \oplus)$.

                \item Sejam $(G, *)$ um grupo, $z\in G$ um elemento fixado e $z^{-1}$ seu inverso. Então a aplica{\c c}{\~a}o $i_z: G\to G$ definida por $i_z(x)=z^{-1}*x*z$, para todo $x \in G$, {\'e} um isomorfismo de grupos.
            \end{enumerate}
        \end{exemplos}
    \end{frame}
\end{document}