%!TEX program = xelatex
%!TEX encoding = ISO-8859-1
\def\ano{2020}
\def\semestre{1}
\def\disciplina{\'Algebra 1}
\def\turma{C}
\def\autor{Jos\'e Ant\^onio O. Freitas}
\def\instituto{MAT-UnB}

\documentclass{beamer}
\usetheme{Madrid}
\usecolortheme{beaver}
% \mode<presentation>
\usepackage{caption}
\usepackage{textpos}
\usepackage{amssymb}
\usepackage{amsmath,amsfonts,amsthm,amstext}
\usepackage[brazil]{babel}
% \usepackage[latin1]{inputenc}
\usepackage{graphicx}
\graphicspath{{/home/jfreitas/GitHub_Repos/video_aulas/logo/}{D:/Dropbox/imagens-latex/}}
\usepackage{enumitem}
\usepackage{multicol}
\usepackage{answers}
\usepackage{tikz,ifthen}
\usetikzlibrary{lindenmayersystems}
\usetikzlibrary[shadings]
\newtheorem{definicao}{Defini\c{c}\~ao}[section]
\newtheorem{definicoes}{Defini\c{c}\~oes}[section]
\newtheorem{exemplo}{Exemplo}[section]
\newtheorem{exemplos}{Exemplos}[section]
\newtheorem{exercicio}{Exerc{\'\i}cio}
\newtheorem{observacao}{Observa{\c c}{\~a}o:}[section]
\newtheorem{observacoes}{Observa{\c c}{\~o}es:}[section]
\newtheorem*{solucao}{Solu{\c c}{\~a}o:}
\newtheorem{proposicao}{Proposi\c{c}\~ao}
\newtheorem{lema}{Lema}
\newtheorem{teorema}{Teorema}
\newtheorem{corolario}{Corol\'ario}
\newenvironment{prova}[1][Prova]{\noindent\textbf{#1:} }{\qedsymbol}%{\ \rule{0.5em}{0.5em}}
\newcommand{\nsub}{\varsubsetneq}
\newcommand{\vaz}{\emptyset}
\newcommand{\im}{{\rm Im\,}}
\newcommand{\sub}{\subseteq}
\newcommand{\n}{\mathbb{N}}
\newcommand{\z}{\mathbb{Z}}
\newcommand{\rac}{\mathbb{Q}}
\newcommand{\real}{\mathbb{R}}
\newcommand{\complex}{\mathbb{C}}
\newcommand{\cp}[1]{\mathbb{#1}}
\newcommand{\ch}{\mbox{\textrm{car\,}}\nobreak}
\newcommand{\vesp}[1]{\vspace{ #1  cm}}
\newcommand{\compcent}[1]{\vcenter{\hbox{$#1\circ$}}}
\newcommand{\comp}{\mathbin{\mathchoice
{\compcent\scriptstyle}{\compcent\scriptstyle}
{\compcent\scriptscriptstyle}{\compcent\scriptscriptstyle}}}

\title{Isomorfismos de Grupos}
\author[\autor]{\autor}
\institute[\instituto]{\instituto}
\date{\today}

\begin{document}
    \begin{frame}
        \maketitle
    \end{frame}

    \logo{\includegraphics[scale=.1]{logo-MAT.png}\vspace*{8.5cm}}

    \begin{frame}
        Considere o grupo multiplicativo $G = \{1, -1\}$ e o grupo $S_2$ das permutações sobre o conjunto $\{1,2\}$. Aqui
        \[
            S_2 = \left\{f_0 = \begin{pmatrix}
                1 & 2\\1 & 2
            \end{pmatrix}; f_1 = \begin{pmatrix}
                1 & 2\\2 & 1
            \end{pmatrix}\right\}.
        \]
    \end{frame}

    \begin{frame}
        Temos
        \begin{table}
            \begin{minipage}{.5\linewidth}
                \caption*{$G$}
                \centering
                \begin{tabular}{|c|c|c|}
                    \hline
                    $\cdot$ & 1 & -1\\
                    \hline
                    1 & 1 & -1\\
                    \hline
                    -1 & -1 & 1\\
                    \hline
                \end{tabular}
            \end{minipage}%
            \pause
            \begin{minipage}{.5\linewidth}
                \caption*{$S_2$}
                \centering
                \begin{tabular}{|c|c|c|}
                    \hline
                    $\cdot$ & $f_0$ & $f_1$\\
                    \hline
                    $f_0$ & $f_0$ & $f_1$\\
                    \hline
                    $f_1$ & $f_1$ & $f_0$\\
                    \hline
                \end{tabular}
            \end{minipage} 
        \end{table}

        \vspace{.4cm}

        Defina $\sigma : G \to S_2$ por
        \begin{center}
            \begin{tabular}{c}
                $\sigma(1) = f_0$\\
                $\sigma(-1) = f_1$.
            \end{tabular}
        \end{center}
    \end{frame}

    \begin{frame}
        Da definição de $\sigma$  é fácil ver que essa função é bijetora. Além disso,
        \begin{center}
            \begin{tabular}{c}
                $\sigma(1) \circ \sigma(1) = f_1 \circ f_1 = f_1 = \sigma(1) = \sigma(1 \cdot 1)$\\
                \\
                $\sigma(1) \circ \sigma(-1) = f_1 \circ f_2 = f_2 = \sigma(-1) = \sigma(1 \cdot -1)$\\
                \\
                $\sigma(-1) \circ \sigma(1) = f_2 \circ f_1 = f_2 = \sigma(-1) = \sigma(-1 \cdot 1)$\\
                \\
                $\sigma(-1) \circ \sigma(-1) = f_2 \circ f_2 = f_1 = \sigma(1) = \sigma(-1 \cdot -1)$\\
            \end{tabular}
        \end{center}
        ou seja, a função $\sigma$ é um homomorfismo de $G$ em $S_2$.

        Como $\sigma$ também é bijetora, então $\sigma$ é um isomorfismo de $G$ em $S_2$. Nesse caso, dizemos que $G$ e $S_2$ são grupos isomorfos e denotamos isso escrevendo $G \cong S_2$.
    \end{frame}

    \begin{frame}
        \begin{definicao}
            Sejam $(G, *)$ e $(H, \triangle)$ grupos. Se existe $f : G \to H$ um isomorfismo, diremos que $G$ e $H$ são \textbf{grupos isomorfos} e denotaremos esse fato escrevendo $G \cong H$.
        \end{definicao}
    \end{frame}

    \begin{frame}
        \begin{proposicao}
            Sejam $G$ e $H$ grupos multiplicativos. Se $f : G \to H$ é um isomorfimos de grupos, então $G$ é comutativo se, e somente se, $H$ é comutativo.
        \end{proposicao}
    \end{frame}

    \begin{frame}
        \begin{exemplos}
            \vspace{.3cm}
            \begin{enumerate}
                \item[1)] Os grupos $\z_6$ e $S_3$ não são isomorfos pois $\z_6$ e comutativo e $S_3$ não é comutativo.

                \vspace{.3cm}

                \item[2)] Considere o grupo $S_6$ das permutações em $\{1, 2, \cdots, 6\}$. Tome
                \[
                    f = \begin{pmatrix}
                        1 & 2 & 3 & 4 & 5 & 6\\
                        2 & 3 & 4 & 5 & 6 & 1
                    \end{pmatrix} \in S_6.
                \]
                Seja $h = [f]$. Então $H \cong \z_6$, onde $\phi : H \to \z_6$ dada por $\phi(f) = \overline{1}$ é um isomorfimo de grupos.
                
                \vspace{.3cm}

            \end{enumerate}
        \end{exemplos}
    \end{frame}

    \begin{frame}
        \begin{proposicao}
            Sejam $G$ e $H$ grupos multiplicativos. Se $f : G \to H$ é um isomorfimos de grupos e se $x \in G$ é tal que $o(x) = h > 0$, então $o(f(x)) = h$.
        \end{proposicao}
    \end{frame}

    \begin{frame}
        Seja $G = [a]$ um grupo cíclico. Dois casos podem ocorrer:

        \textbf{Caso 1:} $a^r \ne a^s$ sempre que $r \ne s$.
    \end{frame}

    \begin{frame}
        \begin{proposicao}
            Se $G = [a]$ é um grupo cíclico que cumpre a condição do \textbf{Caso 1}, então a função $f : \z \to G$ por $f(r) = a^r$ é um isomorfimo de grupos. Ou seja, $G \cong \z$.
        \end{proposicao}
    \end{frame}

    \begin{frame}
        \textbf{Caso 2:} $a^r = a^s$ para algum par de inteiros distintos, $r$ e $s$.
    \end{frame}

    \begin{frame}
        \begin{proposicao}
            Seja $G = [a]$ um grupo cíclico que cumpre a condição do \textbf{Caso 2}. Então existe um inteiro $h > 0$ tal que
            \begin{enumerate}
                \item[i)] $a^h = e$

                \item[ii)] $a^r \ne e$, sempre que $0 < r < h$.
            \end{enumerate}
            Nesse caso, a ordem do grupo $G$ é $h$ e
            \[
                G = [a] = \{e, a, a^2, \cdots, a^{h - 1}\}.
            \]
        \end{proposicao}

        \begin{corolario}
            Seja $G = [a]$ um grupo cíclico de ordem finita igual a $m$. Então a função $f : \z_m \to G$ dada por $f(\overline{x}) = a^x$ é um isomorfimo de grupos.
        \end{corolario}
    \end{frame}
\end{document}