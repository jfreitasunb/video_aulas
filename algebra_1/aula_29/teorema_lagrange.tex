%!TEX program = xelatex
\def\ano{2020}
\def\semestre{1}
\def\disciplina{\'Algebra 1}
\def\turma{C}
\def\autor{Jos\'e Ant\^onio O. Freitas}
\def\instituto{MAT-UnB}

\documentclass{beamer}
\usetheme{Madrid}
\usecolortheme{beaver}
% \mode<presentation>
\usepackage{caption}
\usepackage{textpos}
\usepackage{amssymb}
\usepackage{amsmath,amsfonts,amsthm,amstext}
\usepackage[brazil]{babel}
% \usepackage[latin1]{inputenc}
\usepackage{graphicx}
\graphicspath{{/home/jfreitas/GitHub_Repos/video_aulas/logo/}{D:/Dropbox/imagens-latex/}}
\usepackage{enumitem}
\usepackage{multicol}
\usepackage{answers}
\usepackage{tikz,ifthen}
\usetikzlibrary{lindenmayersystems}
\usetikzlibrary[shadings]
\newtheorem{definicao}{Defini\c{c}\~ao}[section]
\newtheorem{definicoes}{Defini\c{c}\~oes}[section]
\newtheorem{exemplo}{Exemplo}[section]
\newtheorem{exemplos}{Exemplos}[section]
\newtheorem{exercicio}{Exerc{\'\i}cio}
\newtheorem{observacao}{Observa{\c c}{\~a}o:}[section]
\newtheorem{observacoes}{Observa{\c c}{\~o}es:}[section]
\newtheorem*{solucao}{Solu{\c c}{\~a}o:}
\newtheorem{proposicao}{Proposi\c{c}\~ao}
\newtheorem{lema}{Lema}
\newtheorem{teorema}{Teorema}
\newtheorem{corolario}{Corol\'ario}
\newenvironment{prova}[1][Prova]{\noindent\textbf{#1:} }{\qedsymbol}%{\ \rule{0.5em}{0.5em}}
\newcommand{\nsub}{\varsubsetneq}
\newcommand{\vaz}{\emptyset}
\newcommand{\im}{{\rm Im\,}}
\newcommand{\sub}{\subseteq}
\newcommand{\n}{\mathbb{N}}
\newcommand{\z}{\mathbb{Z}}
\newcommand{\rac}{\mathbb{Q}}
\newcommand{\real}{\mathbb{R}}
\newcommand{\complex}{\mathbb{C}}
\newcommand{\cp}[1]{\mathbb{#1}}
\newcommand{\ch}{\mbox{\textrm{car\,}}\nobreak}
\newcommand{\vesp}[1]{\vspace{ #1  cm}}
\newcommand{\compcent}[1]{\vcenter{\hbox{$#1\circ$}}}
\newcommand{\comp}{\mathbin{\mathchoice
{\compcent\scriptstyle}{\compcent\scriptstyle}
{\compcent\scriptscriptstyle}{\compcent\scriptscriptstyle}}}

\title{Teorema de Lagrange}
\author[\autor]{\autor}
\institute[\instituto]{\instituto}
\date{}

\begin{document}
    \begin{frame}
        \maketitle
    \end{frame}

    \logo{\includegraphics[scale=.1]{logo-MAT.png}\vspace*{8.5cm}}

    \begin{frame}
        Seja $G$ um grupo finito. \pause Se $H$ \'e um subgrupo de $G$, \pause ent\~ao existir\'a uma quantidade finita de classes laterais m\'odulo $H$ \pause.
        Assim o conjunto
        \[
            G/H = \{aH \mid a \in G\} \pause
        \]
        \'e finito. \pause

        O n\'umero de elementos de $G/H$ \pause \'e chamado de \textbf{{\'\i}ndice} \pause de $H$ em $G$ \pause e ser\'a denotado por
        \[
            [G : H] = |G/H|.
        \]
    \end{frame}

    \begin{frame}
        \begin{exemplos}
            \begin{enumerate}[label=({\arabic*})]
                \item Seja $G = \{1, -1, i, -i\}$ um grupo \pause e $N = \{1, -1\}$ \pause um subgrupo de $G$. \pause J\'a vimos que as classes laterais de $N$ em $G$ s\~ao \pause
                \[
                    N \quad \mbox{e}\quad iN. \pause
                \]
                Da{\'\i}
                \[
                    G/N = \{N, iN\} \pause
                \]
                e assim $[G : N] = 2$.

                \seti
            \end{enumerate}
        \end{exemplos}
    \end{frame}

    \begin{frame}
        \begin{exemplos}
            \begin{enumerate}[label=({\arabic*})]
                \conti

                \item Seja $G = S_3$. \pause J\'a vimos que se tomamos
                \[
                    Id = \begin{pmatrix}
                        1 & 2 & 3\\
                        1 & 2 & 3
                    \end{pmatrix}, \pause\quad
                    f = \begin{pmatrix}
                        1 & 2 & 3\\
                        2 & 3 & 1
                    \end{pmatrix} \pause \quad \mbox{e}\quad
                    g = \begin{pmatrix}
                        1 & 2 & 3\\
                        1 & 3 & 2
                    \end{pmatrix}
                \]
                ent\~ao
                \[
                    S_3 = \{Id, f, f^2, g, gf, gf^2\}. \pause
                \]
                Considere o subgrupo $H = [\ g\ ] \pause = \{Id, g\}$. Ent\~ao $H$ possui 3 classes laterais que s\~ao
                \[
                    H,\ fH,\ f^2H. \pause
                \]
                Da{\'\i}
                \[
                    G/H = \{H, fH, f^2H\} \pause
                \]
                e ent\~ao $[G : H] = 3$.
            \end{enumerate}
        \end{exemplos}
    \end{frame}

    \begin{frame}
        \begin{teorema}[Teorema de Lagrange]
            Seja $H$ um subgrupo \pause de um grupo finito $G$. \pause Ent\~ao $o(G) = o(H)[G:H]$ \pause e, portanto, $o(H) | o(G)$.
        \end{teorema}
    \end{frame}

    \begin{frame}
        \begin{observacao}
            No grupo $S_4$ \pause considere o seguinte subconjunto: \pause
            \[
                L = \left\{\begin{pmatrix}
                    1 & 2 & 3 & 4\\
                    1 & 2 & 3 & 4
                \end{pmatrix}, \pause \begin{pmatrix}
                    1 & 2 & 3 & 4\\
                    1 & 3 & 4 & 2
                \end{pmatrix}\right\}. \pause
            \]
            Observe que o n\'umero de elementos de L \pause divide $|S_4| = 4! = 24$ \pause mas $L$ n\~ao \'e um subgrupo de $S_4$ \pause pois
            \[
                \begin{pmatrix}
                    1 & 2 & 3 & 4\\
                    1 & 3 & 4 & 2
                \end{pmatrix}^{-1} \pause = \begin{pmatrix}
                    1 & 2 & 3 & 4\\
                    1 & 4 & 2 & 3
                \end{pmatrix} \pause \notin L.
            \]
        \end{observacao}
    \end{frame}

    \begin{frame}
        \begin{corolario}
            Seja $G$ um grupo finito. \pause Ent\~ao a ordem de um elemento $x \in G$ \pause divide a ordem de $G$ \pause e o quociente \'e $[G : H]$, \pause onde $H = [x]$.
        \end{corolario}
    \end{frame}

    \begin{frame}
        \begin{corolario}
            Sejam $G$ um grupo finito \pause e $x \in G$. \pause Ent\~ao
            \[
                x^{o(G)} \pause = e, \pause
            \]
            onde $e$ denota o elemento neutro de $G$.
        \end{corolario}
    \end{frame}

    \begin{frame}
        \begin{corolario}
            Seja $G$ um grupo finito \pause cuja ordem \'e um n\'umero primo. \pause Ent\~ao $G$ \'e um grupo c{\'\i}clico \pause e os \'unicos subgrupos de $G$ \pause s\~ao os triviais, \pause ou seja, $\{e\}$ e $G$.
        \end{corolario}
    \end{frame}
\end{document}
