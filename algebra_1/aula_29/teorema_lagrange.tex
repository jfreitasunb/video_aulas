%!TEX program = xelatex
%!TEX encoding = ISO-8859-1
\def\ano{2020}
\def\semestre{1}
\def\disciplina{\'Algebra 1}
\def\turma{C}
\def\autor{Jos\'e Ant\^onio O. Freitas}
\def\instituto{MAT-UnB}

\documentclass{beamer}
\usetheme{Madrid}
\usecolortheme{beaver}
% \mode<presentation>
\usepackage{caption}
\usepackage{textpos}
\usepackage{amssymb}
\usepackage{amsmath,amsfonts,amsthm,amstext}
\usepackage[brazil]{babel}
% \usepackage[latin1]{inputenc}
\usepackage{graphicx}
\graphicspath{{/home/jfreitas/GitHub_Repos/video_aulas/logo/}{D:/Dropbox/imagens-latex/}}
\usepackage{enumitem}
\usepackage{multicol}
\usepackage{answers}
\usepackage{tikz,ifthen}
\usetikzlibrary{lindenmayersystems}
\usetikzlibrary[shadings]
\newtheorem{definicao}{Defini\c{c}\~ao}[section]
\newtheorem{definicoes}{Defini\c{c}\~oes}[section]
\newtheorem{exemplo}{Exemplo}[section]
\newtheorem{exemplos}{Exemplos}[section]
\newtheorem{exercicio}{Exerc{\'\i}cio}
\newtheorem{observacao}{Observa{\c c}{\~a}o:}[section]
\newtheorem{observacoes}{Observa{\c c}{\~o}es:}[section]
\newtheorem*{solucao}{Solu{\c c}{\~a}o:}
\newtheorem{proposicao}{Proposi\c{c}\~ao}
\newtheorem{lema}{Lema}
\newtheorem{teorema}{Teorema}
\newtheorem{corolario}{Corol\'ario}
\newenvironment{prova}[1][Prova]{\noindent\textbf{#1:} }{\qedsymbol}%{\ \rule{0.5em}{0.5em}}
\newcommand{\nsub}{\varsubsetneq}
\newcommand{\vaz}{\emptyset}
\newcommand{\im}{{\rm Im\,}}
\newcommand{\sub}{\subseteq}
\newcommand{\n}{\mathbb{N}}
\newcommand{\z}{\mathbb{Z}}
\newcommand{\rac}{\mathbb{Q}}
\newcommand{\real}{\mathbb{R}}
\newcommand{\complex}{\mathbb{C}}
\newcommand{\cp}[1]{\mathbb{#1}}
\newcommand{\ch}{\mbox{\textrm{car\,}}\nobreak}
\newcommand{\vesp}[1]{\vspace{ #1  cm}}
\newcommand{\compcent}[1]{\vcenter{\hbox{$#1\circ$}}}
\newcommand{\comp}{\mathbin{\mathchoice
{\compcent\scriptstyle}{\compcent\scriptstyle}
{\compcent\scriptscriptstyle}{\compcent\scriptscriptstyle}}}

\title{Teorema de Lagrange}
\author[\autor]{\autor}
\institute[\instituto]{\instituto}
\date{}

\begin{document}
    \begin{frame}
        \maketitle
    \end{frame}

    \logo{\includegraphics[scale=.1]{logo-MAT.png}\vspace*{8.5cm}}

    \begin{frame}
        Seja $G$ um grupo finito. Se $H$ é um subgrupo de $G$, então existirá uma quantidade finita de classes laterais módulo $H$.
        Assim o conjunto
        \[
            G/H = \{aH \mid a \in G\}
        \]
        é finito.

        O número de elementos de $G/H$ é chamada de \textbf{índice} de $H$ em $G$ e será denotado por
        \[
            [G : H] = |G/H|.
        \]
    \end{frame}

    \begin{frame}
        \begin{exemplos}
            \begin{enumerate}[label=({\arabic*})]
                \item Seja $G = \{1, -1, i, -i\}$ um grupo \pause e $N = \{1, -1\}$. Já vimos que as classes laterais de $N$ em $G$ são
                \[
                    N \quad \mbox{e}\quad iN.
                \]
                Daí
                \[
                    G/N = \{N, iN\}
                \]
                e assim $[G : H] = 2$.

                \seti
            \end{enumerate}
        \end{exemplos}
    \end{frame}
    
    \begin{frame}
        \begin{exemplos}
            \begin{enumerate}[label=({\arabic*})]
                \conti

                \item Seja $G = S_3$. \pause J\'a vimos que se tomamos
                \[
                    Id = \begin{pmatrix}
                        1 & 2 & 3\\
                        1 & 2 & 3
                    \end{pmatrix}, \pause\quad
                    f = \begin{pmatrix}
                        1 & 2 & 3\\
                        2 & 3 & 1
                    \end{pmatrix} \pause \quad \mbox{e}\quad
                    g = \begin{pmatrix}
                        1 & 2 & 3\\
                        1 & 3 & 2
                    \end{pmatrix}
                \]
                ent\~ao
                \[
                    S_3 = \{Id, f, f^2, g, gf, gf^2\}. \pause
                \]
                Considere o subgrupo $H = [\ g\ ] \pause = \{Id, g\}$. Então $H$ possui 3 classes laterais que são
                \[
                    H,\ fH,\ f^2H. 
                \]
                Daí
                \[
                    G/H = \{H, fH, f^2H\}
                \]
                e então $[G : H] = 3$.
            \end{enumerate}
        \end{exemplos}
    \end{frame}

    \begin{frame}
        \begin{teorema}
            Seja $H$ um subgrupo de um grupo finito $G$. Então $o(G) = o(H)[G:H]$ e, portanto, $o(H) | o(G)$.
        \end{teorema}
    \end{frame}
\end{document}