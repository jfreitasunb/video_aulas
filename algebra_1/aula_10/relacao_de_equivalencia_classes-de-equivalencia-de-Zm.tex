%!TEX program = xelatex
% !TEX encoding = ISO-8859-1
\def\ano{2020}
\def\semestre{1}
\def\disciplina{\'Algebra 1}
\def\turma{C}
\def\autor{Jos\'e Ant\^onio O. Freitas}
\def\instituto{MAT-UnB}

\documentclass{beamer}
\usetheme{Madrid}
\usecolortheme{beaver}
% \mode<presentation>
\usepackage{caption}
\usepackage{textpos}
\usepackage{amssymb}
\usepackage{amsmath,amsfonts,amsthm,amstext}
\usepackage[brazil]{babel}
% \usepackage[latin1]{inputenc}
\usepackage{graphicx}
\graphicspath{{/home/jfreitas/GitHub_Repos/video_aulas/logo/}{D:/Dropbox/imagens-latex/}}
\usepackage{enumitem}
\usepackage{multicol}
\usepackage{answers}
\usepackage{tikz,ifthen}
\usetikzlibrary{lindenmayersystems}
\usetikzlibrary[shadings]
\newtheorem{definicao}{Defini\c{c}\~ao}[section]
\newtheorem{definicoes}{Defini\c{c}\~oes}[section]
\newtheorem{exemplo}{Exemplo}[section]
\newtheorem{exemplos}{Exemplos}[section]
\newtheorem{exercicio}{Exerc{\'\i}cio}
\newtheorem{observacao}{Observa{\c c}{\~a}o:}[section]
\newtheorem{observacoes}{Observa{\c c}{\~o}es:}[section]
\newtheorem*{solucao}{Solu{\c c}{\~a}o:}
\newtheorem{proposicao}{Proposi\c{c}\~ao}
\newtheorem{lema}{Lema}
\newtheorem{teorema}{Teorema}
\newtheorem{corolario}{Corol\'ario}
\newenvironment{prova}[1][Prova]{\noindent\textbf{#1:} }{\qedsymbol}%{\ \rule{0.5em}{0.5em}}
\newcommand{\nsub}{\varsubsetneq}
\newcommand{\vaz}{\emptyset}
\newcommand{\im}{{\rm Im\,}}
\newcommand{\sub}{\subseteq}
\newcommand{\n}{\mathbb{N}}
\newcommand{\z}{\mathbb{Z}}
\newcommand{\rac}{\mathbb{Q}}
\newcommand{\real}{\mathbb{R}}
\newcommand{\complex}{\mathbb{C}}
\newcommand{\cp}[1]{\mathbb{#1}}
\newcommand{\ch}{\mbox{\textrm{car\,}}\nobreak}
\newcommand{\vesp}[1]{\vspace{ #1  cm}}
\newcommand{\compcent}[1]{\vcenter{\hbox{$#1\circ$}}}
\newcommand{\comp}{\mathbin{\mathchoice
{\compcent\scriptstyle}{\compcent\scriptstyle}
{\compcent\scriptscriptstyle}{\compcent\scriptscriptstyle}}}

\title{Rela\c{c}\~ao de Equival\^encia - Classes de Equivalência nos Inteiros}
\author[\autor]{\autor}
\institute[\instituto]{\instituto}
\date{\today}

\begin{document}
    \begin{frame}
        \maketitle
    \end{frame}

    \logo{\includegraphics[scale=.1]{logo-MAT.png}\vspace*{8.5cm}}

    \begin{frame}
        \begin{definicao}
            Seja $C$ uma classe de equival{\^e}ncia de uma rela{\c c}{\~a}o de equival{\^e}ncia $R$. Qualquer elemento $y\in C$ {\'e} chamado \textbf{representante} de $C$.
        \end{definicao}

        \begin{proposicao}
            Seja $A$ um conjunto n{\~a}o vazio e $R$ uma rela{\c c}{\~a}o de equival{\^e}ncia em $A$. Ent{\~a}o $A$ {\'e} a uni{\~a}o disjunta das classes $\overline{b}$, $b \in A$, ou seja,
            \[
                A = \bigcup_{b\in A}\overline{b}.
            \]
        \end{proposicao}
        \begin{prova}
            Para todo $b\in A$ temos, pela defini\c{c}\~ao de classe de equival\^encia, que $\overline{b}\subseteq A$. Logo $\bigcup_{b\in A}\overline{b}\subseteq A$. Agora seja $x\in A$. Logo $x \in \overline{x}$ e da{\'\i} $x\in \bigcup_{b\in A}\overline{b}$. Assim $A\subseteq\bigcup_{a\in A}\overline{a}$. Portanto, $A=\bigcup_{b\in A}\overline{b}$.
        \end{prova}
    \end{frame}

    \begin{frame}
        \begin{definicao}
            Sejam $a$, $b \in \z$, $b \neq 0$. Dizemos que $b$ \textbf{divide} $a$ quando existe um inteiro $k$ tal que $a=bk$.
            Nesse caso escrevemos $b \mid a$. Quando $b$ \textbf{n{\~a}o divide} $a$, escrevemos $b\not{\mid}a$.
        \end{definicao}

        \begin{exemplos}
            \begin{enumerate}[label={\arabic*})]
                \item Os inteiros 1 e $-1$ dividem qualquer n{\'u}mero inteiro $a$, pois $a = 1 a$ e $a = (-1)(-a)$.
                \item O n{\'u}mero 0 n{\~a}o divide nenhum inteiro $b$, pois n{\~a}o existe $a \in \z$ tal que $b = 0a$.
                \item Para todo $b\neq 0$, $b$ divide $\pm b$.
                \item Para todo inteiro $b\neq 0$, $b$ divide 0, pois $0 = b0$.
                \item $3 \not{\mid} 8$.
                \item $17 \mid 51$.
            \end{enumerate} 
        \end{exemplos}
    \end{frame}

    \begin{frame}
        \begin{proposicao}
            \begin{enumerate}[label={\roman*})]
                \item $a\mid a$, para todo $a \in \z$.
                \item Se $a\mid b$ e $b\mid a$, $a$, $b > 0$ ent\~ao $a = b$.
                \item Se $a\mid b$ e $b\mid c$, ent{\~a}o $a\mid c$.
                \item Se $a\mid b$ e $a\mid c$, ent{\~a}o $a\mid (bx+cy)$, para todos $x$, $y \in \z$.
            \end{enumerate}
        \end{proposicao}
        \begin{enumerate}
            \item[i)] Imediata.
            
            \item[ii)] De fato, existem $k$, $l \in \z $ tais que $b = ka$ e $a = lb$. Assim $b = klb$, isto \'e, $b(1 - kl) = 0$.
            Como $b \ne 0$ ent\~ao $1 - kl = 0$. Da{\'\i} $kl = 1$ e ent\~ao $k = \pm 1$ e $l = \pm 1$. Mas $a > 0$ e $b > 0$, logo $k = l =1$. Logo $a = b$.

            \item[iii)] De fato, existem $k$, $l \in \z$ tais que $b = ka$ e $c = bl$. Assim  $c = kal = (kl)a$, ou seja, $a\mid c$.
        \end{enumerate}
    \end{frame}
    \begin{frame}
        \begin{enumerate}
            \item[iv)] Temos $b = ka$ e $c = al$, com $k$, $l \in \z$. Da{\'\i} $bx + cy = (ka)x + (al)y = a(kx + ly)$ e como $kx + ly \in \z$ segue que $a \mid (bx + cy)$.
        \end{enumerate}

        \begin{definicao}
            Sejam $a$, $b \in\z$, dizemos que $a$ \textbf{{\'e} congruente \`a} $b$ \textbf{m{\'o}dulo} $m$ se $m \mid (a-b)$. Neste caso, escrevemos $a\equiv_{m} b$ ou $a\equiv b \pmod{m}$.
        \end{definicao}

        \begin{exemplos}
            \begin{enumerate}[label={\arabic*})]
                \item $5\equiv 2 \pmod{3}$, pois $3 \mid (5-2)$.
                \item $3\equiv 1 \pmod{2}$, pois $2\mid (3-1)$.
                \item $3\equiv 9 \pmod{6}$, pois $6\mid (3-9)$.
            \end{enumerate} 
        \end{exemplos}
    \end{frame}

    \begin{frame}
        \begin{proposicao}
            A congru{\^e}ncia m{\'o}dulo $m$ {\'e} uma rela{\c c}{\~a}o de equival{\^e}ncia em $\z$.
        \end{proposicao}
        \begin{prova}
            \begin{enumerate}[label={\roman*})]
                \item Para todo $a \in \z$, $a\equiv a\pmod{m}$ pois $m\mid (a-a)$.
                \item Se $a\equiv b\pmod{m}$, ent{\~a}o $m\mid (a - b)$. Da{\'\i} existe $k \in \z$, tal que $(a - b) = km$. Agora, $(b - a) = -(a - b) = -(km) = (-k)m$, ou seja, $m \mid (b - a)$. Da{\'\i} $b\equiv a \pmod{m}$.
                \item Se $a\equiv b\pmod{m}$ e $b\equiv c\pmod{m}$, ent{\~a}o $m\mid (a-b)$ e $m\mid (b-c)$. Assim, $m\mid [(a-b)+(b-c)]$. Logo, $m\mid (a-c)$, isto {\'e}, $a\equiv c\pmod{m}$.
            \end{enumerate}

            Portanto a congru{\^e}ncia m{\'o}dulo $m$ {\'e} uma rela{\c c}{\~a}o de equival{\^e}ncia.
        \end{prova}
    \end{frame}
    \begin{frame}
        \begin{teorema}
            A rela{\c c}{\~a}o de congru{\^e}ncia m{\'o}dulo $m$ satisfaz as seguintes propriedades:
            \begin{enumerate}[label={\roman*})]
                \item $a_{1}\equiv b_{1}\pmod{m}$ se, e somente se, $a_{1}-b_{1}\equiv 0\pmod{m}$.
                \item Se $a_{1}\equiv b_{1}\pmod{m}$ e $a_{2}\equiv b_{2}\pmod{m}$, ent{\~a}o $a_{1}+a_{2}\equiv b_{1}+b_{2}\pmod{m}$.
                \item Se $a_{1}\equiv b_{2}\pmod{m}$ e $a_{2}\equiv b_{2}\pmod{m}$, ent{\~a}o $a_{1}a_{2}\equiv b_{1}b_{2}\pmod{m}$.\label{item_provado}
                \item Se $a\equiv b\pmod{m}$, ent{\~a}o $ax\equiv bx\pmod{m}$, para todo $x \in \z$.
                \item Vale a lei do cancelamento: se $d \in \z$ e $mdc(d,m) = 1$ ent{\~a}o $ad \equiv bd \pmod{m}$ implica $a\equiv b \pmod{m}$.
            \end{enumerate}
        \end{teorema}
    \end{frame}
    \begin{frame}
        \begin{prova}
            Provemos o item \ref{item_provado}.
            
            Como $a_{1}\equiv b_{2}\pmod{m}$ e $a_{2}\equiv b_{2}\pmod{m}$, existem $k$, $l \in \z$ tais que
            \begin{align*}
                a_1 - b_1 &= km\\
                a_2 - b_2 &= lm,
            \end{align*}
            isto \'e,
            \begin{align*}
                a_1 &= b_1 + km\\
                a_2 &= b_2 + lm,
            \end{align*}
            Assim
            \begin{align*}
                a_1a_2 &= (b_1 + km)(b_2 + lm) \\ &= b_1b_2 + b_1lm + b_2km + klm^2 \\ &= b_1b_2 + \underbrace{(lb_{1}+kb_{2}+klm)}_{\in \z}m
            \end{align*}
            
            Ou seja, $a_1a_2 - b_1b_2 = pm$, onde $p = lb_1 + kb_2 + klm \in \z$. Portanto, $a_1a_2 \equiv b_1b_2 \pmod{m}$.
        \end{prova}
    \end{frame}
\end{document}